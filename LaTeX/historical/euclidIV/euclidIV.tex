\documentclass{article}
\usepackage{amsthm,amsmath,amssymb}
\begin{document}
\title{Book IV of Euclid's {\em Elements} and ancient Greek mosaics}
\author{Jordan Bell}
\date{August 18, 2016}

\maketitle

\section{Introduction}
Book IV of Euclid is attributed to the Pythagoreans, by the Scholion to Euclid 273.3. See Burkert \cite[p.~450]{burkert}

Mueller \cite{mueller}, Chapter 5.






\section{Propositions from Books I--III}
\textbf{I.4}: ``If two triangles have the two sides equal to two sides
respectively, and have the angles contained by the equal straight
lines equal, they will also have the base equal to the base, the
triangle will be equal to the remaining angles respectively, namely those
which the equal sides subtend.''

\textbf{I.9}: ``To bisect a given rectilinear angle.''

\textbf{I.10}: ``To bisect a given finite straight line.''

\textbf{I.11}: ``To draw a straight line at right angles to a given straight
line from a given point on it.''

\textbf{I.12}: ``To a given infinite straight line, from a given point
which is not on it, to draw a perpendicular straight line.''

\textbf{I.13}: ``If a straight line set up on a straight line make angles, it
will make either two right angles or angles equal to two right
angles.''

\textbf{I.23}: ``On a given straight line and at a point on it to construct a
rectilinear angle equal to a given rectilinear angle.''

\textbf{I.26}: ``If two triangles have the two angles equal to two angles
respectively, and one side equal to one side, namely, either the 
side adjoining the equal angles, or that subtending one of the
equal angles, they will also have the remaining sides equal to
the remaining sides and the remaining angle to the remaining
angle.''

\textbf{I.32}: ``In any triangle, if one of the sides be produced, the exterior angle is equal
to the two interior and opposite angles, and the
three interior angles of the triangle are equal to two right angles.''

\textbf{I.36}: ``Parallelograms which are on equal bases and in the same parallels are equal to one another.''

\textbf{I.38}: ``Triangles which are on equal bases and in the same parallels are equal to one another.''

\textbf{I.41}: ``If a parallelogram have the same base with a triangle and
be in the same parallels, the parallelogram is double of the
triangle.''

\textbf{I.46}: ``On a given straight line to describe a square.''

\textbf{I.47}: ``In right-angled triangles the square on the side subtending 
the right angle is equal to the squares on the sides containing
the right angle.''

\textbf{II.11}: ``To cut a given straight line so that the rectangle contained by 
the whole and one of the segments is equal to the square on the remaining segment.''

\textbf{II.14}: ``To construct a square equal to a given rectilinear figure.''

\textbf{III.1}: ``To find the centre of a given circle.''

\textbf{III.16}: ``The straight line drawn at right angles to the diameter
of a circle from its extremity will fall outside the circle, and
into the space between the straight line and the circumference
another straight line cannot be interposed; further the angle
of the semicircle is greater, and the remaining angle less, than
any acute rectilinear angle.''

\textbf{III.18}: ``If a straight line touch a circle, and a straight line be
joined from the centre to the point of contact, the straight line
so joined will be perpendicular to the tangent.''

Book III, Definition 8: ``An \textbf{angle in a segment} is the angle which, when
a point is taken on the circumference of the segment and
straight lines are joined from it to the extremities of the
straight line which is the \textbf{base of the segment}, is contained 
by the straight lines so joined.''

\textbf{III.21}: ``In a circle the angles in the same segment are equal to one another.'' 

\textbf{III.31}: ``In a circle the angle in the semicircle is right, that in a 
greater segment less than a right angle, and that in a less segment 
greater than a right angle; and further the angle of
the greater segment is greater than a right angle, and the angle
of the less segment less than a right angle.'' This means that if $ABC$ is a semicircle and $AC$ is the diameter,
then the angle $ABC$ is right, whatever is the point $B$ on the semicircle.

\textbf{III.32}: ``If a straight line touch a circle, and from the point of
contact there be drawn across, in the circle, a straight line
cutting the circle, the angles which it makes with the tangent
will be equal to the angles in the alternate segments of the 
circle.'' This means that if $ABCD$ is a circle and a line $EF$ touches the circle at the point $B$,
then the angle $FBD$ is equal to the angle $BAD$ and the angle $EBD$ is equal to the angle $DCB$,
whatever are the points $A,C,D$ on the circle. 







\section{Book IV}
\textbf{IV.1}: ``Into a given circle to fit a straight line equal to a given
straight line which is not greater than the diameter of the 
circle.''

\textbf{IV.2}: ``In a given circle to inscribe a triangle equiangular with a given triangle.'' 

\begin{proof}
Let $ABC$ be the given circle and let $DEF$ be the given triangle. 
Using \textbf{III.16}, construct a line $GH$ touching the circle at $A$.

Then using \textbf{I.23}, 
on the line $AH$ at the point $A$ construct an angle $HAC$ equal to the angle $DEF$, and on the line
$GA$ at the point $A$ construct an angle $GAB$ equal to the angle $DFE$. 

Join $BC$. Because the line $GH$ touches the circle at $A$, by \textbf{III.32} the angle $HAC$ is equal to the angle $ABC$ and likewise
the angle $GAB$ is equal to the angle $ACB$. Hence the angle $ABC$ is equal to the angle $DEF$ and the angle
$ACB$ is equal to the angle $DFE$.

By \textbf{I.32}, the three interior angles of a triangle are equal to two right angles, which means
that the angles $ABC, ACD, BAC$ are equal to two right angles and the angles
$DEF,DFE,EDF$ are equal to two right angles. It follows that the angle $BAC$ is equal to the angle
$EDF$, which means that the triangles $ABC$ and $DEF$ are equiangular. 
\end{proof}

\textbf{IV.3}: ``About a given circle to circumscribe a triangle equiangular with a given triangle.''

\begin{proof}
Let $ABC$ be the given circle and let $DEF$ be the given triangle. Produce the line $EF$ in both directions to the points $G,H$. 
Let $K$ be the centre of the circle $ABC$ (\textbf{III.1}) and construct
a line $KB$, where $B$ is any point on the circle. Using
\textbf{I.23}, on the line $KB$ at the point $K$ construct the angle
$BKA$ equal to the angle $DEG$, and on the line 
$KB$ at the point $K$ construct the angle $BKC$ equal to the angle $DFH$.

Using \textbf{III.16}, construct lines $LAM,MBN,NCL$ touching the circle respectively at the points
$A,B,C$. Join $KA,KB,KC$. By \textbf{III.18}, the angle $KAM$ is right and the angle $KBM$ is right.






III.18, I.13, I.32.
\end{proof}

\textbf{IV.4}: ``In a given triangle to inscribe a circle.''

\begin{proof}

\end{proof}

\textbf{IV.5}: ``About a given triangle to circumscribe a circle.''

\begin{proof}

\end{proof}

\textbf{IV.6}: ``In a given circle to inscribe a square.''

\begin{proof}

\end{proof}

\textbf{IV.7}: 

\begin{proof}

\end{proof}







\section{Areas and lengths of equilateral and equiangular polygons}
{\em Meno} 82b--85b

VI.30

XII.1

XIII.1--12, XIII.18, Lemma

XIV.1, XIV.2, Lemma

Babylonian mathematics \cite{friberg}

Heron, {\em Metrica} I.17--25 \cite[pp.~190--]{metrica}. See Heath \cite[pp.~326--329]{HGMII}

{\em Metrica} I.17 calculates the area of an equilateral  triangle $AB\Gamma$ with side length $10$. 
Let $A\Delta$ be the perpendicular to $B\Gamma$, so $\Delta$ is the midpoint of $B\Gamma$. Then
$B\Delta A$ and $\Gamma \Delta A$ are congruent right triangles.
$AB=B\Gamma =2 B\Delta$,  so
$AB^2 = B\Gamma^2 = 4B\Delta^2$.
On the other hand,
by the Pythagorean theorem (\textbf{I.47}), the square on $B\Delta$ and the square
on $A \Delta$ are together equal to the square on $AB$. This means that
$B\Delta^2+A\Delta^2 = AB^2 = 4B\Delta^2$, hence
$A\Delta^2 = 3 B\Delta^2$.
Furthermore,  $B\Gamma^2=4B\Delta^2$, and thus
$B\Gamma^2 : A\Delta^2 = 4:3$ (an ``epitritic ratio'').
Then $A\Delta^2 : 100 = 3:4$, so $A\Delta^2 = 75$, and therefore $A\Delta = \sqrt{75}$. 
In fact, Heron does not calculate $A\Delta$, but from
$B\Gamma^2 : A\Delta^2 = 4:3$ gets 
$B\Gamma^4 : (B\Gamma \cdot A\Delta)^2 = 4:3$. (The fourth power of a line is called here {\em dunamodunamis}; cf. LSJ, s.v.)
The rectangle on the sides $B\Gamma,A\Delta$ is twice the triangle
$AB\Gamma$ (\textbf{I.41}), so 
$B\Gamma^4 : (2AB\Gamma)^2 = 4:3$ and hence
$AB\Gamma^2: B\Gamma^4  = 3:16$.
Then $AB{\Gamma}^2 = 1875$, so
$AB\Gamma = \sqrt{1875} \sim 43 \frac{1}{3}$. 

{\em Metrica} I.18



Archimedes, {\em Measurement of a Circle} \cite{archimedes}. $96$-gon.





Herz-Fischler \cite{herz-fischler}.

Oenopides \cite[p.~131, n.~4]{aristarchus} pentadecagon; cf. Burkert \cite[p.~453]{burkert}

Pappus, {\em Collection} IV.3 \cite[pp.~86--87]{pappusIV}







\section{Tiling and isoperimetry}
According to Proclus, {\em Commentary on the first book of Euclid's Elements} 304--305 \cite[p.~238]{proclus}, it was proved by the Pythagoreans that only triangles,
squares, hexagons can fill the space about a point.

Aristotle, {\em De caelo} III.8, 306b5--7 \cite[p.~177]{aristotle}:

\begin{quote}
And, speaking generally, the attempt to give figures to the simple
elements is irrational, first, because it will be found that they do
not fill the whole (of a space). For, among plane figures, it is agreed
that there are only three which fill up space, the triangle, the square,
and the hexagon; while among solids there are only the pyramid
and the cube.
\end{quote}

Potamo of Alexandria

Simplicius: On Aristotle On the Heavens 3.7--4.6

Heath \cite[pp.~206--213]{HGMII} on Zenodorus

Theon of Alexandria, {\em Commentary on Ptolemy's Syntaxis} I.3 \cite[pp.~386--391]{LCL362}:

\begin{quote}
``In the same way, since the greatest of the various
figures having an equal perimeter is that which has most angles, the circle is the greatest among plane
figures and the sphere among solid.''

We shall give the proof of these propositions in a summary taken from the proofs by Zenodorus
in his book {\em On Isoperimetric Figures}.

{\em Of all reailinear figures having an equal perimeter -- I mean equilateral and equiangular figures -- 
the greatest is that which has most angles.}
\end{quote}









\section{Platonic solids}
{\em Timaeus} 54c--d \cite[p.~212]{timaeus}: 

\begin{quote}
Now all triangles are derived from two, each having one right angle and the other angles acute...
\end{quote}

53d -- 54b \cite[pp.~213--214]{timaeus}: among scalene triangles, the best of them for the construction of bodies is that a pair of which is an equilateral triangle, which has ``the greater
side triple in square of the lesser''; among the isosceles triangles...

Constructs the tetrahedron, octahedron, icosahedron, and cube 54d--55c \cite[pp.~216--218]{timaeus}. 

57c--d \cite[p.~235]{timaeus}






\section{Galois theory}
If $\Omega$ is a set of points in a plane, let $\Lambda_\Omega$ be the set of lines through pairs of distinct points of $\Omega$
and let $\Gamma_\Omega$ be the set of circles whose center is a point of $\Omega$ and whose radius is a distance between two
distinct points of $\Omega$. For example, given distinct points $A,B$, let $\Omega=\{A,B\}$. Let $c_A$ be the circle with center $A$ and radius $AB$,
let $c_B$ be the circle with center $B$ and radius $BA$, and let $l$ be the line through $A$ and $B$.
Then $c_B \cap l = \{A\}$ and $c_A \cap l = \{B\}$, and
$c_A \cap c_B = \{C,D\}$, where $ABC$ and $ABD$ are equilateral triangles (\textbf{I.1}).
Let
\[
\Omega' = \bigcup_{l_1,l_2 \in \Lambda_\Omega, l_1 \neq l_2} (l_1 \cap l_2) \cup
\bigcup_{l \in \Lambda_\Omega, c \in \Gamma_\Omega} (l \cap c) \cup
 \bigcup_{c_1,c_2 \in \Gamma_\Omega, c_1 \neq c_2} (c_1 \cap c_2).
\]
It is straightforward to check that if at least two distinct points belong to $\Omega$ then $\Omega \subset \Omega'$. 
Define $\Omega_1 = \Omega'$, and by induction $\Omega_{n+1} = \Omega_n'$. 
Finally let $\Omega_\infty = \bigcup_{n \geq 1} \Omega_n$. Elements of $\Omega_\infty$ are said to be \textbf{constructible from
$\Omega$}.

Let $A=0$ and let $B=1$ in $\mathbb{C}$ and let $\Omega=\{A,B\}$. 
For $\alpha$ in the algebraic closure of $\mathbb{Q}$, let $m$ be the minimal polynomial of $\alpha$ over $\mathbb{Q}$ and let
$L$ be the splitting field of $m$ over $\mathbb{Q}$. 
It can be proved that
$\alpha \in \Omega_\infty$ if and only if $[L:\mathbb{Q}]$ is a power of $2$ \cite[p.~263, Theorem 10.1.12]{cox}.
One proves that a 
regular and equiangular $n$-gon can be constructed from $\Omega$ if and only if
$\zeta_n=e^{2\pi i/n}$ belongs to $\Omega_\infty$.
The minimal polynomial of $\zeta_n$ over $\mathbb{Q}$ is the cyclotomic polynomial
$\Phi_n$, which has degree $\phi(n)$, where $\phi$ is the Euler phi function, and the splitting field of $\Phi_n$ over $\mathbb{Q}$
is $\mathbb{Q}(\zeta_n)$. 
Thus, a regular and equiangular $n$-gon can be constructed from
$\Omega$ if and only if $[\mathbb{Q}(\zeta_n):\mathbb{Q}]=\phi(n)$ is a power of $2$.
For $n=2^{a} p_1^{a_1} \cdots p_k^{a_k}$,
\begin{align*}
\phi(n) &= \phi(2^a) \phi(p_1^{a_1}) \cdots \phi(p_k^{a_k})\\
&= 2^{a-1} p_1^{a_1-1}(p_1-1) \cdots p_k^{a_k-1} (p_k-1),
\end{align*}
which is a power of $2$ if and only if $a_j=1$, $j=1,\ldots,k$ and $p_j-1$ is a power of $2$, $j=1,\ldots,k$. 
For a prime $p$ to be such that $p-1$ is a power of $2$ it is necessary that $p=2^{2^m}+1$ for some $m \geq 0$; a prime
of this form is called a \textbf{Fermat prime}.  Therefore, an equilateral and equiangular $n$-gon can be constructed from
$\Omega$ if and only if $n$ is the product of a power of $2$ and distinct Fermat primes. 
The known Fermat primes are $3,5,17,257,65537$. Thus
\[
\zeta_3, \zeta_4, \zeta_5, \zeta_6, \zeta_8, \zeta_{10}, \zeta_{12}, \zeta_{15}, \zeta_{16}, \zeta_{17} \in \Omega_\infty
\]
and 
\[
\zeta_7, \zeta_9, \zeta_{11}, \zeta_{13}, \zeta_{14} \not \in \Omega_\infty.
\]
See Cox \cite{cox}






\section{Nature and artifacts}
Theophrastus, {\em De Lapidibus} III.19 \cite[p.~63]{theophrastus}: ``Again, the stone from the neighbourhood
of Miletus, which is angular and contains hexagons, does not burn.''

Pliny, {\em Natural History} XXXVII.20 \cite[p.~225]{LCL419}:
``Many people consider the nature of beryls to
be similar to, if not identical with, that of emeralds.
Beryls are produced in India and are rarely found
elsewhere. All of them are cut by skilled craftsmen
to a smooth hexagonal shape, since their colour,
which is deadened by the dullness of an unbroken
surface, is enhanced by the reflection from the facets.''

Pliny, {\em Natural History} XXXVII.15 \cite[p.~225]{LCL419} describes octahedral diamonds.






Aratus, {\em Phaenomena} 541--543: inscribe a hexagon in the ecliptic circle. (By Euclid \textbf{IV.15}, 
the side of the hexagon equals the radius of the circle.) Each side subtends two constellations on the zodiac. 
cf. 184--185 on an equilateral triangle.









Apis mellifera honeycomb, hexagonal. In {\em De Re Rustica} III.XVI.4--5 \cite[p.~501]{varro}, after stating that nature has given great talent and art to bees, Varro writes:
 
\begin{quote}
Bees are not of a solitary nature, as eagles are, but are like human beings. Even if jackdaws in this respect are the same, still it is not the same case; for in one there is a fellowship
in toil and in building which does not obtain in the other; in the one case there is reason and skill -- it is from these that men learn to toil, to build, to store up food. They have three 
tasks: food, dwelling, toil; and the food is not the same as the wax, nor the honey, nor the dwelling. Does not the chamber in the comb have six angles, the same number as the bee 
has feet? The geometricians prove that this hexagon inscribed in a circular figure encloses the greatest amount of space.
\end{quote}

Pliny, {\em Natural History} XI.12 \cite[p.~451]{LCL353}: ``All the cells are hexagonal, each side being made by one of the bee's six feet.''

Pappus, {\em Collection} V.1--3 \cite[pp.~389--390]{HGMII} writes about why cells of honeycombs are hexagonal; cf.
{\em Collection} VIII, Proposition 19: in a given circle, to construct a hexagon with the same centre as the circle and six other hexagons congruent to the central hexagon, 
each of which has a common side with the central hexagon and has one side which is a chord of the circle. Thomas \cite[pp.~588--593]{LCL362}.








16th cent BC, National Archaeological Museum, Athens, 808, 811. Wooden hexagonal pyxis decorated with repouss\'e gold plates. Mycenae, Grave Circle A, Grave V.
Karo \cite[p.~143]{karo}.

1109 BC, Staatliche Museen zu Berlin, Vorderasiatisches Museum, VA 08255.
Achtseitiges Prisma mit Weihinschrift Tiglatpilesars I. von Assyrien

10th century BC. Kerameikos Archaeological Museum, cinerary urn amphora

1000 BC -- 950 BC, BM 1978,0701.9. Pottery neck-handled amphora.
Proto-Geometric. Attic. Concentric circles.

975 BC -- 950 BC, BM 1978,0701.8. Pottery neck-handled amphora. 
Proto-Geometric. Attic. Concentric circles.

850 BC -- 800 BC. National Archaeological Museum, Athens. A 00216. Double-handled amphora. From Athens, Kerameikos. By the Painter of Athens 216. 

Beginning of the 7th century BC. Archaeological Museum of Thera, Santorini, Inv. no. 1783, Theran amphora with geometric decoration,
Archaic cemetery of Ancient Thera.

691 BC, BM 91032. The Taylor Prism. Neo-Assyrian. Hexagonal clay prism.

673 BC -- 672 BC, BM 121005. The Esarhaddon Prism. Neo-Assyrian. Hexagonal clay prism.

594 BC -- 590 BC, BM G.4264. Athens, silver coin. Obverse: Amphora within circle. Reverse: Incuse square divided into eight triangles.

Dewing 1255, coin

Dewing 1257, coin

Dewing 1453, coin

BM 1886,0507.1

BM 1873,0702.1

BM 1872,0709.367

BM BNK,G.151

SNG v. Aulock 1429; SNG France 2346. Pitane (BC 350) AE 17

SNG M\"unchen 295 var.; SNG Copenhagen 558 var.; SNG von Aulock 1944 var. Erythrai (BC 480-420) Drachm

MMA 2005.278, 2nd-1st century BC. Gold and cabochon garnet ring, hexagonal bezel.

MMA L.2015.72.45, 2nd-1st century BC. Hellenistic Garnet Ring, hexagonal bezel, inscribed
circle.




\section{Mosaics}
Pliny XXXVI.184 \cite{LCL419}

ca. 3500 BC. Staatliche Museen zu Berlin, Vorderasiatisches Museum, ZA 2.19./08686.
Aufstellung des Vorderasiatischen Museums im Pergamonmuseum, Uruk-Raum.
Stiftsmosaik von einer Hoffassade im Inanna-Heiligtum.

Ovadiah \cite{ovadiah}

Dunbabin \cite{dunbabin}

Phrygian Gordion, pebble mosaic, ca. 800 BC. Salzmann \cite{salzmann}, Plate 3,1.

Olynthus, Bellerophon mosaic, first half of the fourth century BC. Salzmann \cite{salzmann}, Plate 13.

Eretria, House of Mosaics, Arimaspi mosaic, second third of the fourth century BC. Salzmann \cite{salzmann}, Plate 26

Pella, House of Dionysos, antechamber with pebble mosaic, last third of the fourth century BC. Salzmann \cite{salzmann},
Plate 37,1. 

Sicyon, pebble mosaic with centaurs, fourth century BC. Inscribed concentric circles in square.






\bibliographystyle{plain}
\bibliography{euclidIV}

\end{document}