\documentclass{article}
\usepackage{amsmath,amssymb}
\usepackage{hyperref}
\begin{document}
\title{Early instances of the martingale}
\author{Jordan Bell}
\date{March 30, 2016}

\maketitle

The following are some early instances of the concept and word martingale. I have not found the word used to talk about betting before
Casanova's autobiography. To push further, one should turn to works on the history of games of chance in Europe. 
It may be useful to suggest one way the term might have started to be used in betting. A martingale is used on a horse to stop the horse from raising its head higher than the rider
wishes. To play a martingale is to keep doubling one's bet each time one loses, perhaps fighting the feeling to run away in despair after losing, so
I suggest (and it would be dumb to take this as more than a suggestion) that playing a martingale connotes controlling oneself and following a system. The only work on the origin
of the word martingale in the mathematical literature  is by Mansuy.\footnote{Roger Mansuy,
{\em The Origins of the Word ``Martingale''},
\url{http://www.emis.ams.org/journals/JEHPS/juin2009/Mansuy.pdf}}


The 13th century fabliau {\em Saint Pierre et le Jongleur}.\footnote{{\em De Saint Piere et du Jougleur}, 
fabliau CXVII in Anatole de Montaiglon and 
Gaston Raynaud, {\em Recueil g\'en\'eral et complet des fabliaux des XIIIe et XIVe si\`ecles}, tome V, pp.~65--79. Translated
in John DuVal and Raymond Eichmann, {\em Fabliaux, Fair and Foul}, 1992, p.~131. See 
Kitty MacGillavry, {\em Le jeu de d\'es dans le fabliau de Saint Pierre}, 
Marche romane: cahiers de l'A.R.U.Lg \textbf{28} (1978), 175--179. See also \url{http://www.arlima.net/no/539}}
Satan assigns a minstrel to guard the souls in hell while he is away. Saint Peter 
comes to hell with three dice, a board, and gold, and wagers  gold for souls in some game of dice.
Saint Peter convinces the minstrel that to get even, he should  double the number of souls bet
each time, and eventually Saint Peter wins all the souls.\footnote{Thomas M. Kavanagh,
{\em Dice, Cards, Wheels: A Different History of French Culture}, pp.~42--43.}

Casanova, talking about a lover ``M. M.'':\footnote{Giacomo Casanova, {\em History of My Life},
translated by Willard R. Trask, p.~124, volume 4, chapter VII.}


\begin{quote}
She made me promise to go to the casino for money to play in parternship with her. I went there
and took all the gold I found, and, determinedly doubling my stakes according to the system known
as the martingale, I won three or four times a day during the rest of the Carnival.
I never lost the sixth card. If I had lost it, I should have been out of funds, which amounted
to two thousand zecchini.
\end{quote}

Later Cassnova writes:\footnote{Giacomo Casanova, {\em History of My Life},
translated by Willard R. Trask, p.~173, volume 4, chapter X.}

\begin{quote}
At this same time I was being ruined at cards. Playing by the martingale, I lost very large sums; urged
on by M. M. herself, I sold all her diamonds, leaving her in possession of only five hundred
zecchini. There was no more question of an elopement. I still played, but for small
stakes, dealing at casinos against poor players. Thus I waited for my luck to come back.
\end{quote}

The entries for MARTINGALE in the 1762 {\em Dictionnaire de l'Acad\'emie Fran\c{c}oise}. First:
\begin{quote}
MARTINGALE. s. f. Terme de man\'ege. Courroie qui tient par un bout \`a la sangle sous le ventre du cheval, \& 
par l'autre \`a la muserole, pour emp\^echer qu'il ne porte au vent.
\end{quote}
Second:
\begin{quote}
MARTINGALE, est aussi un terme de Jeu. {\em Jouer \`a la Martingale}, C'est jouer toujours tout ce 
qu'on a perdu.
\end{quote}

The French dramatist Joseph Servi\`eres writes in 1801 
{\em La martingale, ou Le secret de gagner au jeu}.

The Marquis de Condorcet in 1805:\footnote{M. de Condorcet, {\em El\'emens du calcul des probabilit\'es},
1805, p.~119}
\begin{quote}
Dans plusiers jeux, il arrive au contraire qu'un joueur augmente continuellement sa mise,
de mani\`ere \`a ce qu'un coup favorable le d\'edommage de tout ce qu'il auroit pu perdre dans le
coups pr\'ec\'edens; ce qui s'appelle {\em faire la martingale}. Si on suit le jeu de cette mani\`ere, m\^eme
en jouant contre un banquier qui a quelque avantage, et que le mise ou le nombre des coups
ne soit pas born\'e, on arrive \`a un r\'esultat du m\^eme genre que celui la question de P\'etersbourg. Mais
si le jeu est renferm\'e dans une certaine limite, on trouve seulement qu'en supposant les parties
li\'ees, le joueur qui augmente ainsi sa mise, change la nature du jeu, c'est-\`a-dire, qu'au 
lieu d'un jeu o\`u avec une mise m\'ediocre il avoit une probabilit\'e assez petite de gagner beaucoup,
il a au contraire une probabilit\'e tr\`es-grande de gagner peu en exposant une grande mise.
\end{quote}

Parisot, 1810.\footnote{S\'ebastien A. Parisot, {\em Trait\'e du Calcul conjectural}, 1810, p.~317, chapter V, ``De la martingale''.}

Lacroix in 1816.\footnote{Silvestre Fran\c{c}ois Lacroix, {\em Trait\'e \'el\'ementaire du calcul des probabilit\'es}, 1816,
p.~110, ``Ce que c'est que la {\em martingale}.}

Babbage in a paper read in 1820 and published in 1823.\footnote{Charles Babbage, {\em An Examination of some Questions connected with Games
of Chance}, Transactions of the Royal Society of Edinburgh \textbf{9} (1823), 153--177.}

De Morgan writes:\footnote{Augustus De Morgan, {\em A Budget of Paradoxes}, 1872, pp.~167--168.}
\begin{quote}
About 1830 was published, in the {\em Library of Useful Knowledge}, the tract on {\em Probability}, the joint work
of the late Sir John Lubbock and Mr. Drinkwater (Bethune). It is one of the best elementary openings on the subject.
A binder put my name on the outside (the work was anonymous) and the consequence was that nothing could drive
out of people's heads that it was written by me.
\end{quote}

In this book, Lubbock and Drinkwater write the following:\footnote{p.~17, article 26.}

\begin{quote}
One favourite scheme is so celebrated as to have acquired a particular name; it is called the Martingale, or Double
or Quits, and consists in doubling the last stake after every loss. In order that this may be
permanently successful, the player requires not only an immense capital, but an unlimited permission of staking.
\end{quote}

Charles Sanders Peirce, 1878:\footnote{C. S. Peirce, {\em Illustrations of the Logic of Science. The Doctrine of Chances}, 
The Popular Science Monthly \textbf{12} (1878), March issue, 604--615.}

\begin{quote}
It is an indubitable result of the theory of probabilities that every gambler, if he continues long enough, must ultimately be ruined. Suppose he tries the martingale, which some believe infallible, and which is, as I am informed, disallowed in the gambling-houses.
In this method of playing, he first bets say \$ 1; if he loses it he bets \$ 2; if he loses that he bets \$ 4; if he loses that he bets \$ 8; if he then gains he has lost 1 + 2 + 4 = 7,
and he has gained \$ 1 more; and no matter how many bets he loses, the first one he gains will make him \$ 1 richer than he was in the beginning. In that way,
he will probably gain at first; but, at last, the time will come when the run of luck is so against him that he will not have money enough to double, and must therefore let his bet go. This will {\em probably} happen before he has won as much as he had in the first place, so that this run against him will leave him poorer than he began; some time or other it will be sure to happen. It is true that there is always a possibility of his winning any sum the bank can pay, and we thus come upon a celebrated paradox that, though he is certain to be ruined, the value of his expectation calculated according to the usual rules (which omit this consideration) is large. But, whether a gambler plays in this way or any other, the same thing is true, namely, that if [he] plays long enough he will be sure some time to have such a run against him as to exhaust his entire fortune.
\end{quote}

Venn, 1876.\footnote{John Venn, {\em The Logic of Chance}, second ed., 1876, p.~369, chap. XIV, \S 13.}



\end{document}