% trylinearb.tex    Linear B font
\documentclass{article}
%%\documentclass[12pt]{article}
\usepackage{amsmath}
\usepackage{linearb}

\title{Minoan weights}
\author{Jordan Bell}
\date{March 4, 2017}

\begin{document}
\maketitle

Evans \cite[p.~650]{knossosIV2}, Heraklion Archaeological Museum:

\begin{quote}
Above the floor-level near the West end of the Fifteenth Magazine, and
evidently fallen from an area of the upper system near the North-West
Corner Entrance, was the remarkable stone weight reproduced in Fig.~635.

It is 42 centimetres ($16\frac{1}{2}$ inches) high, of the purple gypsum so much in
use in the last palatial Age, and is somewhat wedge-shaped above, with
a perforation 5-6 centimetres in diameter. It could thus be suspended 
from a rope. Upon both of its sub-triangular faces is an octopus in relief,
the tentacles in each case coiling over its square-cut sides.
\end{quote}

Evans \cite[p.~651]{knossosIV2}:

\begin{quote}
It is clearly a weight, and the tentacles
coiling over the whole surface had a practical
value in making it difficult, without detection,
to reduce its volume.
\end{quote}

Evans \cite[pp.~651--652]{knossosIV2}:

\begin{quote}
The fact that it scales
exactly 29,000 grammes shows that we have here to do with a `light talent'
of a peculiarly Egyptian type. This answers to a somewhat low version
of the Babylonian talent representing 60 {\em minas} of about 490 grammes. The
Knossian weight would itself answer to a {\em mina} of 483\textperiodcentered 33 grammes.

It further appears that the Palace Standard of which we have here the
evidence was approximately repeated in a series of copper ingots. Nineteen
of these were discovered by Professor Halbherr and the Italian Mission in
1903, carefully walled up in four basement compartments of the small
Palace of Hagia Triada, five of them presenting incised signs (Fig.~636).
The average weight of these ingots--which themselves do not greatly vary
in weight--is 29,131\textperiodcentered 6 grammes, and two of them weigh exactly 29
kilogrammes, the amount scaled by the Palace weight from Knossos.
\end{quote}

Evans \cite[p.~653]{knossosIV2}:

\begin{quote}
It is clear, indeed, that this form of ingot, with its sides incurved to
facilitate porterage, had a widespread currency in the Ancient World beyond
the sphere of Minoan enterprise. The large hoard of these found at Serra
Ilixi in Sardinia with inscribed signs had possibly a Cretan connexion, and
those found in Cyprus still come within the Minoan range. But they are 
also seen borne as tribute by Syrians and Nubians. The material of one
of the Hagia Triada specimens analysed by Professor Mosso was nearly
99 per cent. pure copper.
\end{quote}

Evans \cite[pp.~653--656]{knossosIV2}:

\begin{quote}
Numerous examples of smaller weights were discovered belonging to
the late palatial period. The typical shape was a disk, with sides in some
cases slightly rounded off, and the materials were steatite, limestone,
alabaster and, occasionally, lead. In many cases these were engraved
with circular signs of numeration, and it has thus been possible to place
together a consecutive series as shown in Fig.~638.

The larger of these, Fig.~638, {\em a}, of black steatite, found with a late
Palatial lamp of the same material, N.E. of the Palace, is exceptionally
marked with two larger circles, which, as the decimal system was in vogue,
may in each case stand for ten of the units, represented here by four of the
ordinary small circles [ ]. It would therefore be equivalent to 24 units.
The weight, as corrected to its full original volume, scales 1,567\textperiodcentered 47 grammes.
Its diameter is 11\textperiodcentered 5 cm. and height 6\textperiodcentered 5.

Divided by 24, this gives a unit of almost exactly 65\textperiodcentered 5 grammes
({\em c.}~1,008 grains). The unit thus arrived at represents 5 Egyptian gold
units of {\em c.}~13 grammes (=65 grammes). This is 10 Egyptian $\frac{1}{2}$ units of
6\textperiodcentered 5 grammes, the half being often used for calculation--the {\em drachm} as
opposed to the {\em stater}.

Next in gradation is the weight {\em b}, of the same black steatite material,
and also found in a late Palatial deposit. Five of the smaller circles are
here engraved [ ], representing similar 5 half units of 6\textperiodcentered 5 grammes.
The diameter of this weight is 8\textperiodcentered 2 cm., and its height 2\textperiodcentered 9.

The next weight {\em c}, of white limestone, found with the preceding,
presents 4 smaller engraved circles [ ], pointing to a unit of a little over
68 grammes. This answers to 5 Egyptian units of the somewhat full
weight of 13\textperiodcentered 67 grammes. The diameter of this is 6\textperiodcentered 6 cm., and height 3\textperiodcentered 1.

Finally, the small flat weight of coarse alabaster found above the floor-level
of the East Magazines has a single small ring on its upper surface
showing that it represents a unit. Its original weight was 5\textperiodcentered 92 grammes.

One flat disk of fine alabaster from the Palace site, marked with two
small circles, scaling 8\textperiodcentered 54 grammes, belongs to a different system, and clearly
answers to a light-weight Babylonian shekel, and a leaden weight of the
same shape, of 8\textperiodcentered 45 grammes, must be classed with this. Otherwise, the
correspondence of weights of this group, found in a late palatial association--as evidenced by the numbers they bear and the original amount
scaled by them--with the Egyptian gold units, may be regarded as
conclusively established. This might in itself be expected from the close
commercial relations in which Crete at this time stood to Egypt, so well
illustrated in the tombs of the Viziers of Thothmes III and his immediate
successors by the envoys of Keftiu and their offerings. A remarkable
discovery, however, described below, now shows that the Egyptian gold
unit was known in Crete by the very beginning of the Middle Minoan Age.

Amongst other forms of Minoan weights may be mentioned a bronze
ox-head--found in a Late Minoan association of the votive stratum of the
Diktaean Cave--weighing  73\textperiodcentered 62 grammes (1,136\textperiodcentered 5 grains), Fig.~639, apparently
representing 8 {\em kedets} of 9\textperiodcentered 2025 
grammes. Like an Egyptian example of the same form
from Tell-el-Amarna, it is filled with lead.

A haematite example of the well-known `sphendonoid' class--like sling
bullets slightly cut away on the side to enable them to stand--was found on a floor of a basement
room on the South front of the Palace. It lay beneath a wall of the
Re-occupation Period (Fig.~640), and its affinities must be sought in
examples in the same material from L. M. III tombs at Enkomi. It
weighs 12\textperiodcentered 6 grammes (195\textperiodcentered 5 grains), and apparently represents a somewhat
low Egyptian gold unit.
\end{quote}

Evans \cite[pp.~664--665]{knossosIV2}:

\begin{quote}
The true Minoan mediums of currency seem rather 
to have been bars and rings of precious metal. The gold rings, of which
a series was found at Mycenae, answer approximately to a mean weight
of about 8\textperiodcentered 7 grammes (135 grains), which looks like a slight reduction of
the Egyptian {\em kedet} system. That gold bars were in use is demonstrated
by a complete example found in a Cypro-Minoan tomb at Old Salamis,
weighing 72\textperiodcentered 12 grammes (1,113 grains) corresponding
to 8 Egyptian {\em kedets} of 9\textperiodcentered 025 grammes (139\textperiodcentered 125 grains).
A cut section, representing a fraction of such a bar
of pale gold or electrum (two and a half {\em kedets}) was
found at Mycenae (Fig.~653). Similar sections of the
silver bars of Saxon treasure hoards were known as
{\em skillings}. The Mycenae specimen is in fact a true
`{\em skilling}' or `shilling'.

That, however, the older standard in Crete was the
Egyptian has received fresh and striking confirmation
from a quite recent find made on or near the site of
Knossos. This is a `weight-seal' of solid gold and
scaling 12\textperiodcentered 25 grammes (189 grains), which brings it
within the normal limits of the Egyptian gold unit.
\end{quote}








Chadwick \cite[pp.~xi--xii]{chadwick}:

\begin{quote}
This is perhaps the point at which to say something about the Linear A
script, for although it lies strictly outside the scope of this book, some
references have to be made to it. Between the eighteenth and fifteenth
centuries B.C. the Cretans employed an indigenous script, which they used
both for keeping accounts and for dedicatory inscriptions. This was
patently the source from which Linear B was borrowed; indeed it is likely
that the Greeks began by borrowing Minoan scribes, who then adapted
their script to represent the Greek language. Thus we can understand
much of the content of the Linear A tablets; we know how the writing
system works and we can assign approximate values to most of the syllabic
signs. But although we know the meaning of a few words, it has so far
proved impossible to demonstrate convincingly what the underlying
language is. Further progress will depend largely on the discovery and
publication of more texts.
\end{quote}

Chadwick \cite[p.~4]{chadwick}:

\begin{quote}
At what 
date Greek reached the islands is unclear; Thucydides (1.4) talks of Cretan
supremacy in this area, and deduces archaeologically the presence of
Carians (i.e., the people whom he knew as the inhabitants of south-western
Anatolia) in the islands (1.8). Crete was occupied down to the fifteenth
century by people who did not speak Greek, for we have their language in
written form, and although we cannot securely identify it, there is no
doubt that it was not Greek. This is the
language of the clay tablets and other inscriptions in the Linear A script,
which has been found in many parts of Crete, and in traces in the Aegean
islands. The Cretans certainly established themselves outside Crete: the
islands of Keos (K\'ea, off Attica), Kythera (K{\'\i}thira, off Lakonia), Melos
(M{\'\i}los), Rhodes and above all Thera (Santor{\'\i}ni).
\end{quote}

Chadwick \cite[p.~50]{chadwick}:

\begin{quote}
The archaeological evidence for Minoan colonies in the Aegean islands
all dates to an earlier age, and there is no reason to suppose that by the end 
of the fifteenth century any of them were still controlled by Knossos. The
replacement of Minoan by Mycenaean imports, which is documented for
several sites, of course proves only that Greece replaced Crete as the
dominant power in the Aegean at this time, a fact also apparently
witnessed by the contemporary Egyptian monuments.
\end{quote}

Chadwick \cite[pp.~102--103]{chadwick}:

\begin{quote}
As explained in the Preface (p.~xiv), the units of the metric
systems, for which there were doubtless words, are represented in the
script by special signs. In some cases we can guess what these words were,
but since they are never written out syllabically we have no way of verifying
these guesses. The way in which these signs are used differs from that
found in the Minoan Linear A script, since there smaller amounts than
whole units are described by a complicated system of fractional signs. In
Linear B the signs (in some cases clearly the same signs) represent smaller
units, which are of course specified fractions of the major unit. But despite
the difference of usage, it would not be surprising to learn that the basic
units were the same, and we shall see a remarkable point of agreement
later on.

The weights are relatively simple, though we have too little evidence to
complete the lower end of the scale. The largest weight, used of things like
bronze, is a sign representing a balance, transcribed conventionally as \textsc{l}.
The largest unit of weight in use in the classical period was
called {\em talanton} (Latin {\em talentum}, hence our {\em talent}), and since this name too
means a balance, there can be little doubt that it was also the name of the
Mycenaean unit. This is divided into 30 \textsc{m}, the classical talent into 60 {\em minae};
but since the Mycenaean sign is plainly double we can safely call this a 
double-mina ({\em dimnaion}, a word which remained in use as a monetary unit
in Cyprus in classical times). The word {\em mna} (Latinized {\em mina}) is Semitic,
and the sexagesimal system, based on 60 parts to a unit, is also clearly
of Near Eastern origin.

The double-mina was then divided into quarters (\textsc{n}), and this in turn
probably into twelfths (\textsc{p}); the doubt is due to the fact that we find \textsc{p} 12
and \textsc{p} 20, but these are perhaps not reduced to the higher unit (as we often
quote a weight in pounds, and not as so many hundredweight, quarters,
etc.). There is at least one smaller weight, used to give quantites of saffron,
which cannot be certainly fitted to the system. The smaller weights are also
used for gold. The smaller weights of the classical system do not appear to
match: the {\em drachma} is one hundredth of the {\em mina}, the {\em obol} one sixth of the
{\em drachma}.
\end{quote}

\textlinb{\BPtalent} talent.
\textlinb{\BPwtd} double-mina.
\textlinb{\BPwtc} quarter.
 \textlinb{\BPwtd} =$\frac{1}{30}$\textlinb{\BPtalent}.
\textlinb{\BPwtc} =$\frac{1}{4}$\textlinb{\BPwtd}.

Chadwick \cite[pp.~103--104]{chadwick}:

\begin{quote}
Evans found at Knossos a number of weights, the largest
of which is a block of gypsum with octopus decoration. It has been 
disputed whether this is really a weight; if it is, its weight, quoted by Evans
at 29 kg, must be a talent. The more certain examples are flattish cylinders,
sometimes with markings on the top. A large one is marked with two large
circles flanked on each side by two small ones, and Evans plausibly
suggested that this was a notation for 24 units. This gave a unit of
approximately 65.5 g, and other weights could then be shown to fit this:
one with what is probably meant to be five circles weighs 327.02 g
($65 \times 5 = 327.5$), and another weighing 68 g is not far from the supposed
unit.

J. L. Caskey (1970) in his most instructive dig on Keos (or K\'ea) in the
Cyclades found, again in a Minoan context, a number of lead weights.
These too appear to fit the Knossos unit: one with two dots weighs 121.3 g
(if its original weight was 131 g this would represent two units), another
with eight dots weighs 517 g ($8 \times 65.5=524$), and there is an unmarked
example weighing 648.5 g, or close to ten times the unit.
\end{quote}

Chadwick \cite[p.~105]{chadwick}:

\begin{quote}
It is hard to resist the conclusion
that the 65.5 g unit is only one of a number of competing systems current in
the Minoan world, and it may not have been the one selected by the
Mycenaeans as their standard. As the difference between Linear A and
Linear B in the method of expressing weight shows, the units in the system
were probably rearranged, even if the basis remained constant, just as
when Britain introduced decimal coinage in 1972 the pound was kept, but
its relationship to the penny was changed.
\end{quote}



BM 1897,0401.1394, Enkomi, hematite weight,
Late Bronze Age, 1650 BC--1050 BC,
2.5 g.
Crewe U.180.

BM 1897,0401.1395, Enkomi, hematite weight, ``corresponding to three quarters of an Egyptian qedet'',
Late Bronze Age, 1650 BC--1050 BC,
6.63 g.
Crewe U.181.

BM 1897,0401.1396, Enkomi, hematite weight,
Late Bronze Age, 1650 BC--1050 BC,
18.9 g.
Crewe U.182.

BM 1897,0401.1397, Enkomi, hematite weight, ``corresponding to two Egyptian qedets'',
Late Bronze Age, 1650 BC--1050 BC,
18.86 g.
Crewe U.183.

BM 1897,0401.1398, Enkomi, hematite weight,
Late Bronze Age, 1650 BC--1050 BC,
27.1 g.
Crewe U.184.

BM 1897,0401.1399, Enkomi, hematite weight, ``corresponding to three Egyptian qedets'',
Late Bronze Age, 1650 BC--1050 BC,
28.12 g.
Crewe U.185.

BM 1897,0401.1400, Enkomi, hematite weight, ``corresponding to five Egyptian qedets'',
Late Bronze Age, 1650 BC--1050 BC,
46.65 g. 
Crewe U.186.

BM 1897,0401.1401, Enkomi, hematite weight,
Late Bronze Age, 1650 BC--1050 BC,
47.9 g.
Crewe U.187.

BM 1897,0401.1402, Enkomi, hematite weight, ``probably a subdivision of an eastern Mediterranean qedet'',
Late Bronze Age, 1650 BC--1050 BC,
3.3 g.
Crewe U.188.

BM 1897,0401.1564, Enkomi, hematite weight,
Late Bronze Age, 1650 BC--1050 BC,
26.2 g. 
Crewe U.189.

BM 1897,0401.1565, Enkomi, hematite weight,
Late Bronze Age, 1650 BC--1050 BC,
18.1 g.
Crewe U.190.

BM 1897,0401.1566, Enkomi, hematite weight,
Late Bronze Age, 1650 BC--1050 BC,
18 g. 
Crewe U.191.

BM 1897,0401.1567, Enkomi, hematite weight,
Late Bronze Age, 1650 BC--1050 BC,
7.4 g.
Crewe U.192.

BM 1897,0401.1568, Enkomi, hematite weight, ``probably a subdivision of an eastern Mediterranean qedet'',
Late Bronze Age, 1650 BC--1050 BC,
4.36 g.
Crewe U.193.

BM 1897,0401.1569, Enkomi, hematite weight, ``probably a subdivision of an eastern Mediterranean qedet'',
Late Bronze Age, 1650 BC--1050 BC,
2.23 g.
Crewe U.194.

BM 1897,0401.484, Enkomi, Tomb 92, hematite weight,
Late Bronze Age, 1650 BC--1050 BC,
5.07 g.
Crewe 92.31.

BM 1897,0401.485, Enkomi, Tomb 92, hematite weight,
Late Bronze Age, 1650 BC--1050 BC,
6.62 g.
Crewe 92.32.

BM 1897,0401.486, Enkomi, Tomb 92, hematite weight,
Late Bronze Age, 1650 BC--1050 BC,
10.01 g.
Crewe 92.33.

BM 1897,0401.487, Enkomi, Tomb 92, hematite weight,
Late Bronze Age, 1650 BC--1050 BC,
10.14 g.
Crewe 92.34.

BM 1897,0401.488, Enkomi, Tomb 92, hematite weight,
Late Bronze Age, 1650 BC--1050 BC,
18.72 g.
Crewe 92.35.

BM 1897,0401.368, Enkomi, Tomb 67, hematite weight,
Late Bronze Age, 1650 BC--1050 BC,
1.96 g.
Crewe 67.64.

BM 1897,0401.369, Enkomi, Tomb 67, hematite weight,
Late Bronze Age, 1650 BC--1050 BC,
3.45 g.
Crewe 67.65.

BM 1897,0401.370, Enkomi, Tomb 67, hematite weight,
Late Bronze Age, 1650 BC--1050 BC,
5.18 g.
Crewe 67.66.

BM 1897,0401.371, Enkomi, Tomb 67, hematite weight,
Late Bronze Age, 1650 BC--1050 BC,
6.78 g.
Crewe 67.67.

BM 1897,0401.372, Enkomi, Tomb 67, hematite weight,
Late Bronze Age, 1650 BC--1050 BC,
10.15 g.
Crewe 67.68.

BM 1897,0401.373, Enkomi, Tomb 67, hematite weight,
Late Bronze Age, 1650 BC--1050 BC,
19.08 g.
Crewe 67.69.

BM 1897,0401.1529, Enkomi, bronze scale pan,
Late Cypriot III, 1200 BC--1050 BC.
Crewe F.6.

BM 1897,0401.1530, Enkomi, bronze scale pan,
Late Cypriot III, 1200 BC--1050 BC.
Crewe F.7.

BM 1897,0401.1535, Enkomi, copper oxhide-shaped ingot,
Late Cypriot III, 1200 BC--1050 BC,
36.92 kg.
Crewe F.85.






Museum of Underwater Archaeology, Bodrum

Evans \cite{corolla}


Petruso \cite{petruso}








BM 1872,0315.45, Ialysus, 240 grains.




CAHII1, pp.~389--390:

\begin{quote}
Although money, in the sense of coinage, had not yet been 
developed, the system of barter had been simplified by the adoption of certain fixed and generally recognized media of exchange--gold, silver, copper, and grain--in terms of which
other trade goods could be priced with a fair degree of accuracy 
and consistency. Values in metal were usually expressed by weight, the units employed being the {\em deben}, a weight of about 91 grammes, its tenth part, the {\em kit\u{e}} (9\textperiodcentered 1 grammes), and a weight equal to $\frac{1}{12}$ {\em deben} (7\textperiodcentered 6 grammes) which modern scholars have
agreed to call a `piece'.
Since the last-named appears to have
been `a flat, round, piece of metal\dots possibly with an inscription to indicate' its `weight or the name of the issuing authority', it `was
practically a coin'. The ratio of $2:1$ for the values of gold and
silver seems to have remained fairly stable throughout the New Kingdom, dipping momentarily to $1\frac{2}{3}:1$ at the beginning of the reign of Amenophis II because of `the influx of large quantities of
gold as booty and tribute from Palestine and Syria, then recently
conquered'. Copper, with only $\frac{1}{100}$ the value of silver, is always quoted by the {\em deben}, 2 {\em deben} of copper being generally equivalent in value to 1 {\em khar}-sack (2 bushels), of corn which itself was used
as a form of currency. In the Eighteenth Dynasty 8 `pieces'
( $\frac{2}{3}$ {\em deben}) of silver or their equivalent in other commodities would
buy a bull or a cow or the service of a female slave for four days;
6 `pieces', a heifer or 3 arouras (2 acres) of (poor) land; and
$3\frac{1}{2}$ `pieces', a linen garment ({\em d3iw}) of good quality. Later in the
New Kingdom the same garment was priced at $13\frac{3}{4}$--20 {\em deben} of copper, a tunic at 5 {\em deben} of copper,
a calf at 30 {\em deben}, a prime
bull at 130 {\em deben}, and so on.
\end{quote}

CAHII1, p. 514:

\begin{quote}
The uniformity of shapes and techniques in metal-working
over a wide area of the eastern Mediterranean suggests that the industry was partly in the hands of itinerant craftsmen who travelled from place to place by donkey caravan or by ship. The
Bronze Age ship which was wrecked among the Be\c{s} Adalar islands 
off Cape Gelidonya in about 1200 B.C. was carrying not only a stock of ingots of copper and tin, but also an assortment of tools, agricultural implements and household utensils, which may have
been part of the stock-in-trade of itinerant tinkers.
The weights 
they were using were of the Egyptian {\em qedet} standard, which was
probably current throughout the Levant, since it was in use in
Ugarit, Crete and Cyprus, and on the Palestinian coast.
\end{quote}


Glotz \cite[pp.~191--192]{glotz}:

\begin{quote}
No trade  flourishes in ever so small a degree without a regular system of weights and measures. From very early times the  Mesopotamians and the Egyptians had several which crossed the sea and were adopted in the islands of the  \AE{}gean.

The ``Babylonian'' system, which was already in use among the Egyptians of the XlIth Dynasty, is a sexagesimal system, having as unit a light shekel weighing from 7\textperiodcentered 58 to 8\textperiodcentered 42 grams, or on an average 8 grams ; its multiples are the mina of 60 shekels, and the talent of 60 minas (about 28\textperiodcentered 8 kilograms). This
system spread to Crete. A magazine at Knossos contained a truncated pyramid of red limestone over the faces of which twined the tentacles of an octopus in relief. 
The weight of this object is 28\textperiodcentered 6 kilograms ; it is
perhaps the standard of the royal talent, and the reliefs marked on it may, like the stamp on coins, be intended to prevent any fraudulent debasing. The subdivisions of this talent certainly belong to a sexagesimal system, that is to say, one which is both decimal and duodecimal. A marble cylinder discovered at Siteia weighs 1,140 grams, one twentyfth of a talent of 28\textperiodcentered 5
kilograms, and twelve dozen, or a ``gross'', of shekels of 7\textperiodcentered 916 grams. Certain geese of h\ae{}matite or cornelian,
a form of weight well known on the banks of the Nile and in the East, weigh 167\textperiodcentered 18, 2\textperiodcentered 6, and 1\textperiodcentered 63 grams--a fact which seems to indicate a unit of 20 shekels with its sub-multiples of one sixtieth and one hundredth.
\end{quote}







Mycenaean numbers, Ventris \cite[p.~57]{ventris}

    The numbers 1 to 9: \textlinb{\BNi{} \BNii{} \BNiii{} \BNiv{} \BNv{} \BNvi{} \BNvii{} \BNviii{} \BNix}

    The numbers 10 to 90: \textlinb{\BNx{} \BNxx{} \BNxxx{} \BNxl{} \BNl{} \BNlx{} \BNlxx{} \BNlxxx{} \BNxc}

    The numbers 100 to 900: \textlinb{\BNc{} \BNcc{} \BNccc{} \BNcd{} \BNd{} \BNdc{} \BNdcc{} \BNdccc{} \BNcm}

    The number 1000: \textlinb{\BNm}
    


\newcommand{\fluids}{\BPvola\ \BPvolb\ \BPvolcd\ \BPvolcf}
The system of fluid measures: \textlinb{\fluids} 

\newcommand{\commodities}{\BPcloth\ \BPwool\ \BPwheat\ \BPbarley\
                         \BPwine\ \BPolive\ \BPbronze\ \BPgold}
Some commodities: \textlinb{\commodities} 

\newcommand{\menhorses}{\BPman\ \BPwoman\ \BPhorse\ \BPfoal}
Some people and animals: \textlinb{\menhorses}  

\newcommand{\livestock}{\BPpig\ \BPboar\ \BPsow\
                        \BPox\ \BPbull\ \BPcow\
                        \BPsheep\ \BPram\ \BPewe\
                        \BPgoat\ \BPbilly\ \BPnanny}
Some livestock: \textlinb{\livestock}  
                         
                             \newcommand{\weights}{\BPwta\ \BPwtb\ \BPwtc\ \BPwtd\ \BPtalent}
The system of weight measures: \textlinb{\weights} 
                         
\bibliographystyle{plain}
\bibliography{minoan}
    
\end{document}