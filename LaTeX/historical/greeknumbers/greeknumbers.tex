\documentclass{article}
\usepackage[polutonikogreek,english]{babel}
\newcommand{\Gk}[1]{\selectlanguage{polutonikogreek}#1\selectlanguage{english}}

\begin{document}
\title{Greek numbers}
\author{Jordan Bell}
\date{March 4, 2017}

\maketitle

One-tenth Ex. 16:36;
Nu. 18:26, 28; He. 7:1--10.
Two-tenths Lev. 23:13. Three tenths Lev. 14:10.
One-hundredths Neh 5:11.
One-third: appears fourteen times in Revelation, refers to one of three parts which was to be
destroyed Rev. 8.
One-third 2 S. 18:2, one-half Ex. 25:10, 17, one-fourth 1 S. 9:8,
one-fifth Gen 47:24, one-sixth Ezk. 46:14.

Two-thirds ``double portion'' Dt. 21:17, 2 K. 2:9,
four-fifths ``four parts'' Gen. 47:24, nine-tenths ``nine parts'' Neh 11:1.
 

Diogenes Laertius, {\em Vitae philosophorum} 7.58 \cite[p.~198]{LS1}:

\begin{quote}
According to Diogenes [of Babylon] an appellative [\Gk{proshgor'ia}] is a part of language
which signifies a common quality, e.g. `man', `horse'; a name [\Gk{>'onoma}] is a part of
language which indicates a peculiar quality, e.g. `Diogenes', `Socrates'; a
verb is a part of language which, according to Diogenes, signifies a non-compound
predicate, or, as some say, a case-less constituent of a sentence
which signifies something attachable to something or some things, e.g. `I
write', `I speak'. 
\end{quote}

Apollonius Dyscolus, {\em Syntax} 32.2 \cite{householder}

Dionysius of Halicarnassus, {\em De compositione verborum} chap.~II \cite[pp.~71--73]{comp}:

\begin{quote}
Composition is, as the very name indicates, a certain
arrangement of the parts of speech, or elements of diction,
as some call them. These were reckoned as three only by
Theodectes and Aristotle and the philosophers of those times,
who regarded nouns [\Gk{>on'omata}], verbs [\Gk{<p'hmata}] and connectives [\Gk{sund'esmous}]
as the primary
parts of speech. Their successors, particularly the leaders of
the Stoic school, raised the number to four, separating the
articles from the connectives. Then the later inquirers divided
the appellatives from the substantives, and represented the
primary parts of speech as five. Others detached the pronouns
from the nouns, and so introduced a sixth element. Others,
again, divided the adverbs [\Gk{>epirr'hmata}] from the verbs, the prepositions
from the connectives and the participles from the appellatives [\Gk{proshgorik~wn}];
while others introduced still further subdivisions, and so
multiplied the primary parts of speech. The subject would
afford scope for quite a long discussion. Enough to say that 
the combination or juxtaposition of these primary parts, be
they three, or four, or whatever may be their number, forms
the so-called ``members'' (or clauses) of a sentence. Further,
the fitting together of these clauses constitutes what are termed
the ``periods,'' and these make up the complete discourse. The
function of composition is to put words together in an appropriate
order, to assign a suitable connexion to clauses, and to distribute
the whole discourse properly into periods.
\end{quote}

Dionysius Thrax, {\em Tekhne} XI \cite[p.~23]{uhlig}, \cite[p.~176]{tekhne}:

\begin{quote}
There are eight parts of the sentence: noun [\Gk{>'onoma}], verb, participle, article, pronoun, preposition,
adverb [\Gk{>ep'irrhma}], conjunction. For the appellative [\Gk{proshgor'ia}] is a subspecies of the noun.
\end{quote}

Dionysius Thrax, {\em Tekhne} XII \cite[p.~33]{uhlig}, \cite[p.~178]{tekhne}:

\begin{quote}
There are the following subtypes of the noun (these also are referred to as `species'): proper, appellative [\Gk{proshgorik'on}], attached [\Gk{>ep'ijeton}],
relative, quasi-relative, homonymous, synonymous, dionymous, eponymous, ethnic, interrogative, indefinite, anaphoric (also referred to by the names `similative', `de?monstrative', and `correlative'), collective, distributive, inclusive, onomatopoeic, generic, specific, ordinal [\Gk{>arijmhtik'on}],
absolute, participatory.
\end{quote}

Dionysius Thrax, {\em Tekhne} XII, \cite[p.~44]{uhlig},  \cite[p.~180]{tekhne}:

\begin{quote}
\Gk{Taktik`on d'e >esti t`o t'axin dhlo~un, o~<ion pr~wtos de'uteros tr'itos.
>Arijmhtik`on d'e >esti t`o >arijm`on shma~inon, o~<ion e~<is d'uo tre~is.}

An ordinal noun is one which indicates order, such as `first, second,
third'. A numeral noun is one which signifies number, such as `one, two,
three'. 
\end{quote}

Dionysius Thrax, {\em Tekhne} XIX \cite[p.~72]{uhlig}, \cite[p.~183]{tekhne}:

\begin{quote}
An adverb [\Gk{>Ep'irrhm'a}] is a part of the sentence which is uninflected; it qualifies
verbs or is added to verbs.
\end{quote}

Dionysius Thrax, {\em Tekhne} XIX \cite[p.~76]{uhlig}, \cite[p.~184]{tekhne}:

\begin{quote}
\Gk{T`a d`e >arijmo~u dhlwtik'a, o~<ion d'is tr'is tetr'akis.}

Some signify number, for example {\em dis} (twice), {\em tris} (thrice), {\em tetrakis}
(four times). 
\end{quote}

onefold, twofold, threefold

firstly, secondly, thirdly

half, third, quarter, fifth

K\"uhner \cite[p.~621]{kuhner}, \S 181

cardinals, {\em cardinalia}, \Gk{on'omata arijmhtik'a}: answers \Gk{p'osoi}, ``how many?'', one, two, three, four

ordinals, {\em ordinalia}, \Gk{on'omata taktik'a}: answers \Gk{p'ostos}, ``which in order?'', first, second, third, fourth

numeral adverbs: answers ``how many times?'', once, twice, thrice, four times

multiplicative adverbs how many parts: answers ``into how many parts?''

substantive numerals: unit, pair, triply

multiplicatives, \Gk{pollaplasiastik'a arijmhtik'a}: the number of parts of which a whole is composed, answers ``how many fold?'', single, double, triple, quadruple

proportionals, \Gk{analogik'a arijmhtik'a}: answers ``how many times more?''

fractions: half, third, fourth

numeral adverbs: firstly, secondly, thirdly: \Gk{de'uteron, tr'iton}

five ways, six ways: \Gk{pentaq~ws, <exaq~ws}


 \Gk{posapl'asion}. {\em Meno} 83b \cite[p.~118]{waterfieldmeno}: ``How many times as big is it?''



Smyth Art. 347 \cite{smyth}:

\begin{tabular}{l | l | l | l | l}
&&cardinals&ordinals&numeral adverb\\
\hline
1&\Gk{a'}&\Gk{e~<is, m'ia, <'en}&\Gk{pr'wt-os, -h, -on}&\Gk{<'apax}\\
2&\Gk{b'}&\Gk{d'uo}&\Gk{de'uteros}&\Gk{d'is}\\
3&\Gk{g'}&\Gk{tre~is, tr'ia}&\Gk{tr'itos}&\Gk{tr'is}\\
4&\Gk{d'}&\Gk{t'ettares, t'ettara}&\Gk{t'etartos, -h, -on}&\Gk{tetr'akis}\\
5&\Gk{e'}&\Gk{p'ente}&\Gk{p'emptos}&\Gk{pent'akis}\\
6&\Gk{\stigma'}&\Gk{<'ex}&\Gk{<'ektos}&\Gk{<ex'akis}\\
7&\Gk{z'}&\Gk{<ept'a}&\Gk{<'ebdomos}&\Gk{<ept'akis}\\
8&\Gk{h'}&\Gk{>okt'w}&\Gk{>'ogdoos}&\Gk{>okt'akis}\\
9&\Gk{j'}&\Gk{>enn'ea}&\Gk{>'enatos}&\Gk{>en'akis}\\
10&\Gk{i'}&\Gk{d'eka}&\Gk{d'ekatos, -h, -on}&\Gk{dek'akis}\\
11&\Gk{ia'}&\Gk{<'endeka}&\Gk{<end'ekatos}&\Gk{<endek'akis}\\
12&\Gk{ib'}&\Gk{d'wdeka}&\Gk{dwd'ekatos}&\Gk{dwdek'akis}\\
13&\Gk{ig'}&\Gk{tre~is ka`i d'eka}&\Gk{tr'itos ka`i d'ekatos}&\Gk{treiskaidek'akis}\\
14&\Gk{id'}&\Gk{t'ettares ka`i d'eka}&\Gk{t'etartos ka`i d'ekatos}&\Gk{tettareskaidek'akis}\\
15&\Gk{ie'}&\Gk{penteka'ideka}&\Gk{p'emptos ka`i d'ekatos}&\Gk{pentekaidek'akis}\\
16&\Gk{i\stigma'}&\Gk{<ekka'ideka}&\Gk{<'ektos ka`i d'ekatos}&\Gk{<ekkaidek'akis}\\
17&\Gk{iz'}&\Gk{<eptaka'ideka}&\Gk{<'ebdomos ka`i d'ekatos}&\Gk{<eptakaidek'akis}\\
18&\Gk{ih'}&\Gk{>oktwka'ideka}&\Gk{>'ogdoos ka`i d'ekatos}&\Gk{>oktwkaidek'akis}\\
19&\Gk{ij'}&\Gk{>enneaka'ideka}&\Gk{>'enatos ka`i d'ekatos}&\Gk{>enneakaidek'akis}\\
20&\Gk{k'}&\Gk{e>'ikosi}&\Gk{e>ikost'os, -'h, -'on}&\Gk{e>ikos'akis}\\
21&\Gk{ka'}&\Gk{e~<is ka`i e>'ikosi}&\Gk{pr~wtos ka`i e>ikost'os}&\Gk{e>ikos'akis <'apax}\\
30&\Gk{l'}&\Gk{tri{\={a}}'konta}&\Gk{tri\={a}kost'os}&\Gk{tri\={a}kont'akis}\\
40&\Gk{m'}&\Gk{tettar'akonta}&\Gk{tettarakost'os}&\Gk{tettarakont'akis}\\
50&\Gk{n'}&\Gk{pent'hkonta}&\Gk{penthkost'os}&\Gk{penthkont'akis}\\
60&\Gk{x'}&\Gk{<ex'hkonta}&\Gk{<exhkost'os}&\Gk{<exhkont'akis}\\
70&\Gk{o'}&\Gk{<ebdom'hkonta}&\Gk{<ebdomhkost'os}&\Gk{<ebdomhkont'akis}\\
80&\Gk{p'}&\Gk{>ogdo'hkonta}&\Gk{>ogdohkost'os}&\Gk{>ogdohkont'akis}\\
90&\Gk{\qoppa'}&\Gk{>enen'hkonta}&\Gk{>enenhkost'os}&\Gk{>enenhkont'akis}\\
100&\Gk{r'}&\Gk{<ekat'on}&\Gk{<ekatost'os, -'h, -'on}&\Gk{<ekatont'akis}\\
200&\Gk{s'}&\Gk{di\={a}k'osi-oi, -ai, -a}&\Gk{di\={a}kosiost'os}&\Gk{di\={a}kosi'akis}\\
300&\Gk{t'}&\Gk{tri\={a}k'osi-oi, -ai, -a}&\Gk{tri\={a}kosiost'os}&\Gk{tri\={a}kosi'akis}\\
400&\Gk{u'}&\Gk{tetrak'osi-oi, -ai, -a}&\Gk{tetrakosiost'os}&\Gk{tetrakosi'akis}\\
500&\Gk{f'}&\Gk{pentak'osi-oi, -ai, -a}&\Gk{pentakosiost'os}&\Gk{pentakosi'akis}\\
600&\Gk{q'}&\Gk{<exak'osi-oi, -ai, -a}&\Gk{<exakosiost'os}&\Gk{<exakosi'akis}\\
700&\Gk{y'}&\Gk{<eptak'osi-oi, -ai, -a}&\Gk{<eptakosiost'os}&\Gk{<eptakosi'akis}\\
800&\Gk{w'}&\Gk{>oktak'osi-oi, -ai, -a}&\Gk{>oktakosiost'os}&\Gk{>oktakosi'akis}\\
900&\Gk{\sampi'}&\Gk{>enak'osioi}&\Gk{>enakosiost'os}&\Gk{>enakosi'akis}\\
1000&\textsubscript{,}{\Gk{a}}&\Gk{q{\={i}}'li-oi, -ai, -a}&\Gk{q\={i}liost'os, -'h, -'on}&\Gk{q\={i}li'akis}\\
2000&\textsubscript{,}{\Gk{b}}&\Gk{disq{\={i}}'li-oi, -ai, -a}&\Gk{disq{\={i}}liost'os}&\Gk{disq\={i}li'akis}\\
3000&\textsubscript{,}{\Gk{g}}&\Gk{trisq{\={i}}'li-oi, -ai, -a}&\Gk{trisq{\={i}}liost'os}&\Gk{trisq\={i}li'akis}\\
10000&\textsubscript{,}{\Gk{i}}&\Gk{m\={u}'ri-oi, -ai, -a}&\Gk{m\={u}riost'os}&\Gk{m\={u}ri'akis}\\
20000&\textsubscript{,}{\Gk{k}}&\Gk{dism\={u}'rioi}&\Gk{dism\={u}riost'os}&\Gk{dism\={u}ri'akis}\\
100000&\textsubscript{,}{\Gk{r}}&\Gk{dekakism\={u}'rioi}&\Gk{dekakism\={u}riost'os}&\Gk{dekakism\={u}ri'akis}
\end{tabular}

\begin{tabular}{lll}
1&\Gk{<'apax}&\\
2&\Gk{\d`is}&\Gk{d'iqa}\\
3&\Gk{tr'is}&\Gk{tr'iqa}
\end{tabular}

\begin{tabular}{ll}
&substantive numerals\\
1&\Gk{mon'as}\\
2&\Gk{du'as}\\
3&\Gk{tri'as}\\
4&\Gk{tetr'as}\\
5&\Gk{pent'as}\\
6&\Gk{<ex'as}\\
7&\Gk{<ebdom'as}\\
8&\Gk{>ogdo'as}\\
9&\Gk{>enne'as}\\
10&\Gk{dek'as}\\
11&\Gk{<endek'as}\\
12&\Gk{dodek'as}\\
20&\Gk{e>ik'as}\\
40&\Gk{tessarakont'as}\\
100&\Gk{<ekatont'as}\\
1000&\Gk{qili'as}\\
10000&\Gk{muri'as}
\end{tabular}



\begin{tabular}{l | l | l}
&multiplicatives&proportionals\\
1&\Gk{<apl'oos, -o~us}&\\
2&\Gk{dipl'oos, -o~us}&\Gk{dipl'asios}\\
3&\Gk{tripl'oos, -o~us}&\Gk{tripl'asios}\\
4&\Gk{tetrapl'asios}&
\end{tabular}













Nicomachus, {\em Introductio arithmetica} I.18 \cite[p.~214]{nicomachus}:

\begin{quote}
Once more, then; the multiple [\Gk{pollaplas'iwn}] is the species of the greater first and most original by nature, as straightway we shall see, and it is a
number [\Gk{>arijm`os}] which, when it is observed in comparison with another, contains the whole of that number more than once. For example, compared with unity, all the successive numbers beginning with 2 generate in their proper order the regular forms of the multiple; for 2, in the first place, is and is called the double, 3 triple, 4 quadruple, and so on;
for `more than once' means twice, or three times, and so on in succession as far as you like.
\end{quote}

\bibliographystyle{plain}
\bibliography{greeknumbers}

\end{document}