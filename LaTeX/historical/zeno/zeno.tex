\documentclass{article}
\usepackage{amsmath,amssymb,graphicx,subfig,mathrsfs,amsthm,enumitem}
\begin{document}
\title{Zeno of Elea, locomotion, infinity, and time}
\author{Jordan Bell}
\date{March 20, 2017}

\maketitle

\section{Generalities}
Barnes \cite{barnes} on Zeno


\section{Philology}
\textrm{kinesis} is ``motion'', ``movement'', ``change''. But before Aristotle {\em kinesis} means more locomotion or
disturbance
than general change.

Snell \cite[p.~217, chapter 9]{snell}, \cite[pp.~241--244, chapter 10]{snell}


\section{Mathematics}
Heath \cite[pp.~271--283]{HGMI}

Szab\'o \cite{szabo}


\section{Pythagoreans}
Guthrie \cite{sylvanguthrie}

Horky \cite{horky}

Archytas of Tarentum on geometric proportion \cite{archytas}

Philolaus of Croton \cite{philolaus}

Burkert \cite[pp.~285--288]{burkert}

\section{Xenophanes}
Testimonia on Fragment 26 from the peripatetic {\em On Melissus, Xenophanes, and Gorgias} \cite[pp.~204--210]{xenophanes}


\section{Heraclitus}
Heraclitus saying things are and are not is like saying that things are specified by their position and velocity, not just position. (A photo does not tell you everything about an object.)

Cornford \cite[p.~184]{religion}: time is Heraclitus's primary substance. Ap. Sextus Empiricus, {\em Adv. Math.} x.216



\section{Parmenides and Melissus}
The testimonia, A section of Diels and Kranz, on Parmenides are translated in Gallop \cite{gallop}.

Mourelatos \cite[pp.~118--119]{mourelatos}

Parmenides criticizes motion in B8.26--33, and Melissus in B7.7--10.

Melissus 4: ``Nothing that has a beginning and an end is either everlasting or infinite.'' \cite[p.~48]{ancilla}. DK30B4.

Palmer \cite{palmer}


\section{Zeno}
Zeno \cite[pp.~45, 67, 71--85]{lee}

Waterfield \cite{waterfield2009}

Four fragments arguing that if Things are Many a contradiction follows \cite[p.~47]{ancilla}. DK29B.

Algra \cite{algra}

Guthrie \cite{HGPII}

Solmsen \cite[p.~18]{experiments}

Lloyd \cite{lloyd1979}



\section{Anaxagoras}
Fragment 3: ``Nor of the small is there a smallest, but always a smaller (for what-is cannot not be) -- but also
of the large there is always a larger. And [the large] is equal to the small in extent, but in relation to itself each thing is both large and
small.'' Curd \cite[pp.~38--42]{curd2007}

Schofield \cite[Chapter 3]{schofield}



\section{Democritus}
Democritus DK68B155: ``If a cone were cut with a plane parallel to the base, \ldots'' \cite[p.~106]{ancilla}

Lura \cite{luria}

Fragments in Taylor \cite{taylor2010}


\section{Empedocles}



\section{Protagoras}
Protagoras of Abdera DK80B7: tangents to circles \cite[p.~126]{ancilla}.


\section{Gorgias}
Gorgias of Leontini DK82B3: nothing exists, sophisticated argument \cite[p.~128]{ancilla}


\section{Antiphon the Sophist}
Infinite divisibility. Fragments 1, 13 \cite{antiphon}


\section{Diogenes of Apollonia}




\section{Plato}
Friedl\"ander \cite{friedlander}

Taylor \cite{taylor}

Cornford {\em Parmenides} \cite{parmenides}

Cornford {\em Timaeus} \cite{timaeus}

Plato refers to Zeno making his audience think that things are one and many and at rest and at motion in Phaedrus 261d and Parmenides 129e.

Plato's {\em Parmenides} \cite[pp.~93--98, 250--260]{allen}

Cornford \cite[p.~160]{religion} refers to {\em Timaeus} 39b, and writes ``Distance in space is measurable psychologically, by expenditure of strength; but time-distance
can be measured only by counting the rhythmical repitition of the same occurrence.''

Cornford \cite[p.~131]{religion}: the soul can move itself. {\em Laws} 896a. Cornford cites Aetius and Sextus Empiricus.


\section{Aristotle}
Peripatetic {\em On indivisible lines} and {\em On Melissus, Xenophanes, and Gorgias} \cite{hett}

Heath \cite{heatharistotle}

Ross \cite[p.~94]{ross}

Roark \cite{roark}

Bolotin \cite{bolotin}

Cherniss \cite{cherniss}

Aristotle {\em De anima}, Polansky \cite[p.~96]{polansky}

Aristotle {\em Physics} 239b9--14.

Aristotle {\em Prior Analytics} 65b16--21.

\section{Aristoxenus fl.~335 BC}
Levin \cite{levin} and Barker \cite{barker} infinite divisibility in music theory

\section{Heraclides Ponticus c.~390 BC--c.~310 BC}
Sharples \cite{heraclides}

\section{Xenocrates c.~396/395 -- c.~314/313 BC}
Sambursky \cite[p.~91]{sambursky}

Dillon \cite[pp.~111ff]{heirs}: indivisible lines.


\section{Theophrastus of Eresus c.~371 BC -- c.~287 BC},
Sharples \cite{theophrastus}


\section{Praxiphanes fl. c.~300 BC}
Sharples \cite{praxiphanes}

\section{Strato of Lampsacus c.~335 BC -- c.~269 BC}
Sharples \cite{strato}

\section{Eudemus of Rhodes c.~370 BC -- c.~300 BC}
Sharples \cite{eudemus}

Wehlri fragments 37, 60, 62 and 78 \cite{wehrliVIII}.

\section{Diodorus Cronus died c.~284 BC}
Gaskin \cite[pp.~60, 64, 108, 252, 260]{gaskin}


\section{Archimedes c.~287 BC -- c.~212 BC}
Heath \cite{heatharchimedes}

Dijksterhuis \cite{dijksterhuis}

\section{Plutarch c.~46 -- c.~120}
{\em Moralia}, book XIII, {\em De communibus notitiis adversus Stoicos} \cite{moraliaXIII}



\section{Epicurus 341 BC -- 270 BC}
{\em Letter to Herodotus} 57, 61--62. In 38, Epicirus says that there must be a void lest things not move.

Vlastos \cite{vlastos1965}

Milton \cite{milton}


\section{Chrysippus c.~279 BC -- c.~206 BC}
Gould \cite[pp.~112--119, chapter V, \S 1f]{gould}

Bobzien \cite{bobzien}


\section{Polybius c.~200 BC -- c.~118 BC}
{\em Histories}, Book IV, Chapter 40: ``For given infinite time and basins that are limited in volume,
it follows that they will eventually be filled, even if silt barely trickles in. After all, it is a natural law that,
if a finite quantity goes on and on increasing or decreasing -- even if, let us suppose, the amounts involved
are tiny -- the process will necessarily come to an end at some point within the infinite extent of time.''

\section{Asclepiades of Bithynia c.~129/124 BC -- 40 BC}
Vallance \cite{asclepiades}

\section{Antiochus of Ascalon c.~125 BC -- c.~68 BC}
Dillon \cite[p.~82]{dillon}: ``there exists nothing whatever in the nature of things that is an absolute least,
incapable of division.'' {\em Acad. Post.} 27ff.

\section{Varro 116 BC -- 27 BC}
Sedley \cite{stoicgod}

\section{Cicero 106 BC -- 43 BC}
{\em De natura deorum}, I.xxiiii.55: no such thing as an indivisible body

Dillon \cite[p.~170]{heirs}: matter is ``capable of infinite section and division''.

{\em Academica}, I.vii.27: matter and space are infinitely divisible. Antiochus of Ascalon, in Cicero's {\em Academica} 1.27 \cite[p.~98]{academica}:

\begin{quote}
But underlying everything there is a kind of `matter', they think, without any form, and lacking any of those qualities (let's keep using this term and make it more familiar and gentler
on the ear). Everything has been produced or brought about from this, because matter as a whole can receive everything and change in every way and in every part. Matter thus `perishes' into its parts rather than into nothing; and these parts can be infinitely cut or divided since there is no smallest unit in the nature of things, i.e., nothing that can't be
divided. Moreover, everything that is moved is moved through intervals, and these intervals can likewise be infinitely divided.
\end{quote}

\section{Posidonius c.~135 BC -- c.~51 BC}
Fragment 98 \cite[pp.~395--403]{kidd}

\section{Lucretius c.~99 BC -- c.~55 BC}
Lucretius {\em De rerum natura} 2.238--2.239 \cite{melville}.

\section{Philo of Alexandria c.~25 BC -- c.~50}
Goodenough \cite[pp.~127--139]{goodenough}

\section{Alexander of Aphrodisias fl. c.~200}
Quaestiones 1.21 and 1.22 \cite[pp.~74--75]{quaestiones11} and 3.12 \cite[p.~67]{quaestiones216}

{\em Ancient Commentators on Aristotle}


\section{Galen 129--c.~200/216}

\section{Sextus Empiricus c.~160--210}
{\em Against the Physicists} \cite{bett}

{\em Adversus mathematicos I}: {\em Against the grammarians} \cite[pp.~8, 61, 166]{blank}

{\em Outlines of Scepticism} \cite{annas}

Hankinson \cite{sceptics} on moments of time, and on bodies and surfaces in space.

\section{Numenius of Apamea fl. c.~150}
Numenius \cite[p.~58]{numenius}: ``Bodies, containing nothing unchangeable, are naturally subject to change, to dissolution,
and to infinite divisions.''

\section{Plotinus c.~204/205--270}
Wagner \cite{wagner}

Wallis \cite{wallis}

Whittaker \cite{whittaker}

Graeser \cite{graeser}


\section{Dionysius of Alexandria c.~200 -- 264}
Cleve \cite{cleve}



\section{Porphyry 234--c.~305}
Gaiser \cite[p.~482]{gaiser}. Simplicius, {\em In Physicorum}, 454, 6, quotes Porphyry. Take a definite length one cubit long.
Divide it in half. Leave one-half undivided and divide the other again. If we continue dividing, Porphyry says that ``there is a certain
infinite nature enclosed in the cubit, or rather several infinities, one proceeding to the great and one to the small.'' The infinitely large is the increasing
number of segments.

\section{Iamblichus c.~245--c.~325}
{\em The Theology of Arithmetic}, ``On the Dyad'' \cite[p.~45]{iamblichus}: ``length is both infinitely divisible and infinitely extensible.''




\section{Eusebius 260/265--339/340}
{\em Praeparatio evangelica}, book XV, chapter XXII: ``But in fact the whole sentient is one: for how could it be divided? For there can be no correspondence of equal to equal, because the ruling faculty cannot be equal to each and every sensible object. Into how many parts then shall the division be made? Or shall it be divided into as many parts as the number of varieties in the object of sense that enters? And so then each of those parts of the soul will also perceive by its subdivisions, or the parts of the subdivisions will have no perception; but that is impossible. And if any part perceive all the object, since magnitude by its nature is infinitely divisible, the result will be that each man will also have infinite sensations for each sensible object, infinite images, as it were, of the same thing in our ruling faculty.''

\section{Ephrem the Syrian c.~306--373}
Possekel \cite{possekel}


\section{Calcidius fl. c.~400}
van Winden \cite[p.~155]{vanwinden}


\section{Syrianus died c.~437}
Wear \cite{wear}

Syrianus \cite[p.~57]{syrianus13to14}


\section{Proclus 412--485}
Opsomer \cite{opsomer} on the {\em Elements of Physics}

Morrow and Dillon \cite{proclus}

{\em Elements of Theology} \cite{dodds}


\section{Irenaeus of Lyons c.~130--c.~202}
{\em Adversus haereses}, II.1.4: ``These remarks are, in like manner, applicable against the followers of Marcion. For his two gods will also be contained and circumscribed by an immense interval which separates them from one another. But then there is a necessity to suppose a multitude of gods separated by an immense distance from each other on every side, beginning with one another, and ending in one another.''

\section{Clement of Alexandria c.~150--c.~215}
{\em Stromata}

\section{Hippolytus of Rome 170--235}
{\em Refutation of All Heresies}, IV.51.



\section{Origen 184/185--253/254}
{\em De Principiis}

\section{Alexander of Lycopolis fl.~early fourth century}
Infinite divisibility of matter \cite{lycopolis}

\section{Hilary of Poitiers c.~300--c.~368}
{\em De Trinitate}. Meijering \cite{meijering}

\section{Basil the Great 329/330--379}
{\em Hexaemeron}, Homily I, article 4: ``These men who measure the distances of the stars and describe them, both those of the North, always shining brilliantly in our view, and those of the southern pole visible to the inhabitants of the South, but unknown to us; who divide the Northern zone and the circle of the Zodiac into an infinity of parts, who observe with exactitude the course of the stars, their fixed places, their declensions, their return and the time that each takes to make its revolution; these men, I say, have discovered all except one thing: the fact that God is the Creator of the universe, and the just Judge who rewards all the actions of life according to their merit.''

{\em Hexaemeron}, Homily I, article 6: ``The beginning, in effect, is indivisible and instantaneous. The beginning of the road is not yet the road, and that of the house is not yet the house; so the beginning of time is not yet time and not even the least particle of it. If some objector tell us that the beginning is a time, he ought then, as he knows well, to submit it to the division of time -- a beginning, a middle and an end. Now it is ridiculous to imagine a beginning of a beginning. Further, if we divide the beginning into two, we make two instead of one, or rather make several, we really make an infinity, for all that which is divided is divisible to the infinite.''



\section{Gregory of Nyssa c.~335--c.~395}
{\em Against Eunomius}, Book I. See entry for infinity in \cite{maspero}.


\section{Augustine 354--430}
Letter 3 (to Nebridius), article 3.

O'Daly \cite[p.~157]{odaly}.

Knuutila \cite{knuutila}

{\em De Trinitate}, XI, article 17 and XV, chapter 12. In \cite{trinity815}

{\em Confessions}, book XI \cite{confessions}.

\section{Themistius 317--c.~390}
{\em in Physicorum} 91.29--30.



\section{Simplicius}
{\em In Physicorum} 139.27--140.6, Zeno's arguments against plurality. See Curd \cite[pp.~171--186]{monism}

Simplicius \cite{corollaries}


\section{John Philoponus c.~490--c.~570}


\section{Olympiodorus}
Furley \cite{furley1982}


\section{Kalam}
The kalam cosmological argument, in Craig and Sinclair \cite{kalam}

Zimmerman \cite{zimmerman}

Wolfson \cite{wolfsonIV}

\section{An-Nazzam c.~775--c.~846}
Ibrahim An-Nazzam

\section{Al-Kindi c.~801--c.~873}
Al-Kindi \cite{alkindi}

\section{ibn Qurra c.~826--901}
Rashed \cite{rashed2009}

\section{Alfarabi c.~872--950/951}
Alfarabi \cite[pp.~101--111]{alfarabi}


\section{Al-Sijzi c.~945--c.~1020}
Rashed \cite{rashed2000}

\section{Al-Biruni 973--1048}
Letter to Avicenna: ``If the sun is west of the moon in the sky, with a definite space between
them, then even though the moon moves much faster than the sun, it should never be able to catch it.
For the space between them can be conceived as divisible into an infinite number of parts; but how
can a body moving with a finite speed cross an infinite number of spaces?'' \cite[p.~820]{selin}

\section{Avicenna c.~980--1037}
McGinnis \cite{mcginnis}

Rashed \cite{rashed}



\section{Saint Anselm of Canterbury c.~1033 -- 1109},
{\em On the Incarnation of the Word}, \S 15.


\section{Al-Ghazali c.~1058--1111}
Goodman \cite{goodman}


\section{Avempace c.~1085--1138}
Lettinck \cite{lettinck}



\section{Averroes 1126--1198}
Glasner \cite{glasner}

Goldstein \cite{goldstein}


\section{Jewish philosophers}
Saadia Gaon \cite{gaon}

Maimonides, {\em Guide for the Perplexed}, I.73 106a



Hasdai Crescas, in \cite{harvey} and Wolfson \cite{wolfson1929}

Rudavsky \cite{rudavsky}





\section{Gersonides 1288--1344}
Gersonides \cite{gersonides}

Kohler \cite{kohler}


\section{Royal MS 4 A XIV, 12th century}
Royal MS 4 A XIV, {\em Against wens}, ll.~11--13, Storms \cite[p.~155, no.~4]{storms}: ``May you become as small as a linseed grain,/ and much smaller than the hipbone of an
itchmite,/and may you become so small that you become nothing.''


\section{Peter Abelard 1079--1142}
King \cite[p.~94]{abelard}

\section{Hugh of Saint Victor c.~1096--1141}
{\em Didascalion}, chapter 17: ``From this consideration derives the axiom that continuous quantity is
divisible into an infinite number of parts, and discrete quantity multipliable into a product of infinite
size. For such is the vigor of the reason that it divides every length into lengths and every breadth
into breadths, and the like -- and that, to this same reason, a continuity lacking interruption
continues forever.'' \cite[p.~58]{grant1974}

\section{John of Salisbury}
{\em Metalogicon}

\section{William of Conches c.~1090--c.~1154}

\section{Herman of Carinthia c.~1100--c.~1160}
{\em De Essentiis} \cite[p.~252]{essentiis}


\section{William of Auvergne 1180/1190--1249}
William also presents arguments that the view that a continuum, such as time, is infinite results in paradoxes (OO I, 698a-700b).

William had read Aristotle's Physics and agrees with Aristotle that time and motion are coextensive (OO I, 700a). Yet he does not propose Aristotle's definition of time as the number of motion in respect of before and after. Rather, in his account of the essential nature of time he describes time simply as being that flows and does not last, ``that is, it has nothing of itself that lasts in act or potency'' (OO I, 683a; Teske \cite[p.~102]{teske}), {\em De universo}.




\section{Richard Rufus died c.~1260}
Lewis \cite{lewis2012}



\section{Peter of Spain c.~1215-1277}
{\em Syncategoreumata} \cite{hispanus}, chapters 5 and 6


\section{Roger Bacon c.~1214--c.~1292}
Roger Bacon \cite[p.~396]{grant1974}

In his {\em Opus tertium}, {\em Opera quaedam hactenus inedita}, cap 39, pp.~134--135, Roger Bacon writes:\cite[p.~46]{ariew}
\begin{quote}
A body's potential for division cannot be reduced to actuality, purely and completely. It is a potentiality that one can only reduce to actuality impurely and incompletely,
where there is always a mixture with a potential for further actualization; it is always reduced but in such a way that there remains the potential for another division.
That is the potential of the continuum and that which constitutes infinite divisibility; when this potential is reduced by actual division, the possibility of another division is not
excluded. Actually, it is required; in fact, the portion which is the result of divison is a magnitude; hence it is still divisible, and so forth to infinity.
\end{quote}

Roger Bacon, {\em Opus Majus}, part 4, distinction 4, chapter IX \cite[p.~173]{baconI}: argues against the statement that ``the world is composed of an infinite number of material
particles called atoms, as Democritus and Leucippus maintained, by whose position Aristotle and all students of nature have been more hindered than by any other error'': 

\begin{quote}
Yet this error is wholly eliminated by the power
of geometry; for no stronger argument can be used against this error than that the diagonal of the square
in that case and its side would be commensurable, that is, would have a common measure, namely, some aliquot part as a common
measure, the contrary to which Aristotle always teaches. And the truth is clear by the demonstration from the last part of the seventh proposition of the tenth book of the
Elements, where it is shown that if some measure, as a foot or a palm, measures the side, it will not measure the diagonal, nor {\em vice versa}; so that if the diagonal
is ten feet, the side cannot be expressed exactly in feet. And not only does it follow
from this position that they would be commensurable, but also equal. For if the side has ten atoms, or twelve, or more, then let the same number of lines
be drawn from those atoms to the same number in the opposite side, the sides of the square being equal; wherefore just so many lines will occupy the whole
surface of the square; and therefore since the diagonal passes through those lines, and no more can be drawn in the square, the diagonal must receive a single atom from each
line, and therefore there will be no more atoms in the diagonal than in the side, and thus they have an aliquot part as a common measure, and the side has just as many parts
as the diagonal, both of which conclusions are impossible.
\end{quote}


\section{Robert Grosseteste c.~1175--1253}
Lewis \cite{lewis2005} and \cite{lewis2012}



\section{Albertus Magnus c.~1200--1280}
Twetten, Baldner, and Snyder \cite{snyder}

Fox \cite{fox}

Money as infinitely divisible quantity \cite{kaye}

\section{Thomas Aquinas 1225--1274}
{\em Summa Theologica}, prima pars, q. 7, article 3; q. 48, article 4; q. 53, article 2. 

{\em Commentaria in octo libros Physicorum}, articles 69, 377--379 \cite[pp.~188-189]{aquinas}

{\em In libros De generatione et corruptione expositio}, lecture 7, article 56; lecture 4, article 29.


\section{Arnald Villanova c.~1240--1311}
McVaugh on minima natura \cite[p.~97]{villanovaII}


\section{Saint Bonaventure 1221--1274}


\section{Henry of Ghent c.~1217--1293}
Brower-Toland \cite{brower-toland}

\section{Peter John Olivi}
Pasnau \cite{pasnau}

\section{Ramon Llull}
Lohr \cite{lohr2001}

\section{Duns Scotus}
Trifogli \cite{trifogli2004}


\section{Godfrey of Fontaines}
Dales \cite[pp.~185--186, 202--203, 233, 255]{dales}

\section{Henry of Harclay}
Murdoch \cite{harclay}

Dales \cite{dales1984}

\section{Thomas Bradwardine}
Dolnikowski \cite{dolnikowski}


\section{Johannes de Muris}
Busard \cite[p.~35]{demuris}. Porism to Prop. 19: ``the horn-like
angle is infinitely divisible by circular lines, can increase infinitely
by diminishing the circles, and can decrease by augmenting
the circles.''

\section{Walter Chatton}
Murdoch and Synan \cite{synan}

\section{Gerardus Odonis c.~1285--1349}


\section{Nicolas Bonet}

\section{Nicholas of Autrecourt c.~1299--1369}
{\em The tniversal treatise}. 

\section{Robert Kilwardby}
Trifogli \cite{kilwardby}

\section{William of Auxerre}
Tummers \cite{tummers}

\section{Peter of Auvergne}
Galle \cite[pp.~277*--330*]{galle}

\section{John Buridan}
Murdoch and Thijssen \cite{buridan}

Buridan gives an example in his {\em Quaestiones super octo libros Physicorum}, lib. III, quaest. XVIII, fol. 63, col. d, about a cylindrical column dividied into proportional parts \cite[p.~58]{ariew}. In the same work, cols. c, d, Buridan writes ``Assuredly, when I take my book, I take an infinity of parts of my book, for I am taking three parts, 100 parts, 1000 parts, and so forth without end.
But what is impossible, is that one takes an infinity of parts successively, counting one after the other.''



\section{Robert Holcot}
A man is alternately meritorious and sinful in proportional parts of the last hour of his life. This suggests the geometric series
\[
\sum_{n=0}^\infty (-r)^n.
\]
See Murdoch \cite[p.~327]{murdoch1975}. It is from Holkot's {\em In quattuor libros
Sententiarum quaestiones}, book I, qu. 3, fol. Biiiiv, col. 2.


\section{Aegidius Romanus}
Porro \cite{porro}

Trifogli \cite{trifogli1993}

\section{William Crathorn}

\section{Peter Auriol c.~1280--1322}


\section{William of Ockham}
Goddu \cite{goddu}

Murdoch \cite{murdoch1982}

\section{John Bassolis}
John Bassolis in his {\em In Quatuor Sententiarum libros, Quaestiones in Primum Sententiarum}, dist. XLIII, quaest. unica, fol. 213, col. c \cite[p.~99]{ariew}
\begin{quote}
The division of any finite quantity into parts whose magnitudes follow a constant relationship can be pursued to infinity. It is the same with the increase of a quantity by the addition
of similar divisible parts. Divine virtue itself cannot reduce this division or this increase to actuality {\em in facto esse}, but only {\em in fieri}, and this is because the reality
or nature of things repulses this actualization. But this in no way constitutes an objection to our proposition.
\end{quote}

\section{Richard of Middleton}
Richard of Middleton in his commentary on Lombard's {\em Sentences}  {\em Super quatuor libros Sententiarum Petri Lombardi quaestiones subtilissimae}, lib. I, dist. XLIII, art. 1, quaest. IV, vol. 1, p.~386, col. b \cite[p.~79]{ariew}
\begin{quote}
When one states that any continuum is divisible to infinity, I reply that it is true as long as one understands it thus: It can be divided without end, but in such a way that the number
of parts already obtained is always finite. If one admits that it is thus divided, no impossibility results; the existence of an infinite {\em in facto esse} does not result,
only the existence of an infinite {\em in fieri} which one commonly calls an infinite in actuality mixed with
potentiality.
\end{quote}

\section{William Heytesbury}
Wilson \cite{wilson}

Longeway \cite{longeway}

\section{Richard Kilvington},
Kretzmann and Kretzmann \cite{kilvington}

Jung and Podko\'nski \cite{jung}

\section{John Dumbleton}

\section{Walter Burley}
Duhem \cite[p.~57]{ariew} quotes from Walter Burley's {\em Super octo libros physicorum} \cite{burley}, lib. III, tract. II, cap. 4, fol. 70, col. b:
\begin{quote}
What we have just expounded upon proves the truth of the following proposition which is not known by many: Given any line, one can mark off segments whose lengths decrease proportionally, and one can also indicate a point which cannot be reached by a finite operation.
That will occur if one takes as the first segment half the length to the extremity which cannot be reached by a finite operation; one takes as the second segment half the first
segment, and so forth. On the other hand, every point before the extremity can be reached by a finite operation. That can easily be demonstrated geometrically, but for now
we will not insist on its demonstration.
\end{quote}

Spade \cite[pp.~74--75, 117--123]{spade}

\section{Albert of Saxony}
Sarnowsky \cite{sarnowsky}

Biard \cite{saxe}

\section{Walter Odington}

\section{Richard Swineshead}

\section{Nicole Oresme}
{\em Questiones super Physicam} \cite{oresme2013}


\section{Gregory of Rimini}
Cross \cite{rimini}

Thakkar \cite{thakkar}

Gregory of Rimini  \cite[p.~441]{riminiIII}, in the first conclusion of the first article on the first book of his commentary on the {\em Sentences},
says that ``God can make any actually infinite multitude'', and gives the example
of making an infinite number of angels in an hour, and talks about this using proportional parts: in each proportional part of an
hour, God creates and preserves an angel, and at the end of the hour there are infinitely many angels. Rimini also talks about God
creating an infinite magnitude \cite[p.~445]{riminiIII}. Also, creating infinity charity \cite[pp.~446--447]{riminiIII}.

Gregory of Rimini in his {\em In primum Sententiarum}, dists. XLII, XLIII, XLIV, quaest. IV, art. 2, fol. 190, col. c (fol. 175, col. a) \cite[pp.~115--116]{ariew}:
\begin{quote}
When one says, infinity is something never completed, I reply that it is so if its infinitely numerous parts are acquired in equal durations; if, for example, each part of this
infinity were acquired after an hour or a moment, or some other determined quantity of time. In that case, it would have to be that the time would have an infinity of equal parts and,
consequently, that it would be infinite. Since, in any case, it is impossible that an infinite time whose first part is given becomes a past time, an infinity could
not be totally completed or surpassed in this manner.

One says, infinity is something such that when one takes any part of it whatever, there always remains another part to be taken. I reply that this proposition
must be understood as the previous one, by admitting that the parts taken successively are all of the same magnitude and that they are all taken in equal times. If one takes, in some time, a portion of infinity, then in a time equal to the one in which the first part was taken, one takes an equal portion, and one continues in this fashion, there will always remain something
to be taken of this infinity, and it will never be taken in its totality....

But once equal parts of the infinity are not taken in equal times, but in times whose durations decrease in geometric progression... there is no longer any inconsistency in the
infinity being taken in its totality, as long as there is no obstacle of some other nature to this. Similarly, there is no inconsistency in that the infinite multitude of parts of time, in which
the successive parts of the infinite are taken, come to be completely past, as we have already stated. Not only is there no inconsistency in this, but it is necessary that it be.
\end{quote}

Gregory of Rimini, responding to Zeno's paradoxes as stated by writers such as Henricus Hibernicus, Adam Goddam, and Clienton Lengley, in his
{\em In secundum Sententiarum}, dist. II, quaest. II, art. 1, fol. 34, col. c:\cite[p.~57]{ariew} ``In any magnitude there is an infinity of proportional parts, infinity
being taken syncategorematically; a result of this is that none of these parts is the last one.''

Duhem writes \cite[p.~126]{ariew} ``it seems however that Oresme speaks the language of a disciple of Gregory of Rimini, of a defender of the categorematic infinite''. 
In his {\em Trait\'e du Ciel et du Monde},
livre I, fol. 11, col. d (pp.~109--11),
 Nicole Oresme responds to Aristotle's statement in {\em De Caleo} that an infinite body would have to be infinitely heavy: \cite[p.~127]{ariew}
\begin{quote}
But it seems to me that the reason given above is not evident without adding another assumption. For, in accordance with the second reply, I assume a body to be infinite,
and I take or assign in this body a finite portion, spherical in shape, called $A$. Next I take another sphere $B$ from the same section, and of the same shape,
and then another sphere $C$, exactly like $A$ and $B$, proceeding in this manner without stopping. In this way, it appears that there are, in this infinite body,
infinite equal parts $A,B,C,D$, and so on without limit.

Now I posit that in the portion called $A$ there should be distributed the weight of one half-pound, and in $B$ there should likewise be distributed one-half
of another half-pound, and in $C$ one-half of the residue of a pound, and in $D$ half of this remainder, which would be one-sixteenth part of a pound, and so on without
end.

It appears then that the entire infinite body will weigh only one pound, while $A$ will weigh as much as all the other portions, however many, taken
together.
\end{quote}


\section{Adam de Wodeham}
Wood \cite{wodeham}

Courtenay \cite{courtenay1978}

\section{Marsilius of Inghen}
Hoenen \cite{hoenen}

\section{Blasius of Parma}
Biard \cite{blasius}

\section{Jean de Ripa}

\section{John Wycliffe c.~1331--1384}

\section{Patience}
Eldredge \cite{eldredge}

\section{John Dee}
Clulee \cite{johndee}

\nocite{*}

\bibliographystyle{plain}
\bibliography{zeno}

\end{document}