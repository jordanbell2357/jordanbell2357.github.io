\documentclass{article}
\usepackage{amsmath,amssymb,graphicx,subfig,mathrsfs,amsthm,hyperref}
\usepackage[LGR,T1]{fontenc}
\newtheorem{theorem}{Theorem}
\newtheorem{lemma}[theorem]{Lemma}
\newtheorem{proposition}[theorem]{Proposition}
\newtheorem{corollary}[theorem]{Corollary}
\newcommand{\chord}{\ensuremath\mathrm{chord}\,} 
\theoremstyle{definition}
\newtheorem{definition}[theorem]{Definition}
\newtheorem{example}[theorem]{Example}
\begin{document}
\title{Approximating square roots in  antiquity}
\author{Jordan Bell\\ \texttt{jordan.bell@gmail.com}}
\date{\today}

\maketitle

\tableofcontents


\section{Babylonia}
Neugebauer and Sachs \cite[pp.~42--43]{cuneiform}, YBC 7289: in a square of side $2$, 
the diagonal is  $1+\frac{24}{60}+\frac{51}{60^2}+\frac{10}{60^3}$.  
Neugebauer and Sachs suggest that this value was obtained by the following method.
Given $a$, let $\alpha_1$ satisfy  $\alpha_1>\sqrt{a}$. 
Then let $\beta_1$ be such that 
$\sqrt{a}$ is the geometric mean of $\alpha_1$ and $\beta_1$, that is, $\alpha_1:\sqrt{a} = \sqrt{a}:\beta_1$, which
means
$\beta_1 = \frac{a}{\alpha_1}<\sqrt{a}$. 
Then let
$\alpha_2$ be the arithmetic mean of $\alpha_1$ and $\beta_1$,
i.e. $\alpha_2 = \frac{\alpha_1+\beta_1}{2}$,
and as $\alpha_2$ is the arithmetic mean of $\alpha_1$ and $\beta_1$
and $\sqrt{a}$ is the geometric mean of $\alpha_1$ and $\beta_1$ it holds that $\alpha_2>\sqrt{a}$. 
Then let $\beta_2$ be such that $\sqrt{a}$ is the geometric mean of $\alpha_2$ and $\beta_2$, that is,
$\alpha_2:\sqrt{a}=\sqrt{a}:\beta_2$, which means $\beta_2=\frac{a}{\alpha_2}<\sqrt{a}$. 

For $a=2$, take $\alpha_1=\frac{3}{2}=1+\frac{30}{60}$, which satisfies $\alpha_1^2=2+\frac{15}{60}>2$ and so $\alpha_1>\sqrt{2}$.
Then
\[
\beta_1  =\frac{2}{1+\frac{30}{60}}=1+\frac{20}{60}.
\]
Then
\[
\alpha_2 = \frac{1+\frac{30}{60}+1+\frac{20}{60}}{2} = 1 + \frac{25}{60}.
\]
Then 
\[
\beta_2 = \frac{2}{1 + \frac{25}{60}} = 1+\frac{24}{60}+\frac{42}{60^2}+\frac{21}{60^3}
+\frac{10}{60^4}+\cdots.
\]
Then
\begin{align*}
\alpha_3 &= \frac{1 + \frac{25}{60} + 1+\frac{24}{60}+\frac{42}{60^2}+\frac{21}{60^3}
+\frac{10}{60^4}+\cdots}{2}\\
&=1+\frac{24}{60}+\frac{51}{60^2}+\frac{10}{60^3}+\frac{35}{60^4}+\cdots.
\end{align*}

Fowler and Robson \cite{YBC7289}

Neugebauer \cite{neugebauer1931} on square root approximations in Babylonian mathematics

YBC 7243 \cite[pp.~136--139]{cuneiform}, Neugebauer and Sachs, $\sqrt{2} \sim 1;24 51 10$.

BM 15285, B 1 \cite[p.~54]{thureau-dangin}: draw a first square whose side is 1, then draw 
a second square inside that touches the first. Then draw a third square inside the second that touches the second. 
What is the surface of the third square?

AO 6484, Problem 8 \cite[p.~78]{thureau-dangin}: if the diagonal of a square is 10 cubits, what is the side of the square?
Mutiply 10 by $42'30'' = \frac{42}{60}+\frac{30}{60^2}$, getting
$7^\circ 5'=7+\frac{5}{60}$. In fact, the side of the square has length $\sqrt{50}$, so
this amounts to $\sqrt{50} \sim 7+\frac{5}{60}$. 

YAT 6598, Problem 6 \cite[p.~130]{thureau-dangin}: 
for a rectangle whose height is one demi-ninda  2 cubits and  whose width is 2 cubits, what is the diagonal?
(A  {\em ninda} is 12 cubits, a {\em demi-ninda} is 6 cubits, and $x'$ means $x'$ ninda, e.g.
$10'$ means $\frac{1}{6}$ ninda, namely $2$ cubits.) 
Square $10'$, the width, getting $1'40''$.
Divide this by  $40'$, the height, getting $1'40'' \cdot \frac{1}{40'} = 2'30''$. Take half of this, getting
$1'15''$. Add this and the height, getting  $1'15'' + 40'=41'15''$. This is an instance
of
\[
\sqrt{a^2+b^2} \sim a + \frac{1}{2} \cdot b^2 \cdot \frac{1}{a},
\]
with $a=40'$, the height and  $b=10'$, the width. See Weidner \cite{weidner}.

H{\o}yrup \cite{hoyrup}

Friberg \cite{friberg1997} W 23291 \S 4b: area of equilateral triangle triangle with side length 1 is $\frac{26}{60}+\frac{15}{60^2}$. 

Friberg \cite[p.~286]{friberg1997} W 23291 \S 4c; pp.~302--304, VAT 7848 \S 1.

Friberg \cite{friberg} Kassite MS3876, 3

Friberg \cite[p.~548]{reallexikon7}, Ist Sippar 428

Goetze \cite{goetze}, Old Babylonian IM 52916: the height of an equilateral triangle with side length $s$ is $s-\frac{1}{8}s$. (``an eighth is torn out''). The area is
$;26 15$ $s^2$.

Tell Harmal \cite{TellHarmal}, IM 52301

Bruins \cite{bruins1948} and \cite{bruins1950}

Bruins and Rutten \cite{suse}, TMS 3, 27, regular hexagon, no. 31, no. 34

Old Babylonian tablet BM 80209, Problem 2 \cite{BM80209}:  ``if each square-side is [ ... ] 20, what is the transversal?''

Neugebauer \cite{MKT} MKT: II.43; BM 85194, Problem 4

Robson \cite{robson} and \cite{robson1999}







\section{Egypt}
P. P11529, Schubart \cite{P11529}

{\em Rhind Mathematical Papyrus}, Problems 41-60. The area of a regular octagon with side length $a$ is 
$A=2(1+\sqrt{2})a^2$. The length of an apothem is $r=\frac{1}{2}(1+\sqrt{2})a$. 
Thus, if $r=\frac{9}{2}$ then $a=\frac{9}{1+\sqrt{2}}$ and $A=\frac{162}{1+\sqrt{2}}$. In Problem 50, $\sqrt{7/9} \sim 8/9$.

Problem 48: octagon. See Vogel \cite[p.~66]{vogel}.

Problem 58 \cite[p.~167]{egyptian3}

Gillings \cite{gillings}

Parker, Demotic Mathematical Papyri \cite{DMP}, Problem 7:
\[
\sqrt{1500} \sim 38+\frac{2}{3}+\frac{1}{20}.
\]

Demotic Mathematical Papyri, Parker, Problem 32, Problem 36

\[
\sqrt{133+\frac{1}{3}} \sim 11+\frac{1}{2}+\frac{1}{20}.
\]

Parker \cite{parker1969}

P. Dem. Heidelberg 663, Parker \cite{Heidelberg663}

Cairo papyrus JE 89127, Problem 33

Berlin Papyrus 6619

BM 10520, Problem 62:
\[
\sqrt{10} \sim 3+\frac{1}{6}.
\]

Papyrus Berlin 11529

\[
\sqrt{2} \sim 1 + \frac{1}{5}+\frac{1}{7}+\frac{1}{14} = \frac{99}{70}.
\]

Knorr \cite{knorr1982} fractions

Bagnall, wax tablets, TVarie 71 \cite{bagnall}






\section{Music}
Philolaus, Fragment 6a  \cite[pp.~146--147]{philolaus},  from Nicomachus, {\em Manual of Harmonics} 9:

\begin{quote}
The magnitude of harmonia (fitting together) is the fourth ({\em syllaba}) and the fifth ({\em di' oxeian}). The fifth is greater than the fourth
by the ratio $9:8$ [a tone]. For from {\em hypat\={e}} [lowest tone] to the middle string ({\em mes\={e}}) is a fourth, and from the middle string to {\em neat\={e}} [highest tone] is a
fifth, but from {\em neat\={e}} to the third string is a fourth, and from the third string to {\em hypat\={e}} is a fifth. That which is in between the third string and the middle string is the
ratio $9:8$ [a tone], the fourth has the ratio $4:3$, the fifth $3:2$, and the octave ({\em dia pas\={o}n}) $2:1$. Thus the harmonia is five $9:8$ ratios [tones] and two {\em dieses}
 [smaller semitones]. The fifth is three $9:8$ ratios [tones] and a {\em diesis}, and the fourth two $9:8$ ratios [tones] and a {\em diesis}.
\end{quote}

Huffman \cite[p.~164]{philolaus} gives a {\em nihil obstat} for the following suggestion of Tannery. 
From the fifth $3:2$ take away the fourth $4:3$, getting the tone $9:8$, which is lesser than $4:3$.
From $4:3$ take away $9:8$, getting $32:27$, which is greater than $9:8$. From
$32:27$ take away $9:8$, getting the {\em diesis} $256:243$, which is lesser than $9:8$. This procedure can be continued.
From $9:8$ take away $256:243$, getting the {\em apotome} $2187:2048$, which is greater than $256:243$. From
$2187:2048$ take away $256:243$, getting the {\em comma} $531441:524288$, which is lesser than $256:243$. 

Philolaus, Fragment 6b \cite[p.~364]{philolaus}, from Boethius, {\em De Institutione Musica} III.8 (according to Huffman, it is uncertain if this fragment is genuine):

\begin{quote}
Philolaus, then, defined these intervals and intervals smaller than these in the following way: diesis, he says, is the interval by which the ratio
$4:3$ is greater than two tones. The comma is the interval by which the ratio $9:8$ is greater than two dieses, that is than two smaller semitones. Schisma is half of a comma,
 diaschisma half of a diesis, that is a smaller semitone.
\end{quote}

Archytas of Tarentum. Boethius, {\em De Institutione Musica} III.11, a superparticular ratio cannot be divided into equal parts.

Octave is $2:1$, whole tone is $9:8$, fourth is $4:3$, fifth is $3:2$. A semitone satisfies $x^2=2:1$. A tone is a minor semitone and an apotome;
an apotome is $3^7:2^{11}$; an apotome is a minor semitone and a comma; a comma is $3^{12}:2^{19}$.

A superparticular ratio is a pair of numbers $A$ and $B$ such that $A>B$ and
$B$ is a proper divisor of $A-B$.

Archytas's theorem is in {\em Sectio canonis} 3

Archytas's theorem says that an interval whose ratio is epimoric cannot be halved: {\em Sectio canonis} 16, 18; 
Theo of Smyrna 53.1--16, 70.14--19; 
Ptolemy, {\em Harmonics} 24.10--11.

Divide the fourth into three intervals, two of which are equal:
$4:3 = x \cdot x \cdot y$. Take $x$ to be a whole tone: $x=9:8$. Then $y=256:243$. 
$y$ is called the {\em leimma}. cf. {\em diesis}

{\em Sectio canonis}, Postulate ix, notes are related in a ratio of number.

Barker \cite[p.~223]{barker}: Theon of Smyrna, {\em On Mathematics Useful for the Understanding of Plato},
\[
\sqrt{\frac{9}{8}} \sim \frac{17}{16}.
\]

Aristides Quintilianus, {\em De Musica} 95.20ff. \cite[p.~496]{barker}: $\sqrt{\frac{9}{8}} \sim \frac{17}{16}$. 

Ptolemy, {\em Harmonics} I.10 \cite[pp.~297--298]{barker}

Ptolemy, {\em Harmonics} I.11 \cite[pp.~298--299]{barker}

Ptolemy, {\em Harmonics} I.16 \cite[pp.~312--313]{barker}

Nicomachus, {\em Manual of Harmonics}

Macrobius, {\em Commentary on the Dream of Scipio} \cite[pp.~188--189]{macrobius} 2.1.21--23:

\begin{quote}
[21] The ancients chose to call the interval smaller than a tone a semitone, but this must not be taken to mean half a tone
any more than we would call an intermediate vowel a semivowel. [22] The tone by its
very nature cannot be divided equally: inasmuch as it originates in the number nine, which cannot be equally divided,
the tone refuses to be divided into two halves; they have merely called an interval smaller than a full tone a semitone,
but it has been discovered that there is as little difference between it and a full tone as the difference
between the numbers 256 and 243. [23] The early Pythagoreans called the semitone {\em diesis}, but those who came later
decided to use the word {\em diesis} for the interval smaller than the semitone. Plato called the semitone
{\em leimma}.
\end{quote}

Censorinus, {\em De Die Natali} 10.7 \cite[p.~18]{censorinus}: according to Aristoxenus the octave is 6 tones, while
according to the Pythagoreans the octave is 5 tones and 2 semitones, ``so Pythagoras and the mathematicians, who pointed
out that two semi-tones do not necessarily add up to a full tone''.


Proclus, {\em Commentary on Plato's Timaeus} \cite{timaeum3II}

Cohen and Drabkin \cite[p.~286]{drabkin}







\section{Jews}
{\em Sukkah} 8a,b, {\em Eruvin} 23a--b, 57a, 76b, {\em Bava Batra} 101b.





\section{Hippocrates of Chios}
Square lunes: $\sqrt{N}$, $N$ positive integers.






\section{Plato}
Bulmer-Thomas \cite{bulmer-thomas}

Mueller \cite{mueller2003}

{\em Hippias Major} 303c

{\em Parmenides} 149a--c, complete induction and continued fractions. Allen \cite[pp.~238--258]{allen}

{\em Statesman} 266b

{\em Meno} 82b-85b \cite[pp.~102--111]{beresford}. Let $ABCD$ be a square and let $f,u,g,t$ be the midpoints respectively of
$AB,BC,CD,DA$. Socrates asks the slave boy what the area of $ABCD$ is, and to explain what sort of answer he wants he explains that the area of the rectangle
$fBCg$ is $1 \cdot 2$ square feet, and then says that the area of $ABCD$ is $2 \cdot 2$ square feet, which the slave boy then says is $4$ square feet.
The slave boy agrees that there exists a square whose area is double that of $ABCD$, and when asked by Socrates what its area is, says correctly $8$ square feet, and then
says incorrectly that a side of this bigger square is twice a side of $ABCD$. 
Further draw a square $YBWX$ where $A$ is the midpoint of $YB$, $C$ is the midpoint of $BW$, and $D$ is the center of the square.
Socrates explains that the area of $YBWX$ is $4$ times the area of $ABCD$, namely $16$ square feet.
Socrates then states a square whose area is $8$ square feet is twice the area of $ABCD$ and half the area of $YBWX$, and therefore that a side of the desired square is
greater than $BC$ (2 feet) and less than $BW$ (4 feet). When Socrates asks what the slave boy thinks the side of the desired square is, and the slave boy tentatively answers
$3$ feet. Now let $M$ be the midpoint of $YA$, let $K$ be the midpoint of $CW$, and let $MBKL$ be a square, whose sides are thus each $3$ feet.
The slave boy is asked the area of the square $MBKL$ and correctly answers $9$ square feet, which is greater than $8$ square feet. Then 83e--84a \cite[p.~108]{beresford}:

\begin{quote}
Socrates: Ah. So we still haven't got our square of eight square feet; we don't get it from the three-foot line either.

Slave: No, we don't.

Socrates: Well, what line do we get it from? Try and tell us exactly. And if you don't want to use numbers, you can just show us. [He hands the slave his stick.] What line?

Slave: [He stares at the drawing.] Honest to god, Socrates, I don't know!
\end{quote}

Socrates newly draws the square $ABCD$, with each side  2 feet. Further draw a square whose area is $4$ times that of $ABCD$, with $A$ the midpoint of the left side, $C$ the midpoint
of the bottom side, $G$ the midpoint of the right side, and $T$ the midpoint of the top side; $D$ is the center of this square. Then $ACGT$ is a square. Socrates then guides
the slave boy thus: the square $ABCD$ has twice the area of the triangle $ACD$, namely
$ABCD:ACD = 2:1$,
and the square $ACGT$ is made of $4$ triangles each congruent to the triangle $ACD$, namely $ACGT:ACD = 4:1$. Therefore $ACGT:ABCD = 2:1$, and as $ABCD$ is
$4$ square feet this means that $ACGT$ is $8$ square feet, and thus $AC$ is a side of a square twice the square $ABCD$.  
Klein \cite[pp.~99--102]{meno} comments on this passage, and writes, ``At best, this side can only be drawn or `shown.' And Socrates
will hint at this situation at every decisive turn of the search.'' 



{\em Laws} 819d--820d


{\em Theaetetus} 147a--b \cite[p.~22]{theaetetus}, Socrates says, ``Suppose we were asked about some obvious common thing, for instance, what clay is; it would be absurd to answer: potters' clay, and
oven-makers' clay, and brick-makers' clay.'' ``To begin with, it is absurd to imagine that our  answer conveys any meaning to the questioner, when we use the word
`clay', no matter whose clay we call it --  the doll-maker's or any other craftsman's. You do not suppose a man can understand the name of a thing, when he does not
know what the thing is?'' Then (147c), ``And besides, we are going an interminable way round, when our answer might be quite short and simple. In this question about clay,
for instance, the simple and ordinary thing to say is that clay is earth mixed with moisture, never mind whose clay it may be.''

Then  (147d--148b):

\begin{quote}
Theaetetus: Theodorus here was proving to us something about square roots, namely that the sides (or roots) of squares representing three square feet and five square feet
are not commensurable in length with the line representing one foot; and he went on in this way, taking all the separate cases up to the root of seventeen square feet. There for
some reason he stopped. The idea occurred to us, seeing that these square roots were evidently infinite in number, to try to arrive at a single collective term by which we could
designate all these roots.

Socrates: And did you find one?

Theaetetus: I think so; but I should like your opinion.

Socrates: Go on.

Theaetetus: We divided number in general into two classes. Any number which is a product of a number multiplied by itself we likened to the square figure, and we called
such a number `square' or `equilateral'.

Socrates: Well done.

Theaetetus: Any intermediate number, such as 3 or 5 or any number that cannot be obtained by multiplying a number by itself, but has one factor either greater
or less than the other, so that the sides containing the corresponding figure are always unequal, we likened to the oblong figure, and we called it an oblong number.

Socrates: Excellent; and what next?

Theaetetus: All the lines which form the four equal sides of the plane figure representing the equilateral number we defined as {\em length}, while those
which form the sides of squares equal in area to the oblongs we called `{\em roots}' (surds), as not being commensurable with others in length, but only in the
plane areas to which their squares are equal. And there is another distinction of the same sort in the case of solids.
\end{quote}

Brown \cite{brown} on {\em Theaetetus}.


{\em Timaeus} 36b \cite[pp.~71--72]{timaeus}:

\begin{quote}
And he went on to fill up all the intervals of $\frac{4}{3}$ (i.e. fourths) with the interval $\frac{9}{8}$ (the tone), leaving over in each a fraction.
This remaining interval of the fraction had its terms in the numerical proportion of $256$ to $243$ (semitone).
\end{quote}

54c--d \cite[p.~212]{timaeus}: 

\begin{quote}
Now all triangles are derived from two, each having one right angle and the other angles acute...
\end{quote}

53d -- 54b \cite[pp.~213--214]{timaeus}: among scalene triangles, the best of them for the construction of bodies is that a pair of which is an equilateral triangle, which has ``the greater
side triple in square of the lesser''; among the isosceles triangles...

Constructs the tetrahedron, octahedron, icosahedron, and cube 54d--55c \cite[pp.~216--218]{timaeus}. 

57c--d \cite[p.~235]{timaeus}

cf. Chalcidius, {\em On Plato's Timaeus}

{\em Republic} 546

{\em Republic} 546b--d, translated by Thomas \cite[pp.~398--401]{thomasI}.

McNamee and Jacovides \cite{papyrus}

Fossa and Erickson \cite{births}






\section{Aristotle}
{\em Topics} VIII.3, 158b29-35 \cite[p.~80]{aristotle}:

\begin{quote}
In mathematics, too, some things would seem to be not easily
proved for want of a definition, e.g. that the straight line, parallel
to the side, which cuts a plane [a parallelogram] divides similarly
both the line and the area. But, once the definition is stated, the
said property is immediately manifest; for the (operation of)
reciprocal subtraction applicable to both the areas and the lines is
the same (or gives the same result); and this is the definition of the
same ratio.
\end{quote}

Alexander of Aphrodisias \cite[p.~507]{thomasI} writes in his commentary on this passage:

\begin{quote}
For likewise when this is stated it is not obvious; but when the definition of proportion is enunciated it becomes obvious that both the line and the area are cut in the same proportion by the line drawn parallel. For the definition of proportions which those of old time used is this: Magnitudes which have the same alternating subtraction (anthyphairesis) are proportional. But he has called anthyphairesis antanairesis.
\end{quote}





\section{Plutarch}
Plutarch, {\em De animae procreatione in Timaeo} 17 ({\em Moralia} XIII) \cite[pp.~303--309]{chernissI}:

\begin{quote}
What the ``leimma'' is and what is Plato's meaning you will perceive more clearly, however, after having first been reminded briefly of the customary statements in the Pythagorean treatises. For an interval in music is all that is encompassed by two sounds dissimilar in pitch; and of the intervals one is what is called the tone, that by which the fifth is greater than
 the fourth. The harmonists think that this, when divided in two, makes two intervals, each of which they call a semitone; but the Pythagoreans denied that it is divisible into equal
 parts and, as the segments are unequal, name the lesser of them ``leimma'' because it falls short of the half. This is also why among the consonances the fourth is by the former 
 made to consist of two tones and a semitone and by the latter of two and a ``leimma.'' Sense-perception seems to testify in favour of the harmonists but in favour of the
 mathematicians demonstration, the manner of which is
 as follows. It has been found by observation with instruments that the octave has the duple ratio and the fifth the sesquialteran and the fourth the sesquitertian and the tone
 the sesquioctavan. It is possible even now to test the truth of this either by suspending unequal weights from two strings or by making one of two pipes with equal cavities double
 the length of the other, for of the two pipes the larger will sound lower as hypat\^e to n\^et\^e and of the strings the one stretched by the double weight will 
 sound higher than the other as n\^et\^e to hypat\^e. This is an octave. Similarly too, when lengths and weights of three to two are taken, they will produce the fifth and of four to three
 the fourth, the latter of which has sesquitertian ratio and the former sesquialteran. If the inequality of the weights or the
 lengths be made as nine to eight, however, it will produce an interval, that of the tone, not concordant but tuneful because,to put it briefly, the notes it gives, if they are struck 
 successively, sound sweet and agreeable but, if struck together, harsh and painful, whereas in the case of consonances, whether they be struck together or alternately, the sense
  accepts with pleasure the combination of sounds. What is more, they give a rational demonstration of this too. The reason is that in a musical scale the octave is composed of the 
  fifth and the fourth and arithmetically the duple is composed of the sesquialter and the sesquiterce, for twelve is four thirds of nine and half again as much as eight and twice as
   much as six. Therefore the ratio of the duple is composite of the sesquialter and the sesquiterce just as that of the octave is of the fifth and the fourth, but in that case the fifth is 
   greater than the fourth by a tone and in this the sesquialter greater than the sesquiterce by a sesquioctave. It is apparent, then, that the octave
has the duple ratio and the fifth the sesquialteran and the fourth the sesquitertian and the tone the sesquioctavan,
\end{quote}

18 \cite[pp.~309--315]{chernissI}:

\begin{quote}
Now that this has been demonstrated, let us see whether the sesquioctave is susceptible of being
divided in half, for, if it is not, neither is the tone. Since nine and eight, the first numbers producing the sesquioctavan ratio, have no intermediate interval but between them when both
are doubled the intervening number produces two intervals, it is clear that, if these intervals are equal, the sesquioctave is divided in half. But now twice nine is eighteen and twice
 eight sixteen; and between them these numbers contain seventeen, and one of the intervals turns out to be larger and the other smaller, for the former is eighteen seventeenths and
 the second is seventeen sixteenths. It is into unequal parts, then, that the sesquioctave is divided; and, if this is, the tone is also. Neither of its segments, therefore, when it is divided, 
 turns out to be a semitone; but it has rightly been called by the mathematicians ``leimma.''  This is just what Plato says god in filling in the sesquiterces with the sesquioctaves leaves
  a fraction of each of them, the ratio of which is 256 to 243. For let the fourth be taken as expressed by two numbers comprising the sesquitertian
ratio, 256 and 192; and of these let the smaller, 192, be placed at the lowest note of the tetrachord and the larger, 256, at the highest. It is to be proved that, when this is filled in with
 two sesquioctaves, there is left an interval of the size that numerically expressed is 256 to 243. This is so, for, when the lower note has been raised a tone, which is a sesquioctave,
 it amounts to 216; and, when this has been raised again another tone, it amounts to 243, for the latter exceeds 216 by 27 and 216 exceeds 192 by 24, and of these 27 is an eighth
 of 216 and 24 an eighth of 192. Consequently, of these three numbers the largest turns out to be sesquioctavan of the intermediate and the intermediate sesquioctavan of
the smallest; and the interval from the smallest to the largest, i.e. that from 192 to 243, amounts to an interval of two tones filled in with two sesquioctaves. When this is subtracted, 
 however, there remains of the whole as an interval left over what is between 243 and 256, that is thirteen; and this is the very reason why they named this number ``leimma.'' 
So I, for my part, think that Plato's intention is most clearly explained by these numbers.
\end{quote}

19 \cite[pp.~315--317]{chernissI}:

\begin{quote}
As terms of the fourth, however, others put the high note at 288 and the low at 216 and then determine proportionally those that come next, except that they take the ``leimma'' to be
 between the two tones. For, when the lower note has been raised a tone, the result is 243 and, when the higher has been lowered a tone, it is 256, for 213 is nine eighths of 216 and
 288 nine eighths of 256, so that each of the two intervals is that of a tone and there is left what is between 243 and 256; and this is not a semitone but
is less, for 288 exceeds 256 by 32 and 243 exceeds 216 by 27 but 256 exceeds 243 by thirteen, which is less than half of both the excesses 32 and 27. Consequently it turns out
that the fourth consists of two tones and a ``leimma,'' not of two tones and a half. Such, then, is the demonstration of this point. As to the following point, from what has been said
before it is not very difficult either to see why, after Plato had said that there came to be intervals of three to two and of four to three and of nine to eight, when saying that those of four
to three are filled in with those of nine to eight he did not mention those of three to two but omitted them. The reason is that the sesquialter {$\langle$}is greater than{$\rangle$} the 
sesquiterce by the sesquioctave {$\langle$}so that with the sesquioctave's{$\rangle$} addition to the sesquiterce the sesquialter is filled in as well.
\end{quote}

20 \cite[pp.~317--321]{chernissI}:

\begin{quote}
After the exposition of these matters the task of filling in the intervals and inserting the means I should still have left to you for an exercise to do yourselves though no one at all had
happened to have done it before; but now that this has been worked out by many excellent men and especially by Crantor and Clearchus and Theodorus, all of Soli, it is not 
unprofitable to say a few words about the way in which they disagree. For Theodorus unlike those others does not make two rows but sets out the double and the triple numbers one 
after another in a single straight line, relying for this in the first place
upon what is stated to be the cleavage of the substance lengthwise that makes two parts presumably out of one, not four out of two, and in the second place saying that it is suitable
for the insertions of the means to be arranged in this sequence, as otherwise there will be disorder and confusion and transpositions to the very first triple from the first double of the
terms that ought to fill in each of the two. Crantor and his followers, however, are supported by the position of the numbers, paired off with plane numbers over against plane and
square over against square and cubic over against cubic numbers, and
in their being taken not in order but alternately even and (30 b.) odd by {$\langle$}Plato himself{$\rangle$}. For after
putting at the head the unit, which is common to both, he takes eight and next thereafter twenty-seven, all but showing us the position that he assigns to each of the two kinds.
Now, to treat this with greater precision is a task that belongs to others; but what remains is a proper part of our present disquisition.
\end{quote}








\section{Euclid}
Euclid, {\em Elements} \cite{euclidI}, \cite{euclidII}, \cite{euclidIII}

I.32: in a triangle the exterior angle is equal to the sum of the opposite interior angles, and the sum of all the interior angles is two right angles.

I.43: complements in a parallelogram

I.45: to construct a parallelogram with a given angle and area equal to a given rectilinear figure

I.47: Pythagorean theorem

II.2: let $AB$ be a line and let $C$ be point on $AB$, then square on $AB$ is the rectangle with sides $AB,BC$ and the rectangle
$AB,AC$.

II.4: let $AB$ be a line and let $C$ be a point on $AB$, then the square on $AB$ is the square on $AC$ and the square on $CB$
and twice the rectangle on $AC,CB$. 

II.5: let $AB$ be a line divided into unequal segments $AD,DB$ and let $C$ be the midpoint of $AB$. Then the rectangle on $AD,DB$ and the square on $CD$
is the square on $CB$.

II.6: let $AB$ be a line and $C$ its midpoint. Extend $AB$ to $AD$. Then the rectangle on $AD,DB$ and the square on $CB$ is the square on $CD$.

II.10: let $AB$ be a line with $C$ its midpoint, and extend $AB$ to $AD$. Then the square on $AD$ and the square on $DB$ is twice the square on $AC$ and twice the square
on $CD$. 

II.11: to divide a line $AB$ into two segments, the larger $AH$ and the smaller $HB$, such that the square on $AH$ is the rectangle
on $AB,BH$. 

II.14: to construct a square whose area is the a given rectilineal figure.

III.20: let $BC$ be points on the circumference of a circle, with center $O$. Let $A$ be on the circumference. Then
the central angle $BOC$ is twice the inscribed angle $BAC$. 

III.26: let $BC$ and $EF$ be arcs of equal circles and suppose the central angles $BOC$ and $EOF$ are equal, then
the arcs $BC$ and $EF$ are equal, and likewise if the inscribed angles $BAC$ and $EDF$ are equal.

III.27: if the arcs are equal then the angles are equal.

III.28: in two equal circles if chords $CB$ and $EF$ are equal then the arc $CB$ is equal to the arc $EF$.

III.29: in two equal circles if arcs $CB$ and $EF$ are equal, then chords $CB$  and $EF$ are equal.

III.32: let $BC$ be the arc of a circle and $BF$ the tangent at point $B$. Let $D$ lie on the arc determined by $B$ and $C$  not included
in the angle $CBF$. Then the angles $BDC$ and $FBC$ are equal.

III.37: let $D$ be a point outside a circle with center $F$ and let $DA$ be a secant that cuts the circle at $C$. Let $B$ be a point on the circle
such that the rectangle on $AD,DC$ is equal to the square on $DB$.
Then the line $DB$ is tangent to the circle at $B$.

IV.2: to inscribe a given triangle in a circle.

IV.6: to inscribe a given square in a circle.

IV.10: to construct an isosceles triangle where each base angle is twice the summit angle.

IV.11: to inscribe a regular pentagon in a given circle.

IV.12: to circumscribe a regular pentagon about a given circle.

IV.13: to inscribe a circle in a given regular pentagon.

IV.14: to circumscribe a circle about a given regular pentagon.

IV.15: to inscribe a regular hexagon in a circle.

V.15: let $a:b=c:d$. If $a>c$ then $b>d$, if $a=c$ then $b=d$, and if $a<c$ then $b<d$. 

VI.1: if two triangles hae the same altitude then the areas of the triangles have the same ratio as their bases, and likewise for parallelograms.

VI.8: in a right triangle drop a perpendicular from the vertex of the right angle to the hypotenuse. Then the two new right triangles are similar
to the original.

VI.14

VI.16 mean, extreme 

VI.17 mean, extreme

VI.25: to construct a rectilinear figure similar to a given rectilinear figure and with the same area as another given rectilinear
figure.

Plutarch, {\em Quaestiones Convivales} VIII.2.4, 720A \cite[p.~177]{thomasI}:

\begin{quote}
Among the most geometrical theorems, or rather
problems, is this -- given two figures, to apply a third
equal to the one and similar to the other; it was
in virtue of this discovery they say Pythagoras
sacrificed. This is unquestionably more subtle and
elegant than the theorem which he proved that the
square on the hypotenuse is equal to the squares on
the sides about the right angle.
\end{quote}

VI.28

VI.30: mean and extreme ratio

VI.33

X.1

X.2

X.3

X.9

X.21 medial

X.24 medial area

X.36 binomial

X.73 apotome

X.76 minor

XIII.1

XIII.2

XIII.3

XIII.4

XIII.5

XIII.6: apotome

XIII.8: regular pentagon

XIII.9

XIII.10

XIII.11: minor

XIII.12

XIII.13, Lemma 

XIV, Theorem 1

XIV, Lemma to Theorem 3





Knorr \cite{knorr1975}

{\em Optics}, Proposition VIII \cite[pp.~260--261]{drabkin}: if $\alpha,\beta$ are acute angles
and $\alpha<\beta$ then $\tan \alpha:\tan \beta<\alpha:\beta$.









\section{Archimedes}
Archimedes, {\em Measurement of a Circle}, Proposition 3 states that 
if $d$ is the diameter of a circle and $c$ is the circumference of the circle then
$\left(3+\frac{10}{71}\right) d < c < \left(3+\frac{1}{7}\right)d$ \cite[pp.~223--238]{dijksterhuis}.
To prove $c<\left(3+\frac{1}{7}\right)d$, 
it is taken as granted that $\sqrt{3}:1>265:153$.
To prove $c>\left(3+\frac{10}{71}\right)d$, it is taken as granted that
$\sqrt{3}:1<1351:780$. 

Heath \cite{archimedes}

{\em Sand Reckoner}

Knorr \cite[p.~522]{knorr1989}: Eutocius  

Hultsch \cite{naherungswerte}

Hofmann \cite{hofmann}







\section{Aristarchus of Samos}
Aristarchus, {\em On the Sizes and Distances of the Sun and Moon}, Proposition 4 \cite[p.~367]{aristarchus}.
In the proof of this proposition, the following is taken as granted: in a triangle $BAD$ 
where $ADB=90^\circ$ and $BAD=1^\circ$, 
$BAD:45^\circ > BD:DA$. (This is an instance of
$\frac{\tan \beta}{\tan \alpha} < \frac{\beta}{\alpha}$ when $\alpha$ and $\beta$ are acute angles with $\beta<\alpha$; here
$\alpha=\frac{1}{2}ADB=45^\circ$ and $\beta = BAD=1^\circ$.)
It follows that $\tan 1^\circ = BD:DA < \frac{1}{45}$; in fact,
$\tan 1^\circ = ;1,2,50,\ldots$ and $\frac{1}{45}  = ; 1,20$. 
In this proposition, $BD$ is the radius of a circle such that $DA$ touches the circle at $D$. Furthermore, let $BF$ be the radius of the circle
that is perpendicular to $BA$, the line $BA$ cuts this circle at $G$, and $H$ is taken on the circle such that the arc $FD$ is equal
to the arc $HG$. 



Proposition 7 \cite[p.~379]{aristarchus}: let $B$ be the center of a circle with radii $BE,BA$ that are perpendicular. Let 
when $ABEF$ is a square and $AG$ bisects $FE$,
then $FB^2:BE^2 = FG^2:GE^2$.  
But $FB^2:BE^2 = 2:1$, so $FG^2:GE^2 = 2:1$. It is stated that because
$49:25<2:1$, then $FG^2:GE^2 > 49:25$, and then $FG:GE > 7:5$. 
If $BCA$ is a right triangle where $CAB = 3^\circ$, then $AB > 18 BC$, which amounts to
$\sin 3^\circ < \frac{1}{18}$. 
Conversely, let $DKB$ be a right triangle, with $BDK=3^\circ$. Circumscribe this triangle, and in the circumscribed circle, inscribe a
regular hexagon one of whose sides is $BL$. Now, because $DKB$ is a right angle, $BD$ is  a diameter of the circle.
The side of the inscribed hexagon is equal to the radius of the circle ({\em Elements} IV.15). Thus, 
$BD:BL=2:1$. The arc $BL$ is $60^\circ$ and the arc $BK$ is $6^\circ$. But the arc $BL$ has to the arc $BK$ a ratio greater than $BL$ has to $BK$. (This is an instance of
$\alpha:\beta>\chord \alpha:\chord \beta$ when $\alpha$ and $\beta$ are acute angles and $\alpha>\beta$.)
Therefore $10:1 > BL:BK$, and as $BD:BL=2:1$ we get
$BD:BK < 20:1$. This means $\sin 3^\circ = \sin BDK = \frac{BK}{BD} > \frac{1}{20}$. 


Proposition 13 \cite[p.~397]{aristarchus}: $7921:4050>88:45$. This can be found as follows.
$7921=4050+3871$, $4050=3871+179$, thus
$\frac{7921}{4050}=1+\frac{3871}{4050}$ and $\frac{4050}{3871}=1+\frac{179}{3871}$, so
\[
\frac{7921}{4050}=1+\frac{3871}{4050} = 1+\cfrac{1}{1+\cfrac{179}{3871}}.
\] 
Next, 
$3871=21\cdot 179+112$, $179=112+67$, thus
\[
\frac{3871}{179} = 21 +  \frac{112}{179} = 21+\cfrac{1}{1+\cfrac{67}{112}},
\]
hence
\[
\frac{7921}{4050} = 1+\cfrac{1}{1+\cfrac{1}{21+\cfrac{1}{1+\cfrac{67}{112}}}}
\]
Finally, $112=67+45$, whence
\[
\frac{7921}{4050} = 1+\cfrac{1}{1+\cfrac{1}{21+\cfrac{1}{1+\cfrac{1}{1+\cfrac{45}{67}}}}}.
\] 
Then
\[
1+\cfrac{1}{1+\cfrac{1}{21+\cfrac{1}{1+\cfrac{1}{1+0}}}} = \frac{88}{45}
\]
is an approximation from below to $\frac{7921}{4050}$. 

Proposition 15 \cite[p.~407]{aristarchus}: $71755875:61735500>43:37$. This can be found as follows. 
$71755875 = 61735500+10020375$, $61735500 = 6\cdot 10020375 + 1613250$, $10020375 = 6\cdot 1613250+340875$. 
\begin{align*}
\frac{71755875}{61735500}&=1+\cfrac{1}{6+\cfrac{1}{6+\cfrac{340875}{1613250}}}.
\end{align*}
Then 
\[
1+\cfrac{1}{6+\cfrac{1}{6+0}}=\frac{43}{37}
\]
is an approximation from below to $\frac{71755875}{61735500}$. 



Neugebauer \cite{HAMA}







\section{Theodosius}
There are scarcely any detailed modern expositions of spherical trigonometry; one is Ratcliffe \cite{ratcliffe}, Chapter 2.



Theodosius, {\em Sphaerica} III.11: ver Eecke \cite{theodosius} and Heiberg \cite{heiberg1927}





\section{Menelaus}
Menelaus \cite{menelaos}







\section{Eratosthenes}
Goldstein \cite{goldstein}

Neugebauer \cite[pp.~336, 746--748]{HAMA}





\section{Hipparchus}
$\sqrt{9750000} \sim 3122 + \frac{1}{2}$. cf. Heron, {\em Metrica} \cite[pp.~18--20]{heronisIII}
and Ptolemy, {\em Almagest} IV.11. Toomer \cite[p.~211]{almagest}.

Hipparchus, {\em Commentary on Aratus} I.3.5--7 \cite[p.~]{heidel}, cf. Manitius \cite[p.~27]{manitius}, says
the following about Aratus, {\em Phaenomena} 497:

\begin{quote}
In the first place, Aratus seems to me to be mistaken in thinking the latitude of Greek lands to be such that
the ratio of the longest day to the shortest is as 5 to 3; for he says of the summer tropic, `If you measure it as accurately
as possible and divide it into eight parts, five in the daylight will turn above the earth, and three below it.' Now it is
agreed that in Greek lands the gnomon at the equinox is to its midday shadow in the ratio of 4 to 3. Consequently the 
longest day has a length of $14 \frac{3}{5}$ hours and the latitude is approximately $37^\circ$. Where, however,
the longest day is to the shortest as 5 to 3, the longest day has 15 hours and the latitude is approximately $41^\circ$.
Consquently it is evident that the latter ratio does not hold for Greek lands, but rather for the region about the
Hellespont.
\end{quote} 


Hipparchus says
\[
\sin \frac{1}{2}(a-12) 15^\circ = \tan \phi \tan \omega,
\]
where $a$ is the number of hours in the longest day, $\phi$ is the latitude of the place, and $\omega$
is the latitude of the tropic.

Cohen and Drabkin \cite[pp.~82--86]{drabkin}


Neugebauer \cite{HAMA}

Hipparchus, Fragment 41 \cite[p.~91]{dicks}, in the {\em Almagest} I.67.22 and Theon of Alexandria's
{\em Commentary on the Almagest}:

\begin{quote}
I have taken the arc from the northernmost limit to the most southerly, that is the arc between the tropics, as being always $47^\circ$ and more than two-thirds
but less than three-quarters of a degree, which is nearly the same estimate as that of Eratosthenes and which Hipparchus also used; for the arc between the tropics
amounts to almost exactly 11 of the units of which the meridian contains 83.

[Theon's comment.] This ratio is nearly the same as that of Eratosthenes, which Hipparchus also used because it had been accurately mearued; for Eratosthenes determined
the whole circle as being 83 units, and found that part of it which lies between the tropics to be 11 units; and the ratio $360^\circ: 47^\circ 42' 40''$ is the same as $83:11$.
\end{quote}

In fact, $360^\circ: 47^\circ 42' 40'' = 16200:2147$, and using the Euclidean algorithm we get the approximations $7,8,15:2,
83:11,16200:2147$. 








\section{Geographers}
Strabo, {\em Geography} 1.1.8 \cite[p.~40]{strabo}: ``That the inhabited world is an island must be assumed both from the senses as well
as experience.''

1.1.12, Hipparchus Fragment 11 \cite[p.~65]{dicks}:

\begin{quote}
At all events it is a fact that many men have spoken of the
necessity for wide learning in relation to this subject [i.e.
geography]. Hipparchus also rightly points out in his treatise
aginst Eratosthenes that, while geographical knowledge is the
concern of everyone whether layman or scholar, it is impossible
to attain it without consideration of the heavens and of the
observations of eclipses; thus one cannot determine whether
Alexandria in Egypt is north or south of Babylon, or by how
much, without investigation by means of the {\em climata}. Similarly
one cannot decide accurately whether places are situated to
a greater or less degree towards the east or west except by
comparison of [the times of] eclipses of the sun and moon.
This is what Hipparchus says, anyway.
\end{quote}

1.1.20  \cite[p.~45]{strabo}: ``One must assume that the universe is sphere-shaped,
and that the surface of the earth is sphere-shaped, and moreover, what is fundamental
to this, that the motion of the [heavenly] bodies is toward the center.'' 

2.1.29, Hipparchus Fragment 22  \cite[pp.~73--75]{dicks}:

\begin{quote}
Hipparchus, taking these things for granted and having shown,
as he thinks, that according to Eratosthenes Babylon is a little
more than 1000 stades further east than Thapsacus, again
gratuitously fabricates an assumption for his own use in his
next argument; for he says that, if a straight line is assumed
drawn from Thapsacus towards the south, and a line
perpendicular to it from Babylon, a right-angled triangle will be
formed composed of the side drawn from Thapsacus to Babylon,
of the perpendicular drawn from Babylon to the meridian
through Thapsacus, and of the meridian itself through
Thapsacus. In this triangle he makes the hypotenuse the line from
Thapsacus to Babylon, which he says is 4800 stades, and the
perpendicular from Babylon to the meridian through
Thapsacus a little more than 1000 stades, which is the amount by
which the line to Thapsacus [from the Caspian Gates] exceeds
that up to Babylon [from the frontier between Carmania and Persia];
and from this he also calculates the remaining side
about the right angle to be many times longer than the said
perpendicular.
\end{quote}

Using $4800$ stades for the hypotenuse and $1000$ stades for one side,
the other side will be about $4695$ stades, which is indeed much longer than 1000 stades.

2.1.34, Hipparchus Fragment 24 \cite[p.~77]{dicks}:

\begin{quote}
Neither is his subsequent conclusion correct. For, since Eratosthenes
had given the distance from the Caspian Gates to
Babylon as stated above [i.e. 6700 stades], from the Caspian
Gates to Susa 4900 stades, and from Babylon to Susa as
3400 stades, Hipparchus, again starting from the same
hypotheses, says that an obtuse-angled triangle is formed with the 
Caspian Gates, Susa and Babylon at its vertices, having the
obtuse angle at Susa and the lengths of its sides as set out above.
Then he concludes that it will follow from these hypotheses
that the point of intersection of the meridian line through the
Caspian Gates and the parallel through Babylon and Susa is
more than 4400 stades further west than the intersection of
the same parallel with the straight line running from the
Caspian Gates to the borders of Carmania and Persia; and
that, in fact, this latter line makes an angle of about $45^\circ$ with
a direction half-way between the south and the equinoctial
east; and that the river Indus runs parallel to this line, so that
this river also does not flow due south from the mountains, as
Eratosthenes says it does, but in a direction between south and
the equinoctial east, just as it has been drawn in the ancient
maps.
\end{quote}

2.5.7, Eratosthenes Fragment 34 \cite[p.~63]{eratosthenes}:

\begin{quote}
Since, according to Eratosthenes, the equator is 252,000
stadia, one fourth would be 63,000. This is the distance from the equator
to the pole, fifteen sixtieths of the sixty [intervals] of the equator.
From the equator to the summer tropic is four [sixtieths], and this is the
parallel drawn through Syene. Each of these distances is computed from
known measurements. The tropic lies at Syene because there at the
summer solstice a gnomon has no shadow in the middle of the day. The
meridian through Syene is drawn approximately along the course of
the Nile from Mero\"e to Alexandria, which is about 10,000 stadia. It happens
that Syene lies in the middle of that distance, so that from there to
Mero\"e is 5,000.
\end{quote}

Mero\"e is between the 5th and 6th cataracts of the Nile. 


2.5.10  \cite[p.~134]{strabo}: it will make only a small difference if a map of the inhabited earth is drawn on a flat surface at least 7 feet long rather than a sphere with diameter at least 10 feet:

\begin{quote}
It will make only a small difference if we draw the parallels and meridians with straight lines, by which we plainly show the latitudes, winds, and other differences, as well
as the positioning of the parts of the earth relative to each other and the heavens, parallel [lines] for the parallels, and ones at right angles for those at right
angles, for the difference can easily be transferred from what is seen by the eye on a flat surface to the form and size carried around the sphere. We can say that the oblique
circles and their straight lines are analogous. Although the various meridians drawn through the pole converge on the sphere toward a single point, on the surface of the plan
there is no difference if the straight lines converge slightly, but there is often no necessity for this, nor is it obvious when the circumferential and converging lines are 
transferred to the surface of the plan and drawn as straight lines.
\end{quote}

2.5.16, Eratosthenes Fragment 46 \cite[p.~69]{eratosthenes}:

\begin{quote}
Such being the shape of the entire [inhabited world], it appears useful
to take two straight lines, which cut across each other at a right angle,
one going through all the greatest width and the other the length, and
the first will be one of the parallels and the other one of the meridians.
Then one should think of lines parallel to these on either side, which are
used to divide the land and the sea that we happen to use. Thus the
shape will be somewhat more clear, as I have described, according to the
length of the line, with different measurements for both the length and
width, and the terrestrial regions will be better manifested, both in the
east and west as well in as the south and north.
\end{quote}

2.5.38, Hipparchus Fragment 48 \cite[p.~95]{dicks}:

\begin{quote}
In the regions some 400 stades south of the parallel through
Alexandria and Cyrene, where the longest day is 14 equinoctial
hours, Arcturus reaches the zenith, but decline a little towards
the south. In Alexandria the gnomon bears to its equinoctial
shadow a ratio of $5:3$. These regions are 1300 stades south of
Carthage, if it be true that in Carthage the gnomon has a ratio
of $11:7$ for its equinoctial shadow.
\end{quote}

2.5.39, Eratosthenes Fragment 60  \cite[pp.~148--149]{strabo}: ``In the region of Ptolemais -- the one in Phoenicia -- and Sidon and Tyre, the longest day has $14 \frac{1}{4}$ equinoctial hours [Hipparchos, F49]. These regions
are about 1,600 stadia father north than Alexandria and about 700 from Karchedon. In the Peloponnesos and around the middle of the Rhodia,
around Xanthos in Lykia or a little to the south, and also 400 stadia south of Syracuse, the longest day has $14 \frac{1}{2}$ equinoctial hours [Hipparchos, F50]. These places are 3,640 from
Alexandria and $\langle$ 2,740 from Karchedon $\rangle$.''

2.5.40, Eratosthenes Fragment 60 \cite[p.~149]{strabo}: ``In the area around Alexandria Troas, around Amphipolis, Apollonia in Epeiros, and south of Rome but north of Neapolis, the longest day has
15 equinoctial hours [Hipparchos, F51]. The parallel is about 7,000 stadia north of the one through Alexandria next to Egypt and more than 28,800 from the equator, 3,400 from the one through Rhodes,
and 1,500 south of Byzantion, Nikaia and the region around Massalia.''

2.5.41, Eratosthenes Fragment 60  \cite[p.~149]{strabo}: ``In the regions around Byzantion the longest day has $15 \frac{1}{4}$ equinoctial hours and the relationship of the gnomon to its shadow at the summer
solstice is 120 to 42 less a fifth [Hipparchos, F52]. These places are 4,900 [stadia] from [the parallel] through the center of the Rhodia and about 30,300 from the equator.''


2.5.42 \cite[p.~149]{strabo}: ``In the regions 3,800 [stadia] to the north of Byzantion the longest day has 16 equinoctial hours, and thus Cassiopeia appears within the arctic
circle [Hipparchos, F57]. These are the places around the Borysthenes and the southern parts of the Maiotis, about 34,100 from the equator.''

Strabo in Books I and II of the {\em Geography} \cite{strabo} reports  distances
between various locations stated by Eratosthenes, Hipparchus, and Polybius. We organize these
in Table \ref{distances}.
(Thapsacus = Euphrates)

\begin{table}[h!]
\caption{Distances between locations reported in Strabo}
\label{distances}
\begin{tabular}{l l l l}
Lycia, Rhodes&Alexandria&4000 stades&1.2.17\\
Mero\"e&Alexandria&10000 stades&1.4.2\\
Alexandria&Hellespont&8100 stades&1.4.2\\
Hellespont&Borysthenes&5000 stades&1.4.2\\
Caspian Gates&Euphrates&10000 stades&1.4.5\\
Euphrates&Nile&5000 stades&1.4.5\\
Nile&Canopic mouth&1300 stades&1.4.5\\
Canopic mouth&Carthage&13500 stades&1.4.5\\
Mer\"oe&Hellespont&18000 stades&2.1.3\\
Byzantium&Borysthenes&3700 stades&2.1.12\\
Babylon&Thapsacus&4800 stades&2.1.21\\
Caspian Gates&Thapsacus&10000 stades&2.1.24, 2.1.39\\
Carthage meridian&Thapsacus meridian&6300 stades&2.1.39\\
Babylon&Carmania&9200 stades&2.1.23, 2.1.25\\
Thapsacus&Babylon&4800 stades&2.1.26\\
Thapsacus&Armenian Gates&1100 stades&2.1.26\\
Thapsacus&Caspian Gates&10000 stades&2.1.27\\
Babylon&Carmania&9000 stades&2.1.27\\
Thapsacus&Babylon&4800 stades&2.1.27\\
Rhodes&Alexandria&4000 stades&2.1.33\\
Babylon&Caspian Gates&6700 stades&2.1.34\\
Babylon&Carmania&9000 stades&2.1.34\\
Caspian Gates&Susa&4900 stades&2.1.34\\
Susa&Babylon&3400 stades&2.1.34\\
Syene&Mero\"e&5000 stades&2.5.7\\
Mero\"e&Alexandria&5000 stades&2.5.7\\
Rhodia&Byzantium&4900 stades&2.5.8
\end{tabular}
\end{table}




Agathemerus, {\em Sketch of Geography} IV \cite[pp.~69--70]{agathemerus}
states distances  between various places, and says that
the length of the inhabited earth from the Ganges to Gades is 68545 stades. IV.15: 
from the Caspian Gates
to the Euphrates is 10050 stades. IV.18: from Mero\"e to Alexandria is 10000 stades, 
and from Alexandria to Linus in Rhodes is 4500 stades.
IV.19: ``city to city'', from Alexandria to Rhodes is 4670 stades.

Pliny, {\em Natural History} 2.186; Books 3--6; 37.108, Philo on Mero\"e

Pliny, {\em Natural History} 5.36: ``But the most beautiful is the free island of Rhodes, which measures 125, or, if we prefer to believe
Isidore, 103 miles round, and which contains the cities of Lindus, Camirus and Ialysus, and now that
of Rhodes. Its distance from Alexandria in Egypt
is 583 miles according to Isidore, 468 according to Eratosthenes, 500 according to Mucianus; and it is 176 miles from Cyprus.''
The distance given by Eratosthenes corresponds to 3750 stades.




\section{Ptolemy}
Ptolemy, {\em Planetary Hypotheses}, Duke \cite{duke}

Ptolemy, {\em Geography} \cite[p.~90]{geography}: for a right triangle $BEZ$ where the base $EZ$ is $23 \frac{5}{6}$ units and the height
$BE$ is 90 units, the hypotenuse $BZ$ is
\[
BZ \sim 93 \frac{1}{10}.
\]

Pedersen \cite{pedersen} on the {\em Almagest}. See page 60 on the half-angle formula.

$\sin 30^\circ = \frac{1}{2}$, $\sin 45^\circ = \frac{1}{\sqrt{2}}$, and $\sin 36^\circ = \frac{\sqrt{10-\sqrt{20}}}{4}$.
Because
$\sin 30^\circ$ and $\sin 45^\circ$ are given, so is $\sin 75^\circ$, and because
$\sin 36^\circ$ is given, so is $\sin 72^\circ$. Therefore
$\sin 3^\circ$ is given; it turns out that
\[
\sin 3^\circ = \frac{2(1-\sqrt{3})\sqrt{5+\sqrt{5}}+(\sqrt{10}-\sqrt{2})(\sqrt{3}+1)}{16},
\] 
which is $\sin 3^\circ = \frac{3}{60}+\frac{8}{60^2}+\frac{24}{60^3}+\cdots$.
Since $\sin 3^\circ$ is given, so is $\sin \frac{3}{2}^\circ$ and then
$\sin \frac{3}{4}^\circ$.  When $\beta<\alpha<90^\circ$ it holds that
$\frac{\alpha}{\beta}>\frac{\sin \alpha}{\sin \beta}$;
this is proved in {\em Almagest} I.10 \cite[pp.~54--55]{almagest}.
Therefore,
\[
\frac{2}{3} \sin \frac{3}{2}^\circ < \sin 1^\circ < \frac{4}{3}\sin \frac{3}{4}^\circ.
\]
This yields
\[
;1,2,49,28,50,\ldots \; < \; \sin 1^\circ  \; < \; ;1,2,49,48,12,\ldots,
\]
so $\sin 1^\circ = ;1,2,49,\ldots$. In fact, $\sin 1^\circ = ;1,2,49,43,11,14,\ldots$. 

Ptolemy, {\em Almagest} I.10 \cite[p.~49]{almagest}:
\[
\sqrt{4500} \sim 67+\frac{4}{60}+\frac{55}{60^2}.
\]
I.10 \cite[p.~49]{almagest}:
\[
\sqrt{4975+\frac{4}{60}+\frac{15}{60^2}} \sim 70+\frac{32}{60}+\frac{3}{60^2}.
\]

I.11, the table of chords. For a circle with diameter $120$, for
an arc of the circle of $\theta$ degrees, let $\chord \theta$ be the length of the chord 
that joins the endpoints of the arc. This means
$\chord \theta = 120 \sin\left(\frac{\theta \pi}{360} \right)$. 
Now, $\sin\left( \frac{\pi}{4} \right) = \frac{\sqrt{2}}{2}$;
$\frac{\theta \pi}{360} = \frac{\pi}{4}$ is equivalent to
$\theta=90$, so
$\chord 90 = 120 \cdot \frac{\sqrt{2}}{2}$, i.e.
$\chord 90 = 60 \cdot \sqrt{2}$. In the table of chords, for
the arc $90$ the chord is $84+\frac{51}{60}+\frac{10}{60^2}$,
so
\[
\sqrt{2} = \frac{\chord 90}{60} \sim 1+\frac{24}{60}+\frac{51}{60^2}+\frac{10}{60^3}.
\]
Similarly, $\sin\left( \frac{\pi}{3} \right) = \frac{\sqrt{3}}{2}$;
$\frac{\theta \pi}{360} = \frac{\pi}{3}$ is equivalent to $\theta = 120$, so
$\chord 120 = 120 \cdot \frac{\sqrt{3}}{2}$, i.e. $\chord 120 = 60 \cdot \sqrt{3}$. 
In the table of chords, for the arc $120$ the chord is $103+\frac{55}{60}+\frac{23}{60^2}$, so
\[
\sqrt{3} = \frac{\chord 120}{60} \sim 1+\frac{43}{60}+\frac{55}{60^2}+\frac{23}{60^3}.
\]

 
III.5 \cite[p.~158]{almagest}: if $DK=1+\frac{15}{60}$ and $K\Theta Z=62+\frac{10}{60}$, then for
$DK^2+K\Theta Z^2=ZD^2$,
\[
ZD \sim 62+\frac{11}{60}.
\]
III.5 \cite[p.~160]{almagest}: if $ZK=1+\frac{15}{60}$ and $KD=62+\frac{10}{60}$,
then for $ZK^2+KD^2=ZD^2$,
\[
ZD \sim 62+\frac{11}{60}.
\]
III.5 \cite[p.~162]{almagest}: if $DK=1+\frac{15}{60}$ and $KZ=57+\frac{50}{60}$, then for
$DZ^2=DK^2+KZ^2$,
\[
DZ \sim 57+\frac{51}{60}.
\]
IV.6 \cite[pp.~195--196]{almagest}:
\[
\sqrt{291+\frac{14}{60}+\frac{36}{60^2}} \sim 17 + \frac{3}{60}+\frac{57}{60^2}. 
\]
IV.6 \cite[p.~197]{almagest}:
\[
\sqrt{476300+\frac{5}{60}+\frac{32}{60^2}} \sim 690+\frac{8}{60}+\frac{42}{60^2}.
\]
IV.6 \cite[p.~201]{almagest}:
\[
\sqrt{213+\frac{43}{60}+\frac{38}{60^2}} \sim  14+\frac{37}{60}+\frac{10}{60^2}.
\]
IV.6 \cite[pp.~201--202]{almagest}:
\[
\sqrt{474904+\frac{46}{60}+\frac{17}{60^2}} \sim 689+\frac{8}{60}.
\]
V.5 \cite[p.~231]{almagest}: if $BD=49+\frac{41}{60}$ and $DK = 10+\frac{19}{60}$, then for
$BK^2=BD^2-DK^2$,
\[
BK \sim 48 + \frac{36}{60}.
\]
V.6 \cite[p.~234]{almagest}: if $BX=48+\frac{26}{60}$ and $XN=10+\frac{19}{60}$, then for
$BX^2+XN^2=BN^2$,
\[
BN \sim 49+ \frac{31}{60}.
\]
V.6 \cite[p.~234]{almagest}: if $BE=48+\frac{31}{60}$, $LB=5+\frac{5}{60}$, $EL=BE+LB$,
$LH=1+\frac{20}{60}$, then for $EL^2+LH^2=EH^2$,
\[
EH \sim 53+\frac{37}{60}.
\]
V.10 \cite[p.~241]{almagest}: if $BD=49+\frac{41}{60}$ and $DM=2+\frac{38}{60}$, then  for
$BM^2=BD^2-DM^2$,
\[
BM \sim 49+ \frac{37}{60}.
\]
V.10 \cite[p.~242]{almagest}: if $BD=49+\frac{41}{60}$ and $DM=\frac{51}{60}$, then for
$BM^2=BD^2-DM^2$,
\[
BM \sim 49+\frac{41}{60}.
\]
V.13 \cite[p.~250]{almagest}: if $BD=49+\frac{41}{60}$ and $DM=4+\frac{8}{60}$, then for $BM^2=BD^2-DM^2$,
\[
BM \sim 49+\frac{31}{60}.
\]
V.13 \cite[p.~251]{almagest}: if $BL=5+\frac{15}{60}$ and $EB=40+\frac{4}{60}$, then for $EL^2=BL^2+EB^2$,
\[
EL \sim 40+\frac{25}{60}.
\]
V.17 \cite[p.~261]{almagest}: if $ZH=62+\frac{38}{60}$ and $HB=4+\frac{33}{60}$, then for
$ZB^2=ZH^2+HB^2$,
\[
ZB \sim 62+\frac{48}{60}.
\]
V.17 \cite[p.~262]{almagest}: if $BH=G\Theta=6+\frac{56}{60}$, $ZH=64$, and $Z\Theta=56$, then for
$ZB^2=ZH^2+BH^2$ and $ZG^2=Z\Theta^2+G\Theta^2$,
\[
ZB \sim 64+\frac{23}{60}, \qquad ZG \sim 56+\frac{26}{60}.
\]
V.17 \cite[p.~263]{almagest}: if $BE=49+\frac{41}{60}$ and $EH=8+\frac{56}{60}$, then for $BH^2=BE^2-EH^2$,
\[
BH \sim 48+\frac{53}{60}.
\]
V.19 \cite[p.~273]{almagest}:
\[
\sqrt{\left(42+\frac{30}{60}\right)^2 + \left(4+\frac{20}{60} \right)^2} \sim 42+\frac{46}{60},
\quad
\sqrt{\left(47+\frac{30}{60}\right)^2 + \left(4+\frac{20}{60} \right)^2} \sim 47+\frac{44}{60}.
\]
VI.7 \cite[p.~299]{almagest}:
\[
\sqrt{429+\frac{32}{60}} \sim 20+\frac{43}{60},\quad
\sqrt{460+\frac{52}{60}} \sim 21+\frac{28}{60},
\quad 
\sqrt{822+\frac{15}{60}} \sim 28+\frac{41}{60}.
\]
VI.7 \cite[p.~300]{almagest}:
\[
\sqrt{1045+\frac{35}{60}} \sim 32+\frac{20}{60}.
\]
VI.7 \cite[p.~301]{almagest}:
\[
\sqrt{2883+\frac{59}{60}} \sim 53+\frac{42}{60},
\qquad \sqrt{331+\frac{21}{60}} \sim 18+\frac{12}{60},
\]
and
\[
\sqrt{3667+\frac{19}{60}} \sim 60+\frac{34}{60},
\qquad \sqrt{421+\frac{21}{60}} \sim 20+\frac{32}{60}.
\]
IX.10 \cite[p.~463]{almagest}: if $DN=2+\frac{2}{60}$ and $NZ=55+\frac{49}{60}$, then for
$DZ^2=DN^2+NZ^2$,
\[
DZ \sim 55 + \frac{51}{60}.
\]
IX.10 \cite[pp.~465--466]{almagest}: if $ZH=60$ and $HM=5+\frac{7}{60}$, then for $ZM^2=ZH^2-HM^2$,
\[
ZM \sim 59+\frac{47}{60}.
\]
IX.10 \cite[p.~466]{almagest}: $DN=84+\frac{36}{60}$ and $ZN=64+\frac{5}{60}$, then for $ZD^2=ZN^2+DN^2$,
\[
ZD \sim 64+\frac{7}{60}.
\]
X.4 \cite[p.~476]{almagest}: if $ZG=60$ and $GL=\frac{34}{60}$, then for $ZG^2-GL^2=ZL^2$,
\[
ZL \sim 60;
\]
if $ZM=58+\frac{53}{60}$ and $DM=1+\frac{8}{60}$, then for $ZD^2=ZM^2+DM^2$,
\[
ZD \sim 58 + \frac{54}{60}.
\]
X.4 \cite[pp.~477--478]{almagest}: if $GZ=60$ and $GL=\frac{42}{60}$, then for $ZG^2-GL^2=ZL^2$,
\[
ZL \sim 60.
\]
X.4 \cite[p.~478]{almagest}: if $ZM = 58+\frac{58}{60}$ and $DM=1+\frac{24}{60}$, then for $ZD^2=ZM^2+DM^2$,
\[
ZD \sim 58+\frac{59}{60}.
\]
X.7 \cite[p.~488]{almagest}:
\[
\sqrt{19289+\frac{32}{60}} \sim 138+\frac{53}{60}.
\]
X.7 \cite[p.~489]{almagest}:
\[
\sqrt{172+\frac{9}{60}} \sim 13+\frac{7}{60}.
\]
X.7 \cite[p.~493]{almagest}: if $DG=60$ and $DF=4+\frac{9}{60}$, then for $GD^2-DF^2=GF^2$,
\[
GF \sim 59+\frac{51}{60}.
\]
X.7 \cite[p.~495]{almagest}: if $DA=60$ and $DF=3+\frac{58+\frac{1}{2}}{60}$,
then for $DA^2-DF^2=FA^2$,
\[
FA \sim 59 + \frac{50}{60}.
\]
X.7 \cite[p.~496]{almagest}: if $DB=60$ and $DF=3+\frac{52}{60}$, then for
$DB^2-DF^2=BF^2$,
\[
BF \sim 59+\frac{53}{60}.
\]
X.7 \cite[p.~498]{almagest}: if $DG=60$ and $DF=4+\frac{11+\frac{1}{2}}{60}$, then for $DG^2-DF^2=GF^2$,
\[
GF \sim 59+\frac{51}{60}.
\]
X.8 \cite[p.~501]{almagest}: if $DB=60$ and $DM=4+\frac{5}{60}$, then for $DB^2-DM^2=BM^2$,
\[
BM \sim 59+\frac{52}{60}.
\]
XI.1 \cite[p.~512]{almagest}: if $DA=60$ and $DH=2+\frac{39}{60}$, then for $DA^2-DH^2=AH^2$,
\[
AH \sim 59+\frac{56}{60}.
\]
XI.1 \cite[p.~514]{almagest}: if $DG=60$ and $DH=1+\frac{28}{60}$, then for
$GD^2-DH^2=GH^2$,
\[
GH \sim 59+\frac{59}{60}.
\]
XI.1 \cite[p.~516]{almagest}: if $DA=60$ and $DH=2+\frac{41}{60}$, then for $AH^2=AD^2-DH^2$,
\[
AH \sim 59+\frac{56}{60}.
\]
XI.2 \cite[p.~521]{almagest}: if $DB=60$ and $DM=2+\frac{44}{60}$, then for
$DB^2-DM^2=MB^2$,
\[
MB \sim 59+\frac{56}{60}.
\]
XI.5 \cite[p.~528]{almagest}: 
\[
\sqrt{50+\frac{51}{60}} \sim 7+\frac{8}{60}.
\]
XI.5 \cite[p.~529]{almagest}: if $DA=60$ and $DH=2+\frac{57}{60}$, then for $DA^2-DH^2=AH^2$,
\[
AH \sim 59+\frac{56}{60}.
\]
XI.5 \cite[p.~531]{almagest}: if $DB=60$ and $DH=1+\frac{13}{60}$, then for $DB^2-DH^2=BH^2$,
\[
BH \sim 59+\frac{59}{60}.
\]
XI.5 \cite[p.~533]{almagest}: if $DG=60$ and $DH=3+\frac{1}{60}$, then for $DG^2-DH^2=GH^2$,
\[
GH \sim 59+\frac{56}{60}.
\]
XI.5 \cite[p.~534]{almagest}: if $DA=60$ and $DH=2+\frac{52}{60}$, then for $AD^2-DH^2=AH^2$,
\[
AH \sim 59+\frac{56}{60}.
\]
XI.5 \cite[pp.~535--536]{almagest}: if $DB=60$ and $DH=1+\frac{5}{60}$, then for $DB^2-DH^2=BH^2$,
\[
BH \sim 59+\frac{59}{60}.
\]
XI.5 \cite[p.~537]{almagest}: if $DG=60$ and $DH=2+\frac{51}{60}$, then for $DG^2-DH^2=GH^2$,
\[
GH \sim 59+\frac{56}{60}.
\]
XI.6 \cite[p.~540]{almagest}: if $DB=60$ and $DM=3+\frac{25}{60}$, then for $DB^2-DM^2=BM^2$,
\[
BM \sim 59+\frac{54}{60}.
\]
XII.2 \cite[p.~564]{almagest}: 
\[
\sqrt{4+\frac{6}{60}+\frac{45}{60^2}} \sim 2+\frac{1}{60}+\frac{40}{60^2}.
\]
XII.2 \cite[p.~566]{almagest}:
\[
\sqrt{4+\frac{35}{60}+\frac{56}{60^2}} \sim 2+\frac{8}{60}+\frac{40}{60^2}.
\]
XII.2 \cite[p.~568]{almagest}:
\[
\sqrt{3+\frac{39}{60}+\frac{12}{60^2}} \sim 1+\frac{54}{60}+\frac{41}{60^2}.
\]
XII.3 \cite[p.~570]{almagest}:
\[
\sqrt{24+\frac{50}{60}+\frac{9}{60^2}} \sim 4+\frac{59}{60}+\frac{1}{60^2}.
\]
XII.3 \cite[p.~571]{almagest}:
\[
\sqrt{27+\frac{13}{60}+\frac{26}{60^2}} \sim 5+\frac{13}{60}+\frac{4}{60^2}.
\]
XII.3 \cite[p.~571]{almagest}:
\[
\sqrt{22+\frac{33}{60}+\frac{39}{60^2}} \sim 4+\frac{45}{60}.
\]
XII.4 \cite[p.~572]{almagest}:
\[
\sqrt{803+\frac{50}{60}+\frac{50}{60^2}} \sim 28+\frac{21}{60}+\frac{8}{60^2}.
\]
XII.4 \cite[p.~574]{almagest}:
\[
\sqrt{964+\frac{48}{60}+\frac{47}{60^2}} \sim 31+\frac{3}{60}+\frac{41}{60^2}.
\]
XII.4 \cite[pp.~574--575]{almagest}:
\[
\sqrt{672+\frac{13}{60}} \sim 25+\frac{25}{60}+\frac{38}{60^2}.
\]
XII.5 \cite[pp.~575--576]{almagest}:
\[
\sqrt{1057+\frac{51}{60}} \sim 32+\frac{31}{60}+\frac{29}{60^2}.
\]
XII.5 \cite[p.~577]{almagest}:
\[
\sqrt{1093+\frac{16}{60}+\frac{23}{60^2}} \sim 33+\frac{3}{60}+\frac{53}{60^2}.
\]
XII.5 \cite[p.~578]{almagest}:
\[
\sqrt{1022+\frac{54}{60}+\frac{7}{60^2}} \sim 31+\frac{58}{60}+\frac{58}{60^2}.
\]
XII.6 \cite[p.~579]{almagest}:
\[
\sqrt{190+\frac{29}{60}+\frac{31}{60^2}} \sim 13+\frac{48}{60}+\frac{7}{60^2}.
\]
XII.6 \cite[p.~580]{almagest}:
\[
\sqrt{257+\frac{22}{60}+\frac{44}{60^2}} \sim 16+\frac{2}{60}+\frac{35}{60^2}.
\]
XII.6 \cite[p.~581]{almagest}:
\[
\sqrt{160+\frac{21}{60}+\frac{29}{60^2}} \sim 12+\frac{39}{60}+\frac{48}{60^2}.
\]
XII.9 \cite[p.~591]{almagest}: if $BM=1+\frac{13}{60}$ and $MZ=60+\frac{16}{60}$, then for $BZ^2=BM^2+MZ^2$,
\[
BZ \sim 60+\frac{17}{60}.
\]
XIII.4 \cite[p.~608]{almagest}: if $AL=29+\frac{30}{60}$ and $LM=30+\frac{32}{60}$, then for $AM^2=AL^2+LM^2$,
\[
AM \sim 42+\frac{27}{60};
\]
if $AB=60$ and $AK=29+\frac{28}{60}$, then for $AK^2+K\Theta^2=A\Theta^2$,
\[
A\Theta \sim 42+\frac{26}{60}.
\]
XIII.4 \cite[p.~610]{almagest}: if $AL=40+\frac{51}{60}$ and $LM=15+\frac{55}{60}$, then for $AL^2+LM^2=AM^2$,
\[
AM \sim 43+\frac{50}{60};
\]
if $AM=43+\frac{50}{60}$ and $\Theta M = 1+\frac{44}{60}$, then for $AM^2+\Theta M^2=A \Theta^2$,
\[
A\Theta \sim 43+\frac{52}{60}.
\]
XIII.4 \cite[p.~611]{almagest}: if $\Theta K = 15+\frac{55}{60}$ and $AK=40+\frac{45}{60}$, then for
$AK^2+K\Theta^2 = A\Theta^2$,
\[
A\Theta \sim 43+\frac{45}{60}.
\]
XIII.4 \cite[p.~613]{almagest}: if $AM=57+\frac{35}{60}$ and $MK=\frac{22}{60}$, then for $AK^2=AM^2+MK^2$,
\[
AK \sim 57+\frac{35}{60}.
\]
XIII.4 \cite[p.~614]{almagest}: if $AB=57+\frac{31}{60}$ and $BL=4+\frac{36}{60}$, then for $AB^2+BL^2=AL^2$,
\[
AL \sim 57+\frac{42}{60},
\]
and if $L\Theta = 2+\frac{53}{60}$, then for $AL^2+L\Theta^2=A\Theta^2$,
\[
A\Theta \sim 57+\frac{46}{60};
\]
if $AB=119+\frac{51}{60}$ and $BL=4+\frac{36}{60}$, then for $AB^2+BL^2=AL^2$,
\[
AL \sim 53+\frac{13}{60}.
\]
XIII.4 \cite[p.~615]{almagest}: if $AL=53+\frac{13}{60}$ and $\Theta L = 2+\frac{41}{60}$, then for $AL^2+\Theta L^2=
A\Theta^2$,
\[
A\Theta \sim 53+\frac{17}{60};
\]
if $K\Theta = 4+\frac{36}{60}$ and $AK=53+\frac{4}{60}$, then for $AK^2+K\Theta^2=A\Theta^2$,
\[
A\Theta \sim 53+\frac{16}{60}.
\]
XIII.4 \cite[p.~617]{almagest}: if $AB=54+\frac{20}{60}$ and $BL=8+\frac{8}{60}$, then for $AB^2+BL^2=AL^2$,
\[
AL \sim 54+\frac{56}{60},
\]
and if $L\Theta = 1+\frac{46}{60}$, then for $AL^2+L\Theta^2=A\Theta^2$,
\[
A\Theta \sim 54+\frac{58}{60}.
\]
XIII.4 \cite[p.~618]{almagest}: if $AB=49+\frac{20}{60}$ and $BL=8+\frac{8}{60}$, then for
$AB^2+BL^2=AL^2$,
\[
AL \sim 50,
\]
and if $\Theta L = 1+\frac{39}{60}$, then for $AL^2+\Theta L^2=A\Theta^2$,
\[
A\Theta = 50+\frac{2}{60};
\]
if $\Theta K = 8+\frac{8}{60}$ and $AK=49+\frac{22}{60}$, then for $AK^2+K\Theta^2=A\Theta^2$,
\[
A\Theta \sim 50+\frac{2}{60}.
\]
XIII.4 \cite[p.~619]{almagest}: if $KM=1+\frac{6}{60}$ and $AM=38+\frac{6}{60}$, then for $AK^2=AM^2+KM^2$,
\[
AK \sim 38+\frac{7}{60}.
\]
XIII.4 \cite[p.~620]{almagest}: if $AB=38+\frac{5}{60}$ and $BL=27+\frac{56}{60}$, then for $AB^2+BL^2=AL^2$,
\[
AL \sim 47+\frac{14}{60},
\]
and if $\Theta L = 1+\frac{46}{60}$, then for $AL^2+L\Theta^2=A\Theta^2$,
\[
A\Theta \sim 47+\frac{16}{60}.
\]
XIII.4 \cite[p.~621]{almagest}: if $KM=1+\frac{6}{60}$ and $AM=26+\frac{6}{60}$, then for $AK^2=KM^2+AM^2$,
\[
AK \sim 26+\frac{7}{60};
\]
if $AB=26+\frac{4}{60}$ and $BL=27+\frac{56}{60}$, then for $AB^2+BL^2=AL^2$,
\[
AL \sim 38+\frac{12}{60},
\]
and if $L\Theta = 1+\frac{33}{60}$, then for $AL^2+L\Theta^2=A\Theta^2$,
\[
A\Theta \sim 38+\frac{14}{60};
\]
if $K\Theta = 27+\frac{56}{60}$ and $AK=26+\frac{4}{60}$, then for $A\Theta^2=AK^2+K\Theta^2$,
\[
A\Theta \sim 38+\frac{12}{60}.
\]
XIII.4 \cite[p.~626]{almagest}: if $AB=60$ and $BD=43+\frac{10}{60}$, then for $AB^2-BD^2=AD^2$,
\[
AD \sim 41+\frac{40}{60};
\]
if $AD=41+\frac{40}{60}$, $DH=1+\frac{50}{60}$, $DZ=29+\frac{58}{60}$,
then for $AD^2-DH^2=AH^2$ and $ZD^2-DH^2=HZ^2$,
\[
AH \sim 41+\frac{37}{60},\qquad HZ \sim 29+\frac{55}{60};
\]
if $AB=63$ and $BD=22+\frac{30}{60}$, then for $AB^2-DB^2=AD^2$,
\[
AD \sim 58+\frac{51}{60}.
\]
XIII.4 \cite[p.~627]{almagest}: if $DH=2+\frac{34}{60}$, $AD=58+\frac{51}{60}$, and
$DZ=21+\frac{1}{60}$, then for $DA^2-DH^2=AH^2$ and $DZ^2-DH^2=HZ^2$,
\[
AH \sim 58+\frac{47}{60},\qquad ZH \sim 20+\frac{53}{60}.
\]
XIII.4 \cite[p.~628]{almagest}: if $AB=61+\frac{15}{60}$ and $BD=43+\frac{10}{60}$, then for $AB^2-BD^2=AD^2$,
\[
AD \sim 43+\frac{27}{60}.
\]
XIII.4 \cite[p.~629]{almagest}: if $BD=43+\frac{10}{60}$ and $AB=58+\frac{45}{60}$, then for $AB^2-DB^2=AD^2$,
\[
AD \sim 39+\frac{51}{60};
\]
if $AB=69$ and $BD=22+\frac{30}{60}$, then for $AD^2=AB^2-BD^2$,
\[
AD \sim 65+\frac{14}{60}.
\]









\section{Heron of Alexandria}
Bruins, {\em Codex Constantinopolitanus}, fol. 6r--6v \cite[p.~6]{constantinopolitanus}: 
for a square each side of which is 50 feet, find the area and the diagonal. The area is 2500 square feet.
To find the diagonal, double the area, getting 5000 square feet. 
The square root of this is said to be $70 \frac{1}{2} \frac{1}{4}$ feet, which is said to be the diagonal.
 
Bruins, {\em Codex Constantinopolitanus}, fol. 6v \cite[p.~6]{constantinopolitanus}: for a rectangle whose length is 50 feet and whose
length is 30 feet, find the area and the diagonal.
The area is 1500 square feet. To find the diagonal, add the length squared, 2500 square feet,
and the width squared, 900 square feet, getting $3400$ square feet. he square root of this is said to be
$58 \frac{1}{3}$ feet, which is said to be the diagonal.

Bruins, {\em Codex Constantinopolitanus}, fol. 7v \cite[p.~9]{constantinopolitanus}:
for an equilateral triangle each side of which is 30 feet, 
find the diameter of an inscribed circle.
It is stated that the area of the  triangle is 390 square feet.
Generally, the area of a triangle with side length $a$ is $A=\frac{\sqrt{3} a^2}{4}$; for
$a=30$, this satisfies $389<A<390$.
Multiply the area by $4$, getting $1560$ square feet. The perimeter of the triangle is 
$90$ feet. 
Divide the area $1560$ square feet by $90$ feet, getting $17 \frac{1}{3}$ feet. 
It is used that for an equilateral triangle with side length $a$, 
the diameter of the inscribed circle is
$\frac{4A}{3a} = \frac{a}{\sqrt{3}}$.

Bruins, {\em Codex Constantinopolitanus}, fol. 8r \cite[p.~10]{constantinopolitanus}:
 an equilateral triangle whose sides are 30 is  two right triangles each with hypotenuse 30 and base 15.
Square 30, getting 900, and square 15, getting 225, and subtract 225 from 900, getting 675.
The square root of this is the height of the right triangle, and it is stated that $\sqrt{675}$ is ``very near'' to
$25 \frac{51}{52}$. Using 26 as the altitude of the equilateral triangle, the area of the equilateral triangle
390, and four times this is 1560. Divide this by the perimeter of the equilateral triangle, getting $\frac{1560}{90}=17 \frac{1}{3}$,
which is said to be the diameter of the inscribed circle.
This is also worked out using Heron's formula.
For a triangle with sides $a,b,c$ and with $s=\frac{a+b+c}{2}$, Heron's formula states that the area of the triangle is 
$A=\sqrt{s(s-a)(s-b)(s-c)}$.
The radius of the inscribed circle is $r=\frac{A}{s}$, so $r^2 = \frac{s(s-a)(s-b)(s-c)}{s^2}$. 
For $a=b=c$,  this is equivalent to $s^2:s(s-a)  = (s-a)^2:r^2$. For $a=30$ we have 
$s=45$, $s^2=2025$, $s-a=15$,  $(s-a)^2=225$, $s(s-a) = 45 \cdot 15 = 675$, thus
$2025:675 = 225:r^2$. Now, $2025:675 = 225:75$, so $r^2=75$. It is then stated that the square root of $75$ is
$8 \frac{2}{3}$, and that twice this is the diameter, $17 \frac{1}{3}$. 

Bruins, {\em Codex Constantinopolitanus}, fol. 8r \cite[p.~12]{constantinopolitanus}:
Euclid XIII.12 is invoked, which says that the square on the side of an equilateral triangle is three times the square on the radius of the circumscribed circle.
Thus for a triangle with side length 30, it is stated that the radius of the circumscribed circle is $\sqrt{300}$. 
Twice this is the diameter, and it is stated that the diameter is $34 \frac{1}{2} \frac{1}{6}$. 
Thus, $2 \sqrt{300} \sim 34 \frac{2}{3}$.  

Bruins, {\em Codex Constantinopolitanus}, fol. 17r \cite[p.~42]{constantinopolitanus}:
for a pyramid whose base is an equilateral triangle with side length $30$ feet and 
with inclined side $20$ feet, find the altitude and the volume. 
Generally, 
let $ABC$ be an equilateral triangle with center $D$ and let
$DE$ be perpendicular to the plane $ABC$. Then $ABCE$ is a pyramid, with altitude 
$DE$. It holds that $AE^2 = DE^2 + \frac{AB^2}{3}$, and the volume of
the pyramid is a third of the area of the base triangular multiplied by the
altitude of the pyramid ({\em Elements} XII.7).
Here, $AB=30$ and $AE=20$, so $\frac{AB^2}{3}=\frac{900}{30}=300$ and
$AE^2 = 400$. Then $DE^2 = AE^2 - \frac{AB^2}{3} = 400-300=100$, and therefore
$DE=10$. The altitude of the pyramid is thus found to be 10 feet. 
The area of $ABC$ is found thus: square 30, getting 900. A third and a tenth of this is
$300+90=390$. This is the area of the base triangle. (This amounts to
$\frac{\sqrt{3}}{4} \sim \frac{1}{3}+\frac{1}{10}$.)
A third of this is 130, and the product of this and the altitude is $130 \cdot 10=1300$. 

Bruins, {\em Codex Constantinopolitanus}, fol. 20r \cite[p.~58]{constantinopolitanus}:
for a regular decagon with sides $10$ feet, find the area. 
Generally, the area of a regular decagon of side length $a$ is 
$\frac{5 \sqrt{5+2\sqrt{5}}}{2}a^2$. 
Square 10, getting 100,  multiply this by 38, getting 3800, and take a fifth, getting 
760, which is said to be the area.

Bruins, {\em Codex Constantinopolitanus}, fol. 57r---v \cite[p.~160]{constantinopolitanus}:
for an isosceles triangle with sides $8$, 12, 12, the height $h$ satisfies $h^2=12^2-4^2=128$.
It is stated that $\sqrt{128} \sim 11 \frac{1}{4} \frac{1}{22} \frac{1}{44} = \frac{249}{22}$. In fact,
$\frac{249}{22} - \frac{1}{223} < \sqrt{128} <  \frac{249}{22} - \frac{1}{224}$. 

Bruins, {\em Codex Constantinopolitanus}, fol. 57r---v \cite[p.~165]{constantinopolitanus}:
for a hexagonal pyramid whose base edges are 12 feet and whose inclined edges are 35 feet,
find the height and the volume. The height $h$ satisfies $35^2 = h^2+12^2$, so
$h^2=1081$. It is stated that $\sqrt{1081} \sim 32 \frac{1}{2} \frac{1}{4} \frac{1}{8} \frac{1}{64}$, which is $32 \frac{57}{64}$. 
In fact,
$\frac{57}{64}-\frac{1}{82}<\sqrt{1081}<\frac{57}{64}-\frac{1}{83}$. 

Bruins, {\em Codex Constantinopolitanus}, fol. 65r \cite[p.~178]{constantinopolitanus}:
find the diagonal of a square piece of wood having each side 10 feet. $10^2+10^2=200$, and it is stated that the approximate
square root of 200 is $14 \frac{1}{7}$ and that this is the diagonal.




Euclid I.22: if $A,B,C$ are three straight lines and any two taken together are greater
than the remaining, to construct a triangle with sides equal to $A,B,C$. 
Heron's formula says that if $a,b,c$ are the sides of a triangle and 
$s=\frac{a+b+c}{2}$ then the area of the triangle is 
$\sqrt{s(s-a)(s-b)(s-c)}$. Heron's formula is proved in 
Heron's {\em Metrica} I.8 \cite[pp.~18--25]{heronisIII};
Thomas \cite[pp.~470--477]{thomasII} translates this passages, and
Heath \cite[pp.~321--323]{HGMII} presents the proof. Heron gives the example
$7,8,9$. Here, $s=\frac{7+8+9}{2}=\frac{24}{2}=12$. Then
\[
s(s-7)(s-8)(s-9) = 12 \cdot 5 \cdot 4 \cdot 3 = 60 \cdot 4 \cdot 3 = 240 \cdot 3 = 720.
\]
Thus the area of the triangle is $\sqrt{720}$. 

Heron works out an approximate value for $\sqrt{720}$ as follows \cite[pp.~323--326]{HGMII}; cf. \cite{metrica}. The square integer closest to $720$ is $729=27 \cdot 27$. 
Then $\frac{720}{27}=26+\frac{2}{3}$. Then
$27+26+\frac{2}{3}=53+\frac{2}{3}$ and half of this is $26+\frac{1}{2}+\frac{1}{3}$, and this approximates
the square root of $720$. Then
\[
\left( 26+\frac{1}{2}+\frac{1}{3} \right)^2 = 720+\frac{1}{36}
\]
 is a square, and the difference between this square and $720$
is $\frac{1}{36}$, which is smaller than the difference between $729$ and $720$, which is $9$. 
Heron states that this process can be done again to get a difference smaller than $\frac{1}{36}$, but does not do any more iterations.
The next iteration is: $\frac{720}{26+\frac{1}{2}+\frac{1}{3}} =26+\frac{134}{161}$, and then
$26+\frac{1}{2}+\frac{1}{3}+26+\frac{134}{161}=53+\frac{643}{966}$, and half of that is
$26+\frac{1609}{1932}$. Then
\[
\left(26+\frac{1609}{1932}\right)^2 =\frac{2687489281}{3732624}
\]
is a square, and the difference between this square and $720$ is $\frac{1}{3732624}$, which is smaller than the difference
between $720+\frac{1}{36}$ and $720$, which is $\frac{1}{36}$. Cf. 
Bruins, {\em Codex Constantinopolitanus}, fol. 70v \cite[p.~189]{constantinopolitanus}.


{\em Metrica} I.9 \cite[pp.~27--29]{heronisIII}:
\[
\sqrt{63} \sim 7+\frac{1}{2}+\frac{1}{4}+\frac{1}{8}+\frac{1}{16}.
\]


{\em Metrica} I.15 \cite[pp.~41--43]{heronisIII}: for $AB=13$, $B\Gamma=10$,
$\Gamma \Delta=20$, $\Delta A=17$, let $AB\Gamma \Delta$ be a quadrilateral where
$B\Gamma \Delta$ is a right angle. Then $B\Delta^2 = 500$. Let $AE$ be perpendicular to $B\Delta$. For
$s=\frac{AB+B\Delta+\Delta A}{2}$ we have
$AE = \frac{2\sqrt{s(s-AB)(s-B\Delta)(s-\Delta A)}}{B\Delta}$, which means
\begin{align*}
AE^2 &= \frac{4}{500} \cdot \frac{30+B\Delta}{2} \cdot
\frac{4+B\Delta}{2} \cdot \frac{30-B\Delta}{2} \cdot \frac{B\Delta-4}{2}\\
&=\frac{1}{125} \cdot \frac{900-500}{4} \cdot \frac{500-16}{4}\\
&=\frac{484}{5}.
\end{align*}
{\em Metrica} I.17 \cite[p.~49]{heronisIII}: 
\[
\sqrt{1875} \sim 43+\frac{1}{3}.
\]
{\em Metrica} I.18 \cite[pp.~51--53]{heronisIII}: the regular pentagon, with each side $10$.
$5 \sim \frac{81}{16}$, and this yields the approximate value
$166+\frac{2}{3}$ for the area. (See Heath \cite[pp.~326--329]{HGMII} about the regular polygons in the {\em Metrica}.)

{\em Metrica} I.19 \cite[pp.~52--55]{heronisIII}: the regular hexagon, with each side $10$. 

{\em Metrica} I.20 \cite[pp.~54--57]{heronisIII}: the regular heptagon, with each side $10$.

{\em Metrica} I.21 \cite[pp.~57--59]{heronisIII}: the regular octagon, with each side $10$.

{\em Metrica} I.22 \cite[pp.~59--61]{heronisIII}: the regular nonagon, with each side $10$.

{\em Metrica} I.23 \cite[p.~61]{heronisIII}: the regular decagon, with each side $10$.

{\em Metrica} I.24 \cite[p.~63]{heronisIII}: the regular $11$-gon, with each side $10$.

{\em Metrica} I.25 \cite[pp.~63--65]{heronisIII}: the regular $12$-gon, with each side $10$.


{\em Stereometrica} \cite{heronisV}.
I.33, $\sqrt{63}$ \cite[pp.~34--35]{heronisV}: for
$\alpha_1=8$ and $\beta_1=\frac{63}{8}=7+\frac{7}{8}$, we get
$\alpha_2 = \frac{\alpha_1+\beta_1}{2} = 7+\frac{15}{16}=8-\frac{1}{16}$. 
I.63 \cite[p.~65]{heronisV}:
\[
\sqrt{356} \sim 18 + \frac{1}{2} + \frac{1}{4}+\frac{1}{8}.
\]
II.1 \cite[p.~85]{heronisV}: $\sqrt{288} \sim 17$.
II.2 \cite[p.~87]{heronisV}: $\sqrt{144+\frac{1}{2}} \sim 12$. 
II.57 \cite[p.~139]{heronisV}: $\sqrt{288} \sim 17$,
$\sqrt{1224} \sim 35$. 
II.59 \cite[p.~143]{heronisV}: $\sqrt{512} \sim 22+\frac{2}{3}$,
$\sqrt{72} \sim 8 + \frac{1}{2}$,
$\sqrt{1400} \sim 37+\frac{1}{4}+\frac{1}{6}$.
II.60 \cite[p.~147]{heronisV}: $\sqrt{128} \sim 11+\frac{1}{4}+\frac{1}{22}+\frac{1}{44}$,
$\sqrt{593} \sim 24+\frac{1}{4}+\frac{1}{8}$. 
II.63 \cite[pp.~151--153]{heronisV}: $\sqrt{1125} \sim 33+\frac{1}{2}+\frac{1}{22}$,
$\sqrt{108} \sim 10+\frac{1}{3}+\frac{1}{15}$. 
II.64 \cite[p.~153]{heronisV}: $\sqrt{1081} \sim 32+\frac{1}{2}+\frac{1}{4}+\frac{1}{8}+\frac{1}{64}$. 
II.65 \cite[p.~155]{heronisV}: $\sqrt{50} \sim 7+\frac{1}{14}$,
$\sqrt{54} \sim 7 + \frac{1}{3}$. 
II.66 \cite[p.~157]{heronisV}: $\sqrt{75} \sim 8+\frac{1}{2}+\frac{1}{8}+\frac{1}{16}$,
$\sqrt{43+\frac{1}{2}+\frac{1}{4}+\frac{1}{9}} \sim 6+\frac{1}{2}+\frac{1}{9}$,
$\sqrt{356+\frac{1}{18}} \sim 18+\frac{1}{2}+\frac{1}{4}+\frac{1}{9}$.





{\em Geometrica} 5.3 \cite[p.~203]{heronisIV}: 
\[
\sqrt{5000} \sim 70 + \frac{1}{2}+\frac{1}{4}.
\]
6.1 \cite[p.~209]{heronisIV}:
\[
\sqrt{3400} \sim 58 + \frac{1}{3}.
\]
10 \cite[p.~223]{heronisIV}: for an equilateral triangle with sides $a$, the area
is $\frac{a^2 \sqrt{3}}{4}$, and the area is stated to be
approximately $\left(\frac{1}{3} + \frac{1}{10} \right)a^2$. This amounts to
\[
\frac{\sqrt{3}}{4} \sim \frac{1}{3}+\frac{1}{10},\qquad
\sqrt{3} \sim 1+\frac{1}{3}+\frac{2}{5} = 1+\frac{44}{60}.
\]
15 \cite[pp.~286--301]{heronisIV} states approximate values for several surds.
15.3:\[
\sqrt{8+\frac{1}{4}+\frac{1}{8}+\frac{1}{16}} \sim
2+\frac{2}{3}+\frac{1}{4}.
\]
For,
$8+\frac{1}{4}+\frac{1}{8}+\frac{1}{16}=\frac{135}{16}=\frac{1215}{144}$, and a square near to this is
$\frac{1225}{144}=\left(2+\frac{11}{12}\right)^2=\left(2+\frac{2}{3}+\frac{1}{4}\right)^2$. 
15.4:
\[
\sqrt{135} \sim 11+\frac{1}{2}+\frac{1}{14}+\frac{1}{21}=11+\frac{13}{21}.
\]
15.6: 
\[
\sqrt{43+\frac{1}{2}+\frac{1}{4}} \sim 6+\frac{1}{2}+\frac{1}{13}+\frac{1}{26}=6+\frac{8}{13}.
\]
15.10:
\[
\sqrt{6300} \sim 79+\frac{1}{3}+\frac{1}{34}+\frac{1}{102} = 79+\frac{19}{51}.
\]
15.11:
\[
\sqrt{1575} \sim 39 + \frac{2}{3}+\frac{1}{51}= 39+\frac{35}{51}.
\]
15.12:
\[
\sqrt{886 \div \frac{1}{16}} \sim 29+\frac{1}{2}+\frac{1}{4}+\frac{1}{68} = 29 + \frac{39}{51}.
\]
15.13:
\[
\sqrt{2460+\frac{15}{16}} \sim 49 + \frac{1}{2}+\frac{1}{17}+\frac{1}{34}+\frac{1}{51} = 49 + \frac{31}{51}.
\]
15.14:
\[
\sqrt{615+\frac{15}{64}} \sim 24+ \frac{1}{2}+\frac{1}{4}+\frac{1}{51}+\frac{1}{51}+\frac{1}{68} = 24+\frac{41}{51}.
\]
16.34--35 \cite[p.~323]{heronisIV}:
\[
\sqrt{216} \sim 14+\frac{2}{3}+\frac{1}{33} = 14+\frac{23}{33}.
\]
16.36--37 \cite[p.~325]{heronisIV}:
\[
\sqrt{720} \sim 26+\frac{1}{2}+\frac{1}{3} = 26+\frac{5}{6}.
\]
16.38 \cite[p.~327]{heronisIV}:
\[
\sqrt{58+\frac{1}{4}+\frac{1}{8}+\frac{1}{16}} \sim 7 +\frac{2}{3}.
\]
19.4 \cite[p.~359]{heronisIV}:
\[
\sqrt{208} \sim 14 + \frac{1}{3}+\frac{1}{12}.
\]
20.13 \cite[p.~373]{heronisIV}:
\[
\sqrt{444+\frac{1}{3}+\frac{1}{9}} \sim 21 + \frac{1}{12}.
\]

21.14 \cite[p.~383]{heronisIV}: for a regular pentagon with side $a$, the area is
$\frac{a^2}{4}\sqrt{5(5+2\sqrt{5})}$. For $a=35$, 
this passage states the area as $35 \cdot 35 \cdot 12 \cdot \frac{1}{7} = 2100$.

The area of a regular hexagon with side $a$ is $\frac{3\sqrt{3}}{2}a^2$. For $a=30$, 
21.16 states the area as $30 \cdot 30 \cdot 13 \cdot \frac{1}{5} = 2340$. 

(The area of a regular heptagon involves $\cos(\pi/7)$.)

The area of a regular octagon with side $a$ is $2(1+\sqrt{2})a^2$.
For $a=10$, 21.19 \cite[p.~385]{heronisIV} states the area as
$10 \cdot 10 \cdot 29 \cdot \frac{1}{6} = 483+\frac{1}{3}$. 

(The area of a regular nonagon involves $\cos(\pi/9)$.)

The area of a regular decagon with side $a$ is $\frac{5\sqrt{5+2\sqrt{5}}}{2} a^2$. 
For $a=10$, 21.21 states the area as $10\cdot 10 \cdot 15 \cdot \frac{1}{2}=750$. 

(The area of a regular 11-gon involves $\cos(\pi/11)$.)

The area of a regular 12-gon with side $a$ is $3(2+\sqrt{3})a^2$. For $a=10$, 21.23 \cite[p.~387]{heronisIV}
states the area as $10 \cdot 10 \cdot 45 \cdot \frac{1}{4}=1125$. 







\section{Diodorus Siculus}
Diodorus Siculus, {\em Bibliotheca historica} 1.63.3--4 writes the following about the Great Pyramids:

\begin{quote}
These pyramids, which are situated on the side of Egypt which is towards Libya, are one hundred and twenty stades from Memphis and forty-five from the Nile, and by the
immensity of their structures and the skill shown in their execution they fill the beholder with wonder and astonishment. For the largest is in the form of a square and has a base length on each side of seven plethra and a height of over six plethra; it also gradually tapers to the top, where each side is six cubits long.
\end{quote} 

(A {\em plethron} is 100 feet.)
For a pyramid whose base is a square whose sides have length $a$ and whose four other faces are equilateral triangles, let $c$ be the distance from the apex of the pyramid to the
midpoint of one of the sides of the square base. 
Then $c^2+(a/2)^2=a^2$, so $c^2=\frac{3}{4} a^2$. For Diodorus Siculus, $a$ is seven plethra, so $c= \frac{\sqrt{3}}{2} a$. Now, 
$6<\frac{\sqrt{3}}{2} \cdot 7<\frac{97}{16}$.

Plutarch, {\em The Dinner of the Seven Wise Men} 147 ({\em Moralia} II):

\begin{quote}
``Not for this alone,'' said Neiloxenus, ``but he does not try to avoid, as the rest of you do, being a friend of kings and being called such.
In your case, for instance, the king finds much to admire in you, and in particular he was immensely pleased with your method of measuring the pyramid, because, without
making any ado or asking for any instrument, you simply set your walking-stick upright at the edge of the shadow which the pyramid cast, and, two triangles being formed by the 
intercepting of the sun's rays, you demonstrated that the height of the pyramid bore the same relation to the length of the stick as the one shadow to the other....''
\end{quote}

Pliny, {\em Natural History} 36.17:

\begin{quote}
The largest Pyramid occupies seven jugera of ground, and the four angles are equidistant, the face of each side being eight hundred and thirty-three feet in length. The total height from the ground to the summit is seven hundred and twenty-five feet, and the platform on the summit is sixteen feet and a-half in circuit.
\end{quote}

Diogenes Laertius, {\em Lives of Eminent Philosophers} 1.27 writes about Thales: ``Hieronymus informs us that he measured the height of the pyramids by the shadow they cast, taking the observation at the hour when our
shadow is of the same length as ourselves.''







\section{Greek and Roman art}
Villa of Maxentius, Mausoleum of Romulus: hexagon, and area of the interior is half the total.

Athens, Tower of the Winds

Temple of Hadrian,  Maritime Theatre, $\sqrt{3}$. Jacobson \cite{jacobson}

Pompeii \cite{pompei}: I.280,322,326,407,438,458; IV.48, 136, 660, 724, 748; V.505, 843.

Pompeii VII.7.5, House of Triptolemus 

Pompeii VI.9.2, House of Meleager

Herculaneum, Casa dell'Atrio a Mosaico, Casa del Rilievo di Telefo

House in Pella in Macedonia, pebble mosaic

Ostia, House of Cupid and Psyche, Room E \cite{ostiaIV}

Mediana, Moesia Superior: Nymphaeum

Gamzigrad, Serbia: Felix Romuliana temple, hexagonal labyrinth mosaic 

Ammaedara, Ha{\"\i}dra, Tunisia: hexagonal mausoleum

Salzmann \cite{salzmann}

Dunbabin \cite{dunbabin}





\section{Varro}
In {\em De Re Rustica} III.XVI.4--5 \cite[p.~501]{varro}, after stating that nature has given great talent and art to bees, Varro writes:
 
\begin{quote}
Bees are not of a solitary nature, as eagles are, but are like human beings. Even if jackdaws in this respect are the same, still it is not the same case; for in one there is a fellowship
in toil and in building which does not obtain in the other; in the one case there is reason and skill -- it is from these that men learn to toil, to build, to store up food. They have three 
tasks: food, dwelling, toil; and the food is not the same as the wax, nor the honey, nor the dwelling. Does not the chamber in the comb have six angles, the same number as the bee 
has feet? The geometricians prove that this hexagon inscribed in a circular figure encloses the greatest amount of space.
\end{quote}

Aristotle, {\em De caelo} III.8, 306b5--7 \cite[p.~177]{aristotle}:

\begin{quote}
And, speaking generally, the attempt to give figures to the simple
elements is irrational, first, because it will be found that they do
not fill the whole (of a space). For, among plane figures, it is agreed
that there are only three which fill up space, the triangle, the square,
and the hexagon; while among solids there are only the pyramid
and the cube.
\end{quote}


Pappus, {\em Collection} V.1--3 \cite[pp.~589--593]{thomasII} writes about why cells of honeycombs are hexagonal; cf.
{\em Collection} VIII, Proposition 19.





\section{Vitruvius}
Vitruvius, {\em De architectura}, IV.1.11 \cite[p.~93]{vitruvius}, on the Corinthian order:

\begin{quote}
The modular system of this capital should be established in such a way that the lower diameter of the column should equal the height of the capital
with the abacus. The breadth of the abacus is to be proportioned so that the length of diagonals taken from corner to corner should be twice the height of the
capital. In that way the faces of the abacus will have fronts of the right breadth on all sides. The faces should curve inwards from the points of the angles of the abacus
by a ninth of the breadth of the face. At the bottom, the capital should be as wide as the top of the column, disregarding the {\em apothesis} and the astragal.
The height of the abacus should be a seventh that of the capital.
\end{quote}

VI.3.3 \cite[p.~172]{vitruvius}:

\begin{quote}
The lengths and breadths of {\em atria} fall into three categories: the first category is laid out so that when the length
has been divided into five units, three should be allocated to the breadth; in the second, the length should be divided
into three units and two assigned to the width; and in the third, the breadth of the {\em atrium} should be incorporated
in a square in which a diagonal should be drawn; the length of the diagonal should be allocated to the
{\em atrium}.
\end{quote}

The first category is the Tuscan {\em atrium}; the second category is the Corinthian {\em atrium}; 
the third category is the tetrastyle {\em atrium}; the other two types of courtyard are displuviate and {\em testudinatum}.

IX, Introduction 4--5 \cite[p.~243]{vitruvius}:

\begin{quote}
[4] First of all, I will explain one of Plato's many exceptionally useful theorems as he formulated it. If there is a site
or square field, that is, one with equal sides, which we have to double, the solution can be found by drawing lines accurately,
since we will need a type of number that cannot be arrived at by multiplication. The proof of this is as follows: a square site
ten feet long and ten feet wide produces an area of a hundred square feet. If, then, we need to double it and produce a square
of two hundred feet, we must find out how long the side of the square would be to obtain from it the two hundred feet corresponding
to the doubling of the area. Nobody can discover this by calculation: for if we take the number fourteen, multiplication will give a 
hundred and ninety-six square feet; if we take fifteen, it will give two hundred and twenty-five square feet.

[5] Therefore, since we cannot solve this problem arithmetically, a diagonal line should be drawn in the ten foot square
from angle to angle so that it is divided into two triangles of equal size, each fifty feet in area; a square with equal sides
should be drawn along the length of this diagonal. In this way four triangles will be produced in the larger square of the same
size and number of feet as the two triangles of fifty square feet created by the diagonal in the smaller square. The problem of
doubling an area was solved by Plato with this procedure using geometrical methods, as is shown in the diagram at the foot
of the page.
\end{quote}

Then in IX, Introduction 6 \cite[pp.~243--244]{vitruvius}:

\begin{quote}
Again, Pythagoras demonstrated how to devise a set-square without the intervention of workmen; the results which workmen
arrive at when they make set-squares, with considerable effort but without great accuracy, can be arrived at with precision
using the principles and methods derived from his teachings. For if we take three rulers, three, four and five feet long,
and assemble them with their ends touching in the form of a triangle, they will form a perfect set-square. If squares with equal
sides are drawn along the lengths of each ruler, the three-foot side will produce an area of nine square feet, the four-foot
side an area of sixteen square feet and the five-foot side an area of twenty-five square feet.
\end{quote}

 





\section{Columella}
In {\em De Re Rustica} V.I.4--8 \cite[pp.~5--7]{columellaII}, Columella defines  measures of area:

\begin{quote}
But to return to my subject, the extent of every area is reckoned by measurement in {\em feet}, and a foot consists of 16 {\em fingers}. The multiplication of the foot produces 
successively the {\em pace}, the {\em actus}, the {\em clima}, the {\em iugerum}, the {\em stadium} and the {\em centuria}, and afterward still larger measurements. The {\em pace} 
contains five feet. The {\em smallest actus} (as Marcus Varro says) is four feet wide and 120 feet long. The {\em clima} is 60 feet each way. The {\em square actus} is bounded by
120 feet each way; when doubled it forms a {\em iugerum}, and it has derived the name of {\em iugerum} from the fact that it was formed by joining. This {\em actus} the country folk
of the province of Baetica call {\em acnua}; they also call a breadth of 30 feet and a length of 180 feet a {\em porca}. The Gauls give the name  {\em candetum} to areas of a
hundred feet in urban districts but to areas of 150 feet in rural districts they also call a half-{\em iugerum} an {\em arepennis}. Two {\em actus}, as I have said, form a {\em iugerum} 
240 feet long and 120 feet wide, which two numbers multiplied together make 28,800 square feet. Next a {\em stadium} contains 125 paces (that is to say 625 feet) which
multiplied by eight makes 1000 paces, which amount to 5000 feet. We now call an area of 200 {\em iugera} a {\em centuria}, as Varro again states; but formerly the {\em centuria}
was so called because it contained 100 {\em iugera}, but afterwards when it was doubled it retained the same name, just as the tribes were so called because the people were 
divided into three parts but now, though many times more numerous, still keep their old name. It was proper that we should begin by briefly mentioning these facts first, as being 
relevant to and closely connected with the system of calculation which we are going to set forth.
\end{quote}

Then in V.I.8--13 \cite[pp.~9--13]{columellaII} he defines different fractions of the {\em iugerum}:

\begin{quote}
Let us now come to our real purpose. We have not put down all the parts of the {\em iugerum} but only those which enter into the estimation of work done. For it was needless to
follow out the smaller fractions on which no business transaction depends. The {\em iugerum}, therefore, as we have said, contains 28,800 square feet, which number of feet is 
equivalent to 288 {\em scripula}. But to begin with the smallest fraction, the half-{\em scripulum}, the 576th part of a {\em iugerum}, contains 50 feet; it is the haif-{\em scripulum} of
the {\em iugerum}. The 288th part of the {\em iugerum} contains 100 feet; this is a {\em scripulum}. The 144th part contains 200 feet, that is two {\em scripula}. The 72nd part contains 
400 feet and is a {\em sextula}, in which there are four {\em scripula}. The 48th part, containing 600 feet, is a {\em sicilicus}, in which there
are six {\em scripula}. The 24th part, containing 1200 feet, is a {\em semi-uncia}, in which there are 12 {\em scripula}. The 12th part, containing 2400 feet, is the {\em uncia}, in
which there are 24 {\em scripula}. The 6th part, containing 4800 feet, is a {\em sextans}, in which there are 48 {\em scripula}. The 4th part, containing 7200 feet is a {\em quadrans},
in which there are 72 {\em scripula}. The 3rd part, containing 9600 feet, is a {\em triens}, in which there are 96 {\em scripula}. The 3rd part plus the 12th part, containing 12,000 feet,
is the {\em quincunx}, in which there are 120 {\em scripula}. The half of a {\em iugerum}, containing 14,400 feet, is a {\em semis}, in which there are 144 {\em scripula}. A half plus
a 12th part, containing 16,800 feet, is a {\em septunx}, in which there are 168 {\em scripula}. Two-thirds of a {\em iugerum}, containing 19,200 feet, is a {\em bes}, in which there are
192 {\em scripula}. Three-quarters, containing 21,600 feet, is a {\em dodrans}, in which there are 216 {\em scripula}. A half plus a third, containing 24,000 feet, is a {\em dextans},
in which there are 240 {\em scripula}. Two-thirds plus a quarter, containing 26,400 feet, is a {\em deunx}, in which there are 264 {\em scripula}. A {\em iugerum}, containing 28,800
feet, is the {\em as}, in which there are 288 {\em scripula}. If the form of the {\em iugerum} were always rectangular and, when measurements were being taken, were always
240 feet long and 120 feet wide, the calculation would be very quickly done; but since pieces of land of different shapes come to be the subjects of dispute, we will give below
specimens of every kind of shape which we will use as patterns.
\end{quote}

The area $A$ of an equilateral triangle whose sides have length $a$ is
\[
A=\sqrt{3} \cdot \frac{a^2}{4}.
\]
In V.II.5 \cite[pp.~15--17]{columellaII}, Columella writes:

\begin{quote}
But if you have to measure a triangle with three equal sides, you will follow this formula. Suppose the field to be triangular, three hundred feet on every side. Multiply this number by itself and the result is 90,000 feet. Take a third part of this sum, that is
30,000. Likewise take a tenth part, that is 9,000. Add the two numbers together; the result is 39,000. We shall say that this is the total number of square feet in this triangle, which measure makes a {\em iugerum}, plus a {\em triens} ($\frac{1}{3}$), plus a {\em sicilicus} ($\frac{1}{48}$).
\end{quote}

This amounts to 
\[
A \sim \frac{a^2}{3}+\frac{a^2}{10} = a^2 \cdot \frac{13}{30},
\]
which implies $\frac{\sqrt{3}}{4} \sim \frac{13}{30}$, or $\sqrt{3} \sim \frac{26}{15}$. 

The area of a regular hexagon whose sides have length $a$ is
\[
A = \sqrt{3} \cdot \frac{3a^2}{2}.
\]
In V.II.10 \cite[pp.~21--23]{columellaII}, Columella writes:

\begin{quote}
If the area has six angles, it is reduced to square  feet in the following manner. Let there be a hexagon, each side of which measures 30 feet. I multiply one side by itself:
30 times 30 makes 900. Of this sum I take one-third, which is 300, a tenth part of which is 90: total 390. This must be multiplied
by 6, because there are 6 sides: the product is 2310. We shall say, therefore, that this is the number of square feet. It will, then, be equivalent to an {\em uncia}
($\frac{1}{12}$ of a {\em iugerum}) less half a {\em scripulum} ($\frac{1}{596}$) plus $\frac{1}{10}$ of a {\em scripulum}.
\end{quote}

This amounts to
\[
A \sim 6 \cdot \left( \frac{a^2}{3}+\frac{a^2}{10} \right) = a^2 \cdot \frac{39}{15},
\]
which implies $\frac{3\sqrt{3}}{2} \sim  \frac{39}{15}$, or $\sqrt{3} \sim \frac{78}{45}$. 






\section{Frontinus}
Frontinus, {\em De Aquaeductu Urbis Romae} 24--25 \cite{rodgers}:

\begin{quote}
[24] Water pipes have been calibrated to measurement either in digits or in inches. Digits are employed in Campania and in most parts of Italy, but inches are still accepted
as standard in Apulia. (2) A digit, by convention, is one-sixteenth part of a foot, while an inch is one-twelfth. (3) Just as there is a distinction between the inch and the digit, there are
also two kinds of digits. (4) One is called square, the other round. (5) The square digit is larger than the round by three-fourteenths of its own size; the round digit is smaller than the
square by three-elevenths of its size (because, of course, the corners are taken away). [25] Later, a pipe called the 5-pipe ({\em quinaria}) came into use in the City to the exclusion 
of all former sizes. Its origin was based neither on the inch nor on either of the two kinds of digit. Some think that Agrippa was responsible for its introduction, others that this was done
by the lead-workers under the influence of the architect Vitruvius. (2) Those who credit Agrippa with its currency derive its name from the suggestion that into one such pipe were 
combined five of the slender ancient pipes (we might say little tubes) used for distributing the supply of water which in those times was not copious. Those who ascribe the 5-pipe to 
Vitruvius and the lead-workers suppose that its origin lay in producing a cylindrical pipe from a sheet of lead five digits in width. (3) The latter explanation is inexact, because in 
forming a cylindrical shape the inner surface is contracted while the outer surface is extended. (4) Most probable is the explanation that the name of the 5-pipe came from its diameter 
of five quarter-digits, (5) according to a system which remains consistent in pipes of increasing size up as far as the 20-pipe: the diameter of each increases in size by the addition of
one quarter-digit. For example, the 6-pipe has a diameter of six quarter-digits, the 7-pipe has seven, and so on by uniform increment up to a 20-pipe.
\end{quote}

See Rodgers \cite[pp.~209--211]{frontinus}.

26--29 \cite{rodgers}:

\begin{quote}
[26] The size of any pipe is determined either by its diameter, or its circumference, or the measure of its cross-section; from any one of these factors its capacity is evident. (2) That
we may more conveniently distinguish between the inch, the square digit, the round digit, and the 5-pipe itself, we need to treat ``the {\em quinaria}'' (5-pipe equivalent) as a unit of 
capacity, for its size is most accurate and its standard best established. (3) The inch pipe has a diameter of $1 \frac{1}{3}$ digits; its capacity is a little more than $1 \frac{1}{8}$
{\em quinariae}, the fraction being $\frac{1}{8}$ plus $\frac{3}{288}$ plus $\frac{2}{3}$ of another $\frac{1}{288}$. (4) A square digit converted to circular shape has a diameter of
$1 \frac{5}{36}$ digits; its capacity is $\frac{5}{6}$ of a {\em quinaria}. (5) A round digit has a diameter of 1 digit; its capacity is $\frac{23}{36}$ of a {\em quinaria}. [27] Now the
pipes based on the 5-pipe are increased in size in two ways. (2) One is by multiplying the 5-pipes themselves, that is by including the equivalent of several 5-pipes into one opening,
with the size of that opening increasing according to the addition of more 5-pipe equivalents. (3) This approach is more or less limited to instances where a number of {\em quinariae}
have been granted: to avoid tapping the conduit too often, a single pipe is used to lead the water into a delivery-tank, and from here individual persons draw off their respective
shares. [28] The second way does not involve an increase in pipe size related to a necessary number of 5-pipes. Instead, the increase is in the diameter of the pipe itself, a change
which alters both its name and its capacity. Take, for example, the 5-pipe: add a sixth quarter-digit to its diameter, and one has a 6-pipe, (2) but the capacity is not increased by an
entire 5-pipe equivalent (it has only $1 \frac{7}{16}$ {\em quinariae}). (3) By adding quarter-digits to the diameter in the same manner, as already explained, one gets larger pipes, a 7-pipe, an 8-pipe, and so on up to the 20-pipe. [29] Beyond the 20-pipe the gauging is based on the number of square digits which are contained in the cross-section, that is the
opening, of each pipe. From this same number the pipes also take their names. (2) Thus that pipe with an area of 25 square digits is called the 25-pipe; likewise the 30-pipe, and so
on by increase in square digits, up to the 120-pipe. 
\end{quote}

See Rodgers \cite[pp.~212--215]{frontinus}.






\section{Faventinus}
Faventinus, {\em De Diversis Fabricis Architectonicae} 28 \cite[p.~80]{plommer}:

\begin{quote}
Quoniam ad omnes usus normae ratio subtiliter inventa videtur, sine quo nihil utiliter fieri potest, hoc modo erit disponenda. sumantur itaque tres regulae, ita ut duae
sint pedibus binis et tertia habeat pedes duo uncias x. eae regulae aequali crassitudine compositae extremis acuminibus iungantur schema facientes trigoni. sic
fiet perite norma composita.
\end{quote}

A {\em norma} is a set-square, a right triangle.

Faventinus, {\em De Diversis Fabricis Architectonicae} 28 \cite[p.~81]{plommer}:

\begin{quote}
Since the principle of the square was a clever discovery and useful for all purposes -- since, indeed, nothing can be done very practically without it,
this is how you will prepare one. Take three scales, two of them 2 foot long, the third, 2 foot 10 inches. They are all to be of one uniform width, and are to be joined
at the ends to give the shape of a triangle. Your square will thus be made to professional standards.
\end{quote}

cf. {\em tegulae bipedales}






\section{Roman  camps}
Polybius, {\em Histories} VI.19--20 describes the formation of four legions, each of which is
said  to be usually 4200 infantry, when there is special danger 5000 infantry, and 300 cavalry.
Polybius,  VI.26--32 \cite[pp.~324--329, 553]{polybius} describes a Roman  camp for two legions. VI.26 \cite[p.~324]{polybius}: ``No matter where this is done, one simple formula for a camp is employed,
which is adopted at all times and in all places.''
VI.31 \cite[p.~328]{polybius}: ``The result of these dispositions is that the whole camp is laid out as a square, and the arrangement both
of the streets and the general plan gives it the appearance of a town. The rampart is dug on all sides at a distance of 200 feet from the tents, and this empty space
serves a number of important purposes.''
VI.32 \cite[p.~328]{polybius}: ``Given the numbers of cavalry and infantry, and on the assumption that the strength of each legion is either 4,000 or 5,000 men, and given
likewise the depth and length and the number of the maniples and squadrons, and besides these the dimensions of the passages and roads and all other details,
it is possible, for anybody who wishes, to calculate the area and perimeter of the camp.'' After stating that the market and the quaestor's depot should be reduced
if there are exceptionally many allies, VI.32  continues \cite[pp.~328--329]{polybius}:

\begin{quote}
On occasions when the two consuls with their four legions are united in one camp, all we need to do is to imagine two camps similar to the one I have described placed
back to back, the two adjoining at the point where the {\em extraordinarii} infantry are quartered, the troops whom we described as facing the ramparts to the rear of each
camp. In this case the shape of the camp becomes oblong, its area is doubled, and the perimeter of the entire rampart measures half as much again. Whenever the two
consuls happen to encamp together, this is the formation they adopt; when they are apart the only difference is that the market, the quaestor's depot and the {\em praetorium}
are placed between the two legions.
\end{quote}

Walbank \cite[p.~715]{walbankII} determines from Polybius's statements that 
the square camp that is described has sides 2150 feet. 

Josephus, {\em The Jewish War} III.5.1 \cite[p.~599]{josephus}: 

\begin{quote}
The Romans never lay themselves open to a surprise attack; for, whatever hostile territory they may invade, they engage in no battle until they have fortified their camp.
This camp is not erected at random or unevenly; they do not all work at once or in disorderly parties; if the ground is uneven, it is first levelled; 
a site for the camp is then measured out in the form of a square. For this purpose the army is accompanied by a multitude of workmen and of tools for building.
\end{quote}

Indeed, {\em tetragonos} usually means square according to LSJ. Gibbon, Chapter I, ``The camp of a Roman legion
presented the appearance of a fortified city. As soon as the space was marked out, the pioneers carefully levelled
the ground, and removed every impediment that might interrupt its perfect regularity.
Its form was an exact quadrangle; and we may
calculate, that a square of about seven hundred yards was sufficient for the encampment of twenty thousand Romans; though a similar number
of our troops would expose to the enemy a front of more than treble that extent.''

Vegetius, {\em Epitoma Rei Militaris} 
I.22 \cite[p.~24]{vegetius}: ``The camp should be built according to the number of soldiers and baggage-train, lest too great
a multitude be crammed in a small area, or a small force in too large a space be compelled to be spread
out more than is appropriate.''
I.23 \cite[p.~24]{vegetius}: ``Camps should be made sometimes square, sometimes triangular,
sometimes semicircular, according as the nature and demands of the site require.''

II.7 \cite[pp.~38--39]{vegetius}: ``Quartermasters measure out the places in camp according to the square footage
for the soldiers to pitch their tents, or else assign them billets in cities.''

III.8 \cite[p.~80]{vegetius}: ``When these conditions have been carefully and stringently investigated, you may build
the camp square, circular, triangular or oblong, as required by the site. Appearance should not prejudice utility, although those
whose length is one-third longer than the width are deemed more attractive. But surveyors should calculcate
the square footage of the site-plan so that the area enclosed corresponds to the size of the army. Cramped quarters
constrict the defenders, whilst unsuitably wide spaces spread them thinly.''

III.15 \cite[p.~97]{vegetius}: ``We said that 6 ft. ought to lie between each line in depth from the rear, and in fact each
warrior occupies 1 ft. standing still. Therefore, if you draw up six lines, an army of 10,000 men will take up 42 ft.
in depth and a mile in breadth. [If you decide to draw up three lines, an army of 10,000 will take up 21 ft. in depth and two
miles in breadth.] In accordance with this system, it will be possible to draw up even 20,000 or 30,000 infantry without the slightest
difficulty, if you follow the square footage for the size. The general does not go wrong when he knows what space
can hold how many fighting men.'' 







\section{Palladius}
Palladius Rutilius Taurus Aemilianus, {\em Opus agriculturae} II.11, {\em De tabulis uinearum} \cite[p.~54]{palladii}:

\begin{quote}
Tabulas autem pro domini uoluptate uel loci ratione
faciemus siue integrum iugerum continentes seu medium
seu quaternariam tabulam, quae quartam iugeri partem
quadrata conficiet.
\end{quote}

The word {\em tabula} is said by Souter, {\em A Glossary of Later Latin to 600 A.D.}, s.v., to mean a ``stretch (of land) in a vineyard''.

Fitch \cite[p.~75]{fitch}:

\begin{quote}
We shall make the planting-beds in accordance with the owner's inclination or the requirements of the place, covering a whole juger or half or a quarter-bed, which consists of a fourth of a juger in square footage. 
\end{quote}

II.12, {\em De mensura pastini Italica} \cite[p.~55]{palladii}:

\begin{quote}
Mensura uero pastini haec est in tabula quadrata iugerali,
ut centeni octogeni pedes per singula latera dirigantur,
qui multiplicati trecentas uiginti et quattuor decempedas
quadratas per spatium omne conplebunt.
secundum hunc numerum omnia quae uolueris pastinare discuties.
decem et octo enum decempedae decies et octies subputatae
trecentas uiginti quattuor explebunt.
quo exemplo doceberis in maiore agri uel minore mensuram. 
\end{quote}

The word {\em iugeralis} is said by Souter, {\em A Glossary of Later Latin to 600 A.D.}, s.v., to mean ``of the land-measure called iugerum'' or ``very large''. 
Rodgers \cite[p.~96]{palladius}:

\begin{quote}
With 32400 sq.ft., P.'s {\em tabula iugeralis} is larger by 3200 sq.ft. than a normal {\em iugerum} ($240 \times 120$ ft.),
but P. is careful to explain that he calculates his {\em tabula} with 180 ft. on a side. I wonder if he arrived at the length of one side
of his ``squared'' {\em iugerum} by dividing the perimeter of a {\em iugerum} ($2 \times 240+2 \times 120=720$) by four equal
sides ($720/4=180$). He is at pains to tell us that the total area is 324 {\em decempedae quadratae}, and I suppose  it is
possible for him to say (2.11) that the {\em tabula} will contain an {\em integrum iugerum}. With {\em medium} (2.11) he must mean
half the area of a {\em iugerum} (traditionally called an {\em actus}, 120 ft. on a side) or 14400 sq.ft.; I doubt that he would
have been meaning half of 32400 sq.ft., which would be $10 \times \surd 162$ ft. on a side. His {\em quaternaria tabula}, I think,
would be 90 ft. on a side or 8100 sq.ft. (one-fourth of 32400) rather than one-fourth the area of a {\em iugerum}, 7200 sq.ft.,
$10 \times \surd 72$ ft. on a side. No-one, I am sure, would object to these rough approximations, least of all P. himself (for his mathematical inexactitude, see my note
on 3.9.9).
\end{quote}

Fitch \cite[p.~75]{fitch}:

\begin{quote}
In a square planting-bed covering one juger, the measurement of the prepared ground is 180 feet on each straight side; when multiplied this will yield 324 10-foot square units across the whole area. Using this figure, you will divide up all the ground you want to prepare. For 18 10-foot lengths multiplied 18 times will yield 324.
This example will show you how to measure a larger or smaller field.
\end{quote}






\section{Agrimensores}
Folkerts \cite{folkerts}

{\em Podismus} \S 7 \cite[pp.~134--137]{guillaumin} states Heron's formula for the right triangle with sides
$6,8,10$; the area of the triangle is $26$.  

We refer to the tractate in the {\em Corpus agrimensorum} attributed to Epaphroditus and Vitruvius Rufus by {\em EVR}. {\em EVR}  
\S 10 \cite[pp.~140--141]{guillaumin}: let $ABCD$ be a right trapezium where $AB$ and $DC$ are parallel, $ADC$ is a right angle,
$AB=25$ feet, $DC=40$ feet, $DA=30$ feet; call $AB$ the summit, $BC$ the hypotenuse, $DC$ the base, and $AD$ the height.
The recipe given for finding the area
of the right triangle with height $AD$ and hypotenuse $BC$ is the following:
add the base $DC$ and the summit $AB$, getting $65$,  take half of this, getting $32 \frac{1}{2}$, and multiply this
by the height $AD$, getting $975$. The recipe given for finding the
hypotenuse $BC$ is the following: add the 
squares on the summit, the base, and the height, getting $3125$. 
Subtract from this twice the product of the base and the summit, i.e. subtract $2 \cdot 25 \cdot 40 = 2000$ from $3125$, getting $1125$. Then $BC$ is the side
of the square $1125$, namely $BC^2=1125$. That is, 
\[
BC^2 = AB^2+DC^2+AD^2 - 2  DC \cdot AB = (DC-AB)^2 + AD^2.
\]
It is stated that $BC$ is $33 \frac{1}{2}$; indeed, $33^2=1089$ and $34^2=1156$.

{\em EVR} \S 11 \cite[pp.~140--143]{guillaumin}: for an equilateral triangle whose sides are $30$ feet,
multiply a side by itself, getting $30\cdot 30 = 900$. Multiply half a side by itself, getting $15\cdot 15 = 225$. Then take away $225$ from
$900$, getting $675$, which is the area. It is stated that the side of the square $675$ is $26$. (Indeed, $25^2=625$ and $26^2=676$.)
This is the height of the triangle.
Then multiply  the height by half the base, getting $26\cdot 15 = 390$. This is the area of the triangle.

{\em EVR} \S 28 \cite[pp.~158--163]{guillaumin}: for an equilateral triangle whose sides are even numbers,
to find the area. Guillaumin explains that in \S \S 28, 30--37 
figurate numbers are being used:
for the triangular number whose each have $n$ pebbles, the figure contains $\frac{n^2+n}{2}$ pebbles; cf.
Nicomachus, {\em Introductio Arithmetica} II.7--12 \cite[pp~239--249]{nicomachus} and Heath \cite[p.~76]{HGMI}. 
The example is given of the equilateral triangle whose sides are $28$ feet, 
multiply a side by itself, getting $28\cdot 28=784$. Add a side to this, getting $784+28=812$. Take half of this, getting $406$. It is asserted that this
is the area of the triangle. (The height of the triangle is $h=\sqrt{28^2-14^2}=\sqrt{588}$, which satisfies $24<h<24 \frac{1}{4}$. 
Then the area of the triangle is half the product of the base and the height, i.e. $\frac{28\cdot h}{2}$, and using $h=24 \frac{1}{4}$ this is
$339 \frac{1}{2}$.)
Conversely the side of a triangle is found given the area.
Multiply the area by $8$, getting $8\cdot 406 = 3248$. Add $1$ to this, getting $3249$. The side of this square is 
$57$. Remove $1$ from this, getting $56$. Take half of this, getting $28$, which is the side of the triangle. 

For an $a$-gonal number with $n$ pebbles on each side, the figure contains 
\[
\frac{(2+(2n-1)(a-2))^2-(a-4)^2}{8(a-2)}
\]
pebbles; cf. Heath \cite[p.~516]{HGMII}.
Conversely, if the figure contains $P$ pebbles, then
\[
n=\frac{1}{2}\left( \frac{\sqrt{8P(a-2)+(a-4)^2}-2}{a-2}+1\right).
\]
{\em EVR} \S 29 \cite[pp.~164--167]{guillaumin} states that for a pentagon with equal sides, multiply a side by itself,
multiply this by $3$, then add one side, and that this gives the pentagon. If the sides are each $10$ feet, multiply
a side by itself, getting $100$. Multiply this by $3$, getting $300$. Add a side to this, getting $310$. Take
half of this, getting $155$, which is said to be the area of the pentagon. 
Conversely, if the area is $155$, to find the side do the following: multiply the area by $24$, getting
$24 \cdot 155 = 3720$. Add $1$ to this, getting $3721$. Find the side of the square $3721$, which is $61$. 
Remove $1$ from this number, getting $60$. Take a sixth of this, getting $10$, which is the said to be the side of the
pentagon.

{\em EVR} \S 31 \cite[pp.~172--177]{guillaumin}: for a hexagon with equal sides, multiply
a side by itself, multiply this by $4$, add twice a side to this, and then take half of this, and it is asserted
that this gives the pentagon. If the sides are $10$ feet, multiply a side by itself, getting $100$. 
Multiply this by $4$, getting $400$. Add twice a side to this, getting $400+2\cdot 10=420$. Take half
of this, getting $210$. It is asserted that this is the area of the hexagon. Conversely,
given the area of the hexagon, find the side. Multiply the area by $32$, getting $32 \cdot 210 = 6720$.
Add $4$ to this, getting $6724$. Find the side of the square $6724$, which is $82$. Remove $2$ from this,
getting $80$. Take an eighth of this, getting $10$. It is asserted that this is the side of the hexagon.

{\em EVR} \S 32 \cite[pp.~176--179]{guillaumin} states that for a heptagon with equal sides, multiply a side by
itself, multiply this by $5$, remove three times a side from this, and then take half of this, and it is asserted that
this is gives the heptagon. If the sides are $10$ feet,
multiply a side by itself, getting $100$. Multiply this by $5$, getting $500$. Remove three times a side from this,
getting $500-3\cdot 10=470$. Take half of this, getting $235$, which is asserted to be the area of the hexagon.
Conversely,
given the area of the heptagon, find the side. Multiply the area by $40$, getting $40 \cdot 235 = 9400$.
Add $9$ to this, getting $9409$. Find the side of the square $9409$, which is $97$. Add $3$ to this, getting $100$.
Take a tenth of this, getting $10$. It is asserted that this is the side of the heptagon.

{\em EVR} \S 33 \cite[pp.~178--179]{guillaumin} states that for an octagon with equal sides, 
multiply a side by itself, multiply this by $6$, remove four times a side from this, and then take half of this, and it is asserted that this gives the octagon.
If the sides are $10$ feet, multiply a side by itself, getting $100$. Multiply this by $6$, getting $600$. Remove four times a side from this,
getting $600-4\cdot 10=560$. Take half of this, getting $280$, which is asserted to be the area of the octagon. Conversely,
given the area of the octagon, find the side. Multiply the area by $48$, getting $48 \cdot 280 = 13440$. Add $16$ to this, getting $13456$. Find the side
of the square $13456$, which is $116$. Add $4$ to this, getting $120$. Take a twelfth of this, getting $10$, which is asserted to be the side
of the octagon. 

{\em EVR} \S \S 34--37 \cite[pp.~180--187]{guillaumin} treat respectively the enneagon, the decagon, the hendecagon, and the dodecagon. 

{\em De iugeribus metiundis} \S 54 \cite[pp.~198--201]{guillaumin}, cf.  \cite[p.~354--356]{blumeI}:

\begin{quote}
Castrense iugerum quadratas habet perticas CCLXXXVIII, pedes autem quadratos $\overline{\text{X}} \overline{\text{X}} \overline{\text{VIII}}$DCCC, 
id est per latus unum perticas XVIII, quae in quattuor latera faciunt perticas LXXII; habet itaque tabula una quadratas perticas LXXII. Si ergo
fuerit ager tetragonus isopleurus, habens per latus unum perticas L, ita eum metiri oportet ut sciamus quot iugera habeat intra se. Duco unum latus
per aliud: fiunt perticae $\overline{\text{II}}$D, quae faciunt iugera VIII, tabulas II, perticas LII. Itaque castrense iugerum capit k(astrenses) modios III.
\end{quote}

It is first stated that a {\em iugerum} contains $288$ square {\em perticae}. 
A {\em iugerum} is a rectangle with sides 240 feet and 120 feet,  thus whose area is $28800$ square feet. A {\em pertica} is 
a length of $10$ feet; see Balblus, {\em Expositio et ratio omnium formarum} \cite[p.~207]{campbell},
{\em Centuriarum quadratarum deformatio sive mensurarum diversarum ritus} \cite[p.~241]{campbell},
and {\em De mensuris agrorum} \cite[p.~271]{campbell}. (Thus, one iugerum contains $288$ square perticae.)
Next it is asserted that the side of the square $28800$ is $18$ perticae, whose perimeter is $72$ perticae.
In fact, $169^2<28800<170^2$, while
a square with side $18$ perticae contains $32400$ square feet. Guillaumin remarks
that the sides of the iugerum are 24 perticae and 12 perticae, and 
$18$ perticae is the arithmetic mean of these. If a rectangle has sides 
$a$ and $b$ with $b > a$, then the square with side $\frac{a+b}{2}$ has the same perimeter as the rectangle, namely
$a+a+b+b$, and has area $\left(\frac{a+b}{2}\right)^2 = \frac{a^2+b^2+2ab}{4}$, while the rectangle has area $ab$, for which
\[
\frac{a^2+b^2+2ab}{4}-ab = \frac{a^2+b^2-2ab}{4} = \frac{1}{4} (b-a)^2.
\]
Thus, the square with the same perimeter as the rectangle has greater area.
It is stated that one {\em tabula} contains $72$ square perticae, i.e., one {\em tabula} contains $7200$ square feet, namely, one {\em tabula} is a quarter of one iugerum.
Then, for a square field whose sides are $50$ perticae,  find how many iugera it contains. Multiply one side of the square by another, getting 
$2500$ square perticae. As $2500=8\cdot 288+196$, 
this field contains $8$ iugera and $196$ square perticae. As $196=2\cdot 72+52$, the remaining $196$ square feet contain
$2$ {\em tabulae} and $52$ square feet; thus the field contains $8$ iugera, 
$2$ {\em tabulae}, and $52$ square perticae.  

\S 56 \cite[pp.~202--203]{guillaumin}:

\begin{quote}
Ager si fuerit trigonus isopleurus, habens tria latera per
quae sexagenas perticas habeat, duco unum latus per alterius
lateris medietatem, id est LX per XXX: fiunt perticae MDCCC,
quae faciunt iugera VI, tabulam unam.
\end{quote}

If a field is an equilateral triangle whose sides are 60 perticae, multiply one side by half another, giving $60 \cdot 30=1800$ square perticae.
$1800=6 \cdot 288+72$, so this is $6$ iugera 1 tabula.

\S 57 \cite[pp.~202--203]{guillaumin}:

\begin{quote}
Ager si caput bubulum fuerit, id est duo trigona isopleura
iuncta, habentia per latus unum perticas L, unius trigoni latus in
alterius trigoni latus duco, id est L per L: fiunt $\overline{\text{II}}$D, quae sunt
iugera VIII, tabulae IIS, perticae XVI.
\end{quote}

If a field is two joined equilateral triangles (a ``head of beef''), whose sides are 50 perticae, multiply the side of one triangle by the side of the other triangle,
that is $50 \cdot 50=2500=8\cdot 288+2\cdot 72+36+16$. That is, the area is $2500$ square perticae, which is $8$ iugera, $2 \frac{1}{2}$ tabulae, 16 square perticae.

\S 63 \cite[pp.~210--211]{guillaumin}:

\begin{quote}
Ager si fuerit sex angulorum, in quadratos pedes sic
redigitur. Esto exagonum in quo sint per latus unum perticae XXX.
Latus unum in se multiplico, id est tricies triceni: fiunt perticae
DCCCC. Huius summae tertiam partem statuo, id est CCC. Nihilominus
ex eadem pleniori summa decimam partem tollo,  id est
XC. Quae pariter iunctae faciunt CCCXC. Quae sexies ducendae
sunt, quia sex latera habet: quae summa colligit perticas $\overline{\text{II}}$CCCXL.
Tot igitur quadratas perticas in hoc agro esse dicimus.
\end{quote}

For a field that is a hexagon where each side is 30 perticae. 
Multiply one side by itself, getting $30 \times 30=900$. Take a third of $900$, which is $300$, and a tenth of $900$, which is $90$. The sum of these two
is $300+90=390$. Multiply this by $6$, getting $2340$.
The area of the field is $2340$ square perticae. cf. Heron in Heath \cite[p.~327]{HGMII}






Marcus Junius Nipsus, {\em Limitis Repositio} \cite[p.~51]{bouma}:

\begin{quote}
In agris divisis subsiciva fiunt, in quibus trigona, trapezea et
pentagona sunt, et nihil alius nisi modus iugerum adsignatorum et
nomen scriptum est. Actus tamen in base sunt xx. Sic ut puta in
pentagono liis, bis ducti, faciunt cv. Qui in se ducti, faciunt
iugera lxv. Cathetum sic quaerimus semper. Embadum duco quater -- id
est lxv --; fiunt cclx. Huius summae pars vicesima fit xiii; erit
cathetus. In trigono sunt actus xlii, iugera cl. Insequentem actum
iunctum trigono ac trapezeo similiter. Quae si autem fuerint in
trapezeo iugera c, iugera ducta quater, erunt cccc. Horum pars
vicesima -- hoc est xx -- erit basis. Deducto contrario -- id est xx -- 
fit reliquum vii. Erit contraria basis actus vii. Similiter in
reliquius pedibus, si fuerint cc.
\end{quote}

Bouma \cite[p.~73]{bouma} translates:

\begin{quote}
When dividing land, pieces of land remain; these can be triangles,
trapeziums and pentagons; and nothing else but the number of {\em iugera}
assigned and their name has been written down (on the {\em forma}). Yet there are
20 {\em actus} at the base. Thus for instance $52 \frac{1}{2}$ ({\em actus}) in a pentagon make,
multiplied by two, 105 ({\em actus}). Together they comprise 65 {\em iugera}. We
always seek the perpendicular as described below. I multiply the area (of the
pentagon) --  that is 65 {\em iugera} -- by four. This makes 260 ({\em iugera}). From these
the 20th part makes 13. This will be the perpendicular.

In a triangle are 42 {\em actus}, 150 {\em iugera}. Likewise (we want to know) the next
(number of) {\em actus} of triangle and trapezium. If there are 100 {\em iugera} in a 
trapezium, there will be, when multiplied by four, 400 ({\em iugera}). The 20th
part of these (400 {\em iugera}) -- that is 20 -- will be the base. When subtracted
from the opposite base -- that is 20 --, 7 remain: The opposite base will be 7
{\em actus}. The same goes for the other feet, if they are 200.
\end{quote}

For an isosceles trapezium with base $a$, summit $b$, and sides $c$ and $c$, with $b>a$,
let $h$ be its height.
Then $h^2+\left( \frac{b-a}{2} \right)^2 = c^2$, and the area of the trapezium is 
$A=ah+\frac{1}{2}(b-a)h=\frac{1}{2}(b+a)h$. Now,
\[
\frac{b+a}{2} \cdot \frac{c+c}{2} - A
=\frac{b+a}{2} \cdot c -\frac{1}{2}(b+a)h
=\frac{b+a}{2} (c-h).
\]
Thus, $\frac{b+a}{2} \cdot \frac{c+c}{2}$ is greater than the area of the trapezium, as $c>h$.

{\em Vaticanus Palatinus graecus} 367, ff.~94r--97v, no.~23 \cite[p.~51]{fisc}:
for a trapezium with base 16 {\em orgyiai}, summit 
20 {\em orgyiai}, and sides each 25 {\em orgyiai}, the area is said to be
$\frac{16+20}{2} \cdot \frac{25+25}{2} = 18 \cdot 25 = 450$ square {\em orgyiai}, which is 
$2 \frac{1}{4}$ {\em modioi};
an {\em orgyia} is six feet, and a {\em modios} is an area equal to 200 square {\em orgyiai}. 
In fact the area is $54 \sqrt{69}$ square {\em orgyiai}, and $448<54 \sqrt{69}<449$. 









\section{Diophantus of Alexandria}
Let $a$ and $b$ be numbers and let $S$ be those numbers $x$ such that $ax^2-b$ is a square. Diophantus, {\em Arithmetica}, Lemma to VI.15 \cite[p.~238]{diophantus}
states that if $p \in S$ then there is some $q>p$ such that $q \in S$. 
For $a=3$ and $b=11$, it is the case that $3\cdot 25-11=64$ is a square, and take $p=8$. 
Find $q=x+5$ such that $3q^2-11$ is a square, namely $3x^2+30+64$ is a square. It is supposed that it is possible that
this square is $(8-2x)^2$, and so $3x^2+30x+64=64-32x+4x^2$ which is $x^2=62x$, and thus $x=62$.   Then
$q=67$. Then $3\cdot q^2-11=13456=116^2$, so $67 \in S$. 

Heath  \cite[pp.~279--280]{diophantus} explains a generalization of this by Tannery. 
Given that $(p,q)$ is an integral solution of $x^2-Ay^2=1$,
put
\[
p_1=mx-p,\qquad q_1=x+q
\]
with $x \neq 0$ and $m^2 \neq A$.
Analytically, for $(p_1,q_1)$ to be a solution of $x^2-Ay^2=1$ means 
\[
m^2x^2-2mpx+p^2-Ax^2-2Aqx-Aq^2=1,
\]
and because $(p,q)$ is a solution of $x^2-Ay^2=1$ this is equivalent with
\[
m^2x^2-2mpx-Ax^2-2Aqx=0.
\]
This is equivalent with $m^2x-Ax=2mp-2Aq$, which is equivalent with
\[
x =  \frac{2mp+2Aq}{m^2-A},
\]
which finally is  equivalent with
\begin{equation}
p_1=\frac{2m^2p+2Amq}{m^2-A}-p=\frac{2m^2p+2Amq-m^2p+Ap}{m^2-A}
=\frac{(m^2+A)p+2Amq}{m^2-A}
\label{p1}
\end{equation}
and
\begin{equation}
q_1 =  \frac{2mp+2Aq}{m^2-A} + q = 
\frac{2mp+2Aq+m^2q-Aq}{m^2-A}
=\frac{2mp+(m^2+A)q}{m^2-A}.
\label{q1}
\end{equation}
Synthetically, given the final expressions for $p_1$ and $q_1$,
\begin{align*}
p_1^2-Aq_1^2 &= \frac{A^2p^2-2Am^2p^2+m^4p^2-A^3q^2+2A^2m^2q^2
-Am^4q^2}{(m^2-A)^2}\\
&=\frac{(m^2-A)^2 \cdot (p^2-Aq^2)}{(m^2-A)^2}\\
&=p^2-Aq^2\\
&=1.
\end{align*}
Now take $x \neq 0$ and $m=\frac{u}{v}$ and $m^2 \neq A$. 
Then \eqref{p1} and \eqref{q1} are equivalent with
\begin{equation}
p_1 = \frac{pu^2+2Aquv+Apv^2}{u^2-Av^2},\qquad
q_1 = \frac{qu^2+2puv+Aqv^2}{u^2-Av^2}.
\label{p1q1}
\end{equation}
For $p_1,q_2$ to be integral is equivalent with $(u,v)$ being a solution
of $x^2-Ay^2=1$. 
Therefore, if $(p,q)$ and $(u,v)$ are integral solutions of
$x^2-Ay^2=1$ then $(p_1,q_1)$ defined by  \eqref{p1q1} is an integral
solution of $x^2-Ay^2=1$. 







\section{Pappus of Alexandria}
Pappus  \cite[pp.~1057--1059]{pappusIIIi}, {\em Collection} VIII, Proposition 9: 
if a weight of 200 talents needs a force of 40 men to be moved on a horizontal plane,
to find the force needed to move it on a plane inclined at $60^\circ$.
In this calculation,
\[
\sin 60^\circ \sim \frac{104}{120}.
\]

Pappus V, Proposition 42

Pappus V, Proposition 45

Pappus V, Proposition 46

Pappus \cite[p.~1244]{pappusIIIi}, V, Proposition 51, p.~451. 





\section{Theon of Alexandria}
Euclid II.4 \cite[p.~379]{euclidI}: ``If a straight line be cut at random, the square on the whole
is equal to the squares on the segments and twice the rectangle contained by the segments.''
This means that if $x=y+z$ then $x^2=y^2+z^2+2yz$. 

Theon of Alexandria, {\em Commentary on Ptolemy's Syntaxis} \cite[pp.~52--61]{thomasI},
cites this proposition and works out an approximation to 
$\sqrt{4500}$; this is explained by Heath \cite[pp.~60--63]{HGMI}.
$67^2=4489$, and take
\[
\sqrt{4500} = 67 + \frac{x}{60}+\frac{y}{60^2}.
\]
Then
\[
4500 = 4489 + \left(  \frac{x}{60}+\frac{y}{60^2} \right)^2 + 2 \cdot 67 \cdot  \left(  \frac{x}{60}+\frac{y}{60^2} \right),
\]
which is
\[
11 = \frac{x^2}{60^2}+\frac{y^2}{60^4} +  \frac{2xy}{60^3} + \frac{134x}{60} + \frac{134y}{60^2}.
\]
Determine $x$ so that $\frac{134x}{60} < 11$, i.e. $x<\frac{330}{67}$. For $x=4$,
\[
11 = \frac{16}{60^2} + \frac{y^2}{60^4} + \frac{8y}{60^3} + \frac{536}{60} + \frac{134y}{60^2},
\]
then
\[
\frac{7424}{60^2} =\frac{y}{60^2} \left(  \frac{y}{60^2} + \frac{8}{60} +134\right).
\]
When $\frac{y}{60^2} \sim 0$, 
\[
7424 \sim y \cdot \frac{8+60\cdot 134}{60}, \qquad y \sim 
\frac{60 \cdot 7424}{8048},
\]
which is $y \sim 55 + \frac{165}{503}$. Using $y=55$ yields
\[
\sqrt{4500} \sim 67 + \frac{4}{60} + \frac{55}{60^2},
\]
the approximation that appears in {\em Almagest} I.10.








\section{Side and diagonal numbers}
H{\o}yrup \cite[p.~261]{hoyrup}

Heath \cite[pp.~392--402]{euclidI}: Euclid II.9,10. II.9:
\[
a^2+b^2=2\left[ \left(  \frac{1}{2}(a+b) \right)^2
+\left(\frac{1}{2}(a+b)-b\right)^2 \right].
\]
II.10:
\[
(2a+b)^2 + b^2 = 2(a^2+(a+b)^2).
\]

Theon of Smyrna, {\em On Mathematics Useful for the Understanding of Plato} I.31  \cite[pp.~70--75]{dupuis}, 

Theon of Smyrna and Proclus, translated by Thomas \cite[pp.~132--139]{thomasI}.

Proclus, {\em Commentary on Plato's {\em Republic}}, dissertation XIII \cite[pp.~133--135]{festugiereII}.
Waterfield \cite[pp.~107--108]{waterfield} translates Proclus:

\begin{quote}
The Pythagoreans proposed the following elegant theorem
about diameter and side numbers. When to a diameter there is added
the side of which it is the diameter, it becomes a side, while the side,
when added to itself and receiving its own diameter in addition as
well, becomes a diameter. This is proved with the aid of a diagram by
Euclid in the second book of the {\em Elements}. If a straight line is
bisected and a straight line is added to it, the square on the whole
line (that is, including the added line) plus the square on the added
line by itself are together double the square on the half and of the
square on the straight line made up of the half and the added
line.
\end{quote}

Iamblichus, {\em Commentary on Nicomachus} \cite{iamblichus}, IV.144--156.

Cohen and Drabkin \cite[pp.~42--43]{drabkin}.


Proclus, {\em Commentary on the First Book of Euclid's Elements}  \cite{proclus}

Taylor on {\em Timaeus} \cite{taylortimaeus}




\section{Arabic}
Ibn Labban \cite{wisconsin8}






\section{Continued fractions}
Hardy and Wright \cite{hardy}, \S 10.6: 
For $0 \leq \xi_n < 1$,
\begin{align*}
x&=a_0+\xi_0,\\
\frac{1}{\xi_0}&=a_1'=a_1+\xi_1,\\
\frac{1}{\xi_1}&=a_2'=a_2+\xi_2,\\
\frac{1}{\xi_2}&=a_3'=a_3+\xi_3,\\
&\ldots
\end{align*}

Let $x=a_0'=\sqrt{10}$. 
$a_0=[a_0']=3$, $a_0'=a_0+\xi_0$;
$a_1'=\frac{1}{\xi_0}$, $a_1=[a_1']=6$, $a_1'=a_1+\xi_1$;
$a_2'=\frac{1}{\xi_1}$, $a_2=[a_2']=6$, $a_2'=a_2+\xi_2$;
$a_3'=\frac{1}{\xi_2}$, $a_3=[a_3']=6$, $a_3'=a_3+\xi_3$.
Thus,
\[
\sqrt{10} \sim [3,6,6,6] = 3+\cfrac{1}{6+\cfrac{1}{6+\cfrac{1}{6}}}
=\frac{721}{228}.
\]

Fowler \cite{fowler}

Weil \cite{weil}


Euclid X.1,2.






\bibliographystyle{plain}
\bibliography{squareroots}

\end{document}
