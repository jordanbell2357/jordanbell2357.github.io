\documentclass{article}
\usepackage{amsmath,amssymb,mathrsfs,amsthm}
%\usepackage{tikz-cd}
%\usepackage{hyperref}
\newcommand{\inner}[2]{\left\langle #1, #2 \right\rangle}
\newcommand{\tr}{\ensuremath\mathrm{tr}\,} 
\newcommand{\Span}{\ensuremath\mathrm{span}} 
\def\Re{\ensuremath{\mathrm{Re}}\,}
\def\Im{\ensuremath{\mathrm{Im}}\,}
\newcommand{\id}{\ensuremath\mathrm{id}} 
\newcommand{\diam}{\ensuremath\mathrm{diam}} 
\newcommand{\Prog}{\ensuremath\mathrm{Prog}} 
\newcommand{\lcm}{\ensuremath\mathrm{lcm}} 
\newcommand{\supp}{\ensuremath\mathrm{supp}\,}
\newcommand{\grad}{\ensuremath\mathrm{grad}\,}
\newcommand{\dom}{\ensuremath\mathrm{dom}\,}
\newcommand{\upto}{\nearrow}
\newcommand{\downto}{\searrow}
\newcommand{\norm}[1]{\left\Vert #1 \right\Vert}
\theoremstyle{definition}
\newtheorem{theorem}{Theorem}
\newtheorem{lemma}[theorem]{Lemma}
\newtheorem{proposition}[theorem]{Proposition}
\newtheorem{corollary}[theorem]{Corollary}
\theoremstyle{definition}
\newtheorem{definition}[theorem]{Definition}
\newtheorem{example}[theorem]{Example}
\begin{document}
\title{Jointly measurable and progressively measurable stochastic processes}
\author{Jordan Bell}
\date{June 18, 2015}

\maketitle

\section{Jointly measurable stochastic processes}
Let $E=\mathbb{R}^d$  with Borel $\mathscr{E}$, 
let $I = \mathbb{R}_{\geq 0}$, which is a topological  space with the subspace topology inherited from $\mathbb{R}$,
and let
$(\Omega,\mathscr{F},P)$ be a probability space.
For a stochastic process $(X_t)_{t \in I}$ with state space $E$,
we say that
$X$ is \textbf{jointly measurable} if the map
$(t,\omega) \mapsto X_t(\omega)$ is measurable $\mathscr{B}_I \otimes \mathscr{F} \to \mathscr{E}$. 

For $\omega \in \Omega$, the path $t \mapsto X_t(\omega)$ is called
\textbf{left-continuous} if 
for each $t \in I$,
\[
X_s(\omega) \to X_t(\omega),\qquad s \uparrow t.
\] 
We prove that if the paths of a stochastic process are left-continuous then the stochastic process
is jointly measurable.\footnote{cf. Charalambos D. Aliprantis and Kim C. Border, {\em Infinite Dimensional Analysis:
A Hitchhiker's Guide}, third ed., p.~153, Lemma 4.51.}

\begin{theorem}
If $X$ is a stochastic process with state space $E$ and  all the paths of $X$ are left-continuous, then $X$ is
jointly measurable.
\end{theorem}
\begin{proof}
For $n \geq 1$, $t \in I$, and  $\omega \in \Omega$, let 
\[
X^n_t(\omega) = \sum_{k=0}^\infty 1_{[k2^{-n}, (k+1)2^{-n})}(t) X_{k 2^{-n}}(\omega).
\]
Each $X^n$ is measurable $\mathscr{B}_I \otimes \mathscr{F} \to \mathscr{E}$:
for $B \in \mathscr{E}$,
\[
\{(t,\omega) \in I \times \Omega: X^n_t(\omega) \in B\}
=\bigcup_{k=0}^\infty [k2^{-n},(k+1)2^{-n}) \times \{X_{k2^{-n}} \in B\}.
\]
Let $t \in I$.
For each $n$ there is a unique $k_n$ for which $t \in [k_n2^{-n},(k_n+1)2^{-n})$, and 
thus $X_t^n(\omega) = X_{k_n 2^{-n}}(\omega)$. Furthermore, $k_n2^{-n} \uparrow t$, and because 
 $s \mapsto X_s(\omega)$ is left-continuous,  $X_{k_n 2^{-n}}(\omega) \to X_t(\omega)$.
That is, 
$X^n \to X$ pointwise on $I \times \Omega$, 
and because each $X_n$ is measurable 
$\mathscr{B}_I \otimes \mathscr{F} \to \mathscr{E}$ this implies that
$X$ is measurable $\mathscr{B}_I \otimes \mathscr{F} \to \mathscr{E}$.\footnote{Charalambos D. Aliprantis and Kim C. Border, {\em Infinite Dimensional Analysis:
A Hitchhiker's Guide}, third ed., p.~142, Lemma 4.29.} Namely, the stochastic process $(X_t)_{t \in I}$ is jointly measurable, proving the claim.
\end{proof}



\section{Adapted stochastic processes}
Let $\mathscr{F}_I = (\mathscr{F}_t)_{t \in I}$ be a filtration of $\mathscr{F}$. 
A stochastic process $X$ is said to be \textbf{adapted to the filtration $\mathscr{F}_I$}
if for each $t \in I$ the map
\[
\omega \mapsto X_t(\omega), \qquad \Omega \to E,
\]
is measurable $\mathscr{F}_t \to \mathscr{E}$, in other words, for each $t \in I$,
\[
\sigma(X_t) \subset \mathscr{F}_t.
\]
 For a stochastic process $(X_t)_{t \in I}$, the \textbf{natural filtration of $X$} is 
\[
\sigma(X_s: s \leq t).
\]
It is immediate that this is a filtration and that $X$ is adapted to it.


\section{Progressively measurable stochastic processes}
Let $\mathscr{F}_I=(\mathscr{F}_t)_{t \in I}$ be a filtration of $\mathscr{F}$.
A function $X:I \times \Omega \to E$ 
is called \textbf{progressively measurable with respect to the filtration $\mathscr{F}_I$} if for each $t \in I$,
the map
\[
(s,\omega) \mapsto X(s,\omega),\qquad [0,t] \times \Omega \to E,
\]
is measurable $\mathscr{B}_{[0,t]} \otimes \mathscr{F}_t \to \mathscr{E}$.
We denote by $\mathscr{M}^0(\mathscr{F}_I)$ the set of functions
$I \times \Omega \to E$ that are
progressively measurable with respect to the filtration $\mathscr{F}_I$.
We shall  speak about a stochastic process $(X_t)_{t \in I}$ being progressively measurable, by which we mean
that the map $(t,\omega) \mapsto X_t(\omega)$ is progressively measurable.


We denote by $\Prog(\mathscr{F}_I)$ the collection of those subsets $A$ of $I \times \Omega$ such that
for each $t \in I$,
\[
([0,t] \times \Omega) \cap A \in \mathscr{B}_{[0,t]} \otimes \mathscr{F}_t.
\] 
We prove in the following  that this is a $\sigma$-subalgebra of $\mathscr{B}_I \otimes \mathscr{F}$ and that
it is the coarsest $\sigma$-algebra with which all progressively measurable functions are measurable.


\begin{theorem}
Let $\mathscr{F}_I=(\mathscr{F}_t)_{t \in I}$ be a filtration of $\mathscr{F}$.
\begin{enumerate}
\item $\Prog(\mathscr{F}_I)$ is a $\sigma$-subalgebra of $\mathscr{B}_I \otimes \mathscr{F}$, and  is the
$\sigma$-algebra generated by the collection of functions $I \times \Omega \to E$ that 
are progressively measurable with respect to the filtration $\mathscr{F}_I$:
\[
\Prog(\mathscr{F}_I)=\sigma(\mathscr{M}^0(\mathscr{F}_I)).
\]
\item If $X:I \times \Omega \to E$ is progressively measurable with respect to the filtration
$\mathscr{F}_I$, then the stochastic process $(X_t)_{t \in I}$ is jointly measurable and is adapted to the filtration.
\end{enumerate}
\end{theorem}
\begin{proof}
If $A_1,A_2,\ldots \in \Prog(\mathscr{F}_I)$ and $t \in I$ then
\[
([0,t] \times \Omega) \cap \bigcup_{n \geq 1} A_n = \bigcup_{n \geq 1} (([0,t] \times \Omega) \cap A_n),
\]
which is a countable union of elements of the $\sigma$-algebra  $\mathscr{B}_{[0,t]} \otimes \mathscr{F}_t$ and hence
belongs to $\mathscr{B}_{[0,t]} \otimes \mathscr{F}_t$, showing that
$\bigcup_{n \geq 1} A_n \in \Prog(\mathscr{F}_I)$.
If $A_1,A_2 \in  \Prog(\mathscr{F}_I)$ and $t \in I$ then
\[
([0,t] \times \Omega) \cap (A_1 \cap A_2) = (([0,t] \times \Omega) \cap A_1) \cap (([0,t] \times \Omega) \cap A_2),
\]
which is an intersection of two elements of $\mathscr{B}_{[0,t]} \otimes \mathscr{F}_t$ and hence belongs to
$\mathscr{B}_{[0,t]} \otimes \mathscr{F}_t$, showing that
$A_1 \cap A_2 \in \Prog(\mathscr{F}_I)$.
Thus $\Prog((\mathscr{F}_t)_{t \in I})$ is a $\sigma$-algebra.


If $X:I \times \Omega \to E$ is progressively measurable, $B \in \mathscr{E}$, and $t \in I$, then
\[
([0,t] \times \Omega) \cap X^{-1}(B) = \{(s,\omega) \in [0,t] \times \Omega: X(s,\omega) \in B\}.
\]
Because $X$ is progressively measurable, this belongs to $\mathscr{B}_{[0,t]} \otimes \mathscr{F}_t$. This is true for all $t$, hence
$X^{-1}(B) \in \Prog(\mathscr{F}_I)$, which means that
$X$ is measurable $\Prog(\mathscr{F}_I) \to \mathscr{E}$. 

If $X:I \times \Omega \to E$ is measurable $\Prog(\mathscr{F}_I) \to \mathscr{E}$,
$t \in I$, and
$B \in \mathscr{E}$, then
because $X^{-1}(B) \in  \Prog(\mathscr{F}_I)$, we have
$([0,t] \times \Omega) \cap X^{-1}(B) \in \mathscr{B}_{[0,t]} \otimes \mathscr{F}_t$. This is true for all $B \in \mathscr{E}$,
which means that $(s,\omega) \mapsto X(s,\omega)$, $[0,t] \times \Omega \to E$, is measurable
$\mathscr{B}_{[0,t]} \otimes \mathscr{F}_t$, and because this is true for all $t$, $X$ is progressively measurable.
Therefore a function $I \times \Omega \to E$ is progressively measurable if and only if it is measurable
$\Prog(\mathscr{F}_I) \to \mathscr{E}$, which 
shows that $\Prog(\mathscr{F}_I)$  is the coarsest $\sigma$-algebra with which all progressively
measurable functions are measurable.

If $X:I \times \Omega \to E$ is a progressively measurable function and $B \in \mathscr{E}$, 
\[
X^{-1}(B) = \bigcup_{k \geq 1} (([0,k] \times \Omega) \cap X^{-1}(B)).
\]
Because $X$ is progressively measurable,
\[
([0,k] \times \Omega) \cap X^{-1}(B) \in \mathscr{B}_{[0,k]} \otimes
\mathscr{F}_k \subset \mathscr{B}_I \otimes \mathscr{F},
\]
thus $X^{-1}(B)$ is equal to a countable union of elements of $\mathscr{B}_I \otimes \mathscr{F}$ and so itself belongs to
$\mathscr{B}_I \otimes \mathscr{F}$. Therefore $X$ is measurable $\mathscr{B}_I \otimes \mathscr{F} \to \mathscr{E}$, namely $X$ is jointly
measurable.

Because $\Prog(\mathscr{F}_I)$ is the $\sigma$-algebra generated by the collection
of progressively measurable functions and each progressively measurable function is measurable
$\mathscr{B}_I \otimes \mathscr{F}$, 
\[
\Prog(\mathscr{F}_I) \subset \mathscr{B}_I \otimes \mathscr{F},
\]
and so $\Prog(\mathscr{F}_I)$ is indeed a $\sigma$-subalgebra of $\mathscr{B}_I \otimes \mathscr{F}$. 

Let $t \in I$. That $X$ is progressively measurable means that 
\[
(s,\omega) \mapsto X(s,\omega), \qquad [0,t] \times \Omega
\]
is measurable $\mathscr{B}_{[0,t]} \otimes \mathscr{F}_t \to \mathscr{E}$. This implies that for each $s \in [0,t]$ the map
$\omega \mapsto X(s,\omega)$ is measurable $\mathscr{F}_t \to \mathscr{E}$.\footnote{Charalambos D. Aliprantis and Kim C. Border, {\em Infinite Dimensional Analysis:
A Hitchhiker's Guide}, third ed., p.~152, Theorem 4.48.} (Generally, if a function is jointly measurable then it is separately measurable in each argument.)
In particular, $\omega \mapsto X(t,\omega)$ is measurable $\mathscr{F}_t \to \mathscr{E}$, which means that the stochastic process $(X_t)_{t \in I}$ is adapted
to the filtration, completing the proof.
\end{proof}


We now prove that if a stochastic process is adapted and left-continuous then it is progressively
measurable.\footnote{cf. Daniel W. Stroock, {\em Probability Theory: An Analytic View}, second ed.,
p.~267, Lemma 7.1.2.}

\begin{theorem}
Let $(\mathscr{F}_t)_{t \in I}$ be a filtration of $\mathscr{F}$. If $(X_t)_{t \in I}$ is a stochastic
process that is adapted to this filtration and all its paths are left-continuous, then $X$ is progressively measurable with respect to this filtration.
\end{theorem}
\begin{proof}
Write $X(t,\omega)=X_t(\omega)$. 
For $t \in I$,
let $Y$ be the restriction of $X$ to $[0,t] \times \Omega$. We wish to prove that $Y$ is measurable
$\mathscr{B}_{[0,t]} \otimes \mathscr{F}_t \to \mathscr{E}$. 
For $n \geq 1$, define
\[
Y_n(s,\omega) = \sum_{k=0}^{2^n-1} 1_{[kt2^{-n},(k+1)t2^{-n})}(s) Y(kt2^{-n},\omega)
+1_{\{t\}}(s) Y(t,\omega).
\]
Because $X$ is adapted to the filtration,
each $Y_n$ is measurable $\mathscr{B}_{[0,t]} \otimes \mathscr{F}_t \to \mathscr{E}$.
Because $X$ has left-continuous paths, for $(s,\omega) \in [0,t] \times \Omega$,
\[
Y_n(s,\omega) \to Y(s,\omega). 
\]
Since $Y$ is the pointwise limit of $Y_n$, it follows that $Y$ is measurable $\mathscr{B}_{[0,t]} \otimes \mathscr{F}_t \to \mathscr{E}$,
and so $X$ is progressively measurable.
\end{proof}


\section{Stopping times}
Let $\mathscr{F}_I =(\mathscr{F}_t)_{t \in I}$ be a filtration of $\mathscr{F}$. A function $T:\Omega \to [0,\infty]$ is called a \textbf{stopping time with
respect to the filtration $\mathscr{F}_I$} if
\[
\{T \leq t\} \in \mathscr{F}_t, \qquad t \in I.
\]
It is straightforward to prove that a stopping time is measurable $\mathscr{F} \to \mathscr{B}_{ [0,\infty]}$. 
Let
\[
\mathscr{F}_\infty = \sigma(\mathscr{F}_t: t \in I).
\]
We define
\[
\mathscr{F}_T = \{A \in \mathscr{F}_\infty: \textrm{if $t \in I$ then $A \cap \{T \leq t\} \in \mathscr{F}_t$}\}.
\]
It is straightforward to check that $T$ is measurable $\mathscr{F}_T \to \mathscr{B}_{[0,\infty]}$, and in particular
$\{T<\infty\}  \in \mathscr{F}_T$.

For a stochastic process $(X_t)_{t \in I}$ with state space $E$, we define $X_T:\Omega \to E$ by 
\[
X_T(\omega) = 1_{\{T<\infty\}}(\omega) X_{T(\omega)}(\omega). 
\]
We prove that if  $X$ is progressively measurable then $X_T$ is measurable
$\mathscr{F}_T \to \mathscr{E}$.\footnote{Sheng-wu He and Jia-gang Wang and Jia-An Yan,
{\em Semimartingale Theory and Stochastic Calculus}, p.~86, Theorem 3.12.}

\begin{theorem}
If $\mathscr{F}_I  =(\mathscr{F}_t)_{t \in I}$ is a filtration of $\mathscr{F}$,
$(X_t)_{t \in I}$ is a stochastic process that is progressively measurable with respect to $\mathscr{F}_I$,
and $T$ is a stopping time with respect to $\mathscr{F}_I$, then $X_T$ is measurable $\mathscr{F}_T \to \mathscr{E}$.
\label{XT}
\end{theorem}
\begin{proof}
For $t \in I$, using that $T$ is a stopping time we check that $\omega \mapsto T(\omega) \wedge t$ is measurable $\mathscr{F}_t
\to \mathscr{B}_{[0,t]}$, and then $\omega \mapsto (T(\omega) \wedge t,\omega)$,
$\Omega \to [0,t] \times \Omega$,
 is measurable $\mathscr{F}_t \to \mathscr{B}_{[0,t]} \otimes \mathscr{F}_t$.\footnote{Charalambos D. Aliprantis and Kim C. Border, {\em Infinite Dimensional Analysis:
A Hitchhiker's Guide}, third ed., p.~152, Lemma 4.49.}
Because $X$ is progressively
 measurable, $(s,\omega) \mapsto X_s(\omega)$ is measurable
 $\mathscr{B}_{[0,t]} \otimes \mathscr{F}_t \to \mathscr{E}$. Therefore the composition
 \[
 \omega \mapsto X_{T(\omega) \wedge t}(\omega), \qquad \Omega \to E,
 \]
 is measurable $\mathscr{F}_t \to \mathscr{E}$, and a fortiori it is measurable
 $\mathscr{F}_\infty \to \mathscr{E}$.
 We have
 \[
 X_T(\omega) = \lim_{n \to \infty} 1_{\{T \leq n\}}(\omega) X_{T(\omega) \wedge n}(\omega),
 \]
 and because 
 $\omega \mapsto 1_{\{T \leq n\}}(\omega) X_{T(\omega) \wedge n}(\omega)$ is measurable
 $\mathscr{F}_\infty \to \mathscr{E}$, 
 it follows that $\omega \mapsto  X_T(\omega)$  is measurable $\mathscr{F}_\infty \to \mathscr{E}$.
 For $B \in \mathscr{E}$,
 \[
\{X_T \in B\} \cap \{T \leq t\} = \{\omega \in \Omega: X_{T(\omega) \wedge t}(\omega) \in B\} \cap \{T \leq t\} \in \mathscr{F}_t,
 \]
therefore $\{X_T \in B\} \in \mathscr{F}_T$. This means that
$X_T$ is measurable $\mathscr{F}_T \to \mathscr{E}$.
\end{proof}


For a stochastic process $(X_t)_{t \in I}$, a filtration $\mathscr{F}_I =(\mathscr{F}_t)_{t \in I}$, and a stopping time
$T$ with respect to the filtration, we define 
\[
X^T_t(\omega) = X_{T(\omega) \wedge t}(\omega),
\]
and $(X^T_t)_{t \in I}$ is a stochastic process.
We prove that if $X$ is progressively measurable with respect to  $\mathscr{F}_I$ then
the stochastic proces $X^T$ is progressively measurable with respect to $\mathscr{F}_I$.\footnote{Ioannis Karatzas and Steven Shreve,
{\em Brownian Motion and Stochastic Calculus}, 
p.~9, Proposition 2.18.}

\begin{theorem}
If $(X_t)_{t \in I}$ is a stochastic process  that is progressively measurable with respect
to a filtration $\mathscr{F}_I = (\mathscr{F}_t)_{t \in I}$ and 
 $T$ is a stopping time with respect
to $\mathscr{F}_I$, then
$X^T$ is  progressively measurable with respect to $\mathscr{F}_I$.
\end{theorem}
\begin{proof}
Let $t \in I$. 
Because $T$ is a stopping time, for each $s \in [0,t]$ the map $\omega \mapsto T(\omega) \wedge s$ 
is measurable $\mathscr{F}_s \to \mathscr{B}_{[0,t]}$ and a fortiori is measurable $\mathscr{F}_t \to  \mathscr{B}_{[0,t]}$. 
Therefore  $(s,\omega) \mapsto T(\omega) \wedge s$
is measurable $\mathscr{B}_{[0,t]} \otimes \mathscr{F}_t \to \mathscr{B}_{[0,t]}$,\footnote{Charalambos D. Aliprantis and Kim C. Border, {\em Infinite Dimensional Analysis:
A Hitchhiker's Guide}, third ed., p.~152, Theorem 4.48.} and 
$(s,\omega) \mapsto \omega$ is measurable $\mathscr{B}_{[0,t]} \otimes \mathscr{F}_t \to \mathscr{F}_t$. 
This implies that
\[
(s,\omega) \mapsto (T(\omega) \wedge s,\omega), \qquad [0,t] \times \Omega \to [0,t] \times \Omega,
\]
is measurable 
$\mathscr{B}_{[0,t]} \otimes \mathscr{F}_t \to \mathscr{B}_{[0,t]} \otimes \mathscr{F}_t$.\footnote{Charalambos D. Aliprantis and Kim C. Border, {\em Infinite Dimensional Analysis:
A Hitchhiker's Guide}, third ed., p.~152, Lemma 4.49.}
Because $X$ is progressively measurable,
\[
(s,\omega) \mapsto X_s(\omega), \qquad [0,t] \times \Omega \to E,
\]
is measurable $\mathscr{B}_{[0,t]} \otimes \mathscr{F}_t \to \mathscr{E}$.  Therefore
the composition 
\[
(s,\omega) \mapsto X_{T(\omega) \wedge s}(\omega), \qquad [0,t] \times \Omega \to E,
\]
is measurable $\mathscr{B}_{[0,t]} \otimes \mathscr{F}_t  \to \mathscr{E}$, which shows that
$X^T$ is progressively measurable.
\end{proof}



\end{document}