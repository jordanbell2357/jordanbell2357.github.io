\documentclass{article}
\usepackage{amsmath,amssymb,mathrsfs,amsthm}
%\usepackage{tikz-cd}
%\usepackage{hyperref}
\newcommand{\inner}[2]{\left\langle #1, #2 \right\rangle}
\newcommand{\tr}{\ensuremath\mathrm{tr}\,} 
\newcommand{\Span}{\ensuremath\mathrm{span}} 
\def\Re{\ensuremath{\mathrm{Re}}\,}
\def\Im{\ensuremath{\mathrm{Im}}\,}
\newcommand{\id}{\ensuremath\mathrm{id}} 
\newcommand{\var}{\ensuremath\mathrm{var}} 
\newcommand{\Lip}{\ensuremath\mathrm{Lip}} 
\newcommand{\GL}{\ensuremath\mathrm{GL}}
\newcommand{\diam}{\ensuremath\mathrm{diam}} 
\newcommand{\sgn}{\ensuremath\mathrm{sgn}\,} 
\newcommand{\lcm}{\ensuremath\mathrm{lcm}} 
\newcommand{\supp}{\ensuremath\mathrm{supp}\,}
\newcommand{\dom}{\ensuremath\mathrm{dom}\,}
\newcommand{\upto}{\nearrow}
\newcommand{\downto}{\searrow}
\newcommand{\norm}[1]{\left\Vert #1 \right\Vert}
\theoremstyle{definition}
\newtheorem{theorem}{Theorem}
\newtheorem{lemma}[theorem]{Lemma}
\newtheorem{proposition}[theorem]{Proposition}
\newtheorem{corollary}[theorem]{Corollary}
\theoremstyle{definition}
\newtheorem{definition}[theorem]{Definition}
\newtheorem{example}[theorem]{Example}
\begin{document}
\title{Measure theory and Perron-Frobenius operators for continued fractions}
\author{Jordan Bell}
\date{April 18, 2016}

\maketitle


\section{The continued fraction transformation}
For $\xi \in \mathbb{R}$ let $[x]$ be the greatest integer $\leq \xi$, let $R(\xi)=\xi-[\xi]$, and let $\norm{\xi} = \min(R(\xi),1-R(\xi))$, the distance from $\xi$ to a nearest integer.
Let $I=[0,1]$ and define the \textbf{continued fraction transformation} $\tau:I \to I$ by
\[
\tau(x)=\begin{cases}
x^{-1}-[x^{-1}]&x \neq 0\\
0&x=0.
\end{cases}
\]
It is immediate that for $x \in I$, $x \in I \setminus \mathbb{Q}$ if and only if $\tau(x) \in I \setminus \mathbb{Q}$.
For $x \in \mathbb{R}$,
define $a_0(x) = [x]$, and for $n \geq 1$ define $a_n(x) \in \mathbb{Z}_{\geq 1} \cup \{\infty\}$ by
\[
a_n(x) = \left[ \frac{1}{\tau^{n-1}(x-a_0(x))} \right].
\]

For example, let $x=\frac{13}{71}$.
\[
\tau(x) = \frac{71}{13} - \left[\frac{71}{13} \right] = \frac{71}{13} - 5 = \frac{6}{13}.
\]
\[
\tau^2(x) = \frac{13}{6} - \left[\frac{13}{6}\right] = \frac{13}{6}-2= \frac{1}{6}.
\]
\[
\tau^3(x) = \frac{6}{1}-\left[\frac{6}{1}\right] = 0.
\]
Then $\tau^n(x)=0$ for $n \geq 3$. Thus, with $x=\frac{13}{71}$,
\[
a_0(x)=0,\quad a_1(x) =  \left[\frac{71}{13} \right] = 5.
\]
\[
a_2(x) = \left[\frac{1}{\tau(x)}\right] = \left[\frac{13}{6}\right]=2,\quad
a_3(x) = \left[\frac{1}{\tau^2(x)} \right] = \left[\frac{6}{1}\right]=6.
\]
\[
a_4(x)=\left[\frac{1}{\tau^3(x)}\right] = \infty, \qquad a_5(x) = \infty, \qquad \ldots.
\]



\section{Convergents}
For $x \in \Omega = I \setminus \mathbb{Q}$ write $a_n=a_n(x)$, and define
\[
q_{-1}=0,\quad p_{-1}=1,\quad q_0=1,\quad p_0=0,
\]
and for $n \geq 1$,
\[
q_n=a_n q_{n-1}+q_{n-2},\qquad p_n=a_np_{n-1}+p_{n-2}.
\]
Thus
\[
q_1 = a_1 q_0+q_{-1} = a_1,\qquad p_1 = a_1 p_0+p_{-1} = 1.
\]
One proves
\[
p_nq_{n-1}-p_{n-1}q_n=(-1)^{n+1},\qquad n \geq 0.
\]
Also,\footnote{Marius Iosifescu and Cor Kraaikamp,
{\em Metrical Theory of Continued Fractions}, p.~9, Proposition 1.1.1.}
\[
x = \frac{p_n+\tau^n(x) p_{n-1}}{q_n+\tau^n(x) q_{n-1}},\qquad x \in \Omega,\quad n \geq 0.
\]
From this,
\[
x - \frac{p_n}{q_n} = \frac{(-1)^n \tau^n(x)}{q_n(q_n+\tau^n(x) q_{n-1})}.
\]
Now,
\[
a_{n+1} + \tau^{n+1}(x) = \left[\frac{1}{\tau^n(x)} \right] + \frac{1}{\tau^n(x)} - \left[\frac{1}{\tau^n(x)} \right]
=\frac{1}{\tau^n(x)},
\]
and using this,
\begin{align*}
 \frac{\tau^n(x)}{q_n(q_n+\tau^n(x) q_{n-1})}&= \frac{1}{q_n(q_n \cdot (a_{n+1} + \tau^{n+1}(x))+q_{n-1})}\\
 &=\frac{1}{q_n(q_{n+1}+\tau^{n+1}(x) q_n)}.
\end{align*}
Thus
\[
\frac{1}{q_n(q_n+q_{n-1})} < \left| x - \frac{p_n}{q_n} \right| < \frac{1}{q_n q_{n+1}}.
\]

For $n \geq 1$ let 
\[
r_n(x) = \frac{1}{\tau^{n-1}(x)}=a_n+\tau^n(x)
\]
and
\[
s_n = \frac{q_{n-1}}{q_n},\qquad y_n=\frac{1}{s_n}
\]
and
\begin{align*}
u_n &= q_{n-1}^{-2} \left|x-\frac{p_{n-1}}{q_{n-1}}\right|^{-1}\\
&=\frac{1}{q_{n-1}^2} \cdot \frac{q_{n-1}(q_{n-1}+\tau^{n-1}(x) q_{n-2})}{\tau^{n-1}(x)}\\
&=\frac{q_{n-1}+\tau^{n-1}(x)q_{n-2}}{\tau^{n-1}(x) q_{n-1}}\\
&=\frac{q_{n-1} \cdot (a_n+\tau^n(x)) + q_{n-2}}{q_{n-1}}\\
&=a_n+\tau^n(x) + \frac{q_{n-2}}{q_{n-1}}.
\end{align*}
Let $s_0=0$.
It is worth noting that
\[
y_1 \cdots y_n = \frac{q_1}{q_0} \cdots \frac{q_n}{q_{n-1}} = \frac{q_n}{q_0} = q_n.
\]
\[
\frac{1}{s_n} = \frac{q_n}{q_{n-1}} = a_n + \frac{q_{n-2}}{q_{n-1}} = a_n + s_{n-1}.
\]
\[
u_n = a_n+\tau^n(x) + \frac{q_{n-2}}{q_{n-1}} = r_n + s_{n-1}.
\]



\section{Measure theory}
Suppose that $(X,\mathscr{A})$ is a measurable space and $\mu,\nu$ are probability measures on $\mathscr{A}$.
Let
$\mathscr{D} = \{A \in \mathscr{A}: \mu(A)=\nu(A)\}$. 
First, $X \in \mathscr{D}$.  Second, if $A,B \in \mathscr{D}$ and $A \subset B$ then
\[
\mu(B \setminus A) = \mu(B)-\mu(A) = \nu(B)-\nu(A) = \nu(B \setminus A),
\]
so $B \setminus A \in \mathscr{D}$. Third,
suppose that $A_n \in \mathscr{D}$, $n \geq 1$, and $A_n \uparrow A$. Because $\mathscr{A}$ is a $\sigma$-algebra, $A \in \mathscr{A}$,
and then, setting $A_0 = \emptyset$,
\[
\mu(A) = \mu\left( \bigcup_{n \geq 1} (A_n \setminus A_{n-1}) \right) = \sum_{n \geq 1} (\mu(A_n)-\mu(A_{n-1})),
\]
whence $\mu(A)=\nu(A)$. Therefore $\mathscr{D}$ is a Dynkin system. 
\textbf{Dynkin's theorem} says that if $\mathscr{D}$ is a Dynkin system and
$\mathscr{C} \subset \mathscr{D}$ where $\mathscr{C}$ is a $\pi$-system (nonempty and closed under finite intersections),
then $\sigma(\mathscr{C}) \subset \mathscr{D}$.\footnote{Charalambos D. Aliprantis and Kim C. Border,
{\em Infinite Dimensional Analysis: A Hitchhiker's Guide},
third ed., p.~136, Lemma 4.11.}


Suppose now that $\sigma(\mathscr{C})=\mathscr{A}$, that
$\mathscr{C}$ is closed under finite intersections,
and that $\mu(A)=\nu(A)$ for all $A \in \mathscr{C}$.
Then $\mathscr{C} \subset \mathscr{D}$, so by Dynkin's theorem,
$\mathscr{A} =\sigma(\mathscr{C}) \subset \mathscr{D}$, hence
$\mathscr{D}=\mathscr{A}$. That is, for any $A \in \mathscr{A}$, $\mu(A)=\nu(A)$, meaning $\mu=\nu$. 

We shall apply the above  with $(I,\mathscr{B}_I)$, $I=[0,1]$. For
\[
\mathscr{C} = \{(0,u]: 0<u \leq 1\},
\]
it is a fact that
$\sigma(\mathscr{C})=\mathscr{B}_I$. Therefore if $\mu$ and $\nu$ are probability measures on
$\mathscr{B}_I$ such that $\mu((0,u])=\nu((0,u])$ for every $0<u \leq 1$, then
$\mu=\nu$. 


Let $\lambda$ be Lebesgue measure on $I=[0,1]$. Define
\[
d\gamma(x) = \frac{1}{(1+x)\log 2} d\lambda(x),
\]
called the \textbf{Gauss measure}. 
If $\mu$ is a Borel probability measure on $I$,
for measurable $T:I \to I$ and
for $A \in \mathscr{B}_I$ let
\[
T_* \mu (A) = \mu(T^{-1}(A)).
\]
$T_*\mu$, called the \textbf{pushforward of $\mu$ by $T$}, is itself a Borel probability measure on $I$. 
We prove that $\gamma$ is an invariant measure for $\tau$.\footnote{Marius Iosifescu and Cor Kraaikamp,
{\em Metrical Theory of Continued Fractions}, p.~17, Theorem 1.2.1;
Manfred Einsiedler and Thomas Ward, {\em Ergodic Theory with a view towards Number Theory}, 
p.~77, Lemma 3.5.}


\begin{theorem}
$\tau_*\gamma=\gamma$.
\end{theorem}
\begin{proof}
Let $0<u \leq 1$. For $x \in I$, $0<\tau(x) \leq u$ if and only if
$0<\frac{1}{x}-\left[\frac{1}{x}\right] \leq u$ if and only if
$\left[\frac{1}{x}\right] < \frac{1}{x} \leq u + \left[\frac{1}{x}\right]$ if
and only if
$\frac{1}{u + \left[\frac{1}{x}\right]} \leq x < \frac{1}{\left[\frac{1}{x}\right]}$. 
Then, as $0 \not \in \tau^{-1}((0,u])$, 
\[
\tau^{-1}((0,u]) = \bigcup_{i \geq 1} \left[ \frac{1}{u+i},\frac{1}{i}\right).
\]
We calculate
\begin{align*}
\gamma(\tau^{-1}((0,u]))&=\sum_{i \geq 1} \gamma\left( \left[ \frac{1}{u+i},\frac{1}{i}\right)\right)\\
&=\sum_{i \geq 1} \int_{\left[ \frac{1}{u+i},\frac{1}{i}\right)} \frac{1}{(1+x) \log 2} d\lambda(x)\\
&= \frac{1}{\log 2}  \sum_{i \geq 1}\left(\log\left(1+\frac{1}{i}\right)-\log\left(1+\frac{1}{u+i}\right)\right).
\end{align*}
Using
\[
\frac{1+\frac{1}{i}}{1+\frac{1}{u+i}} = \frac{1+\frac{u}{i}}{1+\frac{u}{i+1}},
\]
this is
\begin{align*}
\gamma(\tau^{-1}((0,u]))&= \frac{1}{\log 2}  \sum_{i \geq 1} \left( \log\left(1+\frac{u}{i}\right)-\log\left(1+\frac{u}{i+1}\right) \right)\\
&=\frac{1}{\log 2}  \sum_{i \geq 1} \int_{\frac{u}{i+1}}^{\frac{u}{i}} \frac{1}{1+x} d\lambda(x)\\
&=\gamma((0,u]).
\end{align*}
Because $\gamma(\tau^{-1}((0,u]))=\gamma((0,u])$ for every $0<u \leq 1$, it follows that $\tau_*\gamma=\gamma$.
\end{proof}


We remark that for a set $X$, $X^0$ is a singleton.
For $i \in \mathbb{Z}_{\geq 1}^0$ let $I_0(i)=\Omega$. 
For $n \geq 1$ and $i \in \mathbb{Z}_{\geq 1}^n$, 
let 
\[
I_n(i) = \{\omega \in \Omega: a_k(x) = i_k, 1 \leq k \leq n\}.
\]
For $n \geq 1$ and for $i \in \mathbb{Z}_{\geq 1}^n$, define
\[
[i_1,\ldots,i_n]=\cfrac{1}{i_1+\cfrac{1}{\cdots+\cfrac{1}{i_{n-1}+\cfrac{1}{i_n}}}}.
\]
For $x \in I_n(i)$, 
\[
\frac{p_n(x)}{q_n(x)} = [i_1,\ldots,i_n],
\qquad \frac{p_{n-1}(x)}{q_{n-1}(x)} = [i_1,\ldots,i_{n-1}].
\]


The following is an expression for the sets $I_n(i)$.\footnote{Marius Iosifescu and Cor Kraaikamp,
{\em Metrical Theory of Continued Fractions}, p.~18, Theorem 1.2.2.}

\begin{theorem}
Let $n \geq 1$, $i \in \mathbb{Z}_{\geq 1}^n$,
and define
\[
u_n(i) = \begin{cases}
\frac{p_n+p_{n-1}}{q_n+q_{n-1}}&\textrm{$n$ odd}\\
\frac{p_n}{q_n}&\textrm{$n$ even}
\end{cases}
\]
and
\[
v_n(i) = \begin{cases}
\frac{p_n}{q_n}&\textrm{$n$ odd}\\
\frac{p_n+p_{n-1}}{q_n+q_{n-1}}&\textrm{$n$ even}.
\end{cases}
\]
Then
\[
I_n(i) = \Omega \cap (u_n(i),v_n(i)).
\]
\label{cylinder}
\end{theorem}

From the above, if $n$ is odd and $i \in \mathbb{Z}_{\geq 1}$ then 
\begin{align*}
\lambda(I_n(i)) &= v_n(i)-u_n(i)\\
&= \frac{p_n}{q_n}-\frac{p_n+p_{n-1}}{q_n+q_{n-1}}\\
&=\frac{p_nq_{n-1}-p_{n-1}q_n}{q_n(q_n+q_{n-1})}\\
&=\frac{(-1)^{n+1}}{q_n(q_n+q_{n-1})}\\
&=\frac{1}{q_n(q_n+q_{n-1})},
\end{align*}
and if $n$ is even then  likewise
\[
\lambda(I_n(i))=\frac{1}{q_n(q_n+q_{n-1})}.
\]


Kraaikamp and Iosifescu attribute the following to Torsten Brod\'en, in a 1900 paper.\footnote{Marius Iosifescu and Cor Kraaikamp,
{\em Metrical Theory of Continued Fractions}, p.~21, Corollary 1.2.6.}

\begin{theorem}
For $n \geq 1$, $i \in \mathbb{N}^n$, $x \in I$,
\[
\lambda(\tau^n<x| i) = \frac{x(s_n+1)}{s_nx+1}.
\]
\label{broden}
\end{theorem}
\begin{proof}
We have
\[
\lambda(\tau^n<x| i)=\frac{\lambda((\tau^n<x) \cap I_n(i))}{\lambda(I_n(i))}.
\]
Using
\[
\omega = \frac{p_n+\tau^n(\omega) p_{n-1}}{q_n+\tau^n(\omega) q_{n-1}},\qquad \omega \in \Omega,\quad n \geq 0,
\]
if $n$ is odd then
\begin{align*}
(\tau^n<x) \cap I_n(i) &= \left\{\omega \in \Omega: \frac{p_n+p_{n-1}}{q_n+q_{n-1}}<\omega<\frac{p_n}{q_n},
\tau^n(\omega)<x\right\}\\
&=\left\{ \omega \in \Omega: \frac{p_n+xp_{n-1}}{q_n+xq_{n-1}}<\omega<\frac{p_n}{q_n} \right\}
\end{align*}
and if $n$ is even then
\[
(\tau^n<x) \cap I_n(i) =\left\{ \omega \in \Omega: \frac{p_n}{q_n}<\omega< \frac{p_n+xp_{n-1}}{q_n+xq_{n-1}} \right\}.
\]
Therefore if $n$ is odd,
\begin{align*}
\lambda((\tau^n<x) \cap I_n(i) )&=\frac{p_n}{q_n}-\frac{p_n+xp_{n-1}}{q_n+xq_{n-1}}\\
&=\frac{xp_nq_{n-1}-xp_{n-1}q_n}{q_n(q_n+xq_{n-1})}\\
&=\frac{x}{q_n(q_n+xq_{n-1})}
\end{align*}
and likewise if $n$ is even then
\[
\lambda((\tau^n<x) \cap I_n(i) ) = \frac{x}{q_n(q_n+xq_{n-1})}.
\]
Therefore for $n \geq 1$,
\begin{align*}
\lambda(\tau^n<x| i)&=\frac{x}{q_n(q_n+xq_{n-1})} \cdot q_n(q_n+q_{n-1})\\
&=\frac{x(q_n+q_{n-1})}{q_n+xq_{n-1}}.
\end{align*}
Using $s_n+1 = \frac{q_n+q_{n-1}}{q_n}$ and $s_n x+1=
\frac{xq_{n-1}+q_n}{q_n}$,
\begin{align*}
\lambda(\tau^n<x| i)&=\frac{x q_n(s_n+1)}{q_n(s_nx+1)}\\
&=\frac{x(s_n+1)}{s_nx+1}.
\end{align*}
\end{proof}

For $j \geq 1$ and $s \in I$ define
\[
P_j(s)=\frac{s+1}{(s+j)(s+j+1)}.
\]
We now apply Theorem \ref{broden} to prove the following.\footnote{Marius Iosifescu and Cor Kraaikamp,
{\em Metrical Theory of Continued Fractions}, p.~22, Proposition 1.2.7.}

\begin{theorem}
For $j \geq 1$,
\[
\lambda(a_1=j) = \frac{1}{j(j+1)}.
\]
For $n \geq 1$ and $i \in \mathbb{N}^n$,
\[
\lambda(a_{n+1}=j | i) = P_j(s_n).
\]
\end{theorem}
\begin{proof}
By Theorem \ref{cylinder},
\[
\{\omega \in \Omega: a_1(\omega)=j\} = I_1(j) = \Omega \cap (u_1(j), v_1(j)).
\]
In this case, $q_1=j$, so $u_1(j) = \frac{p_1+p_0}{q_1+q_0}=\frac{1+0}{j+1}
=\frac{1}{j+1}$ and
$v_1(j) = \frac{p_1}{q_1} = \frac{1}{j}$, so 
\[
\{\omega \in \Omega: a_1(\omega)=j\} = \Omega \cap \left(\frac{1}{j+1}, \frac{1}{j} \right).
\]

Now,
\[
a_{n+1}(\omega) = \left[ \frac{1}{\tau^n(\omega)} \right]
=a_1(\tau^n(\omega)).
\]
Thus
\[
\{\omega \in \Omega: a_{n+1}(\omega)=j\}
=\left\{\omega \in \Omega: \tau^n(\omega) \in \left(\frac{1}{j+1},\frac{1}{j}\right)\right\}.
\]
Then using Theorem \ref{broden},
\begin{align*}
\lambda(a_{n+1}=j|i)& = \lambda\left( \tau^n < \frac{1}{j} \big| i \right)-\lambda\left( \tau^n < \frac{1}{j+1} \big| i \right)\\
&=\frac{\frac{1}{j}(s_n+1)}{s_n \frac{1}{j} + 1}-\frac{\frac{1}{j+1}(s_n+1)}{s_n \frac{1}{j+1} + 1}\\
&=\frac{s_n+1}{(s_n+1)(s_n+j+1)}.
\end{align*}
\end{proof}



\section{Perron-Frobenius operators}
For a probability measure $\mu$ on $\mathscr{B}_I$ and for $f \in L^1(\mu)$ let
$d\mu_f = f d\mu$. 
If $\tau_*\mu$ is
absolutely continuous with respect to $\mu$, check that
$\tau_* \mu_f$ is itself absolutely continuous with respect to $\mu$. Then
applying the Radon-Nikodym theorem, let
\[
P_\mu f = \frac{d(\tau_* \mu_f)}{d\mu}.
\]
For $g \in L^\infty(\mu)$,
\[
\int_I g \cdot P_\mu f d\mu=\int_I g d(\tau_* \mu_f)
=\int_I g \circ \tau d\mu_f
=\int_I (g \circ \tau) \cdot f d\mu.
\]
In particular, for $g = 1_A$, $A \in \mathscr{B}_I$,
\[
\int_I 1_A \cdot P_\mu f d\mu = \int_I 1_{\tau^{-1}(A)} \cdot f d\mu.
\]


For $g \in L^\infty(\mu)$,
\[
\int_I g \cdot P_\gamma 1 d\gamma = \int_I g \circ \tau d\gamma
=\int_I g d(\tau_* \gamma),
\]
hence $P_\gamma 1 = 1$ if and only if $\tau_* \gamma$. 

We shall be especially interested in
\[
U=P_\gamma,
\]
where $\gamma$ is the Gauss measure on $I$. We establish almost everywhere an expression for $Uf(x)$.\footnote{Marius Iosifescu and Cor Kraaikamp,
{\em Metrical Theory of Continued Fractions}, p.~59, Proposition 2.1.2.}

\begin{theorem}
For $f \in L^1(\gamma)$, for $\gamma$-almost all $x \in I$,
\[
Uf(x) = \sum_{i \geq 1} P_i(x) f \left( \frac{1}{x+i} \right).
\]
\end{theorem}
\begin{proof}
Let $I_i = \left(\frac{1}{i+1},\frac{1}{i}\right]$ and let $\tau_i$ be the restriction of $\tau:I \to I$ to $I_i$.
For $u \in I_i$, $i \leq \frac{1}{u} < i+1$, hence $\tau_i(u)=\tau(u) = \frac{1}{u}-i$,
i.e. $u=\frac{1}{\tau_i(u)+i}$, i.e. $\tau_i^{-1}(x) = \frac{1}{x+i}$.


For $A \in \mathscr{B}_I$, if $0 \not \in A$ then 
\[
\tau^{-1}(A) = \tau^{-1}\left(\bigcup_{i \geq 1} (A \cap I_i) \right)
=\bigcup_{i \geq 1} \tau^{-1}(A \cap I_i),
\]
and the sets $\tau^{-1}(A \cap I_i)$ are pairwise disjoint, hence
\[
\int_{\tau^{-1}(A)} f d\gamma=\sum_{i \geq 1} \int_{\tau^{-1}(A \cap I_i)}  f d\gamma
=\sum_{i \geq 1} \int_{\tau_i^{-1}(A)}  f d\gamma.
\]
Applying the change of variables formula, as $\frac{d}{dx} \tau_i^{-1}(x) = -(x+i)^{-2}$,
\begin{align*}
\int_{\tau_i^{-1}(A)}  f d\gamma&=\frac{1}{\log 2} \int_{\tau_i^{-1}(A)} \frac{f(u)}{u+1} d\lambda(u)\\
&=\frac{1}{\log 2} \int_A \frac{f \circ \tau_i^{-1}(x)}{\tau_i^{-1}(x)+1} \cdot (x+i)^{-2} d\lambda(x)\\
&=\frac{1}{\log 2} \int_A f\left(\frac{1}{x+i}\right) \cdot \frac{1}{(x+i+1)(x+i)} d\lambda(x)\\
&=\frac{1}{\log 2} \int_A  f\left(\frac{1}{x+i}\right) \cdot P_i(x) \cdot \frac{1}{x+1} d\lambda(x)\\
&=\int_A    f\left(\frac{1}{x+i}\right) \cdot P_i(x) d\gamma(x).
\end{align*}
Therefore
\begin{align*}
\int_{\tau^{-1}(A)} f d\gamma&=\sum_{i \geq 1} \int_A    f\left(\frac{1}{x+i}\right) \cdot P_i(x) d\gamma(x)\\
&= \int_A \sum_{i \geq 1}  f\left(\frac{1}{x+i}\right) \cdot P_i(x) d\gamma(x).
\end{align*}
Then
\[
\int_A P_\gamma f d\gamma =  \int_A \sum_{i \geq 1}  f\left(\frac{1}{x+i}\right) \cdot P_i(x) d\gamma(x).
\]
Because this is true for any $A \in \mathscr{B}_I$ with $0 \not \in A$, it follows that
for $\gamma$-almost all $x \in I$,
\[
P_\gamma f(x) =  \sum_{i \geq 1}  f\left(\frac{1}{x+i}\right) \cdot P_i(x).
\]
\end{proof}


The following gives an expression for $P_\mu f(x)$ under some hypotheses.\footnote{Marius Iosifescu and Cor Kraaikamp,
{\em Metrical Theory of Continued Fractions}, p.~60, Proposition 2.1.3.}

\begin{theorem}
Let $\mu$ be a probability measure on $\mathscr{B}_I$ that is absolutely continuous with respect
to $\lambda$ and suppose that $d\mu = h d\lambda$ with $h(x)>0$ for $\mu$-almost all $x \in I$. Let
$f \in L^1(\mu)$ and define $g(x)=(x+1)h(x)f(x)$. 
For $\mu$-almost all $x \in I$,
\[
P_\mu f(x) = \frac{1}{h(x)} \sum_{i \geq 1} \frac{h((x+i)^{-1})}{(x+i)^2} f\left(\frac{1}{x+i}\right)
=\frac{Ug(x)}{(x+1)h(x)}.
\]
For $n \geq 1$, for $\mu$-almost all $x \in I$,
\[
P_\mu^n f(x) = \frac{U^n g(x)}{(x+1)h(x)}.
\]
\label{Pmu}
\end{theorem}



We prove an expression for $\mu(\tau^{-n}(A))$.\footnote{Marius Iosifescu and Cor Kraaikamp,
{\em Metrical Theory of Continued Fractions}, p.~61, Proposition 2.1.5.}

\begin{theorem}
Let $\mu$ be a probability measure on $\mathscr{B}_I$ that is absolutely continuous with respect
to $\lambda$. Let $h = \frac{d\mu}{d\lambda}$ and let $f(x)=(x+1)h(x)$. For $A \in \mathscr{B}_I$
and $n \geq 1$,
\[
\mu(\tau^{-n}(A)) = \int_A \frac{U^n f(x)}{x+1} d\lambda(x).
\]
\end{theorem}
\begin{proof}
For $n=0$, 
\[
\mu(A) =  \int_A d\mu = 
 \int_A h d\lambda =
\int_A \frac{f(x)}{x+1} d\lambda(x)=\int_A \frac{U^0 f(x)}{x+1} d\lambda(x).
\]
Suppose by hypothesis that the claim is true for some $n \geq 0$.
Then 
\begin{align*}
\mu(\tau^{-n-1}(A))&=\mu(\tau^{-n}(\tau^{-1}(A)))\\
&=\int_{\tau^{-1}(A)} \frac{U^n f(x)}{x+1} d\lambda(x)\\
&=\log 2\cdot \int_{\tau^{-1}(A)} U^n f(x) d\gamma(x)\\
&=\log 2\cdot \int_A U^{n+1}f(x) d\gamma(x)\\
&=\log 2\cdot \int_A \frac{U^{n+1} f(x)}{x+1} d\lambda(x).
\end{align*}
\end{proof}


For $f(x)=\frac{1}{x+1}$ and $A \in \mathscr{B}_I$,
\begin{align*}
\int_A P_\lambda f d\lambda&=\int_{\tau^{-1}(A)} \frac{1}{x+1} d\lambda(x)\\
&=\log 2 \cdot \int_{\tau^{-1}(A)} d\gamma\\
&=\log 2 \cdot \int_A d\gamma\\
&= \int_A f d\lambda.
\end{align*}
Because this is true for all Borel sets $A$,
\[
P_\lambda \frac{1}{x+1} = \frac{1}{x+1}.
\]

For $f \in L^1(\lambda)$ and $x \in I$, let
\[
\Pi_1 f(x) = \frac{1}{(x+1) \log 2} \int_I f d\lambda.
\]
Define
\[
T_0 = P_\lambda - \Pi_1.
\]
For $n \geq 1$, $\Pi_1^n = \Pi_1$. For $f \in L^1(\lambda)$,
\[
P_\lambda \Pi_1 f =  \frac{1}{\log 2} \int_I f d\lambda\cdot P_\lambda \frac{1}{x+1}
= \frac{1}{\log 2} \int_I f d\lambda\cdot\frac{1}{x+1}
=\Pi_1 f(x)
\]
and
\[
\Pi_1 P_\lambda f=\frac{1}{(x+1)\log 2} \int_I P_\lambda f d\lambda
=\frac{1}{(x+1)\log 2} \int_I f d\lambda=\Pi_1 f(x),
\]
hence
\[
P_\lambda \Pi_1 = \Pi_1 = \Pi_1 P_\lambda.
\]
Moreover,
\[
T_0 \Pi_1 = (P_\lambda-\Pi_1) \Pi_1 = P_\lambda \Pi_1 - \Pi_1^2
=0
\]
and
\[
\Pi_1 T_0 = \Pi_1(P_\lambda-\Pi_1) = \Pi_1 P_\lambda - \Pi_1^2
=0.
\]
Because $P_\lambda=\Pi_1+T_0$, using $\Pi_1^2=\Pi_1$,
$T_0 \Pi_1 = 0$, and
$\Pi_1 T_0 = 0$, we have
\[
P_\lambda^n = \Pi_1+T_0^n,\qquad n \geq 1.
\]

Theorem \ref{Pmu} tells us that for $f \in L^1(\lambda)$, 
for $\lambda$-almost all $x \in I$,
\[
P_\lambda f(x) = \sum_{i \geq 1} \frac{1}{(x+i)^2} f\left(\frac{1}{x+i}\right).
\]
With $h(x)=x+1$ and $g=hf$, for $n \geq 1$, for $\lambda$-almost all $x \in I$,
\[
P_\lambda^n f(x) = \frac{U^n g(x)}{x+1}.
\]
Thus
\begin{align*}
U^n g &= h P_\lambda^n f\\
&= h \Pi_1 f + h T_0^n f\\
&= \frac{1}{\log 2} \int_I f d\lambda + h T_0^n f\\
&=\int_I g d\gamma + h T_0^n(g/h).
\end{align*}

Define $I_\gamma:L^1(\gamma) \to L^1(\gamma)$ by 
\[
I_\gamma f = 1 \cdot \int_I f d\gamma.
\]
We have
\[
I_\gamma U f = \int_I P_\gamma f d\gamma = \int_I f d\gamma = I_\gamma f,
\]
meaning $I_\gamma U = I_\gamma$. Furthermore, because
$\tau_* \gamma = \gamma$ we have $P_\gamma 1 = 1$, so
\[
U I_\gamma f =  \int_I f d\gamma \cdot U 1 = \int_I f d\gamma \cdot 1 = I_\gamma f ,
\]
meaning $U I_\gamma = I_\gamma$. 

Let $h(x)=x+1$. $h, \frac{1}{h} \in L^\infty(\gamma)$.
Now define $T:L^1(\gamma) \to L^1(\gamma)$ by
\[
T g = h \cdot T_0(g/h),
\]
which makes sense because $\frac{1}{h} \in L^\infty(\gamma)$. 
Then
\begin{align*}
T^2 g& = T(h \cdot T_0(g/h))\\
&= h \cdot T_0\left(\frac{h \cdot T_0(g/h)}{h}\right)\\
&=h \cdot T_0^2(g/h).
\end{align*}
For $n \geq 1$,
\[
T^n g = h \cdot T_0^n(g/h).
\]
Recapitulating the above, for $n \geq 1$ and $g \in L^1(\gamma)$, 
\[
U^n g = I_\gamma g + h T_0^n(g/h) = I_\gamma g + T^n g,
\]
meaning
\[
U^n = I_\gamma + T^n,\qquad n \geq 1.
\]


It is a fact that $T^n$ converges to $0$ in the strong operator topology on $\mathscr{L}(L^1(\gamma))$, the bounded linear operators
$L^1(\gamma) \to L^1(\gamma)$, that is,
for each $f \in L^1(\gamma)$, $T^n f \to 0$ in $L^1(\gamma)$, i.e.
$\norm{T^n f}_{L^1} \to 0$.\footnote{Marius Iosifescu and Cor Kraaikamp,
{\em Metrical Theory of Continued Fractions}, p.~63, Proposition 2.1.7.}
Then $U^n \to I_\gamma$ in the strong operator topology: for $f \in L^1(\gamma)$,
\[
\int_I \left| U^n f(x) - \int_I f d\gamma \right| d\lambda \to 0.
\]
Iosifescu and Kraaikamp state that has not been determined whether for $\gamma$-almost all $x \in I$,
$U_n f (x) \to I_\gamma f$. 


Let $B(I)$ be the set of bounded Borel measurable functions $f:I \to \mathbb{C}$ and write $\norm{f}_\infty = \sup_{x \in I} |f(x)|$.
For $f \in B(I)$, define for $x \in I$,
\[
U f(x) = \sum_{i \geq 1} P_i(x) f\left(\frac{1}{x+i}\right)
= \sum_{i \geq 1} \frac{x+1}{(x+i)(x+i+1)} f\left(\frac{1}{x+i}\right).
\]
$1 \in B(I)$, and
for $x \in I$,
\[
\sum_{1 \leq i \leq m}  \frac{x+1}{(x+i)(x+i+1)} = \frac{m}{m+x+1},
\]
hence
\[
U1(x) = \sum_{i \geq 1}  \frac{x+1}{(x+i)(x+i+1)} = 1.
\]
For $f \in B(I)$ and $x \in I$,
\[
|Uf(x)| \leq \norm{f}_\infty \cdot U1(x),
\]
hence
\[
\norm{U}_{B(I) \to B(I)} = 1.
\]

Say that $f:I \to \mathbb{R}$ is increasing if $x \leq y$ implies $f(x) \leq f(y)$. 
An increasing function $f:I \to \mathbb{R}$ belongs to $B(I)$. We prove that 
if $f$ is increasing then $Uf$ is decreasing.\footnote{Marius Iosifescu and Cor Kraaikamp,
{\em Metrical Theory of Continued Fractions}, p.~65, Proposition 2.1.11.}

\begin{theorem}
If $f:I \to \mathbb{R}$ is increasing then $Uf$ is decreasing.
\end{theorem}
\begin{proof}
Take $x<y$ and let
\[
S_1 = \sum_{i \geq 1} P_i(y) \left( f\left( \frac{1}{y+i}\right)-f\left(\frac{1}{x+i}\right)\right)
\]
and 
\[
S_2 = \sum_{i \geq 1} (P_i(y)-P_i(x)) f\left(\frac{1}{x+i}\right).
\]
Then
\begin{align*}
Uf(y)-Uf(x) &= \sum_{i \geq 1} \left( P_i(y) f \left(\frac{1}{y+i}\right) - P_i(x) f\left(\frac{1}{x+i}\right)\right)\\
&=S_1+S_2.
\end{align*}
Because $f$ is increasing, $S_1 \leq 0$. 
Using $\sum_{i \geq 1} P_i(u) = 1$ for any $u \in I$, 
\[
\sum_{i \geq 1}(P_i(y)-P_i(x)) f \left(\frac{1}{x+1}\right)  
=0,
\]
and therefore
\begin{align*}
S_2 &= \sum_{i \geq 1} \left(f \left(\frac{1}{x+i} \right) - f \left(\frac{1}{x+1}\right)  \right)(P_i(y)-P_i(x))\\
&=  \left(f \left(\frac{1}{x+2} \right) - f \left(\frac{1}{x+1}\right)  \right)(P_2(y)-P_2(x))\\
&+\sum_{i \geq 3} \left(f \left(\frac{1}{x+i} \right) - f \left(\frac{1}{x+1}\right)  \right)(P_i(y)-P_i(x)).
\end{align*}


For $i \geq 2$, using that $f$ is increasing,
\[
f \left(\frac{1}{x+i} \right) - f \left(\frac{1}{x+1}\right)  \leq  f \left(\frac{1}{x+2} \right) -  f \left(\frac{1}{x+1}\right) \leq 0.
\]
We calculate
\[
P_i'(u) = - \frac{-i^2+i+(u+1)^2}{(u+i)^2(u+i+1)^2}.
\]
The roots of the above rational function are $u=-\sqrt{(i-1)i}-1,\sqrt{(i-1)i}-1$. Thus, $P_i'(u)=0$ if and only if
$u=\sqrt{(i-1)i}-1$. But $\sqrt{(i-1)i}-1 \in I$ if and only if  $i^2-i-1 \geq 0$ and $i^2-i-4 \leq 0$. This is possible if and only
if $i=2$. And
\[
P_i'(0) = \frac{i^2-i-1}{i^2(i+1)^2},
\]
so $P_1'(u) \leq 0$ for all $u \in I$ and for $i \geq 3$, $P_i'(u) \geq 0$ for all $u \in I$. 
For $i=2$, check that if $0 \leq u \leq \sqrt{2}-1$ then
$P_2'(u) \geq 0$ and if $\sqrt{2}-1 \leq u \leq 1$ then $P_2'(u) \leq 0$. 
Then
\begin{align*}
S_2&\leq  \left(f \left(\frac{1}{x+2} \right) - f \left(\frac{1}{x+1}\right)  \right)(P_2(y)-P_2(x))\\
&+\sum_{i \geq 3} \left(f \left(\frac{1}{x+2} \right) - f \left(\frac{1}{x+1}\right)  \right)(P_i(y)-P_i(x))\\
&= \left(f \left(\frac{1}{x+2} \right) - f \left(\frac{1}{x+1}\right)  \right)(P_2(y)-P_2(x))\\
&+\left(f \left(\frac{1}{x+2} \right) - f \left(\frac{1}{x+1}\right)  \right)(-P_1(y)-P_2(y) - (-P_1(x)-P_2(x)))\\
&= \left(f \left(\frac{1}{x+2} \right) - f \left(\frac{1}{x+1}\right)\right)(P_1(x)-P_1(y))\\
&\leq 0.
\end{align*}
We have shown that $S_1 \leq 0$ and $S_2 \leq 0$, so
\[
Uf(y)-Uf(x) = S_1+S_2 \leq 0,
\]
which means that $Uf:I \to \mathbb{R}$ is decreasing.
\end{proof}


For $J=[a,b] \subset I$, 
a \textbf{partition} of $J$ is a sequence $P=(t_0,\ldots,t_n)$ such that $a=t_0<\cdots<t_n=b$. 
For $f:I \to \mathbb{R}$ define
\[
V(f,P) = \sum_{1 \leq i \leq n} |f(t_i)-f(t_{i-1})|.
\]
Define
\[
V_J f = \sup\{V(f,P): \textrm{$P$ is a partition of $J$}\}.
\]
Let $v_f(x) = V_{[0,x]} f$, the \textbf{variation of $f$}. $v_f(1)=V_{[0,1]} f$. 
We say that $f$ has \textbf{bounded variation} if $v_f(1)< \infty$, and denote by $BV(I)$ the set of
functions $f:I \to \mathbb{R}$ with bounded variation. It is a fact that
with the norm
\[
\norm{f}_{BV} = |f(0)| + V_I f,
\]
$BV(I)$ is a Banach algebra.

If $f$ is increasing then $V_I f = f(1)-f(0)$. We will use the following to prove the theorem coming after it.\footnote{Marius Iosifescu and Cor Kraaikamp,
{\em Metrical Theory of Continued Fractions}, p.~66, Proposition 2.1.12.}

\begin{lemma}
If $f:I \to \mathbb{R}$ is increasing then 
\[
V_I (Uf) \leq \frac{1}{2} V_I f.
\]
\end{lemma}
\begin{proof}
Because $Uf$ is decreasing, 
\[
V_I(Uf) = Uf(0)-Uf(1) = \sum_{i \geq 1} \left( P_i(0)f\left(\frac{1}{i}\right) - P_i(1) f\left(\frac{1}{1+i}\right)\right).
\]
As $P_i(u)=\frac{u+1}{(u+i)(u+i+1)}$,
\[
P_i(1) = \frac{2}{(i+1)(i+2)} = 2P_{i+1}(0),
\]
hence 
\begin{align*}
V_I(Uf)&=\sum_{i \geq 1} \left( P_i(0)f\left(\frac{1}{i}\right) - P_i(1) f\left(\frac{1}{1+i}\right)\right)\\
&=\sum_{i \geq 1}  \left( P_i(0)f\left(\frac{1}{i}\right) - P_{i+1}(0) f\left(\frac{1}{1+i}\right)\right)\\
&-\sum_{i \geq 1} P_{i+1}(0) f\left(\frac{1}{1+i}\right)\\
&=P_1(0)f(1) -\sum_{i \geq 1} P_{i+1}(0) f\left(\frac{1}{1+i}\right)\\
&=\frac{1}{2}f(1) - \sum_{i \geq 1} P_{i+1}(0) f\left(\frac{1}{1+i}\right).
\end{align*}
Because $ f\left(\frac{1}{1+i}\right) \geq f(0)$ we have  $- f\left(\frac{1}{1+i}\right) \leq - f(0)$, hence 
\[
V_I(Uf) \leq \frac{1}{2}f(1) -f(0) \sum_{i \geq 1} P_{i+1}(0) = \frac{1}{2}f(1)  - \frac{1}{2}f(0),
\]
using $\sum_{i \geq 1} P_i(0) = 1$ and $P_1(0)=\frac{1}{2}$. 
As $f$ is increasing this means
\[
V_I(Uf) \leq \frac{1}{2}(f(1)-f(0)) = \frac{1}{2} V_I f.
\]
\end{proof}





\begin{theorem}
If $f \in BV(I)$ then
\[
V_I (Uf) \leq \frac{1}{2} V_I f.
\]
\end{theorem}
\begin{proof}
Let 
\[
p_f(x) = \frac{v_f(x)+f(x)-f(0)}{2},\qquad n_f(x) = \frac{v_f(x)-f(x)+f(0)}{2},
\]
the \textbf{positive variation} of $f$ and the \textbf{negative variation} of $f$.
It is a fact that $0 \leq p_f \leq v_f$, $0 \leq n_f \leq v_f$, 
and $p_f$ and $n_f$ are increasing.
Using this,
\begin{align*}
V_I (Uf)&=V_I(Up_f+Un_f)\\
&\leq \frac{1}{2}V_I p_f + \frac{1}{2} V_I n_f\\
&=\frac{1}{2}(p_f(1)-p_f(0)) + \frac{1}{2}(n_f(1)-n_f(0))\\
&=\frac{1}{2}(v_f(1)-v_f(0))\\
&=\frac{1}{2}V_I f.
\end{align*}
\end{proof}


For $f:I \to \mathbb{C}$,  let
\[
s(f) = \sup_{x,y \in I, x \neq y} \frac{|f(x)-f(y)|}{|x-y|}.
\]
We denote by $\Lip(I)$ the set of $f:I \to \mathbb{C}$ such that $s(f)<\infty$.\footnote{Marius Iosifescu and Cor Kraaikamp,
{\em Metrical Theory of Continued Fractions}, p.~67, Proposition 2.1.14.}

\begin{theorem}
For $f \in \Lip(I)$,
\[
s(Uf) \leq (2\zeta(3)-\zeta(2)) s(f).
\]
\end{theorem}
\begin{proof}
Suppose $x,y \in I$, $x > y$. We calculate
\[
\begin{split}
&\frac{Uf(y)-Uf(x)}{y-x}\\
=&\frac{1}{y-x} \sum_{i \geq 1} \left(P_i(y) f\left(\frac{1}{y+i}\right) -P_i(y) f\left(\frac{1}{x+i}\right)\right)\\
& +\frac{1}{y-x} \sum_{i \geq 1} \left( P_i(y) f\left(\frac{1}{x+i}\right) -  P_i(x) f\left(\frac{1}{x+i}\right)\right)\\
=&\sum_{i \geq 1} P_i(y) \cdot \frac{f\left(\frac{1}{y+i}\right) - f\left(\frac{1}{x+i} \right)}{y-x}\\
&+\sum_{i \geq 1} \frac{P_i(y)-P_i(x)}{y-x} f\left(\frac{1}{x+i}\right).
\end{split}
\]
Calculating further,
\begin{align*}
\frac{Uf(y)-Uf(x)}{y-x}&=-\sum_{i \geq 1} P_i(y) \cdot \frac{f\left(\frac{1}{y+i}\right) - f\left(\frac{1}{x+i} \right)}{\frac{1}{y+i}-\frac{1}{x+i}}
\cdot \frac{1}{(x+i)(y+i)}\\
&+\sum_{i \geq 1} \frac{P_i(y)-P_i(x)}{y-x} f\left(\frac{1}{x+i}\right).
\end{align*}
Now, 
\[
P_i(u) =\frac{u+1}{(u+i)(u+i+1)}= \frac{i}{u+i+1}-\frac{i-1}{u+i},
\]
whence 
\[
P_i(y)-P_i(x)=\frac{(x-y)i}{(x+i+1)(y+i+1)} + \frac{(y-x)(i-1)}{(x+i)(y+i)},
\]
therefore
\[
\begin{split}
&\sum_{i \geq 1} \frac{P_i(y)-P_i(x)}{y-x} f\left(\frac{1}{x+i}\right)\\
=&\sum_{i \geq 1} \left(\frac{i-1}{(x+i)(y+i)}
-\frac{i}{(x+i+1)(y+i+1)}\right)f\left(\frac{1}{x+i}\right).
\end{split}
\]
Summation by parts tells us
\[
\sum_{i \geq 1} f_i(g_{i+1}-g_i) = - f_1 g_1 - \sum_{i \geq 1} g_{i+1}(f_{i+1}-f_i),
\]
and here this yields, for $g_i=\frac{i-1}{(x+i)(y+i)}$ and $f_i = f\left(\frac{1}{x+i}\right)$,
\[
\begin{split}
&\sum_{i \geq 1} \left(\frac{i-1}{(x+i)(y+i)}
-\frac{i}{(x+i+1)(y+i+1)}\right)f\left(\frac{1}{x+i}\right)\\
=&\sum_{i \geq 1} g_{i+1}(f_{i+1}-f_i)\\
=&\sum_{i \geq 1} \frac{i}{(x+i+1)(y+i+1)} \left( f \left(\frac{1}{x+i+1}\right) - f \left( \frac{1}{x+i} \right) \right)\\
=&\sum_{i \geq 1} \frac{i}{(x+i+1)(y+i+1)} \cdot \frac{ f \left(\frac{1}{x+i+1}\right) - f \left( \frac{1}{x+i} \right)}{\frac{1}{x+i+1}-\frac{1}{x+i}}
\cdot \frac{-1}{(x+i)(x+i+1)}.
\end{split}
\]
Recapitulating the above,
\[
\begin{split}
&\frac{Uf(y)-Uf(x)}{y-x}\\
=&-\sum_{i \geq 1} P_i(y) \cdot \frac{f\left(\frac{1}{y+i}\right) - f\left(\frac{1}{x+i} \right)}{\frac{1}{y+i}-\frac{1}{x+i}}
\cdot \frac{1}{(x+i)(y+i)}\\
&-\sum_{i \geq 1} \frac{i}{(x+i)(x+i+1)^2 (y+i+1)} \cdot \frac{ f \left(\frac{1}{x+i+1}\right) - f \left( \frac{1}{x+i} \right)}{\frac{1}{x+i+1}-\frac{1}{x+i}}.
\end{split}
\]
Then
\begin{align*}
\left| \frac{Uf(y)-Uf(x)}{y-x} \right|&\leq s(f) \sum_{i \geq 1} P_i(y) \frac{1}{(x+i)(y+i)}\\
&+s(f) \sum_{i \geq 1}  \frac{i}{(x+i)(x+i+1)^2 (y+i+1)}.
\end{align*}
Then, using that $x>y$,
\[
\left| \frac{Uf(y)-Uf(x)}{y-x} \right| \leq s(f) \sum_{i \geq 1} \left( P_i(y) \frac{1}{(y+i)^2} + \frac{i}{(y+i)(y+i+1)^3} \right).
\]
Because $y \in I = [0,1]$, $y \geq 0$ so 
\[
\sum_{i \geq 1} \frac{i}{(y+i)(y+i+1)^3} \leq \sum_{i \geq 1} \frac{1}{(i+1)^3} = -1+\zeta(3).
\]
Let $h(u) = u^2$, with which
\[
\sum_{i \geq 1} P_i(y) \frac{1}{(y+i)^2} = Uh(y).
\]
$h:I \to \mathbb{R}$ is increasing, so $Uh$ is decreasing. 
Because $P_i(0) = \frac{1}{i(i+1)}$,
\[
\sum_{i \geq 1} P_i(y) \frac{1}{(y+i)^2} = Uh(y) \leq Uh(0) = \sum_{i \geq 1} P_i(0) \frac{1}{i^2}
=\sum_{i \geq 1} \frac{1}{i^3(i+1)}.
\]
Doing partial fractions, 
\[
\frac{1}{i^3(i+1)} = \frac{1}{i^3} - \frac{1}{i^2} + \frac{1}{i} - \frac{1}{1+i},
\]
so
\[
\sum_{i \geq 1} \frac{1}{i^3(i+1)} = \zeta(3) - \zeta(2) + 1.
\]
Therefore
\[
\left| \frac{Uf(y)-Uf(x)}{y-x} \right| \leq s(f)  \left(  \zeta(3) - \zeta(2) + 1 - 1 + \zeta(3) \right)
=s(f) (2\zeta(3)-\zeta(2)).
\]
\end{proof}

For example, let 
$f(x)=x$, for which $s(f)=1$. Now,
\[
Uf(x) = \sum_{i \geq 1} P_i(x) \frac{1}{x+i}.
\]
We remind ourselves that
\[
P_i(x) = \frac{x+1}{(x+i)(x+i+1)},\quad P_i'(x) =  \frac{i^2-i-(x+1)^2}{(x+i)^2(x+i+1)^2}.
\]
Then
\begin{align*}
(Uf)'(x)&=\sum_{i \geq 1} \left( P_i'(x) \frac{1}{x+i} - P_i(x) \frac{1}{(x+i)^2} \right)\\
&=\sum_{i \geq 1} \left( \frac{i^2-i-(x+1)^2}{(x+i)^3(x+i+1)^2} - \frac{x+1}{(x+i)^3(x+i+1)}\right)\\
&=\sum_{i \geq 1} \frac{i^2-i-(x+1)^2-(x+1)(x+i+1)}{(x+i)^3(x+i+1)^2}\\
&=\sum_{i \geq 1} \frac{-2x^2-ix-4x+i^2-2i-2}{(x+i)^3(x+i+1)^2}.
\end{align*}
Check that $x \mapsto (Uf)'(x)$ is increasing and negative. Then
$\norm{(Uf)'} \leq |(Uf)'(0)|$, with
\[
(Uf)'(0) = \sum_{i \geq 1} \frac{i^2-2i-2}{i^3(i+1)^2} = -2\zeta(3) + \zeta(2).
\]
Therefore for $f(x)=x$,
\[
s(f) = \norm{(Uf)'}_\infty = 2 \zeta(3)-\zeta(2),
\]
which shows that the above theorem is sharp.





\end{document}