\documentclass{article}
\usepackage{amsmath,amssymb,mathrsfs,amsthm}
%\usepackage{tikz-cd}
%\usepackage{hyperref}
\newcommand{\inner}[2]{\left\langle #1, #2 \right\rangle}
\newcommand{\tr}{\ensuremath\mathrm{tr}\,} 
\newcommand{\Span}{\ensuremath\mathrm{span}} 
\def\Re{\ensuremath{\mathrm{Re}}\,}
\def\Im{\ensuremath{\mathrm{Im}}\,}
\newcommand{\id}{\ensuremath\mathrm{id}} 
\newcommand{\var}{\ensuremath\mathrm{var}} 
\newcommand{\Lip}{\ensuremath\mathrm{Lip}} 
\newcommand{\GL}{\ensuremath\mathrm{GL}}
\newcommand{\diam}{\ensuremath\mathrm{diam}} 
\newcommand{\sgn}{\ensuremath\mathrm{sgn}\,} 
\newcommand{\lcm}{\ensuremath\mathrm{lcm}} 
\newcommand{\supp}{\ensuremath\mathrm{supp}\,}
\newcommand{\dom}{\ensuremath\mathrm{dom}\,}
\newcommand{\upto}{\nearrow}
\newcommand{\downto}{\searrow}
\newcommand{\norm}[1]{\left\Vert #1 \right\Vert}
\theoremstyle{definition}
\newtheorem{theorem}{Theorem}
\newtheorem{lemma}[theorem]{Lemma}
\newtheorem{proposition}[theorem]{Proposition}
\newtheorem{corollary}[theorem]{Corollary}
\theoremstyle{definition}
\newtheorem{definition}[theorem]{Definition}
\newtheorem{example}[theorem]{Example}
\begin{document}
\title{Explicit construction of the $p$-adic numbers}
\author{Jordan Bell}
\date{March 17, 2016}

\maketitle

\section{{\em \textbf{Z}\textsubscript{p}}}
Let $p$ be prime, let $N_p = \{0,\ldots,p-1\}$, and
let $\mathbb{Z}_p$ be the set of maps $x:\mathbb{Z} \to N_p$ such that $x(k)=0$ for all $k<0$. 




\subsection{Addition}
For $x,y \in \mathbb{Z}_p$,
we define $x+y \in \mathbb{Z}_p$ by induction. Define
\[
(x+y)(0) \equiv x(0)+y(0) \pmod{p}, \qquad (x+y)(0) \in N_p.
\]
Assume  for  $k \geq 0$ that there is some 
$A_k \in \mathbb{Z}$ such that
\[
\sum_{j=0}^k (x+y)(j) p^j = A_k p^{k+1} +  \sum_{j=0}^k (x(j)+y(j))p^j.
\]
Define
\[
(x+y)(k+1) \equiv -A_k + x(k+1)+y(k+1) \pmod{p},\qquad (x+y)(k+1) \in N_p,
\]
and then define
$A_{k+1} \in \mathbb{Z}$ by
\[
(x+y)(k+1)= A_{k+1} p - A_k + x(k+1)+y(k+1).
\] 
Then
\begin{align*}
\sum_{j=0}^{k+1}(x+y)(j) p^j &=(x+y)(k+1) p^{k+1} + \sum_{j=0}^k (x+y)(j) p^j\\
&= A_{k+1} p^{k+2} - A_k p^{k+1} + (x(k+1)+y(k+1))p^{k+1}\\
&+A_k p^{k+1} +  \sum_{j=0}^k (x(j)+y(j))p^j\\
&=A_{k+1} p^{k+2} + \sum_{j=0}^{k+1}  (x(j)+y(j))p^j.
\end{align*}
Thus, for each $k \geq 0$, $(x+y)(k) \in N_p$ and 
\begin{equation}
\sum_{j=0}^k (x+y)(j) p^j \equiv \sum_{j=0}^k (x(j)+y(j)) p^j \pmod{p^{k+1}}.
\label{addition}
\end{equation}
It is immediate that $x+y=y+x$.

\begin{lemma}
If $x,y \in \mathbb{Z}_p$ and for each $k \geq 0$,  
\[
\sum_{j=0}^k x(j) p^j \equiv \sum_{j=0}^k y(j) p^j \pmod{p^{k+1}},
\]
then $x=y$.
\end{lemma}
\begin{proof}
Suppose by contradiction that $x \neq y$. Now, $x(0) \equiv y(0) \pmod{p}$
and $x(0),y(0) \in N_p$ so $x(0)=y(0)$. As $x \neq y$, there is  a minimal $k \geq 0$ such that  
$x(k+1) \neq y(k+1)$. On the one hand, 
\[
\sum_{j=0}^{k+1} x(j) p^j = x(k+1) p^{k+1} + \sum_{j=0}^k y(j) p^j,
\]
and on the other hand,
\[
\sum_{j=0}^{k+1} x(j) p^j  \equiv \sum_{j=0}^{k+1} y(j) p^j \pmod{p^{k+2}}.
\]
Then there is some $B$ such that 
\[
x(k+1)p^{k+1} =Cp^{k+2}+ y(k+1) p^{k+1}.
\]
so $x(k+1) - y(k+1) = Bp$. But $-p+1 \leq x(k+1)-y(k+1) \leq p-1$, so $B=0$ and hence
$x(k+1)=y(k+1)$, a contradiction and thus $x=y$.
\end{proof}

Therefore, if $t \in \mathbb{Z}_p$ satisfies, for all $k \geq 0$,
\[
\sum_{j=0}^k t(j) p^j \equiv \sum_{j=0}^k (x(j)+y(j)) p^j \pmod{p^{k+1}}.
\]
then $t=x+y$.
Now let $x,y,z \in \mathbb{Z}_p$. For $k \geq 0$,
\begin{align*}
\sum_{j=0}^k (x+(y+z))(j) p^j &\equiv \sum_{j=0}^k (x(j)+(y+z)(j)) p^j \pmod{p^{k+1}}\\
&=\sum_{j=0}^k (x(j)+y(j)+z(j)) p^j \pmod{p^{k+1}}\\
&\equiv \sum_{j=0}^k ((x+y)(j)+z(j)) p^j \pmod{p^{k+1}},
\end{align*}
which shows that $x+(y+z) = (x+y)+z$. 

Define $t \in \mathbb{Z}_p$ by $t(k)=0$ for all $k \geq 0$. It is immediate that for
$x \in \mathbb{Z}_p$, $x+t=x$, $t+x=x$. 
If $x \neq 0$, let $m \geq 0$ be minimal such that $x(m) \neq 0$, and define
$y \in \mathbb{Z}_p$ by
\[
y(k) =\begin{cases}
0&0 \leq k < m\\
p-x(m)&k=m\\
p-1-x(k)&k >m.
\end{cases}
\]
This makes sense because $1 \leq x(m) \leq p-1$. Then
$x(k)+y(k) =0$ for $0 \leq k < m$, $x(m)+y(m) = p$, and $x(k)+y(k)=p-1$ for $k>m$. For
$k > m$,
\begin{align*}
\sum_{j=0}^k (x(j)+y(j))p^j & = p\cdot p^m + \sum_{j=m+1}^k (p-1)p^j\\
&=p^{m+1} + (p-1) \cdot \frac{p^{k+1}-p^{m+1}}{p-1}\\
&=p^{k+1},
\end{align*}
so 
\[
\sum_{j=0}^k (x(j)+y(j))p^j \equiv \sum_{j=0}^k 0 \cdot p^j \pmod{p^{k+1}},
\]
and it follows that $x+y=0$, $y+x=0$, namely $y=-x$.

We have established that $(\mathbb{Z}_p,+)$ is an abelian group whose identity is 
 $k \mapsto 0$, $k \geq 0$.
 
\begin{lemma}
For $x \in \mathbb{Z}_p$ and $m \geq 1$, 
\[
(p^m x)(k) = \begin{cases}
0&0 \leq k < m\\
x(k-m)&k \geq m.
\end{cases}
\]
\label{timesp}
\end{lemma}
\begin{proof}
For $x \in \mathbb{Z}_p$ and $m \geq 1$ define 
$y(j)=0$ for $0 \leq j <m$ and 
$y(j) = x(j-m)$ for $j \geq m$.
By \eqref{addition}, for $k \geq m$,
\begin{align*}
\sum_{j=0}^k (p^mx)(j) p^j&\equiv \sum_{j=0}^k p^m x(j) p^j \pmod{p^{k+1}}\\
&\equiv \sum_{j=0}^k x(j)p^{j+m} \pmod{p^{k+1}}\\
&\equiv \sum_{j=m}^{m+k} x(j-m) p^j \pmod{p^{k+1}}\\
&\equiv \sum_{j=m}^k x(j-m) p^j \pmod{p^{k+1}}\\
&\equiv \sum_{j=0}^k y(j) p^j \pmod{p^{k+1}}.
\end{align*}
\end{proof}

The following lemma shows that if $x(k)=0$ for $k<m$ then it makes sense to talk about $p^{-m}x \in \mathbb{Z}_p$. 
That is, if $x(k)=0$ for $k<m$ then there is a unique $y \in \mathbb{Z}_p$ such that $p^m y=x$. (For comparison,
it is false that for any $z \in \mathbb{C}$ there is a unique $z^{1/2} \in \mathbb{C}$, or that
for any $n \in \mathbb{Z}$ there is a unique $p^{-1}n \in \mathbb{Z}$.)


\begin{lemma}
Let $x \in \mathbb{Z}_p$ with $x(0)=0$. If $y \in \mathbb{Z}_p$ and $py=x$ then $y(k)=x(k+1)$ for $k \geq 0$.
\label{dividep}
\end{lemma}
\begin{proof}
By Lemma \ref{timesp}, $(py)(0)=0$ and $(py)(k)=y(k-1)$ for $k \geq 1$, and as $py=x$ this means
$x(0)=0$ and $x(k)=y(k-1)$ for $k \geq 1$, i.e. $x(k+1) = y(k)$ for $k \geq 0$. 
\end{proof}
 
 
\subsection{Multiplication}
For $x,y \in \mathbb{Z}_p$, we define $xy \in \mathbb{Z}_p$ by induction.
Define
\[
(xy)(0) \equiv x(0) y(0) \pmod{p},\qquad (xy)(0) \in N_p.
\]
Assume for $k \geq 0$ that there is some $A_k \in \mathbb{Z}$ such that 
\[
\sum_{j=0}^k (xy)(j) p^j = A_k p^{k+1} +\left( \sum_{j=0}^k x(j) p^j \right) \left( \sum_{j=0}^k y(j) p^j\right).
\]
There is some $B \in \mathbb{Z}$ such that 
\[
\begin{split}
&\left( \sum_{j=0}^{k+1} x(j) p^j \right) \left( \sum_{j=0}^{k+1} y(j) p^j\right)\\
=&\left( x(k+1) p^{k+1} + \sum_{j=0}^k x(j) p^j \right) \left( y(k+1) p^{k+1} + \sum_{j=0}^k y(j) p^j\right)\\
=&B p^{k+2} + x(k+1)y(0) p^{k+1} + x(0)y(k+1)p^{k+1} + \left( \sum_{j=0}^k x(j) p^j \right) \left( \sum_{j=0}^k y(j) p^j\right).
\end{split}
\]
Hence
\begin{align*}
\left( \sum_{j=0}^{k+1} x(j) p^j \right) \left( \sum_{j=0}^{k+1} y(j) p^j\right)&=B p^{k+2} + x(k+1)y(0) p^{k+1} + x(0)y(k+1)p^{k+1}\\
&+\sum_{j=0}^k (xy)(j) p^j - A_k p^{k+1}.
\end{align*}
Now define
\[
(xy)(k+1) \equiv x(k+1)y(0) + x(0)y(k+1) - A_k \pmod{p},\qquad (xy)(k+1) \in N_p,
\]
and let $C \in \mathbb{Z}$ such that
\[
(xy)(k+1) = Cp + x(k+1)y(0) + x(0)y(k+1) - A_k,
\]
whence, taking $A_{k+1} = B-C$,
\begin{align*}
\left( \sum_{j=0}^{k+1} x(j) p^j \right) \left( \sum_{j=0}^{k+1} y(j) p^j\right)&=Bp^{k+2} + 
(xy)(k+1)p^{k+1} - Cp^{k+2} + A_kp^{k+1}\\
&+\sum_{j=0}^k (xy)(j) p^j - A_k p^{k+1}\\
&=A_{k+1}p^{k+2} + \sum_{j=0}^{k+1} (xy)(j) p^j.
\end{align*}
Thus, for each $k \geq 0$, $(xy)(k) \in N_p$ and 
\begin{equation}
\sum_{j=0}^k (xy)(j) p^j \equiv \left( \sum_{j=0}^k x(j) p^j \right) \left( \sum_{j=0}^k y(j) p^j\right) \pmod{p^{k+1}}.
\label{multiplication}
\end{equation}
It is immediate that $xy=yz$. 

For $t \in \mathbb{Z}_p$, if for each $k \geq 0$,
\[
\sum_{j=0}^k t(j) p^j \equiv \left( \sum_{j=0}^k x(j) p^j \right) \left( \sum_{j=0}^k y(j) p^j\right) \pmod{p^{k+1}}.
\]
then $t=xy$. Now let $x,y,z \in \mathbb{Z}_p$. For $k \geq 0$,
\begin{align*}
\sum_{j=0}^k (x(yz))(j) p^j &\equiv\left( \sum_{j=0}^k x(j) p^j \right) \left( \sum_{j=0}^k (yz)(j) p^j\right) \pmod{p^{k+1}}\\
&\equiv \left( \sum_{j=0}^k x(j) p^j \right)  \left( \sum_{j=0}^k y(j) p^j \right) \left( \sum_{j=0}^k z(j) p^j\right) \pmod{p^{k+1}}\\
&\equiv \left( \sum_{j=0}^k (xy)(j) p^j\right) \left( \sum_{j=0}^k z(j) p^j \right) \pmod{p^{k+1}}\\
&\equiv  \sum_{j=0}^k ((xy)z)(j) p^j \pmod{p^{k+1}},
\end{align*}
which shows that $x(yz)=(xy)z$. 

Define $u \in \mathbb{Z}_p$ by $u(0)=1$, $u(k)=0$ for $k \geq 1$. It is apparent that for $x \in \mathbb{Z}_p$, $xu=x$
and $ux=x$. 


\subsection{Ring}
For $x,y,z \in \mathbb{Z}_p$ and for $k \geq 0$, using \eqref{addition} and \eqref{multiplication},
\begin{align*}
\sum_{j=0}^k (x(y+z))(j) p^j&\equiv \left( \sum_{j=0}^k x(j) p^j \right)\left( \sum_{j=0}^k (y+z)(j) p^j \right) \pmod{p^{k+1}}\\
&\equiv  \left( \sum_{j=0}^k x(j) p^j \right) \left(\sum_{j=0}^k (y(j)+z(j)) p^j\right) \pmod{p^{k+1}}\\
&\equiv  \left( \sum_{j=0}^k x(j) p^j \right) \left( \sum_{j=0}^k y(j) p^j \right)\\
&+ \left( \sum_{j=0}^k x(j) p^j \right) \left( \sum_{j=0}^k z(j) p^j \right) \pmod{p^{k+1}}\\
&\equiv \sum_{j=0}^k (xy)(j) p^j  + \sum_{j=0}^k (xz)(j) p^j \pmod{p^{k+1}}\\
&\equiv \sum_{j=0}^k (xy+xz)(j) p^j \pmod{p^{k+1}},
\end{align*}
which shows that $x(y+z)=xy+xz$. Therefore $\mathbb{Z}_p$ is a commutative ring with unity
$0 \mapsto 1$, $k \mapsto 0$ for $k \geq 1$. 



\subsection{Integral domain}
Let $\mathbb{Z}_p^*$ be the set of those $x \in \mathbb{Z}_p$ for which there is some $y \in \mathbb{Z}_p$ such that
$xy=1$, namely the set of invertible elements of $\mathbb{Z}_p$. 

\begin{lemma}
Let $x \in \mathbb{Z}_p$. $x \in \mathbb{Z}_p^*$ if and only if $x(0) \neq 0$. 
\label{invertible}
\end{lemma}
\begin{proof}
If $x(0)=0$ and $y \in \mathbb{Z}_p$ then $(xy)(0) \equiv x(0) y(0)  \equiv 0 \pmod{p}$ while
$1(0) \equiv 1 \pmod{p}$, so $xy \neq 1$ and therefore $x \not \in \mathbb{Z}_p^*$.

If $x(0) \neq 0$, we define $y \in \mathbb{Z}_p$ by induction. As $x(0) \neq 0$, it makes sense to define 
\[
y(0) x(0) \equiv 1 \pmod{p},\qquad y(0) \in N_p.
\]
We use \eqref{multiplication} and the fact that $1(0)=1$, $1(k)=0$ for $k \geq 1$. Suppose for $k \geq 0$ that 
there is some $A_k \in \mathbb{Z}$ such that
\[
 \left( \sum_{j=0}^k x(j) p^j \right) \left( \sum_{j=0}^k y(j) p^j\right) = A_k p^{k+1} + 1.
\]
Because $x(0) \neq 0$, it makes sense to define
\[
y(k+1)x(0) + x(k+1)y(0) \equiv -A_k \pmod{p}.
\]
Then
\begin{align*}
 \left( \sum_{j=0}^{k+1} x(j) p^j \right) \left( \sum_{j=0}^{k+1} y(j) p^j\right) &\equiv 
x(k+1)y(0)p^{k+1}+y(k+1)x(0)p^{k+1}\\
& \left( \sum_{j=0}^{k} x(j) p^j \right) \left( \sum_{j=0}^{k} y(j) p^j\right) \pmod{p^{k+2}}\\
&\equiv -A_k p^{k+1}+ A_k p^{k+1} + 1 \pmod{p^{k+2}}\\
&\equiv 1 \pmod{p^{k+2}}.
\end{align*}
This shows that $xy = 1$, thus $x \in \mathbb{Z}_p^*$ and $y = x^{-1}$.
\end{proof}



\begin{theorem}
$\mathbb{Z}_p$ is an integral domain.
\end{theorem}
\begin{proof}
Let $x,y \in \mathbb{Z}_p$ be nonzero. Let $m \geq 0$ be minimal such that $x(m) \neq 0$ and let $n \geq 0$ be minimal such that $y(n) \neq 0$.
Then $(p^{-m}x)(0) \neq 0$ and $(p^{-n}y)(0) \neq 0$, and using
$p^{-m-n} (xy) = p^{-m} x \cdot p^{-n} y$,
\begin{align*}
(xy)(m+n) &\equiv 
(p^{-m-n} (xy))(0) \pmod{p}\\
&\equiv (p^{-m} x)(0)  \cdot (p^{-n} y)(0) \pmod{p}\\
&\not \equiv 0 \pmod{p},
\end{align*}
thus $xy \neq 0$.
\end{proof}





\subsection{{\em p}-adic valuation}
For $x \in \mathbb{Z}_p$, let
\[
v_p(x) = \inf\{k \geq 0: x(k) \neq 0\}. 
\]
$x(k)=0$ for $0 \leq k < v_p(x)$. $v_p(x) = \infty$ if and only if $x=0$.

\begin{lemma}
For $x,y \in \mathbb{Z}_p$, 
\[
v_p(xy) = v_p(x)+v_p(y)
\]
and
\[
v_p(x+y) \geq \min(v_p(x),v_p(y)).
\]
\end{lemma}

Lemma \ref{invertible} says that for $x \in \mathbb{Z}_p$, $x \in \mathbb{Z}_p^*$ if and only if $x(0) \neq 0$.
In other words, 
\[
\mathbb{Z}_p^* = \{x \in \mathbb{Z}_p : v_p(x) = 0\} = \{x \in \mathbb{Z}_p : |x|_p = 1\}.
\]

For $n \geq 1$,
define $\pi_n:\mathbb{Z}_p \to \mathbb{Z}/p^n \mathbb{Z}$ by
\[
\pi_n(x) = \sum_{k=0}^{n-1} x(k) p^k + p^n \mathbb{Z}.
\]
It is apparent that $\pi_n$ is onto.

\begin{lemma}
$\pi_n:\mathbb{Z}_p \to \mathbb{Z}/p^n \mathbb{Z}$ is a ring homomorphism,
and
\[
\ker \pi_n = \{x \in \mathbb{Z}_p : v_p(x) \geq n\} = p^n \mathbb{Z}_p.
\]
\end{lemma}
\begin{proof}
Let $x,y \in \mathbb{Z}_p$. By \eqref{addition},
\[
\sum_{k=0}^{n-1} (x+y)(k) p^k + p^n \mathbb{Z} = \sum_{k=0}^{n-1} x(k) p^k + \sum_{k=0}^{n-1} y(k) p^k + p^n \mathbb{Z},
\]
i.e.
\[
\pi_n(x+y) = \pi_n(x)+\pi_n(y).
\]
By \eqref{multiplication},
\[
\sum_{k=0}^{n-1} (xy)(k) p^k + p^n \mathbb{Z} = \left(\sum_{k=0}^{n-1} x(k) p^k + p^n\mathbb{Z} \right)
 \left(\sum_{k=0}^{n-1} y(k) p^k + p^n\mathbb{Z} \right),
\]
i.e.
\[
\pi_n(xy) = \pi_n(x) \pi_n(y).
\]
For $1 \in \mathbb{Z}_p$, $1(0)=1$, $1(k)=0$ for $k \geq 1$, so 
\[
\pi_n(1) = 1 + p^n\mathbb{Z},
\]
which is the unity of $\mathbb{Z}/p^n \mathbb{Z}$. Therefore $\pi_n$ is a ring homomorphism.

$\pi_n(x)=0$ means
\[
\sum_{k=0}^{n-1} x(k) p^k \in p^n \mathbb{Z}.
\]
But $0 \leq \sum_{k=0}^{n-1} x(k) p^k < \sum_{k=0}^{n-1} (p-1) p^k = p^n-1$, so $\pi_n(x)=0$ if and only if 
$x(k)=0$ for $0 \leq k \leq n-1$. 
\end{proof}

Then for $n \geq 1$,
\begin{align*}
\mathbb{Z}_p &= \bigcup_{j=0}^{p^n-1} (j+p^n \mathbb{Z}_p)\\
&=\bigcup_{j=0}^{p^n-1} \{x \in \mathbb{Z}_p: v_p(x-j) \geq n\}\\
&=\bigcup_{j=0}^{p^n-1} \{x \in \mathbb{Z}_p: |x-j|_p \leq p^{-n}\}\\
&=\bigcup_{j=0}^{p^n-1} \{x \in \mathbb{Z}_p: |x-j|_p < p^{-n+1}\}.
\end{align*}

Because $\mathbb{Z}/p\mathbb{Z}$ is a field and $\pi_1:\mathbb{Z}_p \to \mathbb{Z}/p\mathbb{Z}$ is an onto ring homomorphism,
\[
\ker \pi_1 = p\mathbb{Z}_p
\]
is a maximal ideal in $\mathbb{Z}_p$.



\begin{theorem}
If $I$ is an ideal in $\mathbb{Z}_p$ and $I \neq \{0\}$, then
there is some $n \geq 0$ such that $I=p^n \mathbb{Z}_p$.
\end{theorem}
\begin{proof}
There is some $a \in I$ with minimal $v_p(a) \geq 0$, and as $I \neq \{0\}$, $v_p(a) \neq \infty$. 
Then $(p^{-v_p(a)}a)(0) = a(v_p(a)) \neq 0$, so by Lemma \ref{invertible},
$p^{-v_p(a)} a \in \mathbb{Z}_p^*$. Hence there is some $u \in \mathbb{Z}_p^*$ such that
$p^{-v_p(a)} a = u$, i.e. $p^{v_p(a)} = u^{-1} a$. But $I$ is an ideal and $a \in I$, so
$p^{v_p(a)} \in I$, which shows that $p^{v_p(a)} \mathbb{Z}_p \subset I$. 
Let $x \in I$, $x \neq 0$.
Then there is some $v \in \mathbb{Z}_p^*$ such that $p^{-v_p(x)}x=v$, i.e.
$x = p^{v_p(x)}v$.  
Because $v_p(a)$ is minimal, $v_p(x) \geq v_p(a)$ and so
\[
x = p^{v_p(x)} v = p^{v_p(a)} \cdot p^{v_p(x)-v_p(a)} \in p^{v_p(a)} \mathbb{Z}_p.
\]
Therefore $I = p^{v_p(a)} \mathbb{Z}_p$.
\end{proof}



\section{{\em \textbf{Q}\textsubscript{p}}}
Let $\mathbb{Q}_p$ be the set of maps $x:\mathbb{Z} \to N_p$ such that for some $m \in \mathbb{Z}$, 
$x(k)=0$ for all $k<m$. For $x \in \mathbb{Q}_p$ define
\[
v_p(x) = \inf\{k \in \mathbb{Z} : x(k) \neq 0\}. 
\]
$x(k)=0$ for $k<v_p(x)$, $k \in \mathbb{Z}$. $v_p(x) = \infty$ if and only if $x=0$.
\[
\mathbb{Z}_p = \{x \in \mathbb{Q}_p: v_p(x) \geq 0\}.
\]

For $m \in \mathbb{Z}$ and $x \in \mathbb{Q}_p$, define 
\[
(T_m x)(k) = x(k+m),\qquad k \in \mathbb{Z}.
\]
For  $x \in \mathbb{Q}_p$ with $x(k)=0$ for $k<m$,
if $k<0$ then $k+m<m$ and so 
\[
(T_m x)(k) = x(k+m) = 0,
\]
which means that $T_m x \in \mathbb{Z}_p$.
For $x,y \in \mathbb{Q}_p$ with
$x(k)=0$ and $y(k)=0$ for $k<m$,
$T_m x, T_m y \in \mathbb{Z}_p$ and 
$T_m x + T_m y \in \mathbb{Z}_p$. Define
\[
x+y = T_{-m}(T_m x + T_m y) \in \mathbb{Q}_p.
\]
Check that this makes sense.
Likewise, $T_m x \cdot T_m y \in \mathbb{Z}_p$, and define
\[
xy = T_{-m} (T_m x  \cdot T_m y) \in \mathbb{Q}_p.
\]
Check that this makes sense.
Check that $\mathbb{Q}_p$ is a commutative ring with 
additive identity $k \mapsto 0$ for $k \in \mathbb{Z}$.
and unity $0 \mapsto 1$, $k \mapsto 0$ for $k \neq 0$. Finally,\footnote{For a ring $R$ with $x \in R$, $px = \sum_{k=1}^p x$. It does not make
sense to talk about $px$ before we have $x+y$, and  it is nonsense to talk about $p^{-m}x$ for $x \in \mathbb{Q}_p$ before have
defined addition on $\mathbb{Q}_p$. This is why I defined $T_m$ rather than initially using $x \mapsto p^{-m}x$; it is incorrect and a sloppy
habit to use properties of an object before showing that it exists.}
\[
T_m x = p^{-m}x.
\]

\begin{theorem}
$\mathbb{Q}_p$ is a field, of characteristic $0$.
\end{theorem}



\section{Metric}
For $x \in \mathbb{Q}_p$ define 
\[
|x|_p = p^{-v_p(x)}.
\]
$|x|_p=0$ if and only if $x=0$. For $x,y \in \mathbb{Q}_p$ define
\[
d_p(x,y) = |x-y|_p.
\]
$d_p$ is an \textbf{ultrametric}: 
\[
d_p(x,z) \leq \max(d_p(x,y),d_p(y,z)).
\]


\begin{theorem}
$\mathbb{Q}_p$ is a topological field.
\end{theorem}
\begin{proof}
For $(x,y),(u,v) \in \mathbb{Q}_p \times \mathbb{Q}_p$ let
\[
\rho((x,y),(u,v)) = \max(d_p(x,u),d_p(y,v)).
\]
\[
d_p(x+y,u+v) = |(x-u)+(y-v)|_p = \max(|x-u|_p,|y-v|_p) = \rho((x,y),(u,v)),
\]
which shows that $(x,y) \mapsto x+y$ is continuous $\mathbb{Q}_p \times \mathbb{Q}_p \to \mathbb{Q}_p$.
And
\[
d_p(-x,-y) = |-x-y|_p = |-1|_p |x+y|_p = |x+y|_p = d_p(x,y),
\]
which shows that $x \mapsto -x$ is continuous $\mathbb{Q}_p \to \mathbb{Q}_p$.
For $\rho((x,y),(u,v)) \leq \delta$, $|x-u|_p \leq \delta$ so $|u|_p \leq |x|_p +\delta$ and
\begin{align*}
d_p(xy,uv) &= |xy-uv|_p\\
&= |xy-uy+uy-uv|_p\\
&=\max(|xy-uy|_p,|uy-uv|_p)\\
&=\max(|y|_p |x-u|_p,|u|_p|y- v|_p)\\
&\leq \max(|y|_p \delta, (|x|_p+\delta)\delta),
\end{align*}
which shows that $(x,y) \mapsto xy$ is continuous $\mathbb{Q}_p \times \mathbb{Q}_p \to \mathbb{Q}_p$.
Finally, for $x,y \neq 0$,
\[
d_p(x^{-1},y^{-1}) = |x^{-1}-y^{-1}|_p = |xy|_p^{-1} |y-x|_p,
\]
which shows that $x \mapsto x^{-1}$ is continuous $\mathbb{Q}_p \setminus \{0\} \to \mathbb{Q}_p \setminus \{0\}$.
\end{proof}

For $x \in \mathbb{Q}_p$ and $r>0$, write
\[
B_{<r}(x) = \{y \in \mathbb{Q}_p: |y-x|_p<r\},
\quad 
B_{\leq r}(x) = \{y \in \mathbb{Q}_p: |y-x|_p<r\}.
\]
Thus, for $x \in \mathbb{Q}_p$ and $n \geq 0$,
\[
x+p^n \mathbb{Z} = B_{\leq p^{-n}}(x).
\]

\begin{lemma}
For $x \in \mathbb{Q}_p$, 
\[
\{x+p^n \mathbb{Z}_p: n \geq 0\}
\]
is a local base at $x$.
\label{localbase}
\end{lemma}
\begin{proof}
For $\epsilon>0$, let $p^{-n}<\epsilon$, $n \geq 0$, namely $n > \frac{1}{\log p} \log \frac{1}{\epsilon}$. For this $n$,
\[
x+p^n \mathbb{Z}_p = B_{\leq p^{-n}}(x) \subset B_{<\epsilon}(x).
\]
\end{proof}


\begin{theorem}
$\mathbb{Z}_p$ is a compact subspace of $\mathbb{Q}_p$.
\end{theorem}
\begin{proof}
Let $x_n \in \mathbb{Z}_p$ be a sequence. Because $x_n(0) \in N_p$, $n \geq 0$, there is some
$a(0) \in N_p$ and an infinite subset $I_0$ of $\{n \geq 0\}$ such that 
$x_n(0) = a(0)$ for $n \in I_0$. 
Suppose by induction that for some $N \geq 0$ there are $a(0),\ldots,a(N) \in N_p$ and 
an infinite set $I_N \subset \{n \geq 0\}$ such that
\[
x_n(k) = a(k),\qquad 0 \leq k \leq N, \quad n \in I_N.
\]
But for each $x \in I_N$, $x_n(N+1)$ belongs to the finite set $N_p$, and because $I_N$ is infinite there is
some $a(N+1) \in N_p$ and
an infinite set $I_{N+1} \subset I_N$ such that
$x_n(N+1) = a(N+1)$ for $n \in I_{N+1}$. 
We have thus defined $a \in \mathbb{Z}_p$. 

Let $\alpha_0 \in I_0$, and by induction let $\alpha_n > \alpha_{n-1}$, $\alpha_n \in I_n$; in particular
as $\alpha_0 \geq 0$ we have $\alpha_n \geq n$. 
Then for any $n \geq 0$,
$x_{\alpha_n}(k)=a(k)$ for $0 \leq k \leq n$. 
Take $\epsilon>0$ and let $p^{-m-1}<\epsilon$. For $n \geq m$,
\[
|x_{\alpha_n}-a|_p \leq p^{-n-1} \leq p^{-m-1}< \epsilon,
\]
which shows that the sequence $x_{\alpha_n}$ tends to $a$. This means that
$\mathbb{Z}_p$ is sequentially compact and therefore compact. 
\end{proof}



For $x,y \in \mathbb{Q}_p$,
\[
d_p(px,py) = |px-py|_p = |p|_p |x-y|_p = p^{-1} |x-y|_p,
\]
which shows that $x \mapsto px$ is continuous $\mathbb{Q}_p \to \mathbb{Q}_p$. 
Therefore, the fact that $\mathbb{Z}_p$ is compact implies that
for $n \geq 0$, $p^n \mathbb{Z}_p$ is compact. Then by Lemma \ref{localbase} we get the following.

\begin{theorem}
$\mathbb{Q}_p$ is locally compact.
\end{theorem}




\begin{theorem}
$\mathbb{Q}_p$ is a complete metric space.
\end{theorem}



A topological space $X$ is \textbf{zero-dimensional} 
if there is a base for its topology each element of which is clopen. In a Hausdorff space,
a compact set is closed, and because the sets $p^n \mathbb{Z}_p$ are compact, $n \geq 0$, from
Lemma \ref{localbase} we get the following.

\begin{lemma}
$\mathbb{Q}_p$ is zero-dimensional.
\end{lemma}

It is a fact that if a Hausdorff space is zero-dimensional then it is \textbf{totally disconnected}, so
by the above, $\mathbb{Q}_p$ is totally disconnected.



\section{{\em p}-adic fractional part}
For $x \in \mathbb{Q}_p$, let 
\[
[x]_p = \sum_{k \geq 0} x(k) p^k \in \mathbb{Z}_p
\]
and
\[
\{x\}_p = \sum_{k<0} x(k) p^k \in \mathbb{Z}[1/p] \subset \mathbb{Q}.
\]
We call $\{x\}_p$ the \textbf{$p$-adic fractional part of $x$}.
Then 
\[
x = [x]_p + \{x\}_p \in \mathbb{Q}_p.
\]
Furthermore, as $x(k) \to 0$ as $k \to -\infty$,
\[
0 \leq \{x\}_p < \sum_{k<0} (p-1)p^k = (p-1) \sum_{k=1}^\infty p^{-k} = 1,
\] 
therefore for $x \in \mathbb{Q}_p$,
\[
\{x\}_p \in [0,1) \cap \mathbb{Z}[1/p].
\]

Define the \textbf{Pr\"ufer $p$-group}
\[
\mathbb{Z}(p^\infty) = \{e^{2\pi im p^{-n}} : m,n \geq 0\}.
\]
We assign the Pr\"ufer $p$-group the discrete topology.



Define $\psi_p:\mathbb{Q}_p \to S^1$ by
\[
\psi_p(x) = e^{2\pi i\{x\}_p}.
\]
We prove that this is a homomorphism from the locally compact group $\mathbb{Q}_p$ whose
image is the Pr\"ufer $p$-group and whose kernel is $\mathbb{Z}_p$.\footnote{Alain M. Robert, {\em A Course in $p$-adic Analysis}, p.~42, Proposition 5.4.}



\begin{theorem}
$\psi_p:\mathbb{Q}_p \to S^1$ is a  homomorphism of locally compact groups.
$\psi_p(\mathbb{Q}_p) = \mathbb{Z}(p^\infty)$, and $\ker \psi_p = \mathbb{Z}_p$.
\end{theorem}
\begin{proof}
For $x,y \in \mathbb{Q}_p$,
\begin{align*}
\{x+y\}_p - \{x\}_p - \{y\}_p &= x+y - [x+y]_p - x + [x]_p - y + [y]_p\\
&=[x]_p + [y]_p - [x+y]_p \in \mathbb{Z}_p.
\end{align*}
Check that $\mathbb{Z}[1/p] \cap \mathbb{Z}_p = \mathbb{Z}$. It then follows that
\[
\{x+y\}_p - \{x\}_p - \{y\}_p  \in \mathbb{Z},
\]
therefore $e^{2\pi i(\{x+y\}_p - \{x\}_p - \{y\}_p)}=1$, i.e.
\[
\psi_p(x+y) = e^{2\pi i\{x+y\}_p} = e^{2\pi i\{x\}_p} e^{2\pi i\{y\}_p}=\psi_p(x) \psi_p(y),\qquad x,y \in \mathbb{Q}_p,
\]
namely $\psi_p$ is a homomorphism.

$\psi_p(x)=1$ if and only if $e^{2\pi i\{x\}_p}=1$ if and only if $\{x\}_p \in \mathbb{Z}$. But
$\{x\}_p \in [0,1)$, so $\psi_p(x)=1$ if and only if $\{x\}_p=0$, hence
$\psi_p(x)=1$ if and only if $x \in \mathbb{Z}_p$, namely
\[
\ker \psi_p = \mathbb{Z}_p.
\]

Let $x \in \mathbb{Q}_p$. 
As $\{x\}_p \in \mathbb{Z}[1/p]$, there is some $n \geq 0$ such that $p^n \{x\}_p \in \mathbb{Z}$, so
$\psi_p(x)^{p^n} = 1$, which means that $\psi_p(x) \in \mathbb{Z}[p^\infty]$. 
Let $e^{2\pi imp^{-n}} \in \mathbb{Z}[p^\infty]$, $n,m \geq 0$. 
But $p^{-n} \in \mathbb{Q}_p$ and, whether or not $n >0$,
\[
\psi_p(p^{-n}) = e^{2\pi i\{p^{-n}\}_p} = e^{2\pi ip^{-n}},
\]
and $m p^{-n} \in \mathbb{Q}_p$, and using that $\psi_p$ is a homomorphism,
\[
\psi_p(mp^{-n}) = \psi_p(p^{-n})^m = e^{2\pi imp^{-n}}.
\]
This shows that $\psi_p(\mathbb{Q}_p) = \mathbb{Z}[p^\infty]$.

Finally, let $x \in \mathbb{Q}_p$. For $y \in B_{\leq 1}(x) = x+\mathbb{Z}_p$, 
so there is some $w \in \mathbb{Z}_p$ such that $y=x+w$. But $\psi_p(x+w)
=\psi_p(x) \psi_p(w)=\psi_p(x)$, so
\[
|\psi_p(y)-\psi_p(x)| = |\psi_p(x)-\psi_p(x)|=0,
\]
showing that $\psi_p$ is continuous at $x$. 
\end{proof}

Because $\mathbb{Z}[p^\infty]$ is discrete, it is immediate that $\psi_p$ is an open map.
The \textbf{first isomorphism theorem for topological groups} states that if
$G$ and $H$ are locally compact groups, $f:G \to H$ is a homomorphism of topological groups that is
onto and open, then $G/\ker f$ and $H$ are isomorphic as topological groups. 
Therefore the quotient group $\mathbb{Q}_p / \mathbb{Z}_p$ and the Pr\"ufer group $\mathbb{Z}[p^\infty]$
are isomorphic as topological groups.





\end{document}