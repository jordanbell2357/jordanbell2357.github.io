\documentclass{article}
\usepackage{amsmath,amssymb,graphicx,subfig,mathrsfs,amsthm}
\usepackage[draft]{hyperref}
%\usepackage{tikz-cd}
\newcommand{\inner}[2]{\left\langle #1, #2 \right\rangle}
\def\Re{\ensuremath{\mathrm{Re}}\,}
\newcommand{\tr}{\textrm{tr}} 
\newcommand{\abs}{\textrm{abs}} 
\newcommand{\Span}{\textrm{span}} 
\newcommand{\Hol}{\textrm{Hol}} 
\newcommand{\SA}{B_{\textrm{sa}}(H)} 
\newcommand{\positive}{B_{\textrm{+}}(H)} 
\newcommand{\id}{\textrm{id}} 
\newcommand{\norm}[1]{\left\Vert #1 \right\Vert}
\newtheorem{theorem}{Theorem}
\newtheorem{lemma}[theorem]{Lemma}
\newtheorem{corollary}[theorem]{Corollary}
\begin{document}
\title{The weak topology of locally convex spaces and the weak-* topology of their duals}
\author{Jordan Bell}
\date{April 3, 2014}
\maketitle

\section{Introduction}
These notes  give a summary of results that everyone who does work in functional analysis should know
about the weak topology on locally convex topological vector spaces and the weak-* topology on their dual spaces.
The most striking of the results we prove is Theorem \ref{weakbounded}, which shows that
a subset of a locally convex space is bounded if and only if it is weakly bounded. It is straightforward to prove that if a set
is bounded then it is weakly bounded, but to prove that if a set is weakly bounded then it is bounded we use the Hahn-Banach separation
theorem, the Banach-Alaoglu theorem, and the uniform boundedness principle.

If $X$ is a topological vector space then we will see that the weak topology on it is coarser than the original topology: any set that is open in the original
topology is open in the weak topology. From this it follows that it is easier for a sequence to converge in the weak topology than in the original topology:
 for a sequence to converge to a point means
that it is eventually contained in every neighborhood of the point, and a point has fewer neighborhoods in the weak topology than it does in the original
topology. The weak topology encodes information we may care about, and we may be able to establish
that certain sets are compact in the weak topology that are not compact in the original topology.


In these notes I first define the weak topology on a topological vector space $X$, and show that if $X$ is locally convex then $X$ with the weak topology is also a locally convex space.
 Indeed
a normed space is locally convex, but there are function spaces that we care about that are not normed spaces. For example, the set of holomorphic functions
on the open unit disc is a Fr\'echet space\footnote{A {\em Fr\'echet space} is a complete metrizable locally convex space.
It is a fact that a locally convex
 space is metrizable if and only its topology is induced by countably many of its seminorms. See John B. Conway, {\em A Course
 in Functional Analysis}, second ed., p.~105, chapter IV, Proposition 2.1.}  that is not normable. If $U$ is an open subset of $\mathbb{R}^n$, then $C^k(U)$ is a Fr\'echet space that is not normable, and the set of Schwartz functions on $U$ is also
a Fr\'echet space that is not normable. Moreover, none of the theorems stated for topological vector spaces and locally convex spaces is much easier to prove
in the case of normed spaces, and thus it is not a great waste of time to digest their statements in a larger category
of spaces.

Except for the results about normed spaces, the hypotheses of all the theorems that we present are satisfied for a separable metrizable locally convex space. The hypotheses of every result in this note are satisfied for separable reflexive Banach spaces, for example, $L^p(\mathbb{R}^n)$ for $1<p<\infty$.

If every instance of $\mathbb{C}$ in this text is replaced with $\mathbb{R}$, the resulting text does not require any further changes to be correct.



\section{Topological vector spaces}
\label{TVS}
If $X$ is a topological space and $x \in X$ and $\mathcal{B}$ is a set of open sets, we say that $\mathcal{B}$ is a {\em local basis at $x$} if each element of $\mathcal{B}$ includes $x$ and if for every open set
$U$ that includes $x$ there is some $V \in \mathcal{B}$ such that $V \subseteq U$. If for each $x \in X$ the set $\mathcal{B}_x$ is a local basis at $x$, then $\bigcup_{x \in X} \mathcal{B}_x$ is a basis for the topology of $X$.
If $X$ is a vector space, $\mathcal{B}$ is a set of subsets of $X$, and $x \in X$, we define
\[
x+\mathcal{B} = \{x+N: N \in \mathcal{B}\}, \qquad x+N = \{x+y : y \in N\}.
\]

If $X$ is a vector space over $\mathbb{C}$ with a topology $\mathcal{O}(X)$ such that $(X,\mathcal{O}(X))$ is Hausdorff and such that
addition $X \times X \to X$ and
scalar multiplication $\mathbb{C} \times X \to X$ are continuous, we say that $X$ is a {\em topological vector space}.
It is straightforward to prove that if $\mathcal{B}$ is a local basis at $0$ then $x+\mathcal{B}$ is a local basis at $x$, and so 
$\bigcup_{x \in X} \{x+\mathcal{B}\}$ is a basis for the topology $\mathcal{O}(X)$.
To specify a topology on a vector space it suffices to specify a local basis at $0$: This gives a basis by taking the union of the translates of the local basis over all $x \in X$, and then this basis generates a topology. However, $X$ might not be a topological vector space with the topology thus generated. (That is, if we define a topology on a vector space
by declaring certain sets including the origin to be open, the vector space need not be a topological vector space with this topology.)


A topological vector space $(X,\mathcal{O}(X))$ is said to be {\em locally convex} if there is a local basis at $0$ each element of which is convex.
 If $\nu_\alpha$ is a  family of seminorms on $X$, we define the {\em seminorm  topology  induced by this family} to be
 the coarsest topology on $X$ such that for all $x_0 \in X$ and $\alpha \in I$, the map
$x \mapsto \nu_\alpha(x-x_0)$ is continuous.
We say that a set of seminorms $\nu_\alpha$ is a {\em separating family} if $x \neq 0$ implies that there is some $\alpha$ such that $\nu_\alpha(x)>0$.
 One can prove that a topological vector space is locally convex if and only if its topology is induced by a separating family of  seminorms.\footnote{Walter Rudin,
{\em Functional Analysis}, second ed., p.~27, Theorem 1.36 and Theorem 1.37.}


Let $(X,\mathcal{O}(X))$ be a topological vector space over $\mathbb{C}$. The {\em dual space} $X^*$ is the set of all continuous linear maps
$(X,\mathcal{O}(X)) \to \mathbb{C}$. 
The {\em weak topology} on $X$, which we denote by $\mathcal{O}_w(X)$, is the initial topology for $X^*$. That is, $\mathcal{O}_w(X)$ is the coarsest topology on $X$ so that each element of
$X^*$ is continuous $(X,\mathcal{O}_w(X)) \to \mathbb{C}$. 
Equivalently, the weak topology on $X$ is the seminorm topology induced by the seminorms $|\alpha|$, $\alpha \in X^*$.
The topologies $\mathcal{O}(X)$ and $\mathcal{O}_w(X)$ are comparable, and $\mathcal{O}(X)$ is at least as fine as $\mathcal{O}_w(X)$. 
That is, $\mathcal{O}_w(X) \subseteq \mathcal{O}(X)$. 
A vague rule is that
the smaller $X^*$ is compared to the set of all linear maps $X \to \mathbb{C}$, the smaller $\mathcal{O}_w(X)$ will be compared to $\mathcal{O}(X)$.
If $X^*$ separates $X$
 then $(X,\mathcal{O}_w(X))$ is a locally convex topological vector space. It is locally convex because $\mathcal{O}_w(X)$ is induced by the separating family
of seminorms $|\alpha|$, $\alpha  \in X^*$.


\begin{itemize}
\item We say that $x_k \to x$ {\em weakly} if $x_k \to x$ in $(X,\mathcal{O}_w(X))$: 
for every  neighborhood $N$ of $x$ in $(X,\mathcal{O}_w(X))$, the sequence $x_k$ is eventually in $N$.
\item We say that $A$ is {\em weakly bounded} if $A$ is a bounded subset of $(X,\mathcal{O}_w(X))$:
for every neighborhood $N$ of $0$ in $(X,\mathcal{O}_w(X))$ there is some $c \geq 0$ such that $A \subseteq \{c x: x \in N\}=c
N$.
\end{itemize}


\section{Locally convex spaces}
The {\em Hahn-Banach theorem separation theorem}
 states the following.\footnote{Walter Rudin, {\em Functional Analysis}, second ed., p.~59, Theorem 3.4.}
 
 \begin{theorem}[Hahn-Banach separation theorem]
 If
 \begin{itemize}
 \item $X$ is a locally convex topological vector space over $\mathbb{C}$
 \item $A,B$ are disjoint, nonempty, closed, convex subsets of $X$
 \item $A$ is compact
 \end{itemize}
 then there is some $\lambda \in X^*$ and
$\gamma_1,\gamma_2 \in \mathbb{R}$ such that, for all $a \in A$ and $b \in B$,
\[
\Re \lambda(a) < \gamma_1<\gamma_2 < \Re \lambda(b).
\]
\label{hahnbanach}
\end{theorem}

It follows that if $X$ is locally convex then $X^*$ separates $X$.
We can also use the Hahn-Banach separation theorem to prove that in a locally
convex space, the weak closure of a convex set is equal to its original closure.\footnote{Walter Rudin, {\em Functional Analysis}, second ed.,
p.~66, Theorem 3.12.}

\begin{theorem}
If $E$ is a convex subset of a locally convex space $(X,\mathcal{O}(X))$, then the closure $\overline{E}$ of $E$
in $(X,\mathcal{O}(X))$ is equal to the closure $\overline{E}_w$ of $E$ in $(X,\mathcal{O}_w(X))$.
\label{weakclosure}
\end{theorem}
\begin{proof}
For $A \subseteq X$, denote $A^c = X \setminus A$. We have, as $\mathcal{O}_w(X)$ is coarser than $\mathcal{O}(X)$,
\begin{eqnarray*}
\overline{E}&=&\left( \bigcup_{U \in \mathcal{O}(X), E \subseteq U^c} U \right)^c\\
&\subseteq&\left( \bigcup_{U \in \mathcal{O}_w(X), E \subseteq U^c} U \right)^c\\
&=&\overline{E}_w.
\end{eqnarray*}

In the other direction, let $x_0 \not \in \overline{E}$. If we can show that $x_0 \not \in \overline{E}_w$,
this will show that $\overline{E}_w \subseteq \overline{E}$ and hence that $\overline{E}=\overline{E}_w$.
Let $A=\{x_0\}$ and $B=\overline{E}$, which satisfy the conditions of the Hahn-Banach theorem. Thus
there is some $\lambda \in X^*$ and $\gamma_1,\gamma_2 \in \mathbb{R}$ such that for all
$x \in \overline{E}$,
\[
\Re \lambda(x_0) < \gamma_1 < \gamma_2 < \Re \lambda(x).
\]
Let $U=\{x \in X: \Re \lambda(x)<\gamma_1\}$.
The set $V=\{z \in \mathbb{C}: \Re z < \gamma_1\}$ is an open subset of $\mathbb{C}$ and $U = \lambda^{-1}(V)$, 
so $U \in \mathcal{O}_w(X)$. For every $x \in \overline{E}$ we have $\Re \lambda(x) > \gamma_2>\gamma_1$, and for every
$x \in U$ we have $\Re \lambda(x)<\gamma_1$, so $\overline{E} \cap U = \emptyset$ and in particular
$E \cap U =\emptyset$. The three facts $x_0 \in U, U \in \mathcal{O}_w(X)$, and $E \cap U=\emptyset$ imply
that $x_0 \not \in \overline{E}_w$, completing the proof.
\end{proof}


If a sequence converges weakly, it  need not converge in the original topology.
{\em Mazur's theorem} shows that if a sequence in a metrizable locally
convex space converges weakly then 
there is a sequence in the convex hull of the original sequence that converges to the same limit as the weak limit of the original sequence.\footnote{Walter Rudin, {\em Functional Analysis}, second ed., p.~67, Theorem 3.13. } 


\begin{theorem}[Mazur's theorem]
Let $X$ be a metrizable locally convex space. If $x_n \to x$ weakly, then there is a sequence $y_i \in X$ such that
each $y_i$ is a convex combination of finitely many $x_n$ and such that $y_i \to x$. 
\end{theorem}
\begin{proof}
The {\em convex hull} of a subset $A$ of $X$ is the set of all convex combinations of finitely many elements of
$A$. The  convex hull of a set is convex and contains the set. Let $H$ be the convex hull
of the sequence $x_n$, and let $K$ be the weak closure of $H$.  
Since $x_n \to x$ weakly and $x_n \in H$,
it follows that $x \in K$. As $H$ is convex, Theorem \ref{weakclosure} tells us that $K= \overline{H}$, so
$x \in \overline{H}$. But $X$ is metrizable, so $x$ being in the closure of $H$ implies that there is a sequence $y_i \in H$ such that
$y_i \to x$.
This sequence
$y_i$ satisfies the claim.
\end{proof}



\section{Weak-* topology}
If $X$ is a vector space over $\mathbb{C}$ and $\mathcal{F}$ is a set of linear maps $X \to \mathbb{C}$
that separates $X$, and we give $X$ the initial topology for $\mathcal{F}$,
then one can prove that with this topology $X$ is a locally convex
space whose dual space is $\mathcal{F}$.\footnote{Walter Rudin, {\em Functional Analysis}, second ed, p.~64, Theorem 3.10.} 

Let $X$ be a topological vector space over $\mathbb{C}$, and for $x \in X$, define $f_x:X^* \to \mathbb{C}$
by $f_x(\lambda) = \lambda(x)$. $f_x$ is linear. If $\lambda_1,\lambda_2 \in X^*$ are distinct, then $\lambda_1-\lambda_2 \neq 0$ so
there is some $x \in X$ such that $(\lambda_1-\lambda_2)(x) \neq 0$, which tells us that $f_x(\lambda_1) \neq f_x(\lambda_2)$. Therefore
the set $\{f_x : x\in X\}$ separates $X^*$. Let $\mathcal{O}_X(X^*)$ be the initial topology for $\{f_x: x\in X\}$, and by the previous paragraph
we have that $(X^*,\mathcal{O}_X(X^*))$ is a locally convex space whose dual space is
$\{f_x:x \in X\}$. The topology $\mathcal{O}_X(X^*)$ is called the {\em weak-* topology}
on $X^*$.  We record what we've just said as a theorem to make it easier to look up.

\begin{theorem}
If $X$ is a topological vector space,  then its dual
$X^*$ with the weak-* topology $\mathcal{O}_X(X^*)$ is a locally convex space,
and the dual space of $(X^*,\mathcal{O}_X(X^*))$ is the set of $f_x:X^* \to \mathbb{C}$, where $f_x(\lambda)=\lambda(x)$, $x \in X, \lambda \in X^*$.
\label{weakstar}
\end{theorem}

The {\em Banach-Alaoglu theorem}\footnote{Walter Rudin, {\em Functional Analysis}, second
ed., p.~68, Theorem 3.15.} shows that certain subsets of $X^*$ are weak-* compact, i.e. they are compact subsets of $(X,\mathcal{O}_X(X^*))$. The set $K$ in the statement of the theorem is called
the {\em polar} of the set $V$.

\begin{theorem}[Banach-Alaoglu theorem]
If $X$ be a topological vector space and  $V \subseteq X$ is a neighborhood of $0$, then
\[
K=\{\lambda \in X^*:\textrm{if $x \in V$ then $|\lambda(x)| \leq 1$}\}
\]
 is a compact subset of $(X^*,\mathcal{O}_X(X^*))$.
\label{alaoglu}
\end{theorem}

The following theorem shows that
in a separable topological vector space, a weak-* compact set is weak-* metrizable.\footnote{Walter Rudin, {\em Functional Analysis}, second ed., p.~70, Theorem 3.16.}
This is useful because if possible we would like to characterize a topology
by its convergent sequences rather than by its open sets.

\begin{theorem}
If $X$ is a separable topological vector space and $K$ is a weak-* compact subset of $X^*$, then
$K$ with the subspace topology inherited from  $(X^*,\mathcal{O}_X(X^*))$ is metrizable.
\label{metrizable}
\end{theorem}
\begin{proof}
Let $\{x_n\}$ be a countable dense set in $X$, and define $f_n:X^* \to \mathbb{C}$ by
$f_n(\lambda)=\lambda (x_n)$. For each $n$,   $f_n:(X^*,\mathcal{O}_X(X^*)) \to \mathbb{C}$ is linear and continuous.
For distinct $\lambda_1,\lambda_2 \in X^*$, the set $U=\{x \in X: \lambda_1(x) \neq \lambda_2(x)\}$ is an open subset
of $X$.  As $U$ is a nonempty open set and $\{x_n\}$ is a dense subset of $X$, there is some $x_n \in U$, giving $f_n(\lambda_1) \neq f_n(\lambda_2)$. Therefore
$\{f_n:X^* \to \mathbb{C}\}$ separates $X^*$. It is a fact that if $Y$ is a compact topological space and there is a countable set of
continuous functions $Y \to \mathbb{C}$ that separates $Y$ then $Y$ is metrizable.\footnote{Walter Rudin, {\em Functional Analysis}, second ed.,
p.~63, \S 3.8, (c). In Rudin it is stated for real valued functions.}
 $K$ is a compact topological space with the subspace topology inherited from $\mathcal{O}_X(X^*)$, and hence this topology on
$K$ is metrizable.
\end{proof}

We can combine Theorem \ref{alaoglu} and Theorem \ref{metrizable} to get the following, which  states that the
polar of a neighborhood of $0$ in a separable topological vector space is weak-* sequentially compact.\footnote{Walter Rudin, {\em Functional Analysis}, second ed.,
p.~70, Theorem 3.17.}

\begin{theorem}
Let $X$ is a separable topological vector space. If $V$ is a neighborhood of $0$ and if the sequence $\lambda_n \in X^*$ satisfies
\[
|\lambda_n(x)| \leq 1, \qquad n \geq 1, x \in V,
\]
then there is a subsequence $\lambda_{a(n)}$ and some $\lambda \in X^*$ such that for all $x \in X$,
\[
\lim_{n \to \infty} \lambda_{a(n)}(x)= \lambda(x).
\]
\end{theorem}
\begin{proof}
The Banach-Alaoglu Theorem implies that the polar
\[
K=\{\lambda \in X^*:\textrm{if $x \in V$ then $|\lambda(x)| \leq 1$}\},
\]
is weak-* compact. $K$ with the subspace 
topology inherited from $\mathcal{O}_X(X^*)$ is compact, hence by Theorem \ref{metrizable} it is metrizable. 
Since the sequence $\lambda_n$ is contained in $K$, it has a  subsequence $\lambda_{a(n)}$ that converges weakly 
to some $\lambda \in K$. 
For each $x \in X$, the function $f_x:(X^*,\mathcal{O}_X(X^*)) \to \mathbb{C}$ defined by $f_x(\lambda)=\lambda(x)$
is continuous, hence for all $x \in X$ we have $f_x(\lambda_{a(n)}) \to f_x(\lambda)$, which is the claim.
\end{proof}

The following lemma gives a tractable characterization of weakly bounded sets.\footnote{Walter Rudin, {\em Functional Analysis}, second ed., p.~66.}


\begin{lemma}
If $X$ is a topological vector space and $E$ is a subset of $X$, then $E$ is weakly bounded if and only if 
for all $\lambda \in X^*$ there is some $\gamma(\lambda)$ such that if $x \in E$ then $|\lambda(x)| \leq \gamma(\lambda)$.
\label{weakboundedlemma}
\end{lemma}

The following theorem\footnote{Walter Rudin, {\em Functional Analysis}, second ed.,
p.~70, Theorem 3.18.} shows that in  a locally convex space, the boundedness of a set is equivalent to its weak boundedness.
Its proof involves much of what we have talked about so far, and actually
requires introducing some new terms with which one should probably have at least an  acquaintance. The statement of the theorem does not involve
the weak-* topology on $X^*$, but the proof uses the Banach-Alaoglu theorem.

\begin{theorem}[Weak boundedness is equivalent to boundedness]
If $X$ is locally convex and $E \subseteq X$, then $E$ is bounded in
$(X,\mathcal{O}(X))$ if and only if $E$ is bounded in $(X,\mathcal{O}_w(X))$.
\label{weakbounded}
\end{theorem}
\begin{proof}
Suppose that $E$ is bounded in $(X,\mathcal{O}(X))$ (for every neighborhood of $0$, the set $E$ is contained in some dilation of that neighborhood). Let $N$ be a neighborhood of $0$ in $\mathcal{O}_w(X)$. 
There is some $U \in \mathcal{O}_w(X)$ with $0 \in U \subseteq N$. Now, $U \in \mathcal{O}(X)$, so $N$ is a neighborhood
of $x$ in $\mathcal{O}(X)$. As $E$ is bounded, there is some $\alpha$ such that $E \subseteq \alpha N =  \{\alpha x:x \in N\}$,
which is what it means for  $E$ to be bounded in $(X,\mathcal{O}_w(X))$. 

Suppose that $E$ is bounded in $(X,\mathcal{O}_w(X))$. Let $N$ be a neighborhood of $0$ in 
$(X,\mathcal{O}(X))$. We want to show that there is some $\alpha$ such that $E \subseteq \{\alpha x: x \in N\}$. 
If $\mathcal{B}$ is a local basis at a point in a topological vector space, then every element of $\mathcal{B}$ contains
the closure of some element of $\mathcal{B}$.\footnote{Walter Rudin, {\em Functional Analysis}, second ed., p.~11, Theorem 1.11.}
A subset $A$ of a vector space is said to be {\em balanced} if for every $\alpha \in \mathbb{C}$ with $|\alpha|\leq 1$ we have
$\alpha A \subseteq A$. It is a fact that in a topological vector space, every convex neighborhood of $0$ contains a convex balanced neighborhood of $0$.\footnote{Walter
Rudin, {\em Functional Analysis}, second ed., p.~12, Theorem 1.14.} The purpose of having said all of this is the following: since $N$ is a neighborhood
of $0$ in  $\mathcal{O}(X)$, there is a balanced convex $V$ that is a neighborhood of $0$ in $(X,\mathcal{O}(X))$ such that
$\overline{V} \subseteq N$, where $\overline{V}$ is the closure of $V$ in $(X,\mathcal{O}(X))$.
Moreover, it is a fact that the closure of a balanced set is itself balanced, and the closure of a convex set is itself convex,\footnote{Walter
Rudin, {\em Functional Analysis}, second ed., p.~11, Theorem 1.13.} so $\overline{V}$ is a balanced convex set that is contained in $N$. Thus to prove $E \subseteq \alpha N$
it suffices to prove that $E \subseteq \alpha \overline{V}$, and this is what we will do.

Let
\[
K=\{\lambda \in X^*:\textrm{if $x \in V$ then $|\lambda(x)| \leq 1$}\},
\]
 the polar of $V$. If $x \in V$ and $\lambda \in K$ then $|\lambda(x)| \leq 1$, so 
\[
V \subseteq \{x \in X: \textrm{if $\lambda \in K$ then $|\lambda(x)| \leq 1$}\}=\bigcap_{\lambda \in K} \{x \in X: |\lambda(x)| \leq 1\}.
\]
The right-hand side is an intersection of closed sets in $(X,\mathcal{O}(X))$, so it is closed. Therefore $\overline{V}$ is contained in the right-hand side.
Furthermore, it is a consequence of the Hahn-Banach separation theorem\footnote{Walter
Rudin, {\em Functional Analysis}, second ed., p.~61, Theorem 3.7.}  that if $B$ is a convex balanced closed set in a locally
convex space and $x_0 \not \in B$, then there is some $\lambda \in X^*$ such that $\lambda (x_0) > 1$ and $|\lambda(x)| \leq 1$ for $x \in B$.
Thus, if $x_0 \not \in \overline{V}$, then
there is some $\lambda \in X^*$ such that $\lambda (x_0)>1$ and if 
$x \in \overline{V}$
then $|\lambda(x)| \leq 1$. From this it follows that $\lambda \in K$, and then because $\lambda(x_0)>1$ it follows that
\[
x_0 \not \in \{x \in X: \textrm{if $\lambda \in K$ then $|\lambda(x)| \leq 1$}\}.
\]
We have shown that if $x_0 \not \in \overline{V}$ then $x_0$ is not an element of the polar of $K$, and
therefore 
\[
\overline{V}= \{x \in X: \textrm{if $\lambda \in K$ then $|\lambda(x)| \leq 1$}\}.
\]

By the Banach-Alaoglu theorem, $K$ is a compact set in $(X^*,\mathcal{O}_X(X^*))$, and one checks that $K$ is convex.
The {\em uniform boundedness principle}\footnote{Walter Rudin, {\em Functional Analysis}, second ed., p.~46, Theorem 2.9}
states that if 
\begin{itemize}
\item $W$ and $Z$ are topological vector spaces
\item $K$ is a compact convex set in $W$
\item $\Gamma$ is a set of continuous linear maps $W \to Z$
\item For each $w \in K$, the set $\Gamma(w)=\{g(w):g \in \Gamma\}$ is a bounded subset of $Z$
\end{itemize}
then there is a bounded set $B$ in $Z$ such that if $g \in \Gamma$ then $g(K) \subseteq
B$.
For $x \in E$, define $g_x:X^* \to \mathbb{C}$ by $g_x(\lambda)=\lambda(x)$, and define $\Gamma=\{g_x:x \in E\}$.
Because $E$ is weakly bounded,  for all $\lambda \in X^*$ there is some $\gamma(\lambda)$ such that
if $x \in E$ then $|\lambda(x)| \leq \gamma(\lambda)$; this is by  Lemma \ref{weakboundedlemma}.
Hence for all $\lambda \in X^*$ there is some $\gamma(\lambda)$ such that
 the set $\Gamma(\lambda)$
is contained in the closed disc in $\mathbb{C}$ with radius $\gamma(\lambda)$. 
Thus  $\Gamma(\lambda)$ is bounded in $\mathbb{C}$.
We apply the uniform boundedness principle using $W=(X^*,\mathcal{O}_X(X^*))$ and $Z=\mathbb{C}$. 
We thus obtain that there is some
$0 \leq \gamma<\infty$ such that if $g \in \Gamma$ then $g(K)$ is contained in the closed disc of radius $\gamma$. That is,
for all $x \in E$ and $\lambda \in K$, we have
\[
|g_x(\lambda)| \leq \gamma,
\] 
i.e. for all $x \in E$ and $\lambda \in K$ we have
\[
|\lambda(x)| \leq \gamma.
\]
Therefore,
if $x \in E$ and $\lambda \in K$, then 
\[
\left| \lambda\left( \frac{1}{\gamma} \cdot x\right) \right| = \frac{1}{\gamma} |\lambda(x)| \leq 1,
\]
which means that $\frac{1}{\gamma}\cdot x \in \overline{V}$. This is true for all $x \in E$, so $\frac{1}{\gamma}\cdot E \subseteq \overline{V} \subseteq N$, and
hence $E \subseteq \gamma\cdot N$. Since $N$ was an arbitrary
neighborhood of $0$ in $(X,\mathcal{O}(X))$, we have satisfied  the definition of the set $E$ being bounded in
$(X,\mathcal{O}(X))$, completing the proof.
\end{proof}

The final result we state in this section gives a condition for the dual of a locally convex space to be weak-* separable.\footnote{Walter Rudin, {\em Functional Analysis},
second ed., p.~90, chapter 3, Exercise 28.} We already stated
in Theorem \ref{weakstar} that the dual with the weak-* topology of a topological vector space is itself a locally
convex space.

\begin{theorem}
If $X$ is a separable metrizable locally convex space, then $(X^*,\mathcal{O}_X(X^*))$ is a separable locally convex space.
\end{theorem}



\section{Normed spaces}
Let $X$ be a normed space with norm $\norm{\cdot}$.
The topology on $X$ is the coarsest topology such that for each $x_0 \in X$, the map $x \mapsto \norm{x-x_0}$ is continuous.
A normed vector space is locally convex and is metrizable, with metric $d(x,y)=\norm{x-y}$.


If $X$ and $Y$ are topological vector spaces and $T:X \to Y$ is a linear map, we say that $T$ is {\em bounded} if $E$ being bounded
in $X$ implies that $T(E)$ is bounded in $Y$.\footnote{A statement close to it is proved in Walter Rudin,
{\em Functional Analysis}, second ed., p.~24, Theorem 1.32.}

\begin{theorem}
Let $X$ and $Y$ be normed vector spaces and let $T:X \to Y$ be linear. The following three statements are equivalent:
\begin{itemize}
\item $T$ is continuous.
\item $T$ is bounded.
\item  There is some $\gamma$ such that if $x \in X$ then $\norm{Tx} \leq \gamma \norm{x}$.
\end{itemize}
\label{bounded}
\end{theorem}

If $X$ and $Y$ are normed spaces and $T:X \to Y$ is a bounded linear map,
the {\em operator norm} $\norm{T}$ of $T$ is defined to be the infimum of those $\gamma$ such that
if $x \in X$ then $\norm{Tx} \leq \gamma \norm{x}$. $\norm{T}=\sup_{\norm{x} \leq 1} \norm{Tx}$. 
The set of bounded linear maps $X \to Y$ is denoted $\mathscr{B}(X,Y)$.
 $\mathscr{B}(X,Y)$
is a normed space, and  if $Y$ is a Banach space then $\mathscr{B}(X,Y)$ is a Banach space.\footnote{Walter Rudin, {\em Functional Analysis}, second ed., p.~92, Theorem 4.}
$X^*=\mathscr{B}(X,\mathbb{C})$, and since $\mathbb{C}$ is a Banach space, with the operator
norm $X^*$ is a Banach space.


The following theorem rewrites 
 Theorem \ref{weakbounded} in the terminology of norms.\footnote{Walter Rudin, {\em Functional Analysis}, second ed., p.~71.}

\begin{theorem}[Weak boundedness is equivalent to boundedness]
If $X$ is a normed space and $E$ is a subset of $X$, then
\[
\sup_{x \in E} |\lambda(x)| < \infty
\]
for all $\lambda \in X^*$  if and only if
 there is some $\gamma$ such that  $x \in E$ implies
\[
\norm{x} \leq \gamma.
\]
\label{normbound}
\end{theorem}
\begin{proof}
Suppose that $\sup_{x \in E} |\lambda(x)| < \infty$ holds for all $\lambda \in X^*$. By 
 Lemma \ref{weakboundedlemma}, 
 this means that the set $E$ is weakly bounded. But by Theorem \ref{weakbounded} this implies that $E$ is bounded. 
The closed unit ball is a neighborhood
of $0$ in $(X,\mathcal{O}(X))$, so, as $E$ is bounded, there is some $\gamma$ such that $E \subseteq \{\gamma x: \norm{x} \leq 1\}$.
Hence, if $x \in E$ then $\norm{x} \leq \gamma$.

Suppose that there is some $\gamma$ such that $x \in E$ implies that $\norm{x} \leq \gamma$, and let
$\lambda \in X^*$. Because $\lambda$ is continuous, for $x \in E$ we have
\[
|\lambda(x)| \leq \norm{\lambda} \norm{x} \leq \norm{\lambda} \gamma<\infty.
\]
\end{proof}






\begin{theorem}[Banach-Alaoglu theorem]
If $X$ be a normed vector space, then  $B=\{\lambda \in X^*: \norm{\lambda} \leq 1\}$  is a compact
subset of $(X,\mathcal{O}_X(X^*))$.
\end{theorem}
\begin{proof}
It is a fact\footnote{Walter Rudin, {\em Functional Analysis}, second ed., p.~94, Theorem 4.3.} (proved using the Hahn-Banach extension theorem)
that for every $x \in X$,
\[
\norm{x}=\sup \{|\lambda(x)|: \lambda \in B\}.
\]
From this and Theorem \ref{alaoglu}, it follows that
$B$ is weak-* compact in $X^*$.
\end{proof}

The {\em Eberlein-Smulian theorem} states that a set in a normed space is weakly
compact if and only if the set is weakly sequentially compact.\footnote{Robert E. Megginson,
{\em An Introduction to Banach Space Theory},
p.~248, Theorem 2.8.6.}

\begin{theorem}[Eberlein-Smulian theorem]
If $X$ is a normed space, then
a subset $A$ of $X$ is compact in $\mathcal{O}_w(X)$ if and only if every sequence in $A$ has a subsequence that
converges in $\mathcal{O}_w(X)$ to an element of $A$. 
\end{theorem}


\section{Banach spaces}
We say that a Banach   space $X$ is {\em reflexive} if $X^{**}=X$, where $X^*=\mathscr{B}(X,\mathbb{C})$
and $X^{**}=\mathscr{B}(X^*, \mathbb{C})$. (Although it makes sense to talk about $X^*$ for a normed space, $X^*$ is itself a Banach space
and so too is $X^{**}$, hence if a normed space were reflexive then it would have to be a Banach space.)


 {\em Kakutani's theorem} relates the property of a Banach space being reflexive with
weak compactness.\footnote{Joseph Diestel,
{\em Sequences and Series in Banach Spaces}, p.~18, chapter III.}


\begin{theorem}[Kakutani's theorem]
The closed unit ball in a Banach space is weakly compact if and only if the space is reflexive.
\end{theorem}

Thus, combining the Eberlein-Smulian theorem and Kakutani's theorem we get that a Banach space is reflexive if and only if the closed unit ball is weakly
sequentially compact.


The following theorem states that a Banach space is separable if and only if the closed unit ball in the dual space is weak-* metrizable.\footnote{
John B. Conway, {\em A Course in Functional Analysis}, second ed., p.~134, Theorem 5.1.}
 
 \begin{theorem}
 Let $X$ be a Banach space and let $B=\{\lambda \in X^*: \norm{\lambda} \leq 1\}$. $X$ is separable if and only if $B$ with the subspace topology inherited
 from $(X^*,\mathcal{O}_X(X^*))$ is metrizable.
 \end{theorem}




\section{Hilbert spaces}
Let $H$ be a Hilbert space with inner product $\inner{\cdot}{\cdot}:H \times H \to \mathbb{C}$, linear in the first argument.  
The {\em Riesz representation theorem}\footnote{John B. Conway, {\em A Course in Functional Analysis}, second ed.,
p.~13, Theorem 3.4.} states that if $\lambda \in H^*$, then there is a unique $h_\lambda \in H$ such that
if $h \in H$ then 
\[
\lambda(h)=\inner{h}{h_\lambda},
\]
and $\norm{\lambda}=\norm{h_\lambda}$.
On the other hand, if $h_0 \in H$ then $\lambda(h)=\inner{h}{h_0}$ satisfies, by the Cauchy-Schwarz inequality,
\[
|\lambda_{h_0}(h)| = |\inner{h}{h_0}| \leq \norm{h} \norm{h_0}, 
\]
hence, by Theorem \ref{bounded}, $\lambda_{h_0} \in H^*$. Thus $H^*=\{\lambda_{h_0}: h_0 \in H\}$.
The weak topology $\mathcal{O}_w(H)$ on $H$ is the coarsest topology on $H$ such that
each $\lambda \in H^*$ is continuous $(H,\mathcal{O}_w(H)) \to \mathbb{C}$, hence such that
for each $h_0 \in H$,  the funtion $h \mapsto \inner{h}{h_0}$ is continuous $(H,\mathcal{O}_w(H)) \to \mathbb{C}$.
Thus, a net $h_\alpha \in H$ converges to $h \in H$ in $\mathcal{O}_w(H)$ if and only if for all $h_0 \in H$ we have that
 $\inner{h_\alpha}{h_0} \in \mathbb{C}$ converges to  $\inner{h}{h_0} \in \mathbb{C}$, and this characterizes the weak topology on $H$.





\end{document}