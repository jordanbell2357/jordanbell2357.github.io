\documentclass{article}
\usepackage{amsmath,amssymb,mathrsfs,amsthm}
%\usepackage{tikz-cd}
%\usepackage{hyperref}
\newcommand{\inner}[2]{\left\langle #1, #2 \right\rangle}
\newcommand{\tr}{\ensuremath\mathrm{tr}\,} 
\newcommand{\Span}{\ensuremath\mathrm{span}} 
\def\Re{\ensuremath{\mathrm{Re}}\,}
\def\Im{\ensuremath{\mathrm{Im}}\,}
\newcommand{\id}{\ensuremath\mathrm{id}} 
\newcommand{\var}{\ensuremath\mathrm{var}} 
\newcommand{\Lip}{\ensuremath\mathrm{Lip}} 
\newcommand{\Hilb}{\ensuremath\mathrm{Hilb}} 
\newcommand{\GL}{\ensuremath\mathrm{GL}} 
\newcommand{\diam}{\ensuremath\mathrm{diam}} 
\newcommand{\sgn}{\ensuremath\mathrm{sgn}\,} 
\newcommand{\lcm}{\ensuremath\mathrm{lcm}} 
\newcommand{\supp}{\ensuremath\mathrm{supp}\,}
\newcommand{\dom}{\ensuremath\mathrm{dom}\,}
\newcommand{\upto}{\nearrow}
\newcommand{\downto}{\searrow}
\newcommand{\norm}[1]{\left\Vert #1 \right\Vert}
\newtheorem{theorem}{Theorem}
\newtheorem{lemma}[theorem]{Lemma}
\newtheorem{proposition}[theorem]{Proposition}
\newtheorem{corollary}[theorem]{Corollary}
\theoremstyle{definition}
\newtheorem{definition}[theorem]{Definition}
\newtheorem{example}[theorem]{Example}
\begin{document}
\title{Unbounded operators in a Hilbert space and the Trotter product formula}
\author{Jordan Bell}
\date{August 25, 2015}

\maketitle

\section{Unbounded operators}
Let $H$ be a Hilbert space with inner product $\inner{\cdot}{\cdot}$. We do not assume that $H$ is separable.
By an \textbf{operator in $H$} we mean 
a linear subspace
$\mathscr{D}(T)$ of $H$ and a linear map
$T:\mathscr{D}(T) \to H$.
We define
\[
\mathscr{R}(T) = \{Tx: x \in \mathscr{D}(T)\}.
\]
If $\mathscr{D}(T)$ is dense in $H$ we say that \textbf{$T$ is densely defined}.

Write
\[
\mathscr{G}(T) = \{(x,y) \in H \times H: x \in \mathscr{D}(T), y = Tx\}.
\]
When $\mathscr{G}(T) \subset \mathscr{G}(S)$, we write
\[
T \subset S,
\]
and say that \textbf{$S$ is an extension of $T$}.
If $\mathscr{G}(T)$ is a closed linear subspace of $H \times H$, we say that \textbf{$T$ is closed}.

We say that an operator $T$ in $H$ is \textbf{closable} if there is a closed operator $S$ in $H$ such that
$T \subset S$. If $T$ is closable, one proves that there is a unique closed operator $\overline{T}$ in $H$ with
$T \subset \overline{T}$ and such that if $S$ is a closed operator satisfying $T \subset S$ then
$\overline{T} \subset S$. 

Suppose that $T$ is a densely defined operator in $H$. We define $\mathscr{D}(T^*)$ to be
the set of those $y \in H$ for which 
\[
x \mapsto \inner{Tx}{y}, \qquad x \in \mathscr{D}(T),
\]
is continuous. For $y \in \mathscr{D}(T^*)$, by the Hahn-Banach theorem there is some $\lambda_y
\in H^*$ such that
\[
\lambda_y x = \inner{Tx}{y}, \qquad x \in \mathscr{D}(T).
\]
Next, by the Riesz representation theorem, there is a unique 
$x_y \in H$ such that
\[
\lambda_y x = \inner{x}{x_y}, \qquad x \in H,
\]
and hence
\[
\inner{x}{x_y} = \inner{Tx}{y}, \qquad x \in \mathscr{D}(T).
\]
If $v \in H$ satisfies
\[
\inner{x}{v} = \inner{Tx}{y}, \qquad x \in \mathscr{D}(T),
\]
then 
\[
\inner{x}{v} = \inner{x}{x_y}, \qquad x \in \mathscr{D}(T),
\]
and because $\mathscr{D}(T)$ is dense in $H$ this implies that $v=x_y$.
We define
$T^*:\mathscr{D}(T^*) \to H$ by $T^*y = x_y$, which satisfies
\[
\inner{Tx}{y} = \inner{x}{T^*y}, \qquad x \in \mathscr{D}(T).
\]
$T^*$ is called \textbf{the adjoint of $T$}.
One checks that $\mathscr{D}(T^*)$ is a linear subspace of $H$ and that
$T^*:\mathscr{D}(T^*) \to H$ is a linear map.
We say that \textbf{$T$ is self-adjoint} when $T=T^*$.

For operators $S$ and $T$ in $H$ we define
\[
\mathscr{D}(S+T) = \mathscr{D}(S) \cap \mathscr{D}(T)
\]
and
\[
\mathscr{D}(ST) = \{x \in \mathscr{D}(T): Tx \in \mathscr{D}(S)\}.
\]
One checks that
\[
(R+S)+T = R+(S+T), \qquad (RS)T=R(ST),
\]
and
\[
RT+ST = (R+S)T, \qquad TR+TS \subset T(R+S).
\]
We now determine the adjoint of products of densely defined operators.\footnote{Walter Rudin,
{\em Functional Analysis}, second ed., p.~348, Theorem 13.2.}

\begin{theorem}
If $S$, $T$, and $ST$ are densely defined operators in $H$, then
\[
T^*S^* \subset (ST)^*.
\]
If $S \in \mathscr{B}(H)$, then
\[
T^*S^* = (ST)^*.
\]
\end{theorem}
\begin{proof}
Let $y \in \mathscr{D}(T^*S^*)$ and let $x \in \mathscr{D}(ST)$. 
Then $S^* y \in \mathscr{D}(T^*)$ and $x \in \mathscr{D}(T)$, so
\[
\inner{Tx}{S^*y} = \inner{x}{T^* S^*y}.
\]
On the other hand, $y \in \mathscr{D}(S^*)$, so
\[
\inner{STx}{y} = \inner{Tx}{S^*y}.
\]
Hence
\[
\inner{STx}{y} = \inner{x}{T^* S^*y},
\]
which implies that $(ST)^*y = T^*S^*y$ for each $y \in \mathscr{D}(T^*S^*)$, that is,
$T^*S^* \subset (ST)^*$.

Suppose that $S \in \mathscr{B}(H)$, hence
$S^* \in \mathscr{B}(H)$, for which $\mathscr{D}(S^*)=H$.
Let $y \in \mathscr{D}((ST)^*)$.
For $x \in \mathscr{D}(ST)$,
\[
\inner{Tx}{S^*y} = \inner{STx}{y} = \inner{x}{(ST)^*y}.
\]
This implies that $S^*y \in \mathscr{D}(T^*)$ and hence $y \in \mathscr{D}(T^*S^*)$, showing
\[
\mathscr{D}((ST)^*) \subset \mathscr{D}(T^*S^*).
\] 
\end{proof}

If $T$ is an operator in $H$, we say that \textbf{$T$ is symmetric} if
\[
\inner{Tx}{y} = \inner{x}{Ty}, \qquad x,y \in \mathscr{D}(T).
\]

\begin{theorem}
Let $T$ be a densely defined operator in $H$. $T$ is symmetric if and only if 
$T \subset T^*$.
\end{theorem}
\begin{proof}
Suppose that $T$ is symmetric and let $(y,Ty) \in \mathscr{G}(T)$. For $x \in \mathscr{D}(T)$,
\[
|\inner{Tx}{y}| =  |\inner{x}{Ty}| \leq \norm{x} \norm{Ty},
\]
hence $x \mapsto \inner{Tx}{y}$ is continuous on $\mathscr{D}(T)$, i.e.
 $y \in \mathscr{D}(T^*)$. For $x \in \mathscr{D}(T)$, 
on the one hand,
\[
 \inner{Tx}{y} = \inner{x}{T^* y},
 \]
 and on the other hand,
 \[
 \inner{Tx}{y} = \inner{x}{Ty}.
 \]
 Therefore $\inner{x}{T^*y}=\inner{x}{Ty}$ for all $x \in \mathscr{D}(T)$, and because $\mathscr{D}(T)$ is dense
 in $H$ we get that $T^*y=Ty$, i.e.
 $(y,Ty) \in \mathscr{G}(T^*)$. Therefore
 $\mathscr{G}(T) \subset \mathscr{G}(T^*)$.
 
 Suppose that $\mathscr{G}(T) \subset \mathscr{G}(T^*)$. 
 Let $x,y \in \mathscr{D}(T)$. We have
 $(y,Ty) \in \mathscr{G}(T^*)$, i.e.
 $y \in \mathscr{D}(T^*)$ and
 $T^*y= Ty$. Hence 
  \[
 \inner{Tx}{y} = 
 \inner{x}{T^*y} = \inner{x}{Ty},
 \]
 showing that $T$ is symmetric.
\end{proof}

One proves that if $T$ is a symmetric operator in $H$ then $T$ is closable and $\overline{T}$ is symmetric.
An operator $T$ in $H$ is said to be \textbf{essentially self-adjoint} when $T$ is densely defined, symmetric, and
$\overline{T}$ (which is densely defined) is self-adjoint.

\section{Graphs}
For $(a,b), (c,d) \in H \times H$, we define 
\[
\inner{(a,b)}{(c,d)} = \inner{a}{c}+\inner{b}{d}.
\]
This is an inner product on $H \times H$ with which $H \times H$ is a Hilbert space. 
We define $V:H \times H \to H \times H$ by
\[
V(a,b) = (-b,a), \qquad (a,b) \in H \times H,
\]
which belongs to $\mathscr{B}(H \times H)$. It is immediate that $VV^*=I$ and $V^*V=I$, namely, 
\textbf{$V$ is unitary}. As well, $V^2=-I$, whence if $M$ is a linear subspace of $H \times H$ then $V^2 M=M$. 
The following theorem relates the graphs of a densely defined operator and its adjoint.\footnote{Walter
Rudin, {\em Functional Analysis}, second ed., p.~352, Theorem 13.8.}

\begin{theorem}
Suppose that $T$ is a densely defined operator in $H$. It holds that
\[
\mathscr{G}(T^*) = \left( V \mathscr{G}(T) \right)^\perp.
\]
\label{perp}
\end{theorem}


\begin{theorem}
If $T$ is a densely defined operator in $H$, then $T^*$ is a closed operator.
\label{adjointclosed}
\end{theorem}
\begin{proof}
$V \mathscr{G}(T)$ is a linear subspace of $H \times H$. 
The orthogonal complement of a linear subspace of a Hilbert space is a closed linear subspace of the Hilbert space, and thus
Theorem \ref{perp} tells us that $\mathscr{G}(T^*)$ is a closed linear subspace of $H \times H$, namely,
$T^*$ is a closed operator.
\end{proof}

Let $T$ be a densely defined operator in $H$. If $T$ is self-adjoint, then the above theorem tells us that
$T$ is itself a closed operator. 


\begin{theorem}
Suppose that $T$ is a closed densely defined  operator in $H$. Then
\[
H \times H = V\mathscr{G}(T) \oplus \mathscr{G}(T^*)
\]
is an orthogonal direct sum.
\end{theorem}
\begin{proof}
Generally, if $M$ is a linear subspace of $H \times H$,
\[
H \times H = \overline{M} \oplus M^\perp = \overline{M} \oplus (\overline{M})^\perp
\]
is an orthogonal direct sum. For $M = V \mathscr{G}(T)$, because $\mathscr{G}(T)$ is a closed linear subspace of $H \times H$, so
is $M$. Thus
\[
H \times H = V \mathscr{G}(T) \oplus (V \mathscr{G}(T))^\perp.
\]
By Theorem \ref{perp}, this is
\[
H \times H = V \mathscr{G}(T) \oplus \mathscr{G}(T^*),
\]
proving the claim.
\end{proof}


If $T$ is an operator in $H$ that is one-to-one, we define $\mathscr{D}(T^{-1})=\mathscr{R}(T)$, and
$T^{-1}$ is a densely defined operator with domain $\mathscr{D}(T^{-1})$. 


The following theorem establishes several properties of symmetric densely defined operators.\footnote{Walter
Rudin, {\em Functional Analysis}, second ed., p.~353, Theorem 13.11.} We remind ourselves that if $T$ is an operator in $H$,
the statement $\mathscr{D}(T)=H$ means that $T$ is a linear map $H \to H$, from which it does not follow that $T$ is continuous.

\begin{theorem}
Suppose that $T$ is a densely defined symmetric operator in $H$. Then the following statements are true:
\begin{enumerate}
\item If $\mathscr{D}(T)=H$ then $T$ is self-adjoint and $T \in \mathscr{B}(H)$.
\item If $T$ is self-adjoint and one-to-one, then $\mathscr{R}(T)$ is dense in $H$ and
$T^{-1}$ is densely defined and self-adjoint.
\item If $\mathscr{R}(T)$ is dense in $H$, then $T$ is one-to-one.
\item If $\mathscr{R}(T)=H$, then $T$ is self-adjoint and $T^{-1} \in \mathscr{B}(H)$.
\end{enumerate}
\end{theorem}

If $T \in \mathscr{B}(H)$ then $T^{**}=T$. The following theorem says that this is true for
closed densely defined operators.\footnote{Walter
Rudin, {\em Functional Analysis}, second ed., p.~354, Theorem 13.12.}

\begin{theorem}
If $T$ is a closed densely defined operator in $H$, then $\mathscr{D}(T^*)$ is dense in $H$ and $T^{**}=T$.
\end{theorem}


The following theorem gives statements about $I+T^*T$ when $T$ is a closed densely
defined operator.\footnote{Walter
Rudin, {\em Functional Analysis}, second ed., p.~354, Theorem 13.13.}

\begin{theorem}
Suppose that $T$ is a closed densely defined operator in $H$ and let $Q=I+T^*T$, with
\[
\mathscr{D}(Q) = \mathscr{D}(T^*T) = \{x \in \mathscr{D}(T): Tx \in \mathscr{D}(T^*)\}.
\]
The following statements are true:
\begin{enumerate}
\item $Q:\mathscr{D}(Q) \to H$ is a bijection, and there are $B, C \in \mathscr{B}(H)$ with
$\norm{B} \leq 1$, $B \geq 0$, $\norm{C} \leq 1$, $C=TB$, and
\[
B(I+T^*T) \subset (I+T^*T)B=I.
\]
$T^*T$ is self-adjoint.
\item Let $T_0$ be the restriction of $T$ to $\mathscr{D}(T^*T)$. Then $\mathscr{G}(T_0)$ is dense in
$\mathscr{G}(T)$.
\end{enumerate}
\end{theorem}

Let $T$ be a symmetric operator in $H$. We say that $T$ is \textbf{maximally symmetric} if 
$T \subset S$ and $S$ being symmetric imply that $S=T$. One proves that a self-adjoint operator
is maximally symmetric.\footnote{Walter
Rudin, {\em Functional Analysis}, second ed., p.~356, Theorem 13.15.}


The following theorem is about $T+iI$ when $T$ is a symmetric operator in $H$.\footnote{Walter
Rudin, {\em Functional Analysis}, second ed., p.~356, Theorem 13.16.}

\begin{theorem}
Suppose that $T$ is a symmetric operator in $H$ and let $j$ be $i$ or $-i$. Then:
\begin{enumerate}
\item $\norm{Tx+jx}^2=\norm{x}^2 + \norm{Tx}^2$ for $x \in \mathscr{D}(T)$.
\item $T$ is closed if and only if $\mathscr{R}(T+jI)$ is a closed subset of $H$.
\item $T+jI$ is one-to-one.
\item If $\mathscr{R}(T+jI)=H$ then $T$ is maximally symmetric.
\end{enumerate}
\label{iI}
\end{theorem}


\section{The Cayley transform}
Let $T$ be a symmetric operator in $H$ and define
\[
\mathscr{D}(U) = \mathscr{R}(T+iI).
\] 
Theorem \ref{iI} tells us that $T+iI$ is one-to-one. Because
\[
\mathscr{D}(T-iI)=\mathscr{D}(T)=\mathscr{D}(T-iI)
\]
and $\mathscr{D}((T+iI)^{-1})=\mathscr{R}(T+iI)$,
\begin{align*}
\mathscr{D}((T-iI)(T+iI)^{-1})&=\{x \in \mathscr{R}(T+iI): (T+iI)^{-1}x \in \mathscr{D}(T)\}\\
&=\{x \in \mathscr{R}(T+iI): (T+iI)^{-1}x \in \mathscr{D}(T+iI)\}\\
&=\mathscr{R}(T+iI)\\
&=\mathscr{D}(U).
\end{align*}
We define
\[
U = (T-iI)(T+iI)^{-1}.
\]
$U$ is called the \textbf{Cayley transform of $T$}.

We have
\[
\mathscr{R}(U) = U\mathscr{D}(U) = U\mathscr{R}(T+iI)=(T-iI)(T+iI)^{-1}\mathscr{R}(T+iI)
=(T-iI)\mathscr{D}(T+iI),
\]
and $\mathscr{D}(T+iI)=\mathscr{D}(T)=\mathscr{D}(T-iI)$ so
\[
\mathscr{R}(U) = (T-iI) \mathscr{D}(T-iI) = \mathscr{R}(T-iI).
\]
Also, for $x \in \mathscr{D}(T)$, Theorem \ref{iI} tells us
\[
\norm{(T+iI)x}^2 = \norm{Tx+ix}^2 = \norm{x}^2+\norm{Tx}^2=
\norm{Tx-ix}^2 = \norm{(T-iI)x}^2,
\]
hence for $x \in \mathscr{D}(U)$, for which $(T+iI)^{-1}x \in \mathscr{D}(T+iI)=\mathscr{D}(T)$,
\[
\norm{Ux} = \norm{(T-iI)(T+iI)^{-1}x}
=\norm{(T+iI)(T+iI)^{-1}x}
=\norm{x},
\]
showing that $U$ is an \textbf{isometry in $H$}.


The Cayley transform of a symmetric operator in $H$ (which we do not presume to be densely defined) has the following properties.\footnote{Walter
Rudin, {\em Functional Analysis}, second ed., p.~385, Theorem 13.19.}

\begin{theorem}
Suppose that $T$ is a symmetric operator in $H$. Then:
\begin{enumerate}
\item $U$ is closed if and only if $T$ is closed.
\item $\mathscr{R}(I-U)=\mathscr{D}(T)$, $I-U$ is one-to-one, and 
\[
T=i(I+U)(I-U)^{-1}.
\]
\item $U$ is unitary if and only if $T$ is self-adjoint.
\end{enumerate}

If $V$ is an operator in $H$ that is an isometry and $I-V$ is one-to-one, then there is a symmetric operator $S$ in $H$ such that
$V$ is the Cayley transform of $S$.
\end{theorem}



\section{Resolvents}
Let $T$ be an operator in $H$. The \textbf{resolvent set of $T$}, denoted $\rho(T)$, is the set of those $\lambda \in \mathbb{C}$ such
that $T-\lambda I:\mathscr{D}(T) \to H$ is a bijection and $(T-\lambda I)^{-1} \in \mathscr{B}(H)$. That is, $\lambda \in \rho(T)$ if and
only if there is some $S \in \mathscr{B}(H)$ such that
\[
S(T-\lambda I) \subset (T-\lambda I)S = I.
\]
We call $R:\rho(T) \to \mathscr{B}(H)$ defined by 
\[
R(\lambda) = (T-\lambda I)^{-1}
\]
the \textbf{resolvent  of $T$}.
The \textbf{spectrum of $T$} is $\sigma(T)=\mathbb{C} \setminus \rho(T)$.  It is a fact that 
$\rho(T)$ is open, that $\sigma(T)$ is closed, and that
if $\sigma(T) \neq \mathbb{C}$ then
$T$ is a closed operator, that
\[
R(z)-R(w) = (z-w)R(z)R(w), \qquad z,w \in \rho(T),
\]
and
\[
\frac{d^n R}{dz^n}(z) = n! R^{n+1}(z), \qquad z \in \rho(T).
\]

If $T$ is a self-adjoint operator in $H$, one proves that $\sigma(T) \subset \mathbb{R}$. 


\section{Resolutions of the identity}
Let $(\Omega,\mathscr{S})$ be a measurable space. A \textbf{resolution of the identity} is a function
\[
E:\mathscr{S} \to \mathscr{B}(H)
\]
satisfying:
\begin{enumerate}
\item $E(\emptyset)=0$, $E(\Omega)=I$.
\item For each $a \in \mathscr{S}$, $E(a)$ is a self-adjoint projection.
\item $E(a \cap b) = E(a) E(b)$.
\item If $a \cap b = \emptyset$, then $E(a \cup b) = E(a)+E(b)$.
\item For each $x,y \in H$, the function $E_{x,y}:\mathscr{S} \to \mathbb{C}$ defined by
\[
E_{x,y}(a) = \inner{E(a)x}{y}, \qquad a \in \mathscr{S},
\]
is a complex measure on $\mathscr{S}$.
\end{enumerate}

We check that if $a_n \in \mathscr{S}$ and $E(a_n)=0$ for each $n=1,2,\ldots$, then for $a=\bigcup_{n=1}^\infty a_n$,
$E(a)=0$. 

Let $\{D_i\}$ be a countable collection of open discs that is a base for the topology of $\mathbb{C}$, i.e.,
$\bigcup D_i = \mathbb{C}$ and for each $i,j$ and for $z \in D_i \cap D_j$, there is some $k$ such that
$x \in D_k \subset D_i \cap D_j$. 
Let $f:(\Omega,\mathscr{S}) \to (\mathbb{C},\mathscr{B}_{\mathbb{C}})$ be a measurable function and let
$V$ be the union of those $D_i$ for which $E(f^{-1}(D_i))=0$. Then
$E(f^{-1}(V))=0$. The \textbf{essential range of $f$} is $\mathbb{C} \setminus V$, and we say that
$f$ is \textbf{essentially bounded} if the essential range of $f$ is a bounded subset of $\mathbb{C}$. We define 
the \textbf{essential supremum of $f$} to be
\[
\norm{f}_\infty = \sup\{|\lambda|: \lambda \in \mathbb{C} \setminus V\}.
\]
Now define $B$ to be the collection of bounded  measurable functions $(\Omega,\mathscr{S}) \to (\mathbb{C},\mathscr{B}_{\mathbb{C}})$, which is a Banach
algebra with the norm
\[
\sup\{|f(\omega): \omega \in \Omega\},
\]
for which
\[
N = \{f \in B: \norm{f}_\infty =0\}
\]
is a closed ideal. Then $B/N$ is a Banach algebra, denoted $L^\infty(E)$, with the norm
\[
\norm{f+N}_\infty = \norm{f}_\infty.
\]
The unity of $L^\infty(E)$ is $1+N$.
Because $L^\infty(E)$ is a Banach algebra, it makes sense to speak about the spectrum of an element of $L^\infty(E)$. For
$f+N \in L^\infty(E)$, the spectrum of $f+N$ is the set of those $\lambda \in \mathbb{C}$ for which there is no $g+N \in L^\infty(E)$ satisfying
$(g+N)(f+N-\lambda(1+N))=1+N$. Check that the spectrum of $f+N$ is equal to the essential range of $g$, for any $g \in f+N$. 


A subset $A$ of $\mathscr{B}(H)$ is said to be \textbf{normal} when $ST=TS$ for all $S,T \in A$ and $T \in A$ implies that
$T^* \in A$.\footnote{Walter Rudin, {\em Functional Analysis}, second ed., p.~319, Theorem 12.21.} (To say that $T \in \mathscr{B}(H)$
is normal means that $TT^*=T^*T$, and this is equivalent to the statement that the set $\{T,T^*\}$ is normal.)

\begin{theorem}
If $(\Omega,\mathscr{S})$ is a measurable space and $E:\mathscr{S} \to H$ is a resolution of the identity, then there
is a closed normal subalgebra $A$ of $\mathscr{B}(H)$ and
 a unique isometric
$^*$-isomorphism $\Psi:L^\infty(E) \to A$ such that 
\[
\inner{\Psi(f)x}{y} = \int_{\Omega} f dE_{x,y}, \qquad f \in L^\infty(E), \quad x,y \in H.
\]
Furthermore,
\[
\norm{\Psi(f)x}^2 = \int_{\Omega} |f|^2 dE_{x,x}, \qquad f \in L^\infty(E), \quad x \in H.
\]
\label{starisomorphism}
\end{theorem}

For $f \in L^\infty(E)$, we define
\[
\int_\Omega f dE = \Psi(f).
\]

For $L^\infty(E)$, $\sigma(\Psi(f))$ is equal to the essential range of
 $f$.\footnote{Walter Rudin, {\em Functional Analysis}, second ed., p.~366, Theorem 13.27.}
 




\section{The spectral theorem}
The following is the spectral theorem for self-adjoint operators.\footnote{Walter Rudin, {\em Functional Analysis}, second ed., p.~368, Theorem 13.30.}

\begin{theorem}
If $T$ is a self-adjoint operator in $H$, then there is a unique resolution of the identity
\[
E: \mathscr{B}_{\mathbb{R}} \to \mathscr{B}(H)
\]
such that 
\[
\inner{Tx}{y} = \int_{\mathbb{R}} \lambda dE_{x,y}(\lambda), \qquad x \in \mathscr{D}(T), \quad y \in H.
\]
This resolution of the identity satisfies $E(\sigma(T))=I$. 
\label{spectral}
\end{theorem}

If $T$ is a self-adjoint operator in $H$
applying the spectral theorem and then
 Theorem \ref{starisomorphism},
 we get that there is a closed normal subalgebra $A$ of $\mathscr{B}(H)$ and a unique isometric
 $^*$-isomorphism $\Psi:L^\infty(E) \to A$ such that
 \[
 \inner{\Psi(f)x}{y} = \int_{\sigma(T)} f(\lambda) dE_{x,y}(\lambda), \qquad f \in L^\infty(E), \quad x,y\in H.
 \]
For $t \in \mathbb{R}$ and 
 $f_t:\sigma(T) \to \mathbb{C}$ defined by
 $f_t(\lambda)=e^{it\lambda}$, this defines
 \[
e^{itT} = \Psi(f_t) =  \int_{\sigma(T)} e^{it\lambda} dE(\lambda).
 \]

Because $\Psi$ is a $^*$-homomorphism, for $t \in \mathbb{R}$ we have
\[
\Psi(f_t)^* \Psi(f_t) = \Psi(\overline{f_t}) \Psi(f_t) = \Psi(f_{-t}) \Psi(f_t)
=\Psi(f_{-t} f_t) = \Psi(f_0)=I,
\] 
and likewise $\Psi(f_t) \Psi(f_t)^*=I$, showing that $e^{itT}=\Psi(f_t)$ is unitary. 
We denote by
$\mathscr{U}(H)$
the collection of unitary elements of $\mathscr{B}(H)$. $\mathscr{U}(H)$  is a subgroup of the group of invertible
elements of $\mathscr{B}(H)$.

 
Furthermore, because $\Psi$ is a $^*$-homomorphism, for $t \in \mathbb{R}$ we have
\[
I = \Psi(f_0) = \Psi(f_t f_{-t}) = \Psi(f_t) \Psi(f_{-t}) = e^{itT} e^{i(-t)T},
\] 
and for $s,t \in \mathbb{R}$ we have
\[
e^{isT} e^{itT} = \Psi(f_s) \Psi(f_t) = \Psi(f_s f_t) = \Psi(f_{s+t}) = e^{i(s+t)T},
\]
showing that $t \mapsto e^{itT}$ is a one-parameter group $\mathbb{R} \to \mathscr{B}(H)$. 
 
For $t \in \mathbb{R}$ and $x \in H$, by Theorem \ref{starisomorphism} we have
\[
\norm{\Psi_t x - x}^2 = \norm{\Psi(f_t-1)x}^2 = \int_{\sigma(T)} |f_t-1|^2 dE_{x,x}
=\int_{\sigma(T)} |e^{it\lambda}-1|^2 dE_{x,x}(\lambda).
\]
For each $\lambda \in \sigma(T)$, $|e^{it\lambda}-1|^2 \to 0$ as $t \to 0$, and thus we get
by the dominated convergence theorem 
\[
\int_{\sigma(T)} |e^{it\lambda}-1|^2 dE_{x,x}(\lambda) \to 0, \qquad t \to 0.
\]
That is, for each $x \in H$,
\[
\norm{e^{itT}x-x} \to 0
\]
as $t \to 0$, showing that $t \mapsto e^{itT}$ is \textbf{strongly continuous}, i.e. $t \mapsto e^{itT}$ is continuous
$\mathbb{R} \to \mathscr{B}(H)$ where $\mathscr{B}(H)$ has the strong operator topology.

Conversely, \textbf{Stone's theorem on one-parameter unitary groups}\footnote{cf. Walter Rudin, {\em Functional Analysis}, second ed., p.~382, Theorem 38.}
 states that if $\{U_t: t \in \mathbb{R}\}$ is a strongly continuous one-parameter group of
bounded unitary operators on $H$, then there is a unique self-adjoint operator $A$ in $H$ such that $U_t = e^{itA}$ for each $t \in \mathbb{R}$.

For $t \neq 0$, define $g_t:\sigma(T) \to \mathbb{C}$ by $g_t(\lambda) = \frac{e^{it\lambda}-1}{t}$.
By Theorem \ref{spectral},  for $x \in \mathscr{D}(T)$ and $y \in H$,
\[
\inner{iTx}{y} = i\inner{Tx}{y} = i\int_{\mathbb{R}} \lambda dE_{x,y}(\lambda)
\]
and 
by Theorem \ref{starisomorphism},
\[
\inner{\Psi(g_t)x}{y} = \int_{\sigma(T)} g_t dE_{x,y} = \int_{\sigma(T)} \frac{e^{it\lambda}-1}{t} dE_{x,y}(\lambda),
\]
so
\[
\inner{\Psi(g_t)x-iTx}{y} = \int_{\sigma(T)} \left( \frac{e^{it\lambda}-1}{t} - i\lambda \right) dE_{x,y}(\lambda).
\]
For each $\lambda \in \sigma(T)$, $\frac{e^{it\lambda}-1}{t} - i\lambda \to 0$ as $t \to 0$, and 
for each $t$,
\[
\left|\frac{e^{it\lambda}-1}{t} - i\lambda\right| \leq \left|\frac{e^{it\lambda}-1}{t}\right| + |\lambda|
\leq 2|\lambda|,
\]
and as $x \in \mathscr{D}(T)$, by Theorem \ref{spectral} we have that $\lambda \mapsto |\lambda|$ belongs
 to $L^1(E_{x,y})$. Thus by the dominated convergence theorem,
\[
\inner{\Psi(g_t)x-iTx}{y} = \int_{\sigma(T)} \left( \frac{e^{it\lambda}-1}{t} - i\lambda \right) dE_{x,y}(\lambda) \to 0
\]
as $t \to 0$. In particular, 
\[
\norm{\Psi(g_t)x-iTx}^2 \to 0
\]
as $t \to 0$. That is, for each $x \in \mathscr{D}(T)$,
\[
\frac{e^{itT}x-x}{t} \to iTx
\]
as $t \to 0$. In other words, $iT$ is the \textbf{infinitesimal generator} of the one-parameter group
$e^{itT}$.\footnote{cf. Walter Rudin, {\em Functional Analysis}, second ed., p.~376, Theorem 13.35.} We remark that because
$T^*=T$, the adjoint
of $iT$ is $(iT)^*=\overline{i} T^*=-iT^*=-iT=-(iT)$. 


\section{Trotter product formula}
We remind ourselves that for an operator $T$ in $H$ to be closed means that $\mathscr{G}(T)$ is a closed linear subspace
of $H \times H$.

\begin{theorem}
Let $T$ be an operator in $H$. $T$ is closed if and only if the linear space $\mathscr{D}(T)$ with the norm
\[
\norm{x}_T = \norm{x}+\norm{Tx}.
\]
is a Banach space.
\label{DT}
\end{theorem}


The following is the \textbf{Trotter product formula},
which shows that if $A$, $B$, and $A+B$ are self-adjoint operators  in a  Hilbert space, then
for each $t$, 
$(e^{itA/n} e^{itB/n})^n$ converges strongly to
$e^{it(A+B)}$  as $n \to \infty$.\footnote{Barry Simon, {\em Functional Integration and Quantum Physics}, p.~4, Theorem 1.1; 
Konrad Schm\"udgen, {\em Unbounded Self-adjoint Operators on Hilbert Space}, p.~122, Theorem 6.4.}

\begin{theorem}
Let $H$ be a  Hilbert space, not necessarily separable. If $A$ and $B$ are self-adjoint operators in $H$ such that $A+B$ is a self-adjoint 
operator in $H$, then for each $t \in \mathbb{R}$ and for each $\psi \in H$,
\[
e^{it(A+B)}\psi = \lim_{n \to \infty} \left((e^{itA/n} e^{itb/n})^n \psi \right).
\]
\end{theorem}
\begin{proof}
The claim is immediate for $t=0$, and we prove the claim for $t > 0$; it is straightforward to obtain the claim for $t<0$ using the truth of the claim
for $t>0$.
Let $D = \mathscr{D}(A+B) = \mathscr{D}(A) \cap \mathscr{D}(B)$. Because $A+B$ is self-adjoint, $A+B$ is closed (Theorem \ref{adjointclosed}), so
by Theorem \ref{DT}, the linear space $D$ with the norm $\norm{\phi}_{A+B} = \norm{\phi}+\norm{(A+B)\phi}$ is a Banach space. 
Because $D$ is a Banach space, the uniform boundedness principle\footnote{Walter Rudin, {\em Functional Analysis}, second ed., p.~45, Theorem 2.6.}
tells us that if $\Gamma$ is a collection of bounded linear maps $D \to H$ and if for each $\phi \in D$ the set
$\{\gamma \phi: \gamma \in \Gamma\}$ is bounded in $H$, then the set $\{\norm{\gamma}:\gamma \in \Gamma\}$ is bounded, i.e.
there is some $C$ such that $\norm{\gamma \phi} \leq C\norm{\phi}_{A+B}$ for all $\gamma \in \Gamma$ and all $\phi \in D$.

For $s \in \mathbb{R}$,
let $S_s = e^{is(A+B)}$, $V_s=e^{isA}$, $W_s=e^{isB}$, $U_s=V_sW_s$, which each belong to
$\mathscr{B}(H)$.
For $n \geq 1$, 
\[
\sum_{j=0}^{n-1} U_{t/n}^j(S_{t/n}-U_{t/n})S_{t/n}^{n-j-1}
=U_{t/n}^n-S_{t/n}^n=U_{t/n}^n-S_t,
\]
so, because a product of unitary operators is a unitary operator and a unitary operator has operator norm $1$ and
also using the fact that $S_{t/n}^{n-j-1}=S_{t-\frac{j+1}{n}}$, for $\xi \in H$ we have
\begin{align*}
\norm{(S_t-U_{t/n}^n)\xi}&=\norm{\sum_{j=0}^{n-1} U_{t/n}^j(S_{t/n}-U_{t/n})S_{t/n}^{n-j-1}\xi}\\
&\leq \sum_{j=0}^{n-1} \norm{(S_{t/n}-U_{t/n})S_{t/n}^{n-j-1}\xi}\\
&=\sum_{j=0}^{n-1} \norm{(S_{t/n}-U_{t/n}) S_{t-\frac{j+1}{n}} \xi}\\
&\leq \sum_{j=0}^{n-1} \sup_{0 \leq s \leq t} \norm{(S_{t/n}-U_{t/n})S_s \xi}.
\end{align*}
That is, 
\begin{equation}
\norm{(S_t-U_{t/n}^n)\xi} \leq n\sup_{0 \leq s \leq t} \norm{(S_{t/n}-U_{t/n})S_s \xi},
\qquad \xi \in H, \quad n \geq 1.
\label{tsup}
\end{equation}

Let $\phi \in D$. On the one hand, because $i(A+B)$ is the infinitesimal generator of $\{S_s: s \in \mathbb{R}\}$, we have
\begin{equation}
\frac{S_s-I}{s} \phi \to i(A+B)\phi, \qquad s \downarrow 0.
\label{Slimit}
\end{equation}
On the other hand, for $s \neq 0$ we have, because an infinitesimal generator of a one-parameter group 
commutes with each element of the one-parameter group,
\[
V_s (iB\phi) + V_s \left(\frac{W_s-I}{s}-iB\right)\phi
+\frac{V_s-I}{s}\phi
=\frac{U_s-I}{s}\phi,
\]
and as $V_s$ converges strongly to $I$ as $s \downarrow 0$ and as $iB$ is the infinitesimal generator of
the one-parameter group $\{W_s: s \in \mathbb{R}\}$ and
$iA$ is the infinitesimal generator of the one-parameter group $\{V_s: s \in \mathbb{R}\}$,
\[
V_s (iB\phi) + V_s \left(\frac{W_s-I}{s}-iB\right)\phi
+\frac{V_s-I}{s}\phi \to
iB\phi+iA\phi
\]
as $s \downarrow 0$, i.e.
\begin{equation}
\frac{U_s-I}{s}\phi \to i(A+B)\phi, \qquad s \downarrow 0.
\label{Ulimit}
\end{equation}
Using \eqref{Slimit} and \eqref{Ulimit}, we get that for each $\phi \in D$, 
\[
\frac{S_s-U_s}{s} \phi \to 0, \qquad s \downarrow 0.
\]
Therefore, for each $\phi \in D$, with $s=t/n$ we have
\[
\frac{n}{t}(S_{t/n}-U_{t/n})\phi \to 0, \qquad n \to \infty,
\]
equivalently ($t$ is fixed for this whole theorem),
\begin{equation}
\lim_{n \to \infty}  \norm{n(S_{t/n}-U_{t/n})\phi}=0, \qquad \phi \in D.
\label{Dlimit}
\end{equation}

For each $n \geq 1$, define $\gamma_n:D \to H$ by $\gamma_n = n(S_{t/n}-U_{t/n})$. Each $\gamma_n$ is a linear map, 
and for $\phi \in D$,
\[
\norm{\gamma_n \phi} 
\leq n\norm{S_{t/n}\phi}+n\norm{U_{t/n}\phi}
\leq n\norm{\phi}+n\norm{\phi}
\leq 2n\norm{\phi}_{A+B},
\]
showing that each $\gamma_n$ is a bounded linear map $D \to H$, where $D$ is a Banach space
with the norm $\norm{\phi}_{A+B} = \norm{\phi}+\norm{(A+B)\phi}$. 
Moreover, 
\eqref{Dlimit} shows that for each $\phi \in D$,
there is some $C_\phi$ such that
\[
\norm{\gamma_n \phi} \leq C_\phi, \qquad n \geq 1.
\]
Then applying the uniform boundedness principle, we get that there is some $C>0$ such that for all $n \geq 1$ and for all $\phi \in D$,
\[
\norm{\gamma_n \phi} \leq C \norm{\phi}_{A+B},
\]
i.e. 
\begin{equation}
\norm{n(S_{t/n}-U_{t/n})\phi} \leq C \norm{\phi}_{A+B}, \qquad n \geq 1, \quad \phi \in D.
\label{steinhaus}
\end{equation}

Let $K$ be a compact subset of $D$, where $D$ is a Banach space with the norm $\norm{\phi}_{A+B}=\norm{\phi}+\norm{(A+B)\phi}$. 
Then $K$ is totally bounded, so for any $\epsilon>0$, there are $\phi_1,\ldots,\phi_M \in K$ such that
$K \subset \bigcup_{m=1}^M B_{\epsilon/C}(\phi_m)$.
By \eqref{Dlimit}, for each $m$, $1 \leq m \leq M$, there is some $n_m$ such that when $n \geq n_m$,
\[
\norm{n(S_{t/n}-U_{t/n})\phi_m} \leq \epsilon.
\]
Let $N=\max\{n_1,\ldots,n_M\}$. 
For $n \geq N$ and
for $\phi \in D$, there is some $m$ for which
$\norm{\phi-\phi_m}_{A+B} < \frac{\epsilon}{C}$, 
and using \eqref{steinhaus}, as $\phi-\phi_m \in D$, we get
\begin{align*}
\norm{n(S_{t/n}-U_{t/n})\phi}&\leq \norm{n(S_{t/n}-U_{t/n})(\phi-\phi_m)}
+\norm{n(S_{t/n}-U_{t/n})\phi_m}\\
&\leq C\norm{\phi-\phi_m}_{A+B} + \epsilon\\
&<\epsilon+\epsilon.
\end{align*}
This shows that any compact subset $K$ of $D$ and $\epsilon>0$, there is some $n_\epsilon$ such that if
$n \geq n_\epsilon$ and $\phi \in K$, then 
\begin{equation}
\norm{n(S_{t/n}-U_{t/n})\phi}<\epsilon.
\label{compact}
\end{equation}

Let $\phi \in D$,
let  $s_0 \in \mathbb{R}$, and let $\epsilon>0$. 
Because $s \mapsto S_s$ is strongly continuous $\mathbb{R} \to \mathscr{B}(H)$, there is some $\delta_1>0$ such that
when $|s-s_0|<\delta_1$, $\norm{S_s \phi - S_{s_0} \phi}<\epsilon$, and there is some
$\delta_2>0$ such that when $|s-s_0|<\delta_2$, $\norm{S_s (A+B)\phi- S_{s_0}(A+B)\phi}<\epsilon$, and hence
with $\delta=\min\{\delta_1,\delta_2\}$, when $|s-s_0|<\delta$ we have
\begin{align*}
\norm{S_s \phi - S_{s_0} \phi}_{A+B}&=\norm{S_s \phi - S_{s_0}\phi}+
\norm{(A+B)(S_s \phi - S_{s_0}\phi)}\\
&=\norm{S_s \phi - S_{s_0}\phi}+
\norm{S_s(A+B) \phi - S_{s_0}(A+B)\phi)}\\
&<\epsilon+\epsilon,
\end{align*}
showing that $s \mapsto S_s \phi$ is continuous $\mathbb{R} \to D$. Therefore
$\{S_s \phi: 0 \leq s \leq t\}$ is a compact subset of $D$, so applying \eqref{compact} we get that
for any $\epsilon>0$, there is some $n_\epsilon$ such that if $n \geq n_\epsilon$ and $0 \leq s \leq t$, then
\[
\norm{n(S_{t/n}-U_{t/n}) S_s \phi}<\epsilon,
\]
and therefore if $n \geq n_\epsilon$ then
\begin{equation}
\sup_{0 \leq s \leq t} \norm{n(S_{t/n}-U_{t/n}) S_s \phi} \leq \epsilon.
\label{phiinequality}
\end{equation}

Finally, let $\epsilon>0$. The statement that $A+B$ is self-adjoint in $H$ entails the statement that $D$ is dense in $H$,
so there is some $\phi \in D$ such that $\norm{\phi-\psi}<\epsilon$. 
For $n \geq 1$,
\begin{align*}
\norm{(S_t-U_{t/n}^n) \psi}&\leq \norm{(S_t-U_{t/n}^n) (\psi-\phi)}
+\norm{(S_t-U_{t/n}^n) \phi}\\
&\leq 2 \norm{\psi-\phi}
+\norm{(S_t-U_{t/n}^n) \phi}\\
&<\epsilon+\norm{(S_t-U_{t/n}^n) \phi}.
\end{align*}
Using
 \eqref{tsup} with $\xi=\phi$ and then using \eqref{phiinequality}, there is some $n_\epsilon$ such that
 when $n \geq n_\epsilon$,
\[
\norm{(S_t-U_{t/n}^n)\phi} \leq n\sup_{0 \leq s \leq t} \norm{(S_{t/n}-U_{t/n})S_s \phi} \leq \epsilon.
\]
Therefore for $n \geq n_\epsilon$,
\[
\norm{(S_t-U_{t/n}^n) \psi} < 2\epsilon,
\]
proving the claim.
\end{proof}


\end{document}