\documentclass{article}
\usepackage{amsmath,amssymb,graphicx,subfig,mathrsfs,amsthm}
%\usepackage{tikz-cd}
\usepackage[draft]{hyperref}
\newcommand{\inner}[2]{\left\langle #1, #2 \right\rangle}
\newcommand{\tr}{\ensuremath\mathrm{tr}\,} 
\newcommand{\Span}{\ensuremath\mathrm{span}} 
\def\Re{\ensuremath{\mathrm{Re}}\,}
\def\Im{\ensuremath{\mathrm{Im}}\,}
\newcommand{\id}{\ensuremath\mathrm{id}} 
\newcommand{\rank}{\ensuremath\mathrm{rank\,}} 
\newcommand{\Res}{\ensuremath\mathrm{Res}} 
\newcommand{\diam}{\ensuremath\mathrm{diam}} 
\newcommand{\supp}{\ensuremath\mathrm{supp}\,}
\newcommand{\sech}{\ensuremath\mathrm{sech}\,}
\newcommand{\dom}{\ensuremath\mathrm{dom}\,}
\newcommand{\upto}{\nearrow}
\newcommand{\downto}{\searrow}
\newcommand{\norm}[1]{\left\Vert #1 \right\Vert}
\newtheorem{theorem}{Theorem}
\newtheorem{lemma}[theorem]{Lemma}
\newtheorem{proposition}[theorem]{Proposition}
\newtheorem{corollary}[theorem]{Corollary}
\theoremstyle{definition}
\newtheorem{definition}[theorem]{Definition}
\newtheorem{example}[theorem]{Example}
\begin{document}
\title{A series of  secants}
\author{Jordan Bell}
\date{November 3, 2014}

\maketitle

Let $\mathfrak{H}=\{\tau \in \mathbb{C}: \Im \tau >0\}$. Define $C:\mathfrak{H} \to \mathbb{C}$ by
\[
C(\tau) = 2 \sum_{n=-\infty}^\infty \frac{1}{e^{\pi i n\tau}+q^{-\pi i n\tau}}
=\sum_{n=-\infty}^\infty \sec \pi n \tau, \qquad \tau \in \mathfrak{H}.
\]
We take as granted that $C$ is holomorphic on $\mathfrak{H}$. 


First we calculate the Fourier transform of $x \mapsto \sech \pi x$.\footnote{Elias M. Stein and Rami Shakarchi, {\em Complex Analysis},
p.~81, Example 3.}

\begin{lemma}
For $\xi \in \mathbb{R}$,
\[
\int_{-\infty}^\infty e^{-2\pi i\xi x} \sech \pi x dx = \sech \pi \xi.
\]
\end{lemma}
\begin{proof}
Let $\xi \in \mathbb{R}$ and define
\[
f(z)=\frac{e^{-2\pi iz\xi}}{\cosh \pi z}.
\]
The poles of $f$ are those $z$ at which $\cosh \pi z=0$, thus $z=n i+\frac{i}{2}$, $n \in \mathbb{Z}$.  
Taking $\gamma_R$ to be the contour going from $-R$ to $R$, from $R$ to $R+2i$, from $R+2i$ to $-R+2i$, and from
$-R+2i$ to $-R$, the poles of $f$ inside $\gamma_R$ are $\frac{i}{2}$ and $\frac{3i}{2}$. 
Because $(\cosh \pi z)'=\pi \sinh \pi z$, we work out
\[
\Res_{z=i/2} f(z) = \frac{e^{-2\pi i \cdot \frac{i}{2} \xi}}{\pi \sinh \pi \frac{i}{2}}
=\frac{e^{\pi \xi}}{\pi i \sin \frac{\pi}{2}}
=\frac{e^{\pi \xi}}{\pi i}
\]
and
\[
\Res_{z=3i/2} f(z) = \frac{e^{-2\pi i\cdot \frac{3i}{2} \xi}}{\pi \sinh \pi \frac{3i}{2}}
=\frac{e^{3\pi \xi}}{\pi i \sin \frac{3\pi}{2}}
=\frac{e^{3\pi \xi}}{-\pi i}.
\]

We bound the integrals on the vertical sides as follows.
For $z=-R+iy$,
\[
| \cosh \pi z| = \frac{|e^{\pi z}+e^{-\pi z}|}{2}
\geq \frac{||e^{\pi z}|-|e^{-\pi z}||}{2}
=\frac{|e^{-R\pi} - e^{R\pi}|}{2}
=\frac{e^{R\pi}-e^{-R\pi}}{2},
\]
and, for $0 \leq y \leq 2$,
\[
|e^{-2\pi iz\xi}| = e^{2\pi y\xi} = e^{4\pi \xi}.
\]
For $z=R+iy$,
\[
| \cosh \pi z| = \frac{|e^{\pi z}+e^{-\pi z}|}{2}
\geq \frac{||e^{\pi z}|-|e^{-\pi z}||}{2}
=\frac{|e^{R\pi} - e^{-R\pi}|}{2}
=\frac{e^{R\pi}-e^{-R\pi}}{2},
\]
and, for $0 \leq y \leq 2$,
\[
|e^{-2\pi iz\xi}| = e^{2\pi y\xi} = e^{4\pi \xi}.
\]
Therefore
\[
\left| \int_{-R}^{-R+2i} f(z) dz \right|
\leq \int_{-R}^{-R+2i} |f(z)| dz
\leq 2\cdot  e^{4\pi \xi} \cdot \frac{2}{e^{R\pi}-e^{-R\pi}}
=\frac{e^{4\pi \xi}}{e^{R\pi}-e^{-R\pi}}
\]
and likewise
\[
\left| \int_{R}^{R+2i} f(z) dz \right| \leq \frac{e^{4\pi \xi}}{e^{R\pi}-e^{-R\pi}}.
\]
As $R \to \infty$, each of these tends to $0$.
Therefore,
\[
\int_{-\infty}^\infty f(z) dz + \int_{\infty+2i}^{-\infty+2i} f(z)dz 
=2\pi i \left(\frac{e^{\pi \xi}}{\pi i}+\frac{e^{3\pi \xi}}{-\pi i}\right)
=-2 e^{2\pi \xi}(e^{\pi \xi}-e^{-\pi \xi}),
\]
i.e.,
\[
\int_{-\infty}^\infty f(z) dz = 
 \int_{-\infty+2i}^{\infty+2i} f(z)dz 
 -2 e^{2\pi \xi}(e^{\pi \xi}-e^{-\pi \xi}).
\]
For the top horizontal side, 
\begin{align*}
\int_{-R+2i}^{R+2i} f(z) dz 
&=\int_{-R}^R \frac{e^{-2\pi i(x+2i)\xi}}{\cosh (\pi x+2\pi i)} dx\\
&=\int_{-R}^R \frac{e^{-2\pi ix\xi} e^{4\pi \xi}}{\cosh (\pi x)\cosh(2\pi i)+\sinh(\pi x)\sinh(2\pi i)} dx\\
&=e^{4\pi \xi} \int_{-R}^R \frac{e^{-2\pi ix\xi}}{\cosh \pi x} dx\\
&=e^{4\pi \xi} \int_{-R}^R f(x) dx.
\end{align*}
Writing
\[
I=\int_{-\infty}^\infty f(z) dz,
\]
this gives us
\[
I = e^{4\pi \xi} I  -2 e^{2\pi \xi}(e^{\pi \xi}-e^{-\pi \xi}),
\]
and so
\[
I=-2 e^{2\pi \xi} \frac{e^{\pi \xi}-e^{-\pi \xi}}{1-e^{4\pi \xi}}
=2 \frac{e^{\pi \xi}-e^{-\pi \xi}}{e^{2\pi \xi}-e^{-2\pi \xi}}
=2 \frac{e^{\pi \xi}-e^{-\pi \xi}}{(e^{\pi \xi}-e^{-\pi \xi})(e^{\pi \xi}+e^{-\pi \xi})}
=\sech \pi \xi,
\]
which is what we wanted to show.
\end{proof}


\begin{corollary}
For $t>0$ and $a \in \mathbb{R}$,
\[
\int_{-\infty}^\infty e^{-2\pi i\xi x} e^{-2\pi iax} \sech \frac{\pi x}{t} dx = t \sech (\pi(\xi+a)t),
\qquad \xi \in \mathbb{R}.
\]
\label{sechfourier}
\end{corollary}
\begin{proof}
\begin{align*}
\int_{-\infty}^\infty e^{-2\pi ix\xi} e^{-2\pi iax} \sech \frac{\pi x}{t} dx
&=\int_{-\infty}^\infty e^{-2\pi i(\xi+a)x} \sech \frac{\pi x}{t} dx\\
&=t \int_{-\infty}^\infty e^{-2\pi i(\xi +a) tx} \sech \pi x dx\\
&=t \sech (\pi(\xi+a)t).
\end{align*}
\end{proof}



\begin{theorem}
For all $\tau \in \mathfrak{H}$,
\[
C(\tau) = \frac{i}{\tau} C\left( -\frac{1}{\tau} \right).
\]
\end{theorem}
\begin{proof}
For $f \in L^1(\mathbb{R})$, we define $\widehat{f}:\mathbb{R} \to \mathbb{C}$ by
\[
\widehat{f}(\xi) = \int_{\mathbb{R}} e^{-2\pi i \xi x} f(x) dx, \qquad \xi \in \mathbb{R}.
\]

Following Stein and Shakarchi, for $a>0$,  define $\mathfrak{F}_a$ to be the set of those functions
$f$ defined on some neighborhood of $\mathbb{R}$ in $\mathbb{C}$ such that $f$
is holomorphic on the set $\{z \in \mathbb{C}: |\Im z| < a\}$ and for which there is some $A>0$ such that
\[
|f(x+iy)| \leq \frac{A}{1+x^2}, \qquad x \in \mathbb{R}, \quad |y|<a,
\]
and we set $\mathfrak{F} = \bigcup_{a>0} \mathfrak{F}_a$. 
The \textbf{Poisson summation formula}\footnote{Elias M. Stein and Rami Shakarchi, {\em Complex Analysis},
p.~118, Theorem 2.4.} states that for $f \in \mathfrak{F}$,
\[
\sum_{n \in \mathbb{Z}} f(n) = \sum_{n \in \mathbb{Z}} \widehat{f}(n).
\]

For $z=x+iy$ with
$|y|<\frac{1}{2}$, 
\begin{align*}
|\sech \frac{\pi z}{t}|&=\frac{2}{|e^{\pi (x+iy)}-e^{-\pi(x+iy)}|}\\
&\leq \frac{2}{||e^{\pi(x+iy)}|-|e^{-\pi(x+iy)}||}\\
&=\frac{2}{|e^{\pi x}-e^{-\pi x}|}\\
&=\sech \pi |x|.
\end{align*}
Let $t>0$. 
Because the zeros of $\cosh \pi z$ are $ni+\frac{i}{2}$, $n \in \mathbb{Z}$,
the function $f(z)=\sech \frac{\pi z}{t}$ belongs to $\mathfrak{F}_{\frac{t}{2}}$. 
 Corollary \ref{sechfourier} with $a=0$ gives us
\[
\widehat{f}(\xi) 
=t \sech \pi \xi t,
\]
so applying the Poisson summation formula we get
\[
\sum_{n \in \mathbb{Z}} \sech \frac{\pi n}{t}
=
t\sum_{n \in \mathbb{Z}} \sech \pi nt,
\]
or,
\[
\sum_{n \in \mathbb{Z}} \sec \frac{\pi i n}{t} = t \sum_{n \in \mathbb{Z}}
\sec \pi int,
\]
i.e.,
\[
C\left( \frac{i}{t} \right) = t C(it).
\]
For $\tau=it$ this reads
\[
C(\tau) = \frac{i}{\tau} C\left( -\frac{1}{\tau} \right).
\]
But $\tau \mapsto C(\tau)$ and $\tau \mapsto \frac{i}{\tau} C\left( -\frac{1}{\tau} \right)$ are holomorphic
on $\mathfrak{H}$, so by analytic continuation this identity is true for all $\tau \in \mathfrak{H}$.
\end{proof}


\begin{theorem}
\[
C\left(1-\frac{1}{\tau}\right) \sim \frac{4\tau}{i} e^{\frac{\pi i \tau}{2}}, \qquad \Im \tau \to +\infty.
\]
\end{theorem}
\begin{proof}
Let $t>0$ and define $f(z)=e^{-\pi iz} \sech \frac{\pi z}{t}$, which we check belongs to $\mathfrak{F}_{\frac{t}{2}}$. 
Corollary \ref{sechfourier} with $a=\frac{1}{2}$ tells us that for $t>0$,
\[
\widehat{f}(\xi)=\int_{-\infty}^\infty e^{-2\pi i\xi x} e^{-\pi ix} \sech \frac{\pi x}{t} dx = t \sech \left(\pi\left(\xi+\frac{1}{2} \right)t\right),
\qquad \xi \in \mathbb{R}.
\]
Thus the Poisson summation formula gives, as $(-1)^n = e^{-i\pi n}$,
\[
\sum_{n \in \mathbb{Z}} (-1)^n \sech \frac{\pi n}{t}
=t \sum_{n \in \mathbb{Z}} \sech\left(\pi\left(n+\frac{1}{2}\right)t\right),
\]
or
\[
\sum_{n \in \mathbb{Z}} (-1)^n \sec \frac{\pi i n}{t} = t \sum_{n \in \mathbb{Z}} \sec \left(\pi i \left(n+\frac{1}{2}\right) t\right).
\]
For $\tau=it$ this reads
\[
\sum_{n \in \mathbb{Z}} (-1)^n \sec \frac{\pi n}{\tau}
=\frac{\tau}{i} \sum_{n \in \mathbb{Z}} \sec \left(\pi\left(n+\frac{1}{2}\right) \tau\right).
\]
Now,
\[
\sec \left(\pi n \left(1-\frac{1}{\tau}\right)\right) = 
\frac{1}{\cos \pi n \cos \frac{-\pi n}{\tau}
-\sin \pi n \sin \frac{-\pi n}{\tau}}
=(-1)^n \sec \frac{\pi n}{\tau},
\]
so the above states that for $\tau=it$, $t>0$,
\begin{equation}
C\left(1-\frac{1}{\tau}\right)=
\frac{\tau}{i} \sum_{n \in \mathbb{Z}} \sec \left(\pi\left(n+\frac{1}{2}\right) \tau\right).
\label{overtau}
\end{equation}
We assert that both sides of \eqref{overtau} are holomorphic on $\mathfrak{H}$, and thus by analytic continuation that \eqref{overtau} is true for all
$\tau \in \mathfrak{H}$.

Write $\tau=\sigma+it$.
For $\nu>0$,
\[
\sec \pi \nu \tau =
\frac{2}{e^{i\pi \nu \tau}+e^{-i\pi \nu \tau}}
=\frac{2}{e^{-i\pi \nu \tau}(e^{2\pi i\nu \tau}+1)}
=2e^{i\pi \nu \tau}(1+O(|e^{2\pi i\nu \tau}|)),
\]
or,
\[
\sec \pi \nu \tau = 2e^{i\pi \nu \tau} + O(|e^{3\pi i\nu \tau}|).
\]
Now, 
\[
|e^{\frac{3\pi i \tau}{2}}| = e^{\frac{-3\pi t}{2}},
\]
so,
\[
\sec \pi \nu \tau = 2e^{i\pi \nu \tau} + O(e^{\frac{-3\pi t}{2}}).
\]
For $\nu<0$,
\[
\sec \pi \nu \tau = \sec (-\pi \nu \tau)
=2e^{- i\pi \nu \tau} + O(e^{\frac{-3\pi t}{2}}).
\]
For $\nu=\frac{1}{2}$,
\[
\sec \pi \nu \tau = 2e^{\frac{i\pi  \tau}{2}} + O(e^{\frac{-3\pi t}{2}}),
\]
and for $\nu=-\frac{1}{2}$,
\[
\sec \pi \nu \tau  = 2e^{\frac{i\pi  \tau}{2}} + O(e^{\frac{-3\pi t}{2}}).
\]
It follows that
\[
 \sum_{n \in \mathbb{Z}} \sec \left(\pi\left(n+\frac{1}{2}\right) \tau\right)
 =2e^{\frac{i\pi  \tau}{2}}+2e^{\frac{i\pi  \tau}{2}} + O(e^{\frac{-3\pi t}{2}})
 =4e^{\frac{i\pi  \tau}{2}} + O(e^{\frac{-3\pi t}{2}}).
\]
Using this with \eqref{overtau} yields
\[
C\left(1-\frac{1}{\tau}\right)=\frac{4\tau}{i}e^{\frac{i\pi  \tau}{2}} +
O(|\tau|e^{\frac{-3\pi t}{2}}),
\qquad \tau = \sigma+it,
\]
proving the claim.
\end{proof}


Define $\theta:\mathfrak{H} \to \mathbb{C}$
by
\[
\theta(\tau) = \sum_{n \in \mathbb{Z}} e^{\pi in^2 \tau}, \qquad \tau \in \mathfrak{H}.
\]
By proving that $\frac{C}{\theta^2}$ is a modular form of weight $0$, it follows that it is constant, and one thus
finds that $C=\theta^2$.\footnote{Elias M. Stein and Rami Shakarchi, {\em Complex Analysis},
p.~304.}
One reason that $\theta$ is significant is that, for $q=e^{i\pi \tau}$,
\begin{align*}
\theta(\tau)^2 &= \left(\sum_{n_1 \in \mathbb{Z}} q^{n_1^2} \right)
\left(\sum_{n_2 \in \mathbb{Z}} q^{n_2^2} \right)\\
&=\sum_{(n_1,n_2) \in \mathbb{Z} \times \mathbb{Z}}
q^{n_1^2+n_2^2}\\
&=\sum_{n=0}^\infty r_2(n) q^n,
\end{align*}
where $r_2(n)$ denotes the number of ways that $n$ can be expressed as a sum of two squares. 
We can write $C(\tau)$ as
\begin{align*}
C(\tau) &= 2\sum_{n =-\infty}^\infty \frac{1}{q^n+q^{-n}}\\
&=2\sum_{n=-\infty}^\infty \frac{q^n}{1+q^{2n}}\\
&=1+4\sum_{n=1}^\infty \frac{q^n}{1+q^{2n}}\\
&=1+4 \sum_{n=1}^\infty q^n \frac{1-q^{2n}}{1-q^{4n}}\\
&=1+4 \sum_{n=1}^\infty \left( \frac{q^n}{1-q^{4n}}
-\frac{q^{3n}}{1-q^{4n}}\right).
\end{align*}
Therefore the identity $\theta(\tau)^2 = C(\tau)$ can be written as
\[
\sum_{n=0}^\infty r_2(n) q^n =1+4 \sum_{n=1}^\infty \left( \frac{q^n}{1-q^{4n}}
-\frac{q^{3n}}{1-q^{4n}}\right).
\]
We write
\[
\sum_{n=1}^\infty \frac{q^n}{1-q^{4n}} =\sum_{n=1}^\infty q^n \sum_{m=0}^\infty (q^{4n})^m
=\sum_{n=1}^\infty \sum_{m=0}^\infty q^{n(4m+1)}
=\sum_{k=1}^\infty a(k) q^k,
\]
where $a(k)$ denotes the number of divisors of $k$ of the form $4m+1$,
and
\[
\sum_{n=1}^\infty \frac{q^{3n}}{1-q^{4n}} =\sum_{n=1}^\infty q^{3n} \sum_{m=0}^\infty (q^{4n})^m
=\sum_{n=1}^\infty \sum_{m=0}^\infty q^{n(4m+3)}
=\sum_{k=1}^\infty b(k) q^k,
\]
where $b(k)$ denotes the number of divisors of $k$ of the form $4m+3$. Thus for $n \geq 1$,
\[
r_2(n) = 4(a(n)-b(n)).
\]


\end{document}