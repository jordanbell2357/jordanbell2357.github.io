\documentclass{article}
\usepackage{amsmath,amssymb,graphicx,subfig,mathrsfs,amsthm}
%\usepackage{tikz-cd}
\newcommand{\inner}[2]{\langle #1, #2 \rangle}
\newcommand{\jap}[1]{\left\langle #1 \right\rangle}
\newcommand{\tr}{\textrm{tr}} 
\newcommand{\Span}{\textrm{span}} 
\newcommand{\id}{\textrm{id}} 
\newcommand{\Hom}{\textrm{Hom}} 
\newcommand{\norm}[1]{\left\Vert #1 \right\Vert}
\newtheorem{theorem}{Theorem}
\newtheorem{lemma}[theorem]{Lemma}
\newtheorem{corollary}[theorem]{Corollary}
\begin{document}
\title{The inhomogeneous heat equation on $\mathbb{T}$}
\author{Jordan Bell}
\date{April 3, 2014}

\maketitle
\section{Introduction}
In this note I am working out some material following Steve Shkoller's {\em MAT218: Lecture Notes on Partial Differential Equations}. However, I have written out a number of details that were not in the original notes, and may thus have introduced errors that were not in the notes on which this is
based.

Write
$\mathbb{T}=\mathbb{R}/2\pi \mathbb{Z}$, and
for $1 \leq p < \infty$,
\[
\norm{f}_{L^p}=\left( \frac{1}{2\pi} \int_\mathbb{T} |f(t)|^p dt \right)^{1/p}.
\]
Define
\[
\norm{f}_{H^k}=\left( \sum_{0 \leq j \leq k} \norm{\partial_x^j f}_{L^2}^2 \right)^{1/2}.
\]
If $u$ is a distribution on $\mathbb{T}$, $\partial_x u$ is also a distribution on $\mathbb{T}$, and in particular, if $u \in L^2(\mathbb{T})$ then
$\partial_x u$ is a distribution on $\mathbb{T}$. But if $u \in H^2(\mathbb{T})$, for example, then $\partial_x^2 u$ is an element of  $L^2(\mathbb{T})$, rather than
merely being a distribution.

Fix $T>0$. Let $f \in L^2(0,T;L^2(\mathbb{T}))$ and $g \in H^1(\mathbb{T})$; as $H^1(\mathbb{T}) \subset C^0(\mathbb{T})$, we can
speak about the value of $g$ at every point rather than merely almost all points.

For almost all $t$ and for all $x$, define $f_n$ by
\[
f_n(x,t)=\sum_{k=-n}^n \hat{f}(k,t) e^{ikx},
\]
and for all $x$ define $g_n$ by
\[
g_n(x)=\sum_{k=-n}^n \hat{g}(k) e^{ikx}.
\]
In other words, if $D_n(x)=\sum_{k=-n}^n e^{ikx}$, then
\[
f_n(x,t)=(D_n * f(\cdot,t))(x), \qquad g_n(x)=(D_n * g)(x),
\]
where
\[
(\phi * \psi)(x)=\frac{1}{2\pi} \int_\mathbb{T} \phi(y)\psi(x-y)dy.
\]

\section{Truncation}
For each $n$, assume that there is some $u_n \in C^\infty(0,T; C^\infty(\mathbb{T}))$ such that for almost all $t$ and for  all
$x \in \mathbb{T}$,
\begin{equation}
u_{nt}(x,t)-u_{nxx}(x,t)=f_n(x,t),
\label{npde}
\end{equation}
and for all $x \in \mathbb{T}$,
\[
u_n(x,0)=g_n(x).
\]
We will thus obtain a formula for $u_n$. In fact we will not necessarily have  $u_n \in C^\infty(0,T; C^\infty(\mathbb{T}))$, but once we have
an expression for $u_n$ we can determine the function space of which it is an element. We will then show that there is some $u$ in a certain function
space such that $u_n(x,t)=(D_n * u(\cdot, t))(x)$ for all $x$ and $t$.

For all $t$ and $x$,
\[
u_n(x,t)=\sum_{k \in \mathbb{Z}} \widehat{u_n}(k,t) e^{ikx}.
\]
Then \eqref{npde} becomes the statement that for almost all $t$ and for all $x$,
\[
\sum_{k \in \mathbb{Z}} \widehat{u_n}'(k,t) e^{ikx} + \sum_{k \in \mathbb{Z}} k^2 \widehat{u_n}(k,t) e^{ikx}
=\sum_{k=-n}^n \hat{f}(k,t) e^{ikx}.
\]
If $|k| >n$, then $\widehat{u_n}'(k,t) + k^2 \widehat{u_n}(k,t) = 0$, which  is a linear ordinary differential equation,  whose
solution satisfies $\widehat{u_n}(k,t)=e^{-k^2 t} \widehat{u_n}(k,0)$. Since $u_n(x,0)=g_n(x)$, $\widehat{u_n}(k,0)=0$. Hence
if $|k|>n$ then $\widehat{u_n}(k,t)=0$. If $|k| \leq n$, then for almost all $t$, $\widehat{u_n}'(k,t) + k^2 \widehat{u_n}(k,t) = \hat{f}(k,t)$.
The solution of this is, for all $t$ and for all $x$,
\[
\widehat{u_n}(k,t)=e^{-k^2 t} \widehat{g_n}(k) + e^{-k^2 t}  \int_0^t e^{k^2 s} \widehat{f_n}(k,s) ds.
\]
Hence, for  all $t$ and for all $x$,
\[
u_n(x,t)=\sum_{k=-n}^n \left(e^{-k^2 t} \widehat{g_n}(k) + e^{-k^2 t}  \int_0^t e^{k^2 s} \widehat{f_n}(k,s) ds\right) e^{ikx}.
\]
We merely know
that $\widehat{f_n}(k,t)$ is defined for almost all $t$, thus we only know 
for almost all $t_0 \in (0,T)$ and for all $x$ that $u_{nt}(x,t_0)$ exists. We do have that 
\[
u_n \in C^0(0,T;C^\infty(\mathbb{T})).
\]

\section{𝐻¹}
For almost all $t$, multiply \eqref{npde} by $u_n(x,t)$ and integrate over $\mathbb{T}$. This is,
\[
\int_\mathbb{T} u_{nt}(x,t)u_n(x,t) dx- \int_\mathbb{T} u_{nxx}(x,t) u_n(x,t) dx = \int_\mathbb{T} f_n(x,t) u_n(x,t) dx.
\]
Integrating by parts this becomes
\[
\int_\mathbb{T} u_{nt}(x,t) u_n(x,t) dx + \int_\mathbb{T} u_{nx}(x,t) u_{nx}(x,t) dx = \int_\mathbb{T} f_n(x,t) u_n(x,t) dx,
\]
which is
\[
\pi \cdot \partial_t \frac{1}{2\pi} \int_\mathbb{T} u_n(x,t)^2 dx + 2\pi \cdot \frac{1}{2\pi} \int_\mathbb{T} u_{nx}(x,t)^2 dx = \int_\mathbb{T} f_n(x,t) u_n(x,t) dx.
\]
Writing this using norms,
\[
\pi \cdot \partial_t \norm{u_n(\cdot,t)}_{L^2}^2 + 2\pi \cdot \norm{u_{nx}(\cdot,t)}_{L^2}^2 =  \int_\mathbb{T} f_n(x,t) u_n(x,t) dx.
\]
Integrating from $0$ to $t$, for any $t$,
\[
\pi \cdot \norm{u_n(\cdot,t)}_{L^2}^2 - \pi \cdot \norm{u_n(\cdot,0)}_{L^2}^2
+ 2\pi \int_0^t \norm{u_{nx}(\cdot,s)}_{L^2}^2 ds = \int_0^t  \int_\mathbb{T} f_n(x,s) u_n(x,s) dx ds.
\]

For almost all $s$,
\begin{eqnarray*}
 \int_\mathbb{T} |f_n(x,s) u_n(x,s)| dx &=& 
 2\pi \cdot \frac{1}{2\pi}  \int_\mathbb{T} |f_n(x,s) u_n(x,s)| dx\\
 & \leq& 
 2\pi \cdot \norm{f_n(\cdot,s)}_{L^2} \norm{u_n(\cdot,s)}_{L^2}\\
 &\leq&2\pi \left( \frac{ \norm{f_n(\cdot,s)}_{L^2}^2}{2} + \frac{\norm{u_n(\cdot,s)}_{L^2}^2}{2} \right)\\
 &=&\pi \cdot  \norm{f_n(\cdot,s)}_{L^2}^2 + \pi\cdot \norm{u_n(\cdot,s)}_{L^2}^2.
\end{eqnarray*}
It follows that for all $t$ (not just almost all $t$)
\[
\begin{split}
&\pi \cdot \norm{u_n(\cdot,t)}_{L^2}^2 - \pi \cdot \norm{u_n(\cdot,0)}_{L^2}^2
+ 2\pi \int_0^t \norm{u_{nx}(\cdot,s)}_{L^2}^2 ds\\
 \leq& \int_0^t \pi \cdot  \norm{f_n(\cdot,s)}_{L^2}^2 + \pi\cdot \norm{u_n(\cdot,s)}_{L^2}^2 ds,
 \end{split},
\]
so, as $u_n(x,0)=g_n(x)$,
\[
 \norm{u_n(\cdot,t)}_{L^2}^2+2 \int_0^t \norm{u_{nx}(\cdot,s)}_{L^2}^2 ds \leq  \norm{g_n}_{L^2}^2 +  \int_0^t  \norm{f_n(\cdot,s)}_{L^2}^2  + \norm{u_n(\cdot,s)}_{L^2}^2 ds.
\]
Let
\[
y(t)= \norm{u_n(\cdot,t)}_{L^2}^2+2 \int_0^t \norm{u_{nx}(\cdot,s)}_{L^2}^2 ds.
\]
By the inequality we just established we have, for all $t$,
\begin{eqnarray*}
y(t)& \leq&   \norm{g_n}_{L^2}^2 +  \int_0^t  \norm{f_n(\cdot,s)}_{L^2}^2 ds +\int_0^t \norm{u_n(\cdot,s)}_{L^2}^2 ds\\
&\leq&   \norm{g_n}_{L^2}^2 +  \int_0^t  \norm{f_n(\cdot,s)}_{L^2}^2 ds + \int_0^t y(s) ds.
\end{eqnarray*}
By Gronwall's inequality, we get
\[
 y(t) \leq \left(\norm{g_n}_{L^2}^2 + \int_0^t  \norm{f_n(\cdot,s)}_{L^2}^2 ds \right) e^t.
\]
As $\norm{g_n}_{L^2} \leq \norm{g}_{L^2}$ and $\norm{f_n(\cdot,s)}_{L^2} \leq \norm{f(\cdot,s)}_{L^2}$ (these two
facts follow from Parseval's identity), it follows that
\[
 y(t) \leq \left(\norm{g}_{L^2}^2 + \int_0^t  \norm{f(\cdot,s)}_{L^2}^2 ds \right) e^t.
\]
Therefore,  if $0 \leq t \leq T$ then
\begin{eqnarray*}
\norm{u_n(\cdot,t)}_{L^2}^2+2 \int_0^t \norm{u_{nx}(\cdot,s)}_{L^2}^2 ds& \leq&  \left(\norm{g}_{L^2}^2 + \int_0^t  \norm{f(\cdot,s)}_{L^2}^2 ds \right) e^T\\
&\leq&\left(\norm{g}_{L^2}^2+ \norm{f}_{L^2(0,T;L^2(\mathbb{T})}^2 \right) e^T\\
&=&M.
\end{eqnarray*}
By Parseval's identity,
\[
\sum_{k \in \mathbb{Z}} |\widehat{u_n}(k,t)|^2+2\int_0^t \sum_{k \in \mathbb{Z}} |\widehat{u_{nx}}(k,s)|^2 ds \leq M,
\] 
hence for all $t$,
\[
\sum_{k \in \mathbb{Z}} |\widehat{u_n}(k,t)|^2+2\int_0^t \sum_{k \in \mathbb{Z}} k^2 |\widehat{u_n}(k,s)|^2 ds \leq M.
\]

If $k \leq n \leq m$, then $\widehat{u_n}(k,t)=\widehat{u_m}(k,t)$ for all $t$. Define $\hat{u}(k,t)$ by 
\[
\hat{u}(k,t)
=\lim_{n \to \infty} \widehat{u_n}(k,t)=\widehat{u_k}(k,t).
\]
Thus, for all $t$,
\begin{equation}
\sum_{k \in \mathbb{Z}} |\hat{u}(k,t)|^2+2\int_0^t \sum_{k \in \mathbb{Z}} k^2 |\hat{u}(k,s)|^2 ds \leq M.
\label{k0}
\end{equation}
Then, for some $M'=M'(f,g,T)$,
\[
\int_0^T \sum_{k \in \mathbb{Z}} |\hat{u}(k,t)|^2+ \sum_{k \in \mathbb{Z}} k^2 |\hat{u}(k,t)|^2 dt \leq M'.
\]
It follows that for almost all $t$, there is some $u \in H^1(\mathbb{T})$ whose Fourier coefficients are $\hat{u}(k,t)$, and that we have
\[
\int_0^T \norm{u(\cdot,t)}_{H^1}^2 dt \leq M'.
\]
We have
\[
\lim_{n \to \infty} \int_0^T \norm{u_n(\cdot,t)-u(\cdot,t)}_{H^1}^2 dt=0,
\]
i.e.
\[
\lim_{n \to \infty} \norm{u_n -u}_{L^2(0,T;H^1(\mathbb{T}))}^2 = 0.
\]






\section{𝐻²}
Multiply \eqref{npde} by $u_{nxx}(x,t)$ and integrate over $\mathbb{T}$. We get, for almost all $t$,
\[
\int_\mathbb{T} u_{nt}(x,t)u_{nxx}(x,t) dx- \int_\mathbb{T} u_{nxx}(x,t) u_{nxx}(x,t) dx = \int_\mathbb{T} f_n(x,t) u_{nxx}(x,t) dx.
\]
As
\[
\int_\mathbb{T} u_{nt}(x,t)u_{nxx}(x,t) dx = -\int_\mathbb{T} u_{ntx}(x,t) u_{nx}(x,t)dx
=-\frac{1}{2} \frac{d}{dt} \int_\mathbb{T} u_{nx}(x,t)^2 dx,
\]
we have
\[
-\pi \frac{d}{dt} \norm{u_{nx}(\cdot,t)}_{L^2}^2-2\pi \norm{u_{nxx}(\cdot,t)}_{L^2}^2=\int_\mathbb{T} f_n(x,t) u_{nxx}(x,t) dx.
\]
Integrating from $0$ to $t$,
\[
\begin{split}
&-\pi \norm{u_{nx}(\cdot,t)}_{L^2}^2+\pi \norm{u_{nx}(\cdot,0)}_{L^2}^2-2\pi \int_0^t \norm{u_{nxx}(\cdot,s)}_{L^2}^2 ds\\
=&
\int_0^t \int_\mathbb{T} f_n(x,s) u_{nxx}(x,s) dx.
\end{split}
\]
For almost all $s$,
\begin{eqnarray*}
\int_\mathbb{T} |f_n(x,s) u_{nxx}(x,s)| dx &\leq&
2\pi \norm{f_n(\cdot,s)}_{L^2} \norm{u_{nxx}(\cdot,s)}_{L^2}\\
&\leq& 2\pi \left(\frac{\norm{f_n(\cdot,s)}_{L^2}^2}{2}+ \frac{\norm{u_{nxx}(\cdot,s)}_{L^2}^2}{2} \right)\\
&=&\pi \norm{f_n(\cdot,s)}_{L^2}^2 + \pi \norm{u_{nxx}(\cdot,s)}_{L^2}^2.
\end{eqnarray*}
It follows that, for all $t$,
\[
\begin{split}
&\pi \norm{u_{nx}(\cdot,t)}_{L^2}^2x  + 2\pi \int_0^t \norm{u_{nxx}(\cdot,s)}_{L^2}^2 ds\\
\leq&\pi \norm{g_n'}_{L^2}^2+\pi \int_0^t \norm{f_n(\cdot,s)}_{L^2}^2 + \norm{u_{nxx}(\cdot,s)}_{L^2}^2 ds.
\end{split}
\]
Hence
\begin{eqnarray*}
\norm{u_{nx}(\cdot,t)}_{L^2}^2 +  \int_0^t \norm{u_{nxx}(\cdot,s)}_{L^2}^2 ds& \leq& \norm{g_n'}_{L^2}^2 + \int_0^t \norm{f_n(\cdot,s)}_{L^2}^2 ds\\
&\leq&\norm{g}_{H^1}^2 +  \int_0^t \norm{f(\cdot,s)}_{L^2}^2 ds\\
&\leq&\norm{g}_{H^1}^2 +  \int_0^T \norm{f(\cdot,s)}_{L^2}^2 ds.
\end{eqnarray*}
Using Parseval's identity we have, for  all $t$,
\[
\sum_{k \in \mathbb{Z}} | \widehat{u_{nx}}(k,t)|^2 + \int_0^t \sum_{k \in \mathbb{Z}} |\widehat{u_{nxx}}(k,s)|^2 ds\leq \norm{g}_{H^1}^2 +  \int_0^T \norm{f(\cdot,s)}_{L^2}^2 ds,
\]
hence
\[
\sum_{k \in \mathbb{Z}} k^2 |\widehat{u_n}(k,t)|^2 + \int_0^t \sum_{k \in \mathbb{Z}} k^4 |\widehat{u_n}(k,s)|^2 ds\leq \norm{g}_{H^1}^2 +  \int_0^T \norm{f(\cdot,s)}_{L^2}^2 ds,
\]
so
\[
\sum_{k \in \mathbb{Z}}  k^2| \widehat{u}(k,t)|^2 + \int_0^t \sum_{k \in \mathbb{Z}} k^4 |\widehat{u}(k,s)|^2 ds\leq \norm{g}_{H^1}^2 +  \int_0^T \norm{f(\cdot,s)}_{L^2}^2 ds.
\]
It follows that, for almost all $t$,\footnote{The reason I see that this follows involves the fact that the intersection of two sets of full measure is itself a set of full measure.}
\[
\sum_{k \in \mathbb{Z}}  k^2| \widehat{u}(k,t)|^2 + \sum_{k \in \mathbb{Z}} k^4 |\widehat{u}(k,t)|^2<\infty,
\]
thus $u(\cdot,t) \in H^2(\mathbb{T})$.

We have
\[
\lim_{n \to \infty} \int_0^T \norm{u_n(\cdot,t)-u(\cdot,t)}_{H^2}^2 dt=0,
\]
i.e.
\[
\lim_{n \to \infty} \norm{u_n -u}_{L^2(0,T;H^2(\mathbb{T}))}^2 = 0.
\]

\section{Solution}
For all $t$ we have $u(\cdot,t) \in H^1(\mathbb{T})$, and $H^1(\mathbb{T}) \subset C^0(\mathbb{T})$, so for all $t$ and all $x$, $u(x,t)$ is defined.
The Sobolev embedding tells us that if $k>\alpha+\frac{1}{2}$ then $H^k(\mathbb{T}) \subset C^\alpha(\mathbb{T})$. So, being
specific, we have $H^1(\mathbb{T}) \subset C^{1/4}(\mathbb{T})$. It is a fact that if $h \in C^\alpha(\mathbb{T})$, $\alpha>0$,
then the partial sums of the Fourier series of $h$ converge to $h$ in the supremum norm.

For all $t$ and for each $k$, 
\[
\hat{u}(k,t)=e^{-k^2 t} \widehat{g}(k) + e^{-k^2 t}  \int_0^t e^{k^2 s} \widehat{f}(k,s) ds.
\]
It follows that, for all $x$,
\[
u(x,0)=\lim_{N \to \infty} \sum_{|k| \leq N} \widehat{g}(k)  e^{ikx}.
\]
On the other hand,
\[
g(x)=\lim_{N \to \infty} \sum_{|k| \leq N} \widehat{g}(k)  e^{ikx}.
\]
Thus for all $x$, $u(x,0)=g(x)$. 

We have
\begin{eqnarray*}
\norm{u_t-u_{xx}-f}_{L^2(0,T;L^2(\mathbb{T}))}&\leq&\norm{u_t-u_{nt}}_{L^2(0,T;L^2(\mathbb{T}))}\\
&&+\norm{u_{xx}-u_{nxx}}_{L^2(0,T;L^2(\mathbb{T}))}\\
&&+\norm{f-f_n}_{L^2(0,T;L^2(\mathbb{T}))}\\
&&+\norm{u_{nt}-u_{nxx}-f_n}_{L^2(0,T;L^2(\mathbb{T}))}.
\end{eqnarray*}
Each of the four norms has limit $0$ as $n \to \infty$. Let me work out the first one.
For almost all $t$,
\begin{eqnarray*}
\widehat{u_t}(k,t)-\widehat{u_{nt}}(k,t)&=&\sum_{|k|>n} -k^2 e^{-k^2 t}\hat{g}(k)-k^2 e^{-k^2 t}\int_0^t e^{k^2 s}\hat{f}(k,s) ds\\
&&+e^{-k^2 t} e^{k^2 t} \hat{f}(k,t)\\
&=&\sum_{|k|>n} -k^2 \hat{u}(k,t) + \hat{f}(k,t).
\end{eqnarray*}
Then using Parseval's identity,
\begin{eqnarray*}
\norm{u_t-u_{nt}}_{L^2(0,T;L^2(\mathbb{T}))}^2&=&\int_0^T \sum_{|k|>n} | -k^2 \hat{u}(k,t) + \hat{f}(k,t)|^2 dt \\
&\leq&2\int_0^T \sum_{|k|>n} k^2 |\hat{u}(k,t)|^2 + |\hat{f}(k,t)|^2 dt.
\end{eqnarray*}


 Then,
\[
\norm{u_t-u_{xx}-f}_{L^2(0,T;L^2(\mathbb{T}))}^2=0.
\]
So, for almost all $t$ and for almost all $x$,
\[
u_t(x,t)-u_{xx}(x,t)=f(x,t).
\]

\end{document}