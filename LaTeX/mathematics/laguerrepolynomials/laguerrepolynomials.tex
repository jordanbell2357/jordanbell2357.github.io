\documentclass{article}
\usepackage{amsmath,amssymb,mathrsfs,amsthm}
%\usepackage{tikz-cd}
\usepackage[draft]{hyperref}
\newcommand{\inner}[2]{\left\langle #1, #2 \right\rangle}
\newcommand{\tr}{\ensuremath\mathrm{tr}\,} 
\newcommand{\Span}{\ensuremath\mathrm{span}} 
\def\Re{\ensuremath{\mathrm{Re}}\,}
\def\Im{\ensuremath{\mathrm{Im}}\,}
\newcommand{\id}{\ensuremath\mathrm{id}} 
\newcommand{\var}{\ensuremath\mathrm{var}} 
\newcommand{\Lip}{\ensuremath\mathrm{Lip}} 
\newcommand{\GL}{\ensuremath\mathrm{GL}}
\newcommand{\diam}{\ensuremath\mathrm{diam}} 
\newcommand{\sgn}{\ensuremath\mathrm{sgn}\,} 
\newcommand{\lcm}{\ensuremath\mathrm{lcm}} 
\newcommand{\supp}{\ensuremath\mathrm{supp}\,}
\newcommand{\dom}{\ensuremath\mathrm{dom}\,}
\newcommand{\upto}{\nearrow}
\newcommand{\downto}{\searrow}
\newcommand{\norm}[1]{\left\Vert #1 \right\Vert}
\newtheorem{theorem}{Theorem}
\newtheorem{lemma}[theorem]{Lemma}
\newtheorem{proposition}[theorem]{Proposition}
\newtheorem{corollary}[theorem]{Corollary}
\theoremstyle{definition}
\newtheorem{definition}[theorem]{Definition}
\newtheorem{example}[theorem]{Example}
\begin{document}
\title{Laguerre polynomials and Perron-Frobenius operators}
\author{Jordan Bell}
\date{April 19, 2016}

\maketitle

\section{Laguerre polynomials}
\subsection{Definition and generating functions}
Let $D=\frac{d}{dx}$. For $\alpha>-1$ and $n \geq 0$ let
\[
L_n^\alpha(x) = e^x \frac{x^{-\alpha}}{n!} D^n(e^{-x}x^{n+\alpha}),
\]
called the \textbf{Laguerre polynomials}.
Using the Leibniz rule for $D^n(f \cdot g)$ yields
\[
L_n^\alpha(x)=\sum_{k=0}^n \frac{\Gamma(n+\alpha+1)}{\Gamma(k+\alpha+1)} \frac{(-x)^k}{k!(n-k)!}.
\]
The generating function for the Laguerre polynomials is\footnote{N. N. Lebedev,
{\em Special Functions and Their Applications},
p.~77, \S 4.17.}
\[
w(x,z) = (1-z)^{-\alpha-1} e^{-xz/(1-z)} = \sum_{n=0}^\infty L_n^\alpha(x) z^n,\qquad
|z|<1.
\]
Define
\[
W(x,y,z) = (1-z)^{-1} e^{-(x+y)z/(1-z)} (xyz)^{-\alpha/2} I_\alpha \left( \frac{2 \sqrt{xyz}}{1-z} \right),
\qquad |z|<1,
\]
where 
\[
I_\alpha(x)=i^{-\alpha} J_\alpha(ix) = \sum_{m=0}^\infty \frac{1}{m! \Gamma(m+\alpha+1)} (x/2)^{2m+\alpha}
\]
and
\[
J_\alpha(x) = \sum_{m=0}^\infty \frac{(-1)^m}{m! \Gamma(m+\alpha+1)} (x/2)^{2m+\alpha}.
\]
$W$ satisfies
\[
W(x,y,z) = \sum_{n=0}^\infty 
\frac{n! L_n^\alpha(x) L_n^\alpha(y)}{\Gamma(n+\alpha+1)} z^n.
\]



\subsection{Differential equations satisfied by Laguerre polynomials}
$w$ satisfies the ordinary differential equation
\[
(1-z^2) \partial_z w + (x-(1-z)(1+\alpha))w=0.
\]
This yields, for $n \geq 1$,
\begin{equation}
(n+1)L_{n+1}^\alpha(x)+(x-\alpha-2n-1)L_n^\alpha(x)+(n+\alpha)L_{n-1}^\alpha(x)=0.
\label{4181}
\end{equation}
$w$ also satisfies the ordinary differential equation
\[
(1-t)\partial_x w + tw=0,
\]
which yields, for $n \geq 1$,
\begin{equation}
DL_n^\alpha-DL_{n-1}^\alpha + L_{n-1}^\alpha=0.
\label{4182}
\end{equation}
Using \eqref{4181} and \eqref{4182} gives
\begin{equation}
xDL_n^\alpha = n L_n^\alpha-(n+\alpha)L_{n-1}^\alpha,
\qquad n \geq 1.
\label{4184}
\end{equation}
Using \eqref{4184} and \eqref{4182} we get, for $n \geq 0$,
\begin{equation}
x D^2 L_n^\alpha(x)+(\alpha+1-x)DL_n^\alpha(x)+nL_n^\alpha(x)=0.
\label{4187}
\end{equation}



\subsection{Integral formulas for Laguerre polynomials}
For $\nu>-1$, $a>0$, $b>0$, using the series for $J_\nu$ one calculates\footnote{N. N. Lebedev,
{\em Special Functions and Their Applications},
p.~132, \S 5.15, Example 2.}
\begin{equation}
\int_0^\infty e^{-a^2x^2} J_\nu(bx)x^{\nu+1} dx=\frac{b^\nu}{(2a^2)^{\nu+1}}e^{-\frac{b^2}{4a^2}}.
\label{5152}
\end{equation}
Applying this with $\nu=n+\alpha$, $a=1$, $b=2\sqrt{x}$, $x=\sqrt{t}$ yields
\[
\int_0^\infty e^{-t} J_{n+\alpha}(2\sqrt{xt}) (\sqrt{t})^{n+\alpha+1} \cdot \frac{1}{2\sqrt{t}} dt
=\frac{(2\sqrt{x})^{n+\alpha}}{2^{n+\alpha+1}} e^{-x},
\]
i.e.
\begin{equation}
\int_0^\infty e^{-t} J_{n+\alpha}(2\sqrt{xt}) (\sqrt{xt})^{n+\alpha} dt = e^{-x} x^{n+\alpha}.
\label{4191}
\end{equation}
Now, it is a fact that
\[
\frac{d}{du} u^{\nu/2} J_\nu(2\sqrt{u}) = u^{(\nu-1)/2} J_{\nu-1}(2\sqrt{u}),
\]
and using this and \eqref{4191}, we get that for $\alpha>1$ and $n \geq 0$,
\begin{equation}
L_n^\alpha(x) = \frac{e^x x^{-\alpha/2}}{n!} \int_0^\infty t^{n+\alpha/2} J_\alpha(2\sqrt{xt}) e^{-t} dt.
\label{4193}
\end{equation}



We remind ourselves that for $\alpha>-1$ and $|z|<1$,
\[
(1-z)^{-\alpha-1} e^{-yt/(1-t)} = \sum_{n=0}^\infty L_n^\alpha(y) z^n.
\]
For $|z|<\frac{1}{3}$, using this 
and $e^{-\frac{yt}{1-t}-\frac{y}{2}} = e^{-\frac{2yt+y-yt}{2(1-t)}}=e^{-\frac{y(1+t)}{2(1-t)}}$
one checks that
\[
\begin{split}
&(1-z)^{-\alpha-1} \int_0^\infty e^{-\frac{y(1+t)}{2(1-t)}} y^{\alpha/2} J_\alpha(\sqrt{xy}) dy\\
=&\sum_{n=0}^\infty z^n \int_0^\infty e^{-y/2} y^{\alpha/2} J_\alpha(\sqrt{xy}) L_n^\alpha(y) dy.
\end{split}
\]
Then one gets, for $|z|<1$,
\[
2e^{-x/2} x^{\alpha/2} \sum_{n=0}^\infty (-1)^n L_n^\alpha(x) z^n = \sum_{n=0}^\infty z^n \int_0^\infty e^{-y/2}
y^{\alpha/2} J_\alpha(\sqrt{xy}) L_n^\alpha(y) dy.
\]
Therefore for $\alpha>-1$ and $n \geq 0$,
\begin{equation}
e^{-x/2} x^{\alpha/2} L_n^\alpha(x) = \frac{(-1)^n}{2} \int_0^\infty J_\alpha(\sqrt{xy}) e^{-y/2} y^{\alpha/2}
L_n^\alpha(y)dy.
\label{4203}
\end{equation}



\subsection{Orthogonality of Laguerre polynomials}
Let
\[
\rho_\alpha(x)=e^{-x} x^\alpha.
\]
Let
\[
u_n = \rho_\alpha^{1/2} L_n^\alpha,\qquad n \geq 0.
\]
$u_n$ satisfies the differential equation 
\[
(xu_n')'+\left(n+\frac{\alpha+1}{2}-\frac{x}{4}-\frac{\alpha^2}{4x}\right) u_n = 0.
\]
Using this we get
\[
x(u_n'u_m-u_m'u_n)\bigg|_0^\infty + (n-m) \int_0^\infty u_m u_n dx = 0.
\]
Then
\begin{equation}
(n-m) \int_0^\infty u_m u_n dx = 0.
\label{4211}
\end{equation}
Using \eqref{4181} yields for $n \geq 2$,
\[
\begin{split}
&n (L_n^\alpha)^2 - (n+\alpha)(L_{n-1}^\alpha)^2-(n+1)L_{n+1}^\alpha L_{n-1}^\alpha
+2L_n^\alpha L_{n-1}^\alpha + (n+\alpha-1)L_n^\alpha L_{n-2}^\alpha=0.
\end{split}
\]
Using this and \eqref{4211}, for $n \geq 2$,
\[
n\int_0^\infty e^{-x} x^\alpha L_n^\alpha(x)^2 dx = (n+\alpha)\int_0^\infty e^{-x} x^\alpha L_{n-1}^\alpha(x)^2 dx.
\]
Iterating this, for $n \geq 2$,
\begin{align*}
\int_0^\infty e^{-x} x^\alpha L_n^\alpha(x)^2 dx &=\frac{(n+\alpha)(n+\alpha-1)\cdots(\alpha+2)}{n(n-2)\cdots 3 \cdot 2} 
\int_0^\infty e^{-x} x^\alpha L_1^\alpha(x)^2 dx\\
&=\frac{\Gamma(n+\alpha+1)}{n!}.
\end{align*}



\subsection{Asymptotics for Laguerre polynomials}
It can be proved that for $\alpha>-1$, with $N=n+\frac{\alpha+1}{2}$,\footnote{N. N. Lebedev,
{\em Special Functions and Their Applications}, p.~87, \S 4.22.} for $x \in \mathbb{R}_{\geq 0}$,
\[
L_n^\alpha(x) \sim \frac{\Gamma(n+\alpha+1)}{n!} e^{x/2} (Nx)^{-\alpha/2} J_\alpha(2\sqrt{Nx}),
\qquad n \to \infty.
\]


\subsection{Laguerre expansions}
Suppose that $f:\mathbb{R}_{>0} \to \mathbb{R}$ is piecewise smooth in every interval
$[x_1,x_2]$, $0<x_1<x_2<\infty$, and 
$f \in L^2(d\rho_\alpha)$. 
Let 
\[
c_n(f) = \frac{n!}{\Gamma(n+\alpha+1)} \int_0^\infty f(x) L_n^\alpha(x) \rho_\alpha(x) dx,
\]
$\rho_\alpha(x)=e^{-x} x^\alpha$.
It can be proved that\footnote{N. N. Lebedev,
{\em Special Functions and Their Applications}, p.~88, \S 4.23, Theorem 3.}
if $f$ is continuous at $x$ then
\[
\sum_{n=0}^N c_n(f) L_n^\alpha(x) \to f(x),\qquad N \to \infty,
\]
and if $f$ is not continuous at $x$ then
\[
\sum_{n=0}^N c_n(f) L_n^\alpha(x) \to \frac{f(x+0)}{2}+\frac{f(x-0)}{2},\qquad N \to \infty,
\]
which makes sense because $f$ is a priori piecewise continuous.



Let $\nu>-\frac{1}{2}(\alpha+1)$ and $f(x)=x^\nu$. 
Integrating by parts,
\begin{align*}
c_n(f)&=\frac{n!}{\Gamma(n+\alpha+1)} \int_0^\infty x^{\nu+\alpha} L_n^\alpha(x) e^{-x} \\
&=\frac{1}{\Gamma(n+\alpha+1)} \int_0^\infty x^\nu D^n(e^{-x}x^{n+\alpha}) dx\\
&=(-1)^n \frac{\Gamma(\nu+\alpha+1)\Gamma(\nu+1)}{\Gamma(n+\alpha+1)\Gamma(\nu-n+1)}.
\end{align*}
Thus
\[
x^\nu = \Gamma(\nu+\alpha+1)\Gamma(\nu+1) \sum_{n=0}^\infty \frac{(-1)^n L_n^\alpha(x)}{\Gamma(n+\alpha+1)
\Gamma(\nu-n+1)}.
\]
For $p$ a positive integer,
\[
x^p = \Gamma(p+\alpha+1) \cdot p! \sum_{n=0}^p \frac{(-1)^n L_n^\alpha(x)}{\Gamma(n+\alpha+1) \cdot (p-n)!}.
\]

Define
\[
f(x)=(ax)^{-\alpha/2} J_\alpha(2\sqrt{ax}),\qquad \alpha>-1,\quad a>0,\quad x>0.
\]
Using 
\[
(1-z)^{-\alpha-1} e^{-xz/(1-z)} = \sum_{n=0}^\infty L_n^\alpha(x) z^n,\qquad
|z|<1,
\]
we obtain, as $e^{-\frac{xz}{1-z}-x}= e^{-x/(1-z)}$,
\[
\begin{split}
&(1-z)^{-\alpha-1} \int_0^\infty e^{-x/(1-z)} (x/a)^{\alpha/2} J_\alpha(2\sqrt{ax}) dx\\
=&\int_0^\infty e^{-x} (x/a)^{\alpha/2} J_\alpha(2\sqrt{ax}) \sum_{n=0}^\infty L_n^\alpha(x) z^n dx\\
=&\sum_{n=0}^\infty \left(\int_0^\infty f(x) L_n^\alpha(x) \rho_\alpha(x) dx\right) z^n.
\end{split}
\]
Doing the change of variable $2\sqrt{ax}=by$ with $b>0$ and then applying \eqref{4191} with
$A^2=\frac{b^2}{4a(1-z)}$ and $\nu=\alpha$,
\[
\begin{split}
&(1-z)^{-\alpha-1} \int_0^\infty e^{-x/(1-z)} (x/a)^{\alpha/2} J_\alpha(2\sqrt{ax}) dx\\
=&(1-z)^{-\alpha-1} (2a)^{-\alpha-1} b^{\alpha+2} \int_0^\infty e^{-\frac{b^2 y^2}{4a(1-z)}} 
J_\alpha(by) y^{\alpha+1} dy\\
=&(1-z)^{-\alpha-1} (2a)^{-\alpha-1} b^{\alpha+2} \cdot \frac{b^\alpha}{(2A^2)^{\alpha+1}} e^{-\frac{b^2}{4A^2}}\\
=&(1-z)^{-\alpha-1} (2a)^{-\alpha-1} b^{\alpha+2} \cdot b^{-\alpha-2} (2a(1-z))^{\alpha+1} e^{-a(1-z)}\\
=& e^{-a(1-z)}\\
=&e^{-a} \sum_{n=0}^\infty \frac{(az)^n}{n!}.
\end{split}
\]
Therefore
\begin{align*}
e^{-a} \sum_{n=0}^\infty \frac{a^n}{n!} z^n&=\sum_{n=0}^\infty \left(\int_0^\infty f(x) L_n^\alpha(x) \rho_\alpha(x) dx\right) z^n,
\end{align*}
whence, for $n \geq 0$,
\[
c_n(f) = \frac{n!}{\Gamma(n+\alpha+1)} \int_0^\infty f(x) L_n^\alpha(x) \rho_\alpha(x) dx
=\frac{n!}{\Gamma(n+\alpha+1)} e^{-a} \frac{a^n}{n!}.
\]
Therefore, for $\alpha>-1$, $a>0$, $x>0$,
\[
(ax)^{-\alpha/2} J_\alpha(2\sqrt{ax}) = \sum_{n=0}^\infty c_n(f) L_n^\alpha(x)
=e^{-a} \sum_{n=0}^\infty \frac{a^n}{\Gamma(n+\alpha+1)} L_n^\alpha(x).
\]



\section{Integral operators}
We remind ourselves that, for $\alpha=1$,
\[
u_n(x)=\rho_1(x)^{1/2} L_n^1(x) = e^{-x/2} x^{1/2} L_n^1(x).
\]
$\{u_n: n \geq 0\}$ is an orthonormal basis for $L^2(\mathbb{R}_{\geq 0})$.


For $x,y \in \mathbb{R}_{> 0}$ define
\[
k(x,y)= k_x(y) = k^x(y) = \frac{J_1(2 \sqrt{xy})}{((e^x-1)(e^y-1))^{1/2}}.
\]
For $\phi \in L^2(\mathbb{R}_{\geq 0})$ and $y \in \mathbb{R}_{> 0}$, define
\begin{align*}
K\phi(y) &= \int_{\mathbb{R}_{\geq 0}} k_y(x) \phi(x) dx.
\end{align*}
We have established, with $\alpha=1$,
\[
J_1(2\sqrt{xy}) = (xy)^{1/2} e^{-x} \sum_{n=0}^\infty \frac{x^n}{(n+1)!} L_n^1(y).
\]
Hence
\begin{align*}
\int_0^\infty k_y(x) \phi(x) dx&=\int_0^\infty \phi(x) (e^x-1)^{-1/2} (e^y-1)^{-1/2} (xy)^{1/2}
e^{-x}\\
&\cdot \sum_{n=0}^\infty \frac{x^n}{(n+1)!} L_n^\alpha(y) dx\\
&=\sum_{n=0}^\infty \frac{(e^y-1)^{-1/2} y^{1/2} L_n^1(y)}{(n+1)!} \int_0^\infty \phi(x) (e^x-1)^{-1/2}
x^{1/2} e^{-x} x^n dx\\
&=\sum_{n=0}^\infty q_n(y)  \inner{\phi}{p_n},
\end{align*}
for
\[
p_n(x)=\frac{1}{(n+1)!} (e^x-1)^{-1/2} e^{-x} x^{n+\frac{1}{2}}
=\frac{1}{(n+1)!} e^{-x/2} (e^x-1)^{-1/2} x^n u_n(x)
\]
and
\[
q_n(y)=(e^y-1)^{-1/2} y^{1/2} L_n^1(y) = (1-e^{-y})^{-1/2} u_n(y).
\]
Then
\[
K\phi = \sum_{n=0}^\infty q_n \inner{\phi}{p_n}.
\]



The following states the trace of the operator $K:L^2(\mathbb{R}_{\geq 0}) \to L^2(\mathbb{R}_{\geq 0})$.\footnote{cf.
A. A. Kirillov, {\em Elements of the Theory of Representations},
p.~211, \S 13, Theorem 2.}



\begin{theorem}
$\tr K=\int_0^\infty k(x,x) dx=\int_0^\infty \frac{J_1(2x)}{(e^x-1)} dx=0.7711\ldots$.
\end{theorem}







\section{Hardy spaces}
For $x \in \mathbb{R}$ let $P_x=\{z \in \mathbb{C}: \Re z > x\}$.
Let $H$ be the collection of holomorphic functions $f:P_{-1/2} \to \mathbb{C}$ such that for
any $x>-\frac{1}{2}$, $f|P_x$ is bounded and such that
\[
\int_{\mathbb{R}} \left| f\left( -\frac{1}{2}+iy\right) \right|^2 dy<\infty.
\]


Define $M:L^2(\mathbb{R}_{\geq 0}) \to H$, for $\phi \in L^2(\mathbb{R}_{\geq 0})$, by
\[
M\phi(z) = \int_{\mathbb{R}_{\geq 0}} e^{-zs -s/2} \phi(s) ds.
\]
For $f \in H$ define
\[
P_\lambda f(z) = \sum_{k \geq 1} \frac{1}{(z+k)^2} f \left( \frac{1}{z+k} \right)\qquad \Re z > -\frac{1}{2},
\]
called a \textbf{Perron-Frobenius operator}.
$\lambda$ denotes Lebesgue measure. 


Let 
\[
h(s)=\left(\frac{1-e^{-s}}{s} \right)^{1/2}
\]
for $s \in \mathbb{R}_{>0}$, with
$h(0)=1$.
 Because $h \in L^\infty(\mu)$, it makes sense to define
 $S:L^2(\mathbb{R}_{\geq 0}) \to L^2(\mathbb{R}_{\geq 0})$ by
\[
S \phi(s) = h \phi,\qquad \phi \in L^2(\mathbb{R}_{\geq 0}).
\]

Define $A:H \to L^2(\mathbb{R}_{\geq 0})$ by 
\[
A = S \circ M^{-1}.
\]


We prove that $P_\lambda$ and $K$ are conjugate.\footnote{Marius Iosifescu and Cor Kraaikamp,
{\em Metrical Theory of Continued Fractions}, p.~9, Proposition 1.1.1.}



\begin{theorem}
$P_\lambda = A^{-1}KA$.
\end{theorem}
\begin{proof}
Let $\phi \in L^2(\mathbb{R}_{\geq 0})$ and set $f=M\phi$. Then
\[
A^{-1}KAf = A^{-1}KS\phi.
\]
We calculate
\begin{align*}
(S^{-1}KS\phi)(x)&=h(x)^{-1} \int_{\mathbb{R}_{\geq 0}} k_x(y) \cdot h(y) \cdot \phi(y) dy\\
&=\left(\frac{x}{1-e^{-x}} \right)^{1/2}
\int_0^\infty 
 \frac{J_1(2 \sqrt{xy})}{((e^x-1)(e^y-1))^{1/2}}
 \cdot \left(\frac{1-e^{-y}}{y} \right)^{1/2}
 \cdot \phi(y) dy\\
 &=\int_0^\infty \left( \frac{x}{y} \right)^{1/2} \frac{e^{x/2}}{(e^x-1)^{1/2}} \frac{(e^y-1)^{1/2}}{e^{y/2}}  \frac{J_1(2 \sqrt{xy})}{((e^x-1)(e^y-1))^{1/2}}
 \phi(y) dy\\
 &=\int_0^\infty  \left( \frac{x}{y} \right)^{1/2} \frac{e^{(x-y)/2}}{e^x-1} J_1(2 \sqrt{xy})\phi(y) dy.
\end{align*}
Then
\[
\begin{split}
&(MS^{-1}KS\phi)(z)\\
=&\int_{\mathbb{R}_{\geq 0}} e^{-zx -x/2} (S^{-1}KS\phi)(x) dx\\
=&\int_0^\infty e^{-zx-x/2} \left(  \left( \frac{x}{y} \right)^{1/2} \frac{e^{(x-y)/2}}{e^x-1} J_1(2 \sqrt{xy})\phi(y) dy \right) dx\\
=&
\end{split}
\]
It is a fact that for $\Re z>-1$ and for $t \geq 0$,
\[
\sum_{k \geq 0} (z+k)^{-2} \exp\left(-\frac{t}{z+k} \right) = \int_0^\infty (s t^{-1})^{1/2} e^{-zs} \frac{J_1(2\sqrt{st})}{e^s-1} ds.
\]
Using this,
\begin{align*}
(MS^{-1}KS\phi)(z)&=\int_0^\infty e^{-y/2} \left( \int_0^\infty  (xy^{-1})^{1/2}  e^{-zx} \frac{J_1(2\sqrt{xy})}{e^x-1} dx\right) \phi(y) dy\\
&=\int_0^\infty e^{-y/2} \sum_{k \geq 1} (z+k)^{-2} \exp\left(-\frac{y}{z+k} \right) \cdot \phi(y) dy\\
&=\sum_{k \geq 1} (z+k)^{-2} \left( \int_0^\infty  \exp\left(-\frac{y}{z+k}-\frac{y}{2} \right)  \phi(y) dy\right)\\
&=\sum_{k \geq 1} (z+k)^{-2} \cdot M\phi\left(\frac{1}{z+k}\right).
\end{align*}
Thus, as $f=M\phi$,
\[
(MS^{-1}KSM^{-1}f)(z) = \sum_{k \geq 1} (z+k)^{-2} f\left(\frac{1}{z+k}\right) = P_\lambda f(z),
\]
that is,
\[
A^{-1}KAf(z) = P_\lambda f(z).
\]

\end{proof}

\end{document}