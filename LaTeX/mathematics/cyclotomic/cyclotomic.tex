\documentclass{article}
\usepackage{amsmath,amssymb,mathrsfs,amsthm}
\usepackage[T1]{fontenc}
%\usepackage{tikz-cd}
%\usepackage{hyperref}
\newcommand{\inner}[2]{\left\langle #1, #2 \right\rangle}
\newcommand{\tr}{\ensuremath\mathrm{tr}\,} 
\newcommand{\Span}{\ensuremath\mathrm{span}} 
\def\Re{\ensuremath{\mathrm{Re}}\,}
\def\Im{\ensuremath{\mathrm{Im}}\,}
\newcommand{\id}{\ensuremath\mathrm{id}} 
\newcommand{\var}{\ensuremath\mathrm{var}} 
\newcommand{\Lip}{\ensuremath\mathrm{Lip}} 
\newcommand{\GL}{\ensuremath\mathrm{GL}} 
\newcommand{\Gal}{\ensuremath\mathrm{Gal}} 
\newcommand{\diam}{\ensuremath\mathrm{diam}} 
\newcommand{\sgn}{\ensuremath\mathrm{sgn}\,} 
\newcommand{\lcm}{\ensuremath\mathrm{lcm}} 
\newcommand{\supp}{\ensuremath\mathrm{supp}\,}
\newcommand{\dom}{\ensuremath\mathrm{dom}\,}
\newcommand{\upto}{\nearrow}
\newcommand{\downto}{\searrow}
\newcommand{\norm}[1]{\left\Vert #1 \right\Vert}
\theoremstyle{definition}
\newtheorem{theorem}{Theorem}
\newtheorem{lemma}[theorem]{Lemma}
\newtheorem{proposition}[theorem]{Proposition}
\newtheorem{corollary}[theorem]{Corollary}
\theoremstyle{definition}
\newtheorem{definition}[theorem]{Definition}
\newtheorem{example}[theorem]{Example}
\begin{document}
\title{Cyclotomic polynomials}
\author{Jordan Bell}
\date{April 12, 2017}

\maketitle

\section{Preliminaries}
By an arithmetical function we mean a function whose domain contains the positive integers. We say that an arithmetical function $f$ is
\textbf{multiplicative} when $\gcd(n,m)=1$ implies $f(nm)=f(n)f(m)$, and that it is \textbf{completely multiplicative} when $f(nm)=f(n)f(m)$ for all $n,m \geq 1$.

Write
\[
U_n=\{e^{2\pi ik/n}: 1 \leq k \leq n\} = \{e^{2\pi ik/n}: 0 \leq k \leq n-1\},
\]
 the  $n$th roots of unity. 
For $n>1$, there is an element  $\zeta$ of $U_n$ with $\zeta \neq 1$. Because $\xi \mapsto \zeta \xi$ is a bijection $U_n \to U_n$ we have
$\zeta \sum_{\xi \in U_n} \xi = \sum_{\xi \in U_n} \xi$,
hence $(1-\zeta) \sum_{\xi \in U_n} \xi = 0$. But $\zeta \neq 1$, which means that
\[
\sum_{k=0}^{n-1} e^{2\pi ik/n} = \sum_{\xi \in U_n} \xi = 0,\qquad n>1.
\]


Write
\[
\Delta_n = \{e^{2\pi ik/n}: 1 \leq k \leq n, \gcd(k,n)=1\},
\]
 the primitive $n$th roots
of unity. 
Let $\phi$ be the \textbf{Euler phi function}:
\[
\phi(n) = |\{k: 1 \leq k \leq n, \gcd(k,n)=1\}|=|\Delta_n|.
\]
$\phi$ is multiplicative, and for prime $p$ and for $r \geq 1$, $\phi(p^r)=p^{r-1}(p-1)$.



Let $\mu$ be the \textbf{M\"obius function}:
\[
\mu(n) =\sum_{1\le k\le n, \gcd(k,n)=1} e^{2\pi ik/n}
=
 \sum_{\xi \in \Delta_n} \xi.
\]
For $p$ prime, as $\Delta_p = U_p \setminus \{1\}$,
\[
\mu(p) = -1+\sum_{\xi \in U_p} \xi  = 0-1=-1.
\]
For $r \geq 2$, as $\Delta_{p^r} = U_{p^r} \setminus U_{p^{r-1}}$,
\[
\mu(p^r) =-\sum_{\xi \in U_{p^{r-1}}} \xi+ \sum_{\xi \in U_{p^r}} \xi = -0+0=0.
\]
Furthermore, one proves  that $\mu$ is multiplicative. Thus
\[
\mu(n) =
\begin{cases}
1&\textrm{$n$ is a square-free integer with an even number of prime factors}\\
-1&\textrm{$n$ is a square-free integer with an odd number of prime factors}\\
0&\textrm{otherwise}.
\end{cases}
\]





The \textbf{M\"obius inversion formula} states that if $f$ and $g$ are arithmetic functions satisfying
\[
g(n) = \sum_{d \mid n} f(d),\qquad n \geq 1,
\]
then
\[
f(n) = \sum_{d \mid n} \mu(n/d) g(d), \qquad n \geq 1.
\]

We can write
\[
U_n = \bigcup_{d \mid n} \Delta_d,
\]
and $\Delta_d \cap \Delta_e = \emptyset$ for $d \neq e$. 
So
\[
n = \sum_{d \mid n} \phi(d).
\]
Therefore by the M\"obius inversion formula,
\[
\phi(n) = \sum_{d \mid n} d \cdot \mu(n/d).
\]
Also, for $n > 1$,
\begin{equation}
\sum_{d \mid n} \mu(d) = \sum_{d \mid n} \sum_{\xi \in \Delta_d} \xi = \sum_{\xi \in U_n} \xi = 0.
\label{mobiussum}
\end{equation}

Let
\[
d(n) = \sum_{d \mid n} 1,
\]
the number of divisors of $n$, for example, $d(6) = 4$. Let
\[
\omega(n)=\sum_{p \mid n} 1,
\]
the number of prime divisors of $n$: for $n=p_1^{\alpha_1} \cdots p_r^{\alpha_r}$, $\alpha_1,\ldots,\alpha_r \geq 1$,
we have $\omega(n)=r$, for example
$\omega(12)=\omega(2^2 \cdot 3) = 2$. 








\section{Definition and basic properties of cyclotomic polynomials}
For $n \geq 1$, let
\[
\Phi_n(x) =\prod_{1\le k\le n, \gcd(k,n)=1}
(x-e^{2\pi i k/n})
= \prod_{\xi \in \Delta_n} (x-\xi),
\]
the \textbf{$n$th cyclotomic polynomial}.
The first of the following two identities was found by Euler \cite[pp.~199--200, Chap. III, \S VI]{weil}.

\begin{lemma}
For $n \geq 1$,
\[
x^n-1 = \prod_{d \mid n} \Phi_d(x),
\]
and for $x \not \in U_n$,
\[
\Phi_n(x) = \prod_{d \mid n} (x^d-1)^{\mu(n/d)}.
\]
\label{mobiuslemma}
\end{lemma}
\begin{proof}
For $F_n(x) = x^n-1$, 
each of $e^{2\pi ik/n}$, $1 \leq k \leq n$, is a distinct root of $F_n(x)$, so
\begin{align*}
x^n-1 &= \prod_{1 \leq k \leq n} (x-e^{2\pi ik/n}) \\
&=\prod_{d \mid n} \prod_{1\le k\le n, \gcd(k,n)=d}
(x-e^{2\pi i k/n})\\
&=\prod_{d \mid n}\prod_{1\le j \le n/d, \gcd(j,n/d)=1} (x-e^{2\pi ijd/n})\\
&=\prod_{d \mid n} \Phi_{n/d}(x)\\
&=\prod_{d \mid n} \Phi_d(x).
\end{align*}
That is, $\log F_n = \sum_{d \mid n} \log \Phi_d$. Therefore applying the M\"obius inversion formula yields
$\log \Phi_n = \sum_{d \mid n} \mu(n/d) \log F_d$ and so
$\Phi_n = \prod_{d \mid n} F_{d}^{\mu(n/d)}$.
\end{proof}


\begin{lemma}
When $p$ is a prime,
\[
\Phi_p(x)  = x^{p-1}+\cdots+x+1.
\]
When $p$ is an odd prime,
\[
\Phi_{2p}(x)  = x^{p-1}-x^{p-2}+x^{p-3}-\cdots+x^2-x+1.
\]
\end{lemma}
\begin{proof}
When $p$ is a prime, $x^p-1 = \Phi_1(x) \cdot \Phi_p(x)$, i.e. 
\[
\Phi_p(x) =\frac{x^p-1}{\Phi_1(x)}= \frac{x^p-1}{x-1} = x^{p-1}+\cdots+x+1.
\]

When $p$ is an odd prime, 
\[
\Phi_{2p}(x) = \frac{x^{2p}-1}{\Phi_1(x) \Phi_2(x) \Phi_p(x)}
 = \frac{x^{2p}-1}{(x^p-1) \Phi_2(x)}
  = 
\frac{(x^p-1)(x^p+1)}{(x^p-1) (x+1)}
=\frac{x^p+1}{x+1},
\]
and because $(x+1)(x^{p-1}-x^{p-2}+x^{p-3}-\cdots+x^2-x+1) = x^p+1$,
\[
\Phi_{2p}(x) = x^{p-1}-x^{p-2}+x^{p-3}-\cdots+x^2-x+1.
\]
\end{proof}




\begin{lemma}
If $p$ is a prime and $m \geq 1$, 
\[
\Phi_{pm}(x)=
\begin{cases}
\Phi_m(x^p)& p | m\\
\Phi_m(x^p)/\Phi_m(x)&p \nmid m.
\end{cases}
\]
For $k \geq 1$,
\[
\Phi_{p^k m}(x)=
\begin{cases}
\Phi_m(x^{p^k})& p | m\\
\Phi_m(x^{p^k})/\Phi_m(x^{p^{k-1}})&p \nmid m,
\end{cases}
\]
\label{pmPhi}
\end{lemma}
\begin{proof}
Using Lemma \ref{mobiuslemma},
\begin{align*}
\Phi_{pm}(x)&= \prod_{d \mid (pm)} (x^d-1)^{\mu(pm/d)}\\
&= \prod_{d \mid (pm), p|d} (x^d-1)^{\mu(pm/d)} \cdot  \prod_{d \mid (pm), p \nmid d} (x^d-1)^{\mu(pm/d)}\\
&= \prod_{e \mid m} (x^{pe}-1)^{\mu(m/e)} \cdot  \prod_{d \mid (pm), p \nmid d} (x^d-1)^{\mu(pm/d)}\\
&=\Phi_m(x^p) \cdot \prod_{d \mid (pm), p \nmid d} (x^d-1)^{\mu(pm/d)}.
\end{align*}
If $m=ap$ and $d|(pm)$ and $p \nmid d$,  then $\mu(pm/d)=\mu(ap^2/d)=0$ and
\[
\Phi_{pm}(x) = \Phi_m(x^p) \cdot \prod_{d \mid a} (x^d-1)^{\mu(ap^2/d)} = \Phi_m(x^p).
\]
If $p \nmid m$ and $d \mid (pm)$ and $p \nmid d$, then $\mu(pm/d) = \mu(p) \mu(m/d)=-\mu(m/d)$ and
\[
\Phi_{pm}(x) = \Phi_m(x^p) \cdot  \prod_{d \mid (pm), p \nmid d} (x^d-1)^{\mu(pm/d)} = 
\Phi_m(x^p) \cdot \prod_{d \mid m} (x^d-1)^{-\mu(m/d)}.
\]

For $k \geq 2$,
\[
\Phi_{p^k m}(x) = \Phi_{p \cdot p^{k-1}m}(x) = \Phi_{p^{k-1}m}(x^p)
=\cdots = \Phi_{pm}(x^{p^{k-1}}),
\]
and using the expression we obtained for $\Phi_{pm}(x)$ we get
the expression stated for $\Phi_{p^km}(x)$. 
\end{proof}



\begin{lemma}
For $n=p_1^{\alpha_1} \cdots p_r^{\alpha_r}$, where $p_i$ are prime and $\alpha_i \geq 1$, and
$N=p_1 \cdots p_r$,
\[
\Phi_n(x) = \Phi_N(x^{n/N}).
\]
\label{radical}
\end{lemma}
\begin{proof}
If $d|n$ and $d \nmid N$ then $\mu(d)=0$, hence
\begin{align*}
\Phi_n(x)&=\prod_{d \mid n} (x^{n/d}-1)^{\mu(d)}\\
&=\prod_{d \mid N} (x^{n/d}-1)^{\mu(d)}\\
&=\prod_{d \mid N} ((x^{n/N})^{N/d}-1)^{\mu(d)}\\
&=\Phi_N(x^{n/N}).
\end{align*}
\end{proof}


\begin{lemma}
If $n > 1$ then 
\[
 \Phi_n(x^{-1}) =x^{-\phi(n)} \Phi_n(x).
\]
\label{reciprocal}
\end{lemma}
\begin{proof}
\[
\Phi_n(x^{-1}) = \prod_{d \mid n} (x^{-d}-1)^{\mu(n/d)} = \prod_{d \mid n} (1-x^d)^{\mu(n/d)} (x^{-d})^{\mu(n/d)},
\]
hence
\[
\Phi_n(x^{-1}) = \prod_{d \mid n} (-x^{-d})^{\mu(n/d)} \cdot \prod_{d \mid n} (x^d-1)^{\mu(n/d)}.
\]
Because $n>1$ it holds that $\sum_{d \mid n} \mu(n/d) = 0$, and using this and
$\sum_{d \mid n} d\cdot \mu(n/d) = \phi(n)$ yields
\[
\Phi_n(x^{-1}) = x^{-\phi(n)} \Phi_n(x).
\]
\end{proof}



\begin{lemma}
If $r>1$ is odd then
\[
\Phi_{2r}(x) = \Phi_r(-x).
\]
\label{twiceodd}
\end{lemma}
\begin{proof}
Because $r$ is odd, if $d_1,\ldots,d_l$ are the divisors of $r$ then
\[
d_1,\ldots,d_l,2d_1,\ldots,2d_l
\]
 are the divisors of $2r$, so
\begin{align*}
\Phi_{2r}(x)&=\prod_{d \mid (2r)} (x^d-1)^{\mu(2r/d)}\\
&=\prod_{d \mid r} (x^d-1)^{\mu(2r/d)} \cdot \prod_{d|r} (x^{2d}-1)^{\mu(2r/(2d))}\\
&=\prod_{d \mid r} (x^d-1)^{\mu(2r/d)} (x^{2d}-1)^{\mu(r/d)}\\
&=\prod_{d \mid r} (x^d-1)^{\mu(2) \mu(r/d)+\mu(r/d)} (x^d+1)^{\mu(r/d)}\\
&=\prod_{d \mid r} (x^d+1)^{\mu(r/d)}.
\end{align*}
Because $r$ is odd, any divisor $d$ of $r$ is odd and then $x^d+1 = -((-x)^d-1)$, so
\[
\Phi_{2r}(x) = \prod_{d \mid r} (-1)^{\mu(r/d)} ((-x)^d-1)^{\mu(r/d)}
=(-1)^{\phi(r)} \cdot \prod_{d \mid r} ((-x)^d-1)^{\mu(r/d)}.
\] 
Because $r$ is odd and $>1$, $\phi(r)$ is even, so
we have obtained the claim.
\end{proof}


\begin{theorem}
$\Phi_n \in \mathbb{Z}[x]$.
\label{Zx}
\end{theorem}
\begin{proof}
It is a fact that if $R$ is a unital commutative ring, $f \in R[x]$ is a monic polynomial and
$g \in R[x]$ is a polynomial, then there are $q,r \in R[x]$ with
\[
g = qf+r,
\]
$r=0$ or $\deg r < \deg f$.

First, $\Phi_1(x)=x-1 \in \mathbb{Z}[x]$. For $n>1$, assume that
$\Phi_d(x) \in \mathbb{Z}[x]$ for $1 \leq d < n$. Then let
\[
f = \prod_{d \mid n, d<n} \Phi_d,
\]
which by hypothesis belongs to $\mathbb{Z}[x]$. Since each $\Phi_d$ is monic, so is $f$. 
On the one hand,
since $g(x)=x^n-1 \in \mathbb{Z}[x]$, there are $q,r \in \mathbb{Z}[x]$ with
$g = q f + r$ and
$r=0$ or $\deg r < \deg f$.
On the other hand, by Lemma \ref{mobiuslemma} we have
$g = \Phi_n f \in \mathbb{C}[x]$. 
Thus $\Phi_n f = qf+r \in \mathbb{C}[x]$, 
so $r = f \cdot (\Phi_n - q) \in \mathbb{C}[x]$. If $\Phi_n \neq q$ then $\deg r = \deg f + \deg (\Phi_n-q) \geq \deg f$, contradicting that
$r=0$ or $\deg r < \deg f$. Therefore $\Phi_n = q \in \mathbb{C}[x]$, and because $q \in \mathbb{Z}[x]$ this means that
$\Phi_n \in \mathbb{Z}[x]$. 
\end{proof}

In fact, it can be proved that
$\Phi_n$ is irreducible in $\mathbb{Q}[x]$. Gauss states in entry 40 of his mathematical diary, dated
October 9, 1796, that 
$\Phi_p$ is irreducible in $\mathbb{Q}[x]$ when $p$ is prime, and he proves this
in {\em Disqisitiones Arithmeticae}, Art. 341.
Gauss further states in entry 136 of his mathematical diary, dated June 12, 1808, that for any $n$,
$\Phi_n$ is irreducible in $\mathbb{Q}[x]$, and Kronecker proves this in his
1854 {\em M\'emoire sur les facteurs irr\'eductibles de l'expression $x^n-1$}.
Gauss's work on cyclotomic polynomials is surveyed by Neumann \cite{neumannDA}.
For any $\xi \in \Delta_n$, $\Phi_n(\xi)=0$, and since $\Phi_n$ is irreducible in $\mathbb{Q}[x]$ and is monic, $\Phi$ is the minimal
polynomial of $\xi$ over $\mathbb{Q}$, which implies that $[\mathbb{Q}(\xi):\mathbb{Q}]=\deg \Phi_n=\phi(n)$.

There is a group isomorphism $\Gal(\mathbb{Q}(\xi)/\mathbb{Q}) \to (\mathbb{Z}/n)^*$ \cite[p.~596, Theorem 26]{dummit}.

The \textbf{discriminant} \cite[p.~12, Proposition 2.7]{washington}:
\[
d(\mathbb{Q}(e^{2\pi i/n})) = \frac{(-1)^{\phi(n)/2} n^{\phi(n)}}{\prod_{p \mid n} p^{\phi(n)/(p-1)}}. 
\]


It can be proved that $\mathcal{O}_{\mathbb{Q}(e^{2\pi i/n})} = \mathbb{Z}[e^{2\pi i/n}]$ \cite[p.~60, Proposition 10.2]{neukirch}.







Let $p$ be prime, let $q=p^r$ for $r \geq 1$,  let $\mathbb{F}_q$ be a finite field with $q$ elements,
and for $n \geq 1$ with $\gcd(n,q)=1$, let $\nu$ be the multiplicative order of $q$ modulo $n$: $\nu$ is the minimum positive
integer satisfying $q^\nu \equiv 1 \pmod{n}$. It can be proved that
there are monic, degree $\nu$, irreducible polynomials
$P_1,\ldots,P_{\phi(n)/\nu} \in \mathbb{F}_q[x]$ such that
$\Phi_n = P_1 \cdots P_{\phi(n)/\nu} \in \mathbb{F}_q[x]$ \cite[p.~65, Theorem 2.47]{lidl}; cf.
Bourbaki \cite[p.~581]{commutative} on Kummer.
In particular, $q$ is a generator of the multiplicative group $(\mathbb{Z}/n)^*$ if and only if
$\nu=\phi(n)$ if and only if $\Phi_n$ is irreducible in $\mathbb{F}_q[x]$. We remark that $(\mathbb{Z}/n)^*$ is cyclic if and only if
$n$ is $2$, $4$, some power of an odd prime, or twice some power of an odd prime (Gauss, {\em Disquisitiones Arithmeticae}, Art. 89--92).
This 
 follows from (i) the multiplicative group $(\mathbb{Z}/n)^*$ is isomorphic with the direct product
$(\mathbb{Z}/p_1^{\alpha_1})^* \times \cdots \times (\mathbb{Z}/p_r^{\alpha_r})^*$ for $n=p_1^{\alpha_1} \cdots p_r^{\alpha_r}$,
(ii) $(\mathbb{Z}/2^\alpha)^*$ is isomorphic with $\mathbb{Z}/2 \times \mathbb{Z}/2^{\alpha-2}$, $\alpha \geq 2$,
and (iii) $(\mathbb{Z}/p^\alpha)^*$ is a cyclic group with $p^{\alpha-1}(p-1)$ elements when $p$ is an odd
prime, $\alpha \geq 1$ \cite[p.~314, Corollary 20]{dummit}.








\section{Special values}
\begin{lemma}
$\Phi_1(0)=-1$, and for $n \geq 2$, $\Phi_n(0)=1$.
\label{Phi0}
\end{lemma}
\begin{proof}
$\Phi_1(x)=x-1$, so $\Phi_1(0)=-1$. For $n \geq 2$, using \eqref{mobiussum},
\[
\Phi_n(0) = \prod_{d \mid n} (-1)^{\mu(n/d)} = (-1)^{\sum_{d \mid n} \mu(n/d)} = (-1)^{\sum_{d \mid n} \mu(d)}
=(-1)^0 = 1.
\]
\end{proof}

Let $\Lambda$ be the \textbf{von Mangoldt function}: $\Lambda(n)=\log p$ if $n=p^\alpha$ for some prime $p$ and some integer
$\alpha \geq 1$, and is $\Lambda(n)=0$ otherwise. Thus $\Lambda(2)=\log 2$, $\Lambda(8)=\log 2$, $\Lambda(3)=\log 3$,
and $\Lambda(6) = 0$. One sees that
\[
\log n = \sum_{d \mid n} \Lambda(d).
\]
Therefore by the M\"obius inversion formula,
\[
\Lambda(n) = \sum_{d \mid n} \mu(n/d) \log d.
\]

\begin{theorem}
For $n>1$,
\[
\Phi_n(1) =  e^{\Lambda(n)}
\]
and
\[
\Phi_n'(1) =\frac{1}{2} e^{\Lambda(n)}  \phi(n).
\]
\label{specialvalues}
\end{theorem}
\begin{proof}
For $n>1$,
\[
x^{n-1}+\cdots+x+1 = \prod_{d \mid n, d>1} \Phi_d(x),
\]
hence
\[
\log n = \sum_{d \mid n, d>1} \log \Phi_d(1).
\]
Therefore by the M\"obius inversion formula,
\[
\log \Phi_n(1) = \sum_{d \mid n, d>1} \mu(n/d)  \log d = \sum_{d \mid n} \mu(n/d) \log d = \Lambda(n).
\]


Because $x^n-1 = \prod_{d \mid n} \Phi_d(x)$, taking the logarithm and then taking the derivative
yields
\[
\frac{nx^{n-1}}{x^n-1} = \sum_{d \mid n} \frac{\Phi_d'(x)}{\Phi_d(x)}.
\]
$\Phi_1(x) = x-1$ and so $\frac{\Phi_1'(x)}{\Phi_1(x)} = \frac{1}{x-1}$, hence
\[
\frac{nx^{n-1}}{x^n-1}  - \frac{1}{x-1} = \sum_{d \mid n, d>1} \frac{\Phi_d'(x)}{\Phi_d(x)},
\]
i.e.
\[
\frac{nx^{n-1} - (x^{n-1}+x^{n-2}+\cdots+x+1)}{x^n-1} = \sum_{d \mid n, d>1} \frac{\Phi_d'(x)}{\Phi_d(x)}.
\]
Doing polynomial long division we find
\[
\frac{(n-1)x^{n-1} - x^{n-2} - \cdots - x - 1}{x-1} = (n-1)x^{n-2} + (n-2)x^{n-3} + \cdots + 2x+1.
\]
Hence
\[
\frac{(n-1)x^{n-2} + (n-2)x^{n-3} + \cdots + 2x+1}{x^{n-1}+x^{n-2}+\cdots+x+1} =  \sum_{d \mid n, d>1} \frac{\Phi_d'(x)}{\Phi_d(x)},
\]
and for $x=1$ this is
\[
\frac{n-1}{2} = \sum_{d \mid n, d>1} \frac{\Phi_d'(1)}{\Phi_d(1)}.
\]
By the M\"obius inversion formula,
\[
\frac{\Phi_n'(1)}{\Phi_n(1)} = \sum_{d \mid n, d>1} \mu(n/d) \cdot  \frac{d-1}{2},
\]
and using (i) $\Phi_n(1) = e^{\Lambda(n)}$ for $n>1$, (ii) $\sum_{d \mid n} \mu(n/d) = 0$ for $n>1$,
and (iii) $\sum_{d \mid n} d \cdot \mu(n/d) = \phi(n)$, we have
\[
\Phi_n'(1) =  e^{\Lambda(n)} \frac{1}{2} \sum_{d \mid n} \mu(n/d) \cdot d -e^{\Lambda(n)} \frac{1}{2} \sum_{d \mid n} \mu(n/d)
=\frac{1}{2} e^{\Lambda(n)} \phi(n).
\]
\end{proof}





Because $\Phi_n \in \mathbb{Z}[x]$, it is the case that $\Phi_n(-i) = \overline{\Phi_n(i)}$.


\begin{theorem}
$\Phi_1(i)=i-1$, $\Phi_2(i) = i+1$, $\Phi_4(i)=0$, and otherwise
we have the following.

\begin{itemize}
\item If $n$ is odd and has a prime factor $p \equiv 1 \pmod{4}$, then $\Phi_n(i)=1$.
\item If $p \equiv 3 \pmod{4}$ is prime and $k \geq 1$ is odd, then
$\Phi_{p^k}(i) = i$.
\item If $p \equiv 3 \pmod{4}$ is prime and $k \geq 1$ is even, then
$\Phi_{p^k}(i) = -i$.
\item If $p \equiv 3 \pmod{4}$ is prime and $k \geq 1$ is odd, then
$\Phi_{2p^k}(i)=-i$. 
\item If $p \equiv 3 \pmod{4}$ is prime and $k \geq 1$ is even, then
$\Phi_{2p^k}(i)=i$.
\item If $p,q \equiv 3 \pmod{4}$ are distinct primes and $k, l \geq 1$, then
$\Phi_{p^k q^l}(i) = -1$.
\item If $p,q \equiv 3 \pmod{4}$ are distinct primes and $k, l \geq 1$, then
$\Phi_{2p^k q^l}(i) = -1$.
\item If $p$ is an odd prime and $k \geq 1$, then $\Phi_{4p^k}(i)=p$. 
\item If $\omega(n) \geq 3$ then $\Phi_n(i)=1$. 
\end{itemize}
\end{theorem}
\begin{proof}
$\Phi_1(x)=x-1$, $\Phi_2(x) = x+1$, so $\Phi_1(i)=i-1$ and  $\Phi_2(i)=i+1$.
As $i \in \Delta_4$,
$\Phi_4(i)=0$.

Suppose that $n$ is odd, that $p \equiv 1 \pmod{4}$ is a prime factor of $n$, and write $n=p^km$ with $\gcd(m,p)=1$. 
Lemma \ref{pmPhi} tells us
\[
\Phi_n(x) = \Phi_{p^k m}(x) = \frac{\Phi_m(x^{p^k})}{\Phi_m(x^{p^{k-1}})},
\]
and as $p^{k-1} \equiv 1 \pmod{4}$ and $i^4=1$, this yields
\[
\Phi_n(i) = \frac{\Phi_m(i)}{\Phi_m(i)} = 1.
\]

Suppose that $n$ is odd, that $p \equiv 3 \pmod{4}$ is a prime factor of $n$, and write $n=p^km$ with $\gcd(m,p)=1$. 
If $k$ is odd then $p^k \equiv 3 \pmod{4}$, so
\[
\Phi_n(i) = \frac{\Phi_m(i^{p^k})}{\Phi_m(i^{p^{k-1}})} = \frac{\Phi_m(i^3)}{\Phi_m(i)} = 
\frac{\Phi_m(-i)}{\Phi_m(i)},
\]
and if $m=1$ then
\[
\Phi_n(i) = \frac{\Phi_1(-i)}{\Phi_1(i)} = \frac{-i-1}{i-1}=i.
\]
If $k$ is even then $p^k \equiv 1 \pmod{4}$, so
\[
\Phi_n(i) = \frac{\Phi_m(i)}{\Phi_m(-i)},
\]
and if $m=1$ then
$\Phi_n(i) = -i$.

Suppose that $n=2^k$, $k \geq 3$.
Lemma \ref{radical} tells us that
\[
\Phi_n(x)=\Phi_2(x^{n/2})=\Phi_2(x^{2^{k-1}}) = x^{2^{k-1}} + 1,
\]
thus
\[
\Phi_n(i) = i^{2^{k-1}}+1 = 1+1=2.
\]

Suppose that $n=2m$ with $m>1$ odd. Lemma \ref{twiceodd} tells us
$\Phi_n(x) = \Phi_{2m}(x) = \Phi_m(-x)$, so 
$\Phi_n(i) = \Phi_m(-i)$.

Suppose that $n=2^k m$ with $k \geq 2$ and $m>1$ odd. Lemma \ref{pmPhi} tells us
\[
\Phi_{2^k m}(x) = \Phi_{2^{k-1} \cdot 2m} (x) = \Phi_{2m}(x^{2^{k-1}}),
\]
and then Lemma \ref{twiceodd} tells us $\Phi_{2m}(x^{2^{k-1}}) = \Phi_m(-x^{2^{k-1}})$. For $k=2$ this yields
\[
\Phi_{4m}(i) =\Phi_m(1),
\]
and for $k>2$,
\[
\Phi_n(i) =  \Phi_m(-i^{2^{k-1}}) = \Phi_m(-1).
\]



\end{proof}


Kurshan and Odlyzko \cite{kurshan}



Montgomery and Vaughan \cite[pp.~131--132, Exercise 9]{montgomery}.

\begin{theorem}
If $n=\prod_{p \leq y, p \equiv 2,3 \pmod{5}} p$ with $\omega(n)$  odd, then
\[
|\Phi_n(e^{2\pi i/5})| = \left( \frac{1+\sqrt{5}}{2} \right)^{d(n)/2}.
\]
\label{demr}
\end{theorem}
\begin{proof}
Write $e(x)=e^{2\pi ix}$, let $d \mid n$, $d>1$, and write $d=p_1\cdots p_k \cdot q_1 \cdots q_l$ where
$p_1,\ldots,p_k \equiv 2 \pmod{5}$ and $q_1,\ldots,q_l \equiv 3 \pmod{5}$ are prime. Then
$\omega(d) = k+l$ and, as $2^3 \equiv 3 \pmod{5}$,
\[
d \equiv 2^k 3^l \equiv 2^k 2^{3l} \equiv 2^{k+l} (-1)^l \pmod{5}.
\]
If $\omega(d) \equiv 0 \pmod{4}$ then $2^{k+l} \equiv 1 \pmod{5}$
and if $\omega(d) \equiv 2 \pmod{4}$ then
$2^{k+l}  \equiv -1 \pmod{5}$, and therefore if $\omega(d)$ is even then 
$d \equiv 1 \pmod{5}$ or $d \equiv -1 \pmod{5}$. Since $|e(-1/5)-1|=|e(1/5)-1|$, we have
$|e(d/5)-1| = |e(1/5)-1|$.

If $\omega(d) \equiv 1 \pmod{4}$ then $2^{k+l} \equiv 2 \pmod{5}$ and if $\omega(d) \equiv 3 \pmod{4}$ then
$2^{k+l} \equiv -2 \pmod{5}$, and therefore if $\omega(d)$ is odd then
$d \equiv 2 \pmod{5}$ or $d \equiv -2 \pmod{5}$. Since $|e(-2/5)-1|=|e(2/5)-1|$, we have
$|e(d/5)-1| = |e(2/5)-1|$.

Now using Lemma \ref{mobiuslemma} and $|e(1/5)-1|^{-1} =  |e(2/5)-1|$,
\begin{align*}
|\Phi_n(e(1/5))|&=\prod_{d \mid n} |e(d/5)-1|^{\mu(n/d)}\\
&=\prod_{d \mid n, \textrm{$\omega(d)$ even}}  |e(1/5)-1|^{-1}
\cdot \prod_{d \mid n, \textrm{$\omega(d)$ odd}}  |e(2/5)-1|.
\end{align*}
Hence, for $\omega(n)=2\nu+1$ and for $A= |e(1/5)-1|^{-1}$ and $B=|e(2/5)-1|$,
\begin{align*}
\log |\Phi_n(e(1/5))|&=\sum_{r=0}^\nu \binom{2\nu+1}{2r} \log A + \sum_{r=0}^\nu \binom{2\nu+1}{2r+1} \log B\\
&=2^{2\nu} \log A + 2^{2\nu} \log B\\
&=\log((AB)^{2^{\omega(n)} /2}),
\end{align*}
and using $d(n)=\sum_{r=0}^{\omega(n)} \binom{\omega(n)}{r} = 2^{\omega(n)}$ this is $|\Phi_n(e(1/5))| = (AB)^{d(n)/2}$.
Finally,
\[
AB = \frac{|e(2/5)-1|}{|e(1/5)-1|} =|e(1/5)+1|= \frac{1+\sqrt{5}}{2}.
\]
\end{proof}







\section{Primes in arithmetic progressions}
For prime $p$, $p \nmid n$, the following theorem
relates the order of an element of the multiplicative group $(\mathbb{Z}/p)^*$ 
with $\Phi_n$ \cite[p.~13, Lemma 2.9]{washington}.
We remind ourselves that $\Phi_n \in \mathbb{Z}[x]$ (Theorem \ref{Zx}), and so
$\Phi_n(a) \in \mathbb{Z}$ for $a \in \mathbb{Z}$.

\begin{lemma}
Let $p$ be prime, $p  \nmid n$, and $a \in \mathbb{Z}$. Then $p \mid \Phi_n(a)$ if and only if
$n$ is the multiplicative order of $a$ modulo $p$.
\label{multiplicativeorder}
\end{lemma}
\begin{proof}
Suppose that $p \mid \Phi_n(a)$. Now, let
$b \in \mathbb{Z}$ with 
 $p \mid \Phi_n(b)$. By Lemma \ref{mobiuslemma}, 
$b^n-1 = \prod_{d \mid n} \Phi_d(b)$,
and because $\Phi_n(b) \equiv 0 \pmod{p}$ this yields
$b^n-1 \equiv 0 \pmod{p}$, i.e. $b^n \equiv 1 \pmod{p}$; in particular, $p \nmid b$.
Let $\nu = \min \{k > 0: a^k \equiv 1 \pmod{p}\}$, the multiplicative order of $a$ modulo $p$, so $\nu \mid n$, and
suppose by contradiction that $\nu<n$. Using $x^\nu - 1 = \prod_{d \mid \nu} \Phi_d(x)$ we have
$b^\nu - 1 = \prod_{d \mid \nu} \Phi_d(b)$.
Using this with $b=a$, as $a^\nu \equiv 1 \pmod{p}$ and because $p$ is prime it follows
that for some $d_0 \leq \nu < n$, $\Phi_{d_0}(a) \equiv 0 \pmod{p}$. As $\nu \mid n$, 
\[
b^n-1 = \Phi_n(b) \Phi_{d_0}(b)  \cdot \prod_{d \mid n, d \neq d_0, n} \Phi_d(b).
\]
Applying the above with $b=a$ yields $a^n-1 \equiv 0 \pmod{p^2}$. 
Moreover, by the binomial theorem, $\Phi_n(a+p) \equiv \Phi_n(a) \equiv 0 \pmod{p}$ and
$\Phi_{d_0}(a+p) \equiv \Phi_{d_0}(a) \equiv 0 \pmod{p}$, so applying the above with $b=a+p$ yields
$(a+p)^n - 1 \equiv 0 \pmod{p^2}$.
But by the binomial theorem, $(a+p)^n -1 = \sum_{j=0}^n \binom{n}{j} a^{n-j} p^j - 1$, whence
$(a+p)^n - 1 \equiv a^n + na^{n-1}p - 1 \pmod{p^2}$,
hence $a^n + na^{n-1}p -1 \equiv 0 \pmod{p^2}$.
Together with $a^n-1 \equiv 0 \pmod{p^2}$ this yields
$na^{n-1}p \equiv 0 \pmod{p^2}$, i.e. $na^{n-1} \equiv 0 \pmod{p}$, contradicting that $p \nmid n, a$. 
Therefore $\nu = n$.

Suppose that $a^n \equiv 1 \pmod{p}$ and that $a^\nu \not \equiv 1 \pmod{p}$ for $0<\nu<n$. 
As $ \prod_{d \mid n} \Phi_d(a) = a^n - 1 \equiv 0 \pmod{p}$, 
there is some $d_0 \mid n$ for which $\Phi_{d_0}(a) \equiv 0 \pmod{p}$. Suppose by contradiction that
$d_0<n$. As $d_0 \mid n$,
\[
a^{d_0}-1 = \prod_{d \mid d_0} \Phi_d(a) = \Phi_{d_0}(a) \cdot \prod_{d \mid d_0, d<d_0} \Phi_d(a) \equiv 0 \pmod{p},
\]
contradicting that $a^\nu \not \equiv 1 \pmod{p}$ for $0<\nu<n$. Therefore $\Phi_n(a) \equiv 0 \pmod{p}$, i.e.
$p \mid \Phi_n(a)$.
\end{proof}


\begin{lemma}
Let $p$ be prime, $p \nmid n$. There is some $a \in \mathbb{Z}$ such that $p \mid \Phi_n(a)$ if and only if
$p \equiv 1 \pmod{n}$. 
\label{aPhi}
\end{lemma}
\begin{proof}
Suppose that $a \in \mathbb{Z}$ and $p \mid \Phi_n(a)$. Then by Lemma \ref{multiplicativeorder},
$n$ is the multiplicative order of $a$ modulo $p$. As the multiplicative group
$(\mathbb{Z}/p)^*$ has $p-1$ elements, this implies that $n \mid (p-1)$, i.e. $p-1 \equiv 0 \pmod{n}$.

Suppose that $p \equiv 1 \pmod{n}$, i.e. $n \mid (p-1)$. 
Because $(\mathbb{Z}/p)^*$ is a cyclic group with $p-1$ elements, it is a fact that 
there is some $a \in \mathbb{Z}$, $a+p\mathbb{Z} \in (\mathbb{Z}/p)^*$, whose multiplicative order modulo
$p$ is $n$. (Generally, if $G$ is a cyclic group with $m$ elements and $n$ divides $m$ then there is some
$g \in G$ with order $n$.) Then by Lemma \ref{multiplicativeorder}, $p \mid \Phi_n(a)$.
\end{proof}


We now use Lemma \ref{aPhi} to prove an instance of Dirichlet's theorem on primes in arithmetic progressions \cite[p.~13, Lemma 2.9]{washington}.


\begin{theorem}
For any $n \geq 1$, there are infinitely many primes $p$ with $p \equiv 1 \pmod{n}$.
\end{theorem}
\begin{proof}
The claim for $n=1$ follows from the claim for $n=2$. 
For $n \geq 2$, by Lemma \ref{Phi0}, $\Phi_n(0)=1$, namely the constant coefficient of $\Phi_n(x)$ is $1$. 
Suppose by contradiction that there are at most finitely many such primes $p_1,\ldots,p_t$
and let $M = np_1\cdots p_t$. For $N \in \mathbb{Z}$, $\Phi_n(NM) \equiv 1 \pmod{M}$ and
from $M \mid (\Phi_n(NM) - 1)$ it follows that $p_i \mid (\Phi_n(NM)-1)$, $1 \leq i \leq t$, and
$n \mid (\Phi_n(NM)-1)$. Hence if $p$ is a prime factor
of $\Phi_n(NM)$ then $p \neq p_i$, $1 \leq i \leq t$, and $p \nmid n$.
As $\Phi_n$ is a  monic polynomial that is not a constant,
for all sufficiently large $N$, $\Phi_n(NM)$ is an integer $\geq 2$ and thus has a prime factor $p$, amd we have
 established that $p \nmid n$. Therefore
Lemma \ref{aPhi} tells us that $p \equiv 1 \pmod{n}$. 
But we have also established that $p \neq p_i$, $1 \leq i \leq r$, a contradiction. Therefore
there are infinitely many primes $p$ with $p \equiv 1 \pmod{n}$.
\end{proof}

One can prove that for any integers $n,b \geq 2$ it holds that
\[
\frac{1}{2} \cdot b^{\phi(n)} \leq \Phi_n(b) \leq 2\cdot b^{\phi(n)}.
\]
Using this, Thangadurai and Vatwani \cite{thangadurai} prove that
for $n \geq 2$, the least prime $p \equiv 1 \pmod{n}$ satisfies
\[
p \leq 2^{\phi(n)+1}-1.
\]



\section{Zsigmondy's theorem}

\cite[pp.~167--169, \S 8.3.1]{everest}


\section{Newton's identities and Ramanujan sums}
For positive integers $n$ and $n$,
let
\[
c_n(k) = \sum_{1 \leq j \leq n, \gcd(n,j)=1} e^{2\pi ijk/n} = \sum_{\xi \in \Delta_n} \xi^k,
\]
called a \textbf{Ramanujan sum}.


\begin{lemma}
\[
c_n(k) = \sum_{d \mid \gcd(n,k)} d\cdot \mu(n/d).
\]
\label{ramanujanmobius}
\end{lemma}
\begin{proof}
Let
\[
\eta_n(k) = \sum_{j=1}^n e^{2\pi ijk/n} 
=\begin{cases}
0&n \nmid k\\
m&n \mid k.
\end{cases}
\]
We can write $\eta_n(k)$ as
\[
\eta_n(k) = \sum_{d \mid n} c_d(k),
\]
so by the M\"obius inversion formula, 
\[
c_n(k) = \sum_{d \mid n} \mu(n/d) \eta_d(k).
\]
\end{proof}



\begin{theorem}
For $n>1$ and for $|x|<1$,
\[
\Phi_n(x) = \exp\left(
-\sum_{m=1}^\infty \frac{c_n(m)}{m} x^m\right).
\]
\label{ramanujanexp}
\end{theorem}
\begin{proof}
Using that $\xi \mapsto \xi^{-1}$ is a bijection $\Delta_n \to \Delta_n$,
\begin{align*}
 \frac{d}{dx} \log \Phi_n(x)&=\frac{d}{dx} \sum_{\xi \in \Delta_n} \log (x-\xi)\\
&=\sum_{\xi \in \Delta_n} \frac{1}{x-\xi}\\
&=\sum_{\xi \in \Delta_n} -\frac{1}{\xi} \cdot \frac{1}{1-\frac{x}{\xi}}\\
&=- \sum_{\xi \in \Delta_n} \frac{1}{\xi} \sum_{m=0}^\infty \left(\frac{x}{\xi}\right)^m\\
&=-\sum_{m=0}^\infty x^m \sum_{\xi \in \Delta_n} \xi^{m+1}.
\end{align*}
Because $n>1$, $\Phi_n(0)=1$, and integrating,
\[
\Phi_n(x) = \exp\left( -\sum_{m=0}^\infty \frac{x^{m+1}}{m+1} \sum_{\xi \in \Delta_n} \xi^{m+1}\right)
=\exp\left(-\sum_{m=1}^\infty \frac{x^m}{m} c_n(m)\right).
\]
\end{proof}


A formula due to H\"older \cite[p.~110, Theorem 4.1]{montgomery} is that  
\begin{equation}
c_n(k) = \frac{\mu(n/\gcd(n,k))\cdot \phi(n)}{\phi(n/\gcd(n,k))}.
\label{holder}
\end{equation}

This identity is used to prove the following lemma that we  use later.

\begin{lemma}
If $n$ is square-free then $k \mapsto \mu(n) c_n(k)$ is multiplicative. 
\label{squarefree}
\end{lemma}



\begin{lemma}
For $n \geq 1$ and $\Re s>1$,
\[
\sum_{k=1}^\infty c_n(k) k^{-s} = \zeta(s) \cdot \sum_{d \mid n} \mu(n/d) d^{1-s}.
\]
\label{ramanujandirichlet}
\end{lemma}
\begin{proof}
By Lemma \ref{ramanujanmobius},
\begin{align*}
\sum_{k=1}^\infty c_n(k) k^{-s} &=\sum_{k=1}^\infty k^{-s} \sum_{d \mid n, d \mid k} \mu(n/d) d\\
&=\sum_{d \mid n} \sum_{m=1}^\infty (md)^{-s} \mu(n/d) d\\
&=\sum_{d \mid n} \sum_{m=1}^\infty m^{-s} d^{-s} \mu(n/d) d\\
&=\sum_{m=1}^\infty m^{-s} \sum_{d \mid n} d^{-s} \mu(n/d) d\\
&=\zeta(s) \cdot \sum_{d \mid n} \mu(n/d) d^{1-s}.
\end{align*}
\end{proof}



Write
\[
\prod_{j=1}^n (x-\alpha_j) = \sum_{k=0}^n (-1)^k s_k x^{n-k},
\]
and put, for $k \geq 1$,
\[
p_k = \sum_{j=1}^n \alpha_j^k.
\]
\textbf{Newton's identities} \cite[p.~32, Proposition 3.4]{escofier} state that for $k \geq 1$,
\begin{equation}
p_k = \sum_{j=1}^{k-1} (-1)^{j-1} s_j p_{k-j} + (-1)^{k-1} k s_k.
\label{newtonpk}
\end{equation}
Write
\[
\Phi_n(x) = \sum_{k=0}^{\phi(n)} a_n(k) x^k.
\]
Let $n > 1$, and for  integer $j$ define
\[
\chi_1(j) = \begin{cases}
1&\gcd(n,j)=1\\
0&\gcd(n,j)>1,
\end{cases}
\]
namely the \textbf{principal Dirichlet character modulo $n$}. 
We can then write 
\[
\Phi_n(x) = \prod_{1 \leq k \leq n, \gcd(n,k)=1} (x-e^{2\pi ik/n})
=x^{-n+\phi(n)}  \prod_{j=1}^n (x-\alpha_j)
\]
for
 $\alpha_j = \chi_1(j) e^{2\pi ij/n}$, and thus
 \[
x^{n-\phi(n)} \Phi_n(x) = \prod_{j=1}^n (x-\alpha_j).
 \]
Because $\chi_1(j)^k=\chi_1(j)$ for $k \geq 1$,
\[
p_k = \sum_{j=1}^n \alpha_j^k = \sum_{j=1}^n \chi_1(j) e^{2\pi ijk/n}
=\sum_{1 \leq j \leq n, \gcd(n,j)=1} e^{2\pi ijk/n} = c_n(k).
\]
Now, from
\[
x^{n-\phi(n)} \sum_{k=1}^{\phi(n)}  a_n(k) x^k =  \sum_{k=0}^n (-1)^k s_k x^{n-k}
\]
we have, for $0 \leq k \leq n$,
\[
(-1)^k s_k = a_n(\phi(n)-k).
\]
In fact by Lemma \ref{palindrome}, $a_n(\phi(n)-k)=a_n(k)$, so $a_n(k) = (-1)^k s_k$. 
Thus \eqref{newtonpk} yields the following,
and in particular 
\[
a_n(1) = -c_n(1) = -\mu(n).
\]

\begin{theorem}
For $n \geq 1$ and $k \geq 1$,
\[
ka_n(k) = -c_n(k) - \sum_{j=1}^{k-1} a_n(j) c_n(k-j).
\]
\end{theorem}





Let $n$ be a product of distinct odd primes and for $a \in \mathbb{Z}$ let $\chi(a) = \left( \frac{a}{n}\right)$ be the \textbf{Jacobi symbol}. 
Dedekind, in Supplement I to Dirichlet's {\em Vorlesungen \"uber Zahlentheorie}  \cite[pp.~208--210]{dirichlet}, \S 116, proves that
\begin{equation}
\sum_{1 \leq j \leq n} \chi(j) e^{2\pi ij h/n} = \chi(h) i^{(n-1)^2/4} \sqrt{n};
\label{gausssum}
\end{equation}
this is proved earlier by 
Gauss in his {\em Summatio quarumdam serierum singularium} \cite[pp.~9--45]{gaussII}, dated 1808.
The expression $G(h,\chi)=\sum_{1 \leq j \leq n} \chi(j) e^{2\pi ij h/n}$ is called a \textbf{Gauss sum}.
Dedekind, in Supplement VII to Dirichlet's {\em Vorlesungen}, says what amounts to
the following.
Define
\[
A_n(x) = \prod_{1 \leq a \leq n, \chi(a)=1} (x-e^{2\pi ia/n}) = \sum_j \alpha_n(j) x^j
\] 
and
\[
B_n(x) = \prod_{1 \leq b \leq n, \chi(b)=-1} (x-e^{2\pi ib/n}) = \sum_j \beta_n(j) x^j,
\]
and write
\[
S_n(k) =  \sum_{1 \leq a \leq n, \chi(a)=1} e^{2\pi ik a/n},
\qquad T_n(k) = \sum_{1 \leq b \leq n, \chi(b)=-1} e^{2\pi ik b/n}.
\]
Then
\[
\Phi_n(x) = A_n(x) B_n(x),\qquad c_n(k) = S_n(k) + T_n(k),
\]
and by \eqref{gausssum}, writing
\[
n^*=(-1)^{(n-1)/2} n,
\]
we have
\[
S_n(k) - T_n(k) =\sum_{1 \leq j \leq n} \chi(j) e^{2\pi ik j/n} = \chi(k)  \sqrt{n^*},
\]
hence
\[
2S_n(k) = c_n(k) +  \chi(k)   \sqrt{n^*},
\quad
2T_n(k) = c_n(k) -  \chi(k)  \sqrt{n^*}.
\]
We have  established in Lemma \ref{ramanujanmobius} that $c_n(k) \in \mathbb{Z}$, so this shows that
$S_n(k), T_n(k) \in \mathbb{Q}(\sqrt{n^*})$.
Newton's identities yield for $k \geq 1$,
\[
S_n(k) = -\sum_{j=1}^{k-1}  \alpha_n(n-j) S_n(k-j) - k\alpha_n(n-k)
\]
and
\[
T_n(k) = -\sum_{j=1}^{k-1}  \beta_n(n-j) T_n(k-j) - k\beta_n(n-k),
\]
and it follows that $\alpha_n(k),\beta_n(k) \in \mathbb{Q}( \sqrt{n^*})$.
Furthermore, $\alpha_n(k),\beta_n(k)$ are algebraic integers, so
$\alpha_n(k),\beta_n(k) \in \mathcal{O}_{\mathbb{Q}(\sqrt{n^*})}$. 
If $D$ is a square-free, it is a fact \cite[p.~698, \S 15.3]{dummit} that 
$\mathcal{O}_{\mathbb{Q}(\sqrt{D})} = \mathbb{Z}[\omega]$ for 
\[
\omega = \begin{cases}
\sqrt{D}&D \equiv 2,3 \pmod{4}\\
\frac{1+\sqrt{D}}{2}&D \equiv 1 \pmod{4},
\end{cases}
\]
and $n^*=(-1)^{(n-1)/2} n \equiv 1 \pmod{4}$, we have
$\mathcal{O}_{\mathbb{Q}(\sqrt{n^*})} = \mathbb{Z}[(1+\sqrt{n^*})/2]$. Thus
$\alpha_n(k),\beta_n(k) \in  \mathbb{Z}[(1+\sqrt{n^*})/2]$. 


It is a fact that $\mathbb{Q}( \sqrt{n^*}) \subset \mathbb{Q}(e^{2\pi i/n})$ \cite[p.~19, Proposition 5.13]{kato2}









Gauss, {\em Disquisitiones Arithmeticae}, Art. 357






\section{Algebraic theorems about coefficients of cyclotomic polynomials}
For $n \geq 1$, we write 
\[
\Phi_n(x) = \sum_{k=0}^{\phi(n)} a_n(k) x^k.
\]
Let 
\[
A(n) = \max_{0 \leq k \leq \phi(n)} |a_n(k)|
\]
and
\[
S(n) = \sum_{k=0}^{\phi(n)} |a_n(k)|.
\]
It is immediate that $A(n) \leq S(n)$. 

\begin{lemma}
For $n>1$ and for $0 \leq k \leq \phi(n)$,
\[
a_n(\phi(n)-k) = a_n(k).
\]
\label{palindrome}
\end{lemma}
\begin{proof}
For $P(x) = \sum_{j=0}^n a(j) x^j$,  check that $a(j) = a(n-j)$ for each $0 \leq j \leq n$ is equivalent to
$x^n P(x^{-1}) = P(x)$.  But because $n>1$, by Lemma \ref{reciprocal} we have $\Phi_n(x^{-1})=x^{-\phi(n)} \Phi_n(x)$,
so we obtain the claim.
\end{proof}

Migotti \cite{migotti} proves the following, and also calculates $a_{105}(7)=-2$.
The following is also proved by Bang \cite{bang}; cf. Beiter \cite{beiter1964}.

\begin{theorem}[Bang]
For odd primes $p<q$,
\[
a_{pq}(k) \in \{0,-1,1\}.
\] 
\end{theorem}
\begin{proof}
By Lemma \ref{mobiuslemma},
\begin{align*}
\Phi_{pq}(x)&=\frac{(x^{pq}-1)(x-1)}{(x^p-1)(x^q-1)}\\
&=\frac{(1-x)\sum_{\alpha=0}^{p-1} x^{\alpha q}}{1-x^p}\\
&=(1-x) \sum_{0 \leq \alpha \leq p-1}  x^{\alpha q} \cdot \sum_{\beta \geq 0} x^{\beta p}\\
&=\sum_{0 \leq \alpha \leq p-1, \beta \geq 0} x^{\alpha q+\beta p}
-\sum_{0 \leq \alpha \leq p-1, \beta \geq 0} x^{\alpha q+\beta p+1}\\
&=\sum_{0 \leq \alpha \leq p-1, \beta \geq 0, 0 \leq \delta \leq 1} (-1)^\delta x^{\alpha q+\beta p+\delta}.
\end{align*}
Suppose by contradiction that $\alpha_1 q+\beta_1 p + \delta_1 = 
\alpha_2 q +\beta_2 p +\delta_2$ with $\delta_1=\delta_2$. Then 
$q(\alpha_1-\alpha_2) = p(\beta_2-\beta_1)$, which implies that $p$ divides $\alpha_1-\alpha_2$. But
$0 \leq \alpha_1,\alpha_2 \leq p-1$ means $0 \leq |\alpha_1-\alpha_2| \leq p-1$, so $\alpha_1-\alpha_2=0$ and thence
$\beta_2-\beta_1=0$, which means that $(\alpha_1,\beta_1,\delta_1)=(\alpha_2,\beta_2,\delta_2)$. 
Therefore, for $0 \leq k \leq \phi(pq)$ there are zero, one, or two triples $(\alpha,\beta,\delta)$ such that
$k=\alpha q + \beta p + \delta$; if there are two such triples, then one has $\delta=0$ and one has $\delta=1$.
If there are no such triples, then $a_n(k)=0$. If there is one such triple $(\alpha,\beta,\delta)$, then $a_n(k)=(-1)^\delta$.
If there are two such triples, then $a_n(k) = (-1)^0+(-1)^1 = 0$. 
\end{proof}


Lam and Leung \cite{lam1996} determine the following explicit formula.

\begin{theorem}[Lam and Leung]
Suppose that $p<q$ are primes. Then there are  nonnegative integers $r,s$ such
$(p-1)(q-1)=rp+sq$, and for
$0 \leq k \leq \phi(pq)=(p-1)(q-1)$,
\[
a_{pq}(k) = \begin{cases}
1&\textrm{$0 \leq i \leq r$, $0 \leq j \leq s$ with $k=ip+jq$}\\
-1&\textrm{$r+1 \leq i \leq q-1$, $s+1 \leq j \leq p-1$ with $k+pq=ip+jq$}\\
0&\textrm{otherwise}
\end{cases}
\]
Furthermore,
\[
|\{k: 0 \leq k \leq \phi(pq), a_{pq}(k)=1\}| = (r+1)(s+1)
\]
and
\[
|\{k: 0 \leq k \leq \phi(pq), a_{pq}(k)=-1\}| = (p-s-1)(q-r-1).
\]
\end{theorem}
\begin{proof}
Because $\gcd(p,q)=1$, there is some $0 \leq r \leq q-1$ such that
\[
rp   \equiv -p+1  \pmod{q}.
\]
If $r=q-1$ then we get
from the above that $1 \equiv 0 \pmod{q}$, which is false because $q \neq 1$, so in fact
$0 \leq r \leq q-2$. 
Now,
\[
s=\frac{(p-1)(q-1)-rp}{q}=\frac{pq-p-q+1-rp}{q} 
\]
is an integer and
\[
s = \frac{p(q-r-1)-q+1}{q} \geq \frac{-q+1}{q}>-1,
\]
hence $s \geq 0$. Also,
$s \leq \frac{(p-1)(q-1)}{q} < p-1$, so $s \leq p-2$. We then have
\[
rp+sq=rp+(p-1)(q-1)-rp = (p-1)(q-1).
\]
 
For $\xi \in \Delta_{pq}$, because $\Phi_q(\xi^p)=0$ and $\Phi_p(\xi^q)=0$,
\[
\sum_{i=0}^r (\xi^p)^i = - \sum_{i=r+1}^{q-1} (\xi^p)^i,
\qquad
\sum_{j=0}^s (\xi^q)^j = - \sum_{j=s+1}^{p-1} (\xi^q)^j.
\]
(Because $0 \leq r \leq q-2$ and $0 \leq s \leq p-2$, each of the above four sums has a nonempty index
set.)
From this we have
\[
\left( \sum_{i=0}^r (\xi^p)^i \right) \left(\sum_{j=0}^s (\xi^q)^j \right)
-\left(\sum_{i=r+1}^{q-1} (\xi^p)^i\right) \left( \sum_{j=s+1}^{p-1} (\xi^q)^j\right)=0.
\]
Because $\xi^{-pq}=1$, 
this implies that each $\xi \in \Delta_{pq}$ is a zero of the polynomial
\[
f(x)  = \left( \sum_{i=0}^r x^{ip} \right) \left(\sum_{j=0}^s x^{jq} \right)
-\left(\sum_{i=r+1}^{q-1} x^{ip} \right) \left( \sum_{j=s+1}^{p-1} x^{jq} \right) x^{-pq};
\]
that this is indeed a polynomial follows from
\[
(r+1)p+(s+1)q-pq=rp+sq+p+q-pq=1.
\]
The first product is a monic polynomial of degree $rp+sq=\phi(pq)$. The second product is a polynomial
of degree
\[
(q-1)p+(p-1)q-pq=-p-q+pq=\phi(pq)-1.
\]
Therefore $f(x)$ is a monic polynomial of degree
$\phi(pq)$. Because each $\xi \in \Delta_{pq}$ is a zero of $f(x)$ and $f(x)$ is monic,
$f(x)=\Phi_{pq}(x)$. 
\end{proof}

Carlitz \cite{carlitz1966} proves the following.

\begin{theorem}
Let $p<q$ be primes,
let
\[
qu \equiv -1 \pmod{p},\qquad 0<u<p,
\]
let 
$\theta(pq)$ be the number of terms of $\Phi_{pq}$ with nonzero coefficients,
and
let $\theta_0(pq)$ be the number of terms of $\Phi_{pq}$ with positive coefficients.
Then
\[
\theta(pq) = 2\theta_0(pq)-1
\]
and
\[
\theta_0(pq) = (p-u)(uq+1)/p.
\]
\end{theorem}

Cobeli, Gallot, Moree and Zaharescu \cite{cobeli} give an exposition of $a_{pqr}(k)$ where $p<q<r$ are primes,
$p$ is fixed, and $q,r$ are free.






Bang \cite{bang} proves the following.

\begin{theorem}[Bang]
For odd primes $p<q<r$,
\[
A(pqr) \leq p-1.
\] 
\end{theorem}

Beiter \cite{beiter1968} proves the following improvement for a case of the above theorem.
If $p,q,r$, $3<p<q<r$, are odd primes for which
either $q \equiv \pm 1 \pmod{p}$ or $r \equiv \pm 1 \pmod{p}$, then
\[
A(pqr) \leq \frac{1}{2}(p+1).
\]



Bloom \cite{bloom} proves the following.

\begin{theorem}[Bloom]
For odd primes $p<q<r<s$,
\[
A(pqrs) \leq p(p-1)(pq-1).
\]
\end{theorem}

Gallot and Moree \cite{ternary}



The following is from Lehmer \cite{lehmer}, who says that it appears in an unpublished letter of Schur to Landau;
cf. Bourbaki \cite[V.~165, \S 11, Exercise 19]{bourbaki}.

\begin{theorem}[Schur]
For any odd $m \geq 3$ there are primes $p_1<p_2<\cdots<p_m$, with $p_1+p_2>p_m$.
For such primes,
\[
a_{p_1p_2\cdots p_m}(p_m) = -m+1.
\]
\end{theorem}
\begin{proof}
Write
\[
\pi(x) = |\{p: \textrm{$p$ is prime and $p \leq x$}\}|.
\]
For  $m \geq 3$, suppose by contradiction that if
$p_1<p_2<\cdots<p_m$ are primes then $p_1+p_2 \leq p_m$, and thus
$2p_1<p_m$.
For $k \geq 1$, 
as there are infinitely many primes, let $p_1$ be the least prime $> k$, and let
$k \leq p_1<p_2<\cdots<p_m$. 
Then
\[
\pi(2k)-\pi(k) = \pi(2k) - \pi(p_1) +1 \leq \pi(2p_1) - \pi(p_1) + 1
\leq (m-1)+1 = m.
\]
This yields, for $j \geq 1$,
\[
\pi(2^j) \leq m+\pi(2^{j-1}) \leq m+m+\pi(2^{j-2}) \leq 
\cdots \leq jm.
\]
But the prime number theorem tells us 
\[
\pi(2^j) \sim \frac{2^j}{j \log 2},\qquad j \to \infty,
\]
with which we get a contradiction. 

Let $m \geq 3$ be odd and let $p_1<p_2<\cdots<p_m$ be primes satisfying $p_1+p_2>p_m$,
and let $n=p_1  p_2 \cdots p_m$. Since $p_1+p_2>p_m$, for $1 \leq j,k \leq m$ we have $p_j+p_k \geq p_m+1$. 
It follows that if $d$ is a divisor of $n$ aside from $1$ and $p_1,\ldots,p_m$, and $\mu(n/d) \neq 0$, then
\[
(x^d-1)^{\mu(n/d)} \in x^{p_m+1} \mathbb{Z}[x].
\]
Therefore
\begin{align*}
\Phi_n(x)+x^{p_m+1} \mathbb{Z}[x]&=\prod_{d|n} (x^d-1)^{\mu(n/d)}+x^{p_m+1} \mathbb{Z}[x]\\
&=\prod_{d|n, \mu(d/n) \neq 0} (x^d-1)^{\mu(n/d)}+x^{p_m+1} \mathbb{Z}[x]\\
&=(x-1)^{-1} \cdot \prod_{j=1}^m (x^{p_j}-1)^{\mu(n/p_j)}+x^{p_m+1} \mathbb{Z}[x]\\
&=(x-1)^{-1} \cdot  \prod_{j=1}^m (x^{p_j}-1) +x^{p_m+1} \mathbb{Z}[x]\\
&=(x-1)^{-1} \cdot (-1+x^{p_1}+\cdots+x^{p_m}) +x^{p_m+1} \mathbb{Z}[x].
\end{align*}
Now,
\[
\begin{split}
&(x-1)^{-1} \cdot (-1+x^{p_1}+\cdots-x^{p_m}) +x^{p_m+1} \mathbb{Z}[x]\\
=&(1+x+x^2+\cdots+x^{p_m}) \cdot (1-x^{p_1}-\cdots-x^{p_m}) +x^{p_m+1} \mathbb{Z}[x].
\end{split}
\]
For $1 \leq i \leq m$, 
there is one and only one
$0 \leq j \leq p_m$ such that 
$p_i+j=p_m$. This implies that the coefficient of $x^{p_m}$ in the above expression is
$-m+1$. 
\end{proof}

Lehmer also states that in Rolf Bungers' 1934 dissertation, {\em \"Uber die Koeffizienten von Kreisteilungspolynomen} (University of G\"ottingen), 
it is proved that if there exist infinitely many twin primes then 
for any $M$ there are primes $p<q<r$ such that $A(pqr) \geq M$. Lehmer proves this without
the hypothesis that there are infinitely many twin primes.



For power series $A(x) = \sum_{k=0}^\infty a_k x^k$ and $B(x)=\sum_{k=0}^\infty b_k x^k$, write
\[
A \preceq B
\]
if $|a_k| \leq b_k$ for all $k$.
For power series $A,B,P,Q$ with $A \preceq P$ and $B \preceq Q$,
\[
|a_k+b_k| \leq |a_k|+|b_k| \leq p_k+q_k,
\]
so $A+B \preceq P+Q$, and
\[
\left| \sum_{i+j=k} a_i b_j \right| \leq \sum_{i+j=k} |a_i b_j|
\leq \sum_{i+j=k} p_i q_j,
\]
so $AB \preceq PQ$. 

Now,  
\[
x^d-1 \preceq \sum_{k=0}^\infty x^{kd},
\quad 1 \preceq \sum_{k=0}^\infty x^{kd},
\quad (x^d-1)^{-1} \preceq \sum_{k=0}^\infty x^{kd},
\]
 and since $\mu(n/d) \in \{0,1,-1\}$,
\begin{equation}
\Phi_n(x) = \prod_{d \mid n} (x^d-1)^{\mu(n/d)}
\preceq
\prod_{d \mid n} \left( \sum_{k=0}^\infty x^{kd} \right)
=\prod_{d \mid n} \frac{1}{1-x^d}.
\label{dproduct}
\end{equation}
Hence, because $1 \preceq \frac{1}{1-x^j}$,
\[
\Phi_n(x)  
\preceq \prod_{j=1}^\infty \frac{1}{1-x^j}.
\]
Let $n \mapsto p(n)$ be the partition function, the number of ways of writing
$n$ as a sum of positive integers, where the order does not matter. $p(0)=1$ and $p(n)=0$ for $n<0$, and
for example, $p(4)=5$ because $4=4,3+1,2+2,2+1+1,1+1+1+1$. It is a  fact that for $|x|<1$,
\[
 \prod_{j=1}^\infty \frac{1}{1-x^j} = \sum_{k=0}^\infty p(k) x^k,
\]
found by Euler.

\begin{theorem}
\[
|a_n(k)| \leq p(k),
\]
and so 
\[
A(n) = \max_{0 \leq k \leq \phi(n)} |a_n(k)| \leq
\max_{0 \leq k \leq \phi(n)} p(k)
\leq p(\phi(n)) \leq p(n).
\]
\label{partition}
\end{theorem}

It is proved by Hardy and Ramanujan \cite[p.~166, Chapter VII]{chandra} that
for $K=\pi \sqrt{\frac{2}{3}}$ and $\lambda_n = \sqrt{n-\frac{1}{24}}$, 
\[
p(n) = \frac{e^{K\lambda_n}}{4\sqrt{3}\cdot \lambda_n^2} + O\left( \frac{e^{K\lambda_n}}{\lambda_n^3}\right),
\qquad n \to \infty.
\]
This implies 
\[
p(n) \sim \frac{e^{K \sqrt{n}}}{4\sqrt{3} \cdot n}, \qquad n \to \infty.
\]
Therefore,
\[
A(n) = O\left(\frac{e^{K\sqrt{n}}}{n}\right),\qquad
S(n) = O(e^{K\sqrt{n}}),
\qquad n \to \infty
\]



Now let 
\[
Q_n(x) = \prod_{d|n} (1+x^d+x^{2d}+\cdots+x^{n-d}).
\]
It is straightforward that for $0 \leq k < n$, the coefficient of $x^k$ in $Q_n(x)$
is equal to the coefficient of $x^k$ in $ \prod_{d \mid n} \frac{1}{1-x^d}$. 
For $n>1$, because the degree of $\Phi_n(x)$ is $\phi(n) < n$, using \eqref{dproduct} we get
\[
\Phi_n(x) \preceq Q_n(x).
\]
Let
\[
d(n) = \sum_{d \mid n} 1,
\]
the number of positive integer divisors of $n$. 
It is straightforward that
\[
\prod_{d \mid n} d = n^{d(n)/2},
\]
so
\[
Q_n(1) = \prod_{d \mid n} \frac{n}{d} = \prod_{d \mid n} d = n^{d(n)/2}.
\]
But from $\Phi_n(x) \preceq Q_n(x)$ we have that $S(n)$ is $\leq$ the sum of the coefficients of 
the polynomial $Q_n(x)$, i.e.
\[
S(n) \leq Q_n(1) = n^{d(n)/2}.
\]
This is found by Bateman \cite{bateman}; cf.  \cite[p.~64, Exercise 7]{montgomery}.

\begin{theorem}[Bateman]
\[
S(n) \leq \exp\left(\frac{1}{2}d(n) \log n\right).
\]
\end{theorem}

A result due to Wigert \cite[p.~19, Theorem 6]{chandra}, proved using the prime number theorem, is that
\[
\limsup_{n \to \infty} \log d(n) \cdot \frac{\log \log n}{\log n} = \log 2.
\]
Thus, for each $\epsilon>0$, there is some $n_\epsilon$ such that when $n \geq n_\epsilon$,
\[
\log d(n) \cdot \frac{\log \log n}{\log n} \leq \log 2 +\epsilon,
\]
so
\[
\log d(n) \leq \frac{\log n}{\log \log n} (\epsilon+\log 2).
\]
Then
\[
\log S(n) \leq \frac{d(n)}{2} \cdot \log n 
\leq \frac{\log n}{2} \exp\left( \frac{\log n}{\log \log n} (\epsilon+\log 2)\right).
\]



Wirsing \cite{wirsing}

Konyagin, Maier and Wirsing \cite{konyagin}

Maier \cite{maier1990}, \cite{maier1995}, \cite{maier2006},  \cite{maier2008}

Bachman \cite{bachman}

Bzd\k{e}ga \cite{bzdega}

Nicolas and Terjanian \cite{nicolas}




Let $\Psi_n(x) = \frac{x^n-1}{\Phi_n(x)}$, i.e. $\Psi_n(x) = \prod_{d \mid n, d<n} \Phi_d(x)$, which belongs to $\mathbb{Z}[x]$ and is monic.
Moree \cite{moree2009} proves the following.







\section{Analytic theorems about coefficients of cyclotomic polynomials}
Erd\H{o}s \cite{erdos1956}



Erd\H{o}s and Vaughan \cite{erdos1974} prove the following.


Vaughan \cite{vaughan} proves the next theorem. Vaughan's original proof is complicated and delightful,
and we first outline it and 
then give a radically
simplified proof using Theorem \ref{demr}, attributed to  Saffari by 
Montgomery and Vaughan \cite[pp.~131--132, Exercise 9]{montgomery}.

For $n=\prod_{p \leq y, p \equiv 2,3 \pmod{5}} p$ with $\omega(n)$ odd, let
$c_m=-\frac{c_n(m)}{m}$. Because $n$ is square-free and $\mu(n)=-1$, it follows from
Lemma \ref{squarefree} that
$m \mapsto c_m$ is multiplicative. Because $c_m = O(m^{-1})$, the following Euler product expansions
hold \cite[p.~20, Theorem 1.9]{montgomery}:
\[
\sum_{m=1}^\infty c_m m^{-s} = \prod_p \sum_{k=0}^\infty c_{p^k} p^{-ks},\qquad
\Re s>0
\]
and
\[
\sum_{m=1}^\infty \chi(m) c_m m^{-s} = \prod_p \sum_{k=0}^\infty \chi(p^k) c_{p^k} p^{-ks},
\qquad \Re s>0,
\]
where $\chi$ is the quadratic Dirichlet character modulo $5$.
Using H\"older's formula \eqref{holder} one works out that for $p \mid n$,
\[
\sum_{k=0}^\infty c_{p^k} p^{-ks} = \frac{1-p^{-s}}{1-p^{-(s+1)}}
\]
and for $p \nmid n$,
\[
\sum_{k=0}^\infty c_{p^k} p^{-ks} = \frac{1}{1-p^{-(s+1)}},
\]
thus
\[
\sum_{m=1}^\infty c_m m^{-s} = \zeta(1+s) \prod_{p \mid n} (1-p^{-s}), \qquad \Re s>0.
\]
Using H\"older's formula and that $\chi$ is completely multiplicative, one works out that for $p \mid n$,
\[
\sum_{k=0}^\infty \chi(p^k) c_{p^k} p^{-ks} = \frac{1+p^{-s}}{1-\chi(p) p^{-(s+1)}}
\]
and for $p \nmid n$,
\[
\sum_{k=0}^\infty \chi(p^k) c_{p^k} p^{-ks} = \frac{1}{1-\chi(p) p^{-(s+1)}},
\]
thus
\[
\sum_{m=1}^\infty \chi(m) c_m m^{-s} = L(1+s,\chi) \prod_{p \mid n} (1+p^{-s}),\qquad \Re s>0.
\]
Using (i) the fact that the Gauss sum $\sum_{r=1}^4 \chi(r) e^{2\pi ira/5}$ is equal to $\chi(a) \sqrt{5}$, (ii) the fact that
$c_{5m}=\frac{c_m}{5}$, and (iii) $e^{2\pi im/5} + e^{2\pi i \cdot 4m/5} = 2 \cdot \Re e^{2\pi im/5}$, one works out that
for $x>0$,
\[
4 \cdot \Re \sum_{m=1}^\infty c_m e^{2\pi im/5} e^{-m/x} = \sum_{m=1}^\infty c_m \left( \sqrt{5}\cdot \chi(m) e^{-m/x}
+e^{-5m/x}-e^{-m/x}\right).
\]
Using this and the above Euler product expansions we get for $s>0$,
\[
\begin{split}
&\int_0^\infty \left( \Re \sum_{m=1}^\infty c_m e(m/5) e^{-m/x} \right) x^{-s-1} dx\\
=&\frac{\Gamma(s)}{4} \left( \sqrt{5} \cdot L(1+s,\chi)  \prod_{p \mid n} (1+p^{-s})
-(1- 5^{-s}) \zeta(1+s)   \prod_{p \mid n} (1-p^{-s}) \right).
\end{split}
\]
For $x>0$, writing $f(x) = \Re \sum_{m=1}^\infty c_m e^{2\pi im/5} e^{-m/x}$,
one has for $0<\sigma<1$,
\[
\int_0^\infty f(x) x^{-\sigma-1} dx \leq \frac{1}{1-\sigma} +  \frac{1}{\sigma} \sup_{x \geq 1} f(x),
\]
so
\[
\begin{split}
&\sup_{x \geq 1} f(x)\\
\geq &\sigma \int_0^\infty f(x) x^{-\sigma-1} dx - \frac{\sigma}{1-\sigma}\\
=&\frac{\sigma\Gamma(\sigma)}{4} \left( \sqrt{5} \cdot L(1+\sigma,\chi)  \prod_{p \mid n} (1+p^{-\sigma})
-(1- 5^{-\sigma}) \zeta(1+\sigma)   \prod_{p \mid n} (1-p^{-\sigma}) \right)\\
& - \frac{\sigma}{1-\sigma}.
\end{split}
\]
As $\sigma \to 0$ we have $\sigma \Gamma(\sigma) = 1+O(\sigma)$, $(1-5^{-\sigma})\zeta(1+\sigma) = \log 5 + O(\sigma)$, and
$1-p^{-\sigma} = \sigma \log p+O(\sigma^2)$, thus
\[
\sup_{x \geq 1} f(x) \geq \frac{1}{4} \cdot \sqrt{5} \cdot L(1,\chi) \cdot 2^{\omega(n)}= \frac{1}{4} \cdot \sqrt{5} \cdot L(1,\chi) \cdot d(n).
\]
But  Theorem \ref{ramanujanexp} tells us that for $|z|<1$,
\[
|\Phi_n(z)| = \exp\left( \Re \sum_{m=1}^\infty c_m z^m\right),
\]
so $|\Phi_n(e^{2\pi i/5} e^{-1/x})| = e^{f(x)}$ and thus
\[
\sup_{|z|<1} |\Phi_n(z)| \geq \exp\left( \frac{1}{4} \cdot \sqrt{5} \cdot L(1,\chi) \cdot d(n) \right).
\]
As $\chi$ is the quadratic Dirichlet character modulo $5$, it is a fact that
$L(1,\chi)$ can be explicitly evaluated (this is an instance of Dirichlet's class number formula),
and using this one checks
that $\exp\left( \frac{1}{2} \cdot \sqrt{5} \cdot L(1,\chi)\right) = \frac{1+\sqrt{5}}{2}$.
Therefore
\[
\sup_{|z|<1} |\Phi_n(z)| \geq \left( \frac{1+\sqrt{5}}{2}\right)^{d(n)/2}.
\]


\begin{theorem}[Vaughan]
If $n=\prod_{p \leq y, p \equiv 2,3 \pmod{5}} p$ with $\omega(n)$  odd, then
\[
|\Phi_n(e^{2\pi i/5})| = \left( \frac{1+\sqrt{5}}{2} \right)^{d(n)/2}.
\]
There are infinitely many $n$ such that 
\[
\log A(n) > \exp\left( \frac{(\log 2)(\log n)}{\log \log n}\right).
\]
\end{theorem}
\begin{proof}


\end{proof}



Vaughan further proves the following.

\begin{theorem}[Vaughan]
There is some $C$ such that for infinitely many $k$,
\[
\log \max_{n \geq 1} |a_n(k)| \geq C k^{1/2} (\log k)^{-1/4}.
\]
\end{theorem}






\section{Fourier analysis}
Let $\mathbb{T} = \mathbb{R} / \mathbb{Z}$.
For $p \geq 1$, define
\[
\norm{f}_{L^p} = \left( \int_0^1 |f(x)|^p dx \right)^{1/p}
\]
and $\norm{f}_{L^\infty} = \sup_{x \in [0,1]} |f(x)|$.
By Jensen's inequality,
if $1 \leq p \leq q \leq \infty$ then
\[
\norm{f}_{L^p} \leq \norm{f}_{L^q}.
\]

For $f \in L^1(\mathbb{T})$, define $\widehat{f}:\mathbb{Z} \to \mathbb{C}$ by
\[
\widehat{f}(k) = \int_0^1 e^{-2\pi i kx} f(x) dx.
\]

Define
\[
\norm{\widehat{f}}_{\ell^p} = \left( \sum_{k \in \mathbb{Z}} |\widehat{f}(k)|^p \right)^{1/p}
\]
and $\norm{\widehat{f}}_{\ell^\infty} = \sup_{k \in \mathbb{Z}} |\widehat{f}(k)|$.
For $1 \leq p \leq q \leq \infty$,
\[
\norm{\widehat{f}}_{\ell^q} \leq \norm{\widehat{f}}_{\ell^p}.
\]


Plancherel's theorem  tells us that
\[
\norm{f}_{L^2} = \norm{\widehat{f}}_{\ell^2}.
\]
 The Hausorff-Young inequality states that for $1 \leq p \leq 2$ and $\frac{1}{p}+\frac{1}{q}=1$,
 \[
 \norm{\widehat{f}}_{\ell^q} \leq \norm{f}_{L^p}.
 \]

\textbf{Nikolsky's inequality} \cite[p.~102, Theorem 2.6]{devore} says that if
$\widehat{f}(k) = 0$ for $|k|>n$, namely $f$ is a trigonometric polynomial of degree $n$, then 
for $0<p \leq q \leq \infty$ and for $r \geq \frac{p}{2}$ an integer,
\[
\norm{f}_{L^q} \leq (2nr+1)^{\frac{1}{p}-\frac{1}{q}} \norm{f}_{L^p}.
\]
On the other hand, using Jensen's inequality for sums one proves that if $f$ is a trigonometric polynomial of degree $n$, then for
 $1 \leq p \leq q \leq \infty$, 
 \[
 \norm{\widehat{f}}_{\ell^p} \leq (2n+1)^{\frac{1}{p}-\frac{1}{q}} \norm{\widehat{f}}_{\ell^q}.
 \]

For $f:\mathbb{T} \to \mathbb{C}$, define
\[
\norm{\hat{f}}_{\ell^0} =|\supp \hat{f}| =  \left| \{n \in \mathbb{Z}: \hat{f}(n) \neq 0\} \right|.
\]
McGehee, Pigno and Smith \cite{mcgehee} prove that there is some $K$ such that for all $N$, if
$n_1,\ldots,n_N$ are distinct integers and $c_1,\ldots,c_N \in \mathbb{C}$ satisfy $|c_k| \geq 1$, then
\[
\norm{\sum_{k=1}^N c_k e^{2\pi in_k t}}_{L^1} \geq K \log N.
\]
That is, if $f:\mathbb{T} \to \mathbb{C}$ is a trigonometric polynomial with $|\hat{f}(n)| \geq 1$ when $\hat{f}(n) \neq 0$, then
\[
\norm{f}_{L^1} \geq K \log \norm{\hat{f}}_{\ell^0}.
\]



For $F:\mathbb{Z}/N \to \mathbb{C}$, define $\widehat{F}:\mathbb{Z}/N \to \mathbb{C}$ by
\[
\widehat{F}(k) = \frac{1}{N} \sum_{j=0}^{N-1} F(j) e^{-2\pi ijk/N},\qquad 0 \leq k \leq N-1.
\]
One checks that \cite[pp.~109--110, \S 4.1]{montgomery}
\[
F(j) = \sum_{k=0}^{N-1} \widehat{F}(k) e^{2\pi ijk/N},\qquad 0 \leq j \leq N-1
\]
and
\[
\sum_{k=0}^{N-1} |\widehat{F}(k)|^2 = \frac{1}{N} \sum_{j=0}^{N-1} |F(j)|^2.
\]
For $a_0,\ldots,a_{N-1} \in \mathbb{C}$, define $f:\mathbb{T} \to \mathbb{C}$ by
\[
f(x) = \sum_{k=0}^{N-1} a_k e^{2\pi ikx}
\]
and define $F:\mathbb{Z}/N \to \mathbb{C}$ by
\[
F(j) =f(j/N)= \sum_{k=0}^{N-1} a_k e^{2\pi i kj/N}, \qquad 0 \leq j \leq N-1,
\]
for which we calculate $\widehat{F}(k)=a_k$, for $0 \leq k \leq N-1$. 
Then
\[
\sum_{k=0}^{N-1} |a_k|^2  = \sum_{k=0}^{N-1} |\widehat{F}(k)|^2 
=\frac{1}{N} \sum_{j=0}^{N-1} |F(j)|^2 = \frac{1}{N} \sum_{j=0}^{N-1} |f(j/N)|^2.
\]




Carlitz \cite{carlitz1967}





\section{Algebraic topology}
Musiker and Reiner \cite{reiner}

Meshulam \cite{meshulam}


\nocite{*}

\bibliographystyle{plain}
\bibliography{cyclotomic}

\end{document}