\documentclass{article}
\usepackage{amsmath,amssymb,graphicx,subfig,mathrsfs,amsthm}
%\usepackage{tikz-cd}
\usepackage[draft]{hyperref}
\newcommand{\inner}[2]{\left\langle #1, #2 \right\rangle}
\newcommand{\tr}{\ensuremath\mathrm{tr}\,} 
\newcommand{\Span}{\ensuremath\mathrm{span}} 
\def\Re{\ensuremath{\mathrm{Re}}\,}
\def\Im{\ensuremath{\mathrm{Im}}\,}
\newcommand{\id}{\ensuremath\mathrm{id}} 
\newcommand{\rank}{\ensuremath\mathrm{rank\,}} 
\newcommand{\diam}{\ensuremath\mathrm{diam}} 
\newcommand{\osc}{\ensuremath\mathrm{osc}} 
\newcommand{\co}{\ensuremath\mathrm{co}\,} 
\newcommand{\supp}{\ensuremath\mathrm{supp}\,}
\newcommand{\Hom}{\ensuremath\mathrm{Hom}}
\newcommand{\ext}{\ensuremath\mathrm{ext}\,}
\newcommand{\ba}{\ensuremath\mathrm{ba}\,}
\newcommand{\cl}{\ensuremath\mathrm{cl}\,}
\newcommand{\dom}{\ensuremath\mathrm{dom}\,}
\newcommand{\Cyl}{\ensuremath\mathrm{Cyl}\,}
\newcommand{\extreals}{\overline{\mathbb{R}}}
\newcommand{\upto}{\nearrow}
\newcommand{\downto}{\searrow}
\newcommand{\norm}[1]{\left\Vert #1 \right\Vert}
\newtheorem{theorem}{Theorem}
\newtheorem{lemma}[theorem]{Lemma}
\newtheorem{proposition}[theorem]{Proposition}
\newtheorem{corollary}[theorem]{Corollary}
\theoremstyle{definition}
\newtheorem{definition}[theorem]{Definition}
\newtheorem{example}[theorem]{Example}
\begin{document}
\title{The Gottschalk-Hedlund theorem, cocycles, and small divisors}
\author{Jordan Bell}
\date{July 23, 2014}

\maketitle

\section{Introduction}
This note consists of my working through details in the paper {\em Resonances and small divisors} by \'Etienne Ghys.\footnote{\url{http://perso.ens-lyon.fr/ghys/articles/resonancesmall.pdf}} Aside from containing mathematics, Ghys makes thoughtful remarks about the history of physics, unlike the typically thoughtless statements 
people make about the Ptolemaic system. He insightfully states ``Kepler's zeroth law'': ``If the orbit of a planet is bounded, then it is
periodic.'' I can certainly draw a three dimensional bounded curve that is not closed, but that curve is not the orbit of a planet. It is also intellectually lazy to scorn Kepler's correspondence between
orbits and the Platonic solids (``Kepler's fourth law'').

\section{Almost periodic functions}
Suppose that $f:\mathbb{R} \to \mathbb{C}$ is continuous. For $\epsilon>0$, we call $T \in \mathbb{R}$ an \textbf{$\epsilon$-period of $f$} if 
\[
|f(t+T)-f(t)|<\epsilon, \qquad t \in \mathbb{R}.
\]
$T$ is a period of $f$ if and only if it is an $\epsilon$-period for all $\epsilon>0$.

We say that $f$ is \textbf{almost periodic} if for every $\epsilon>0$ there is some $M_\epsilon>0$ such that if $I$ is an interval 
of length $>M_\epsilon$ then there is an $\epsilon$-period in $I$.

If $f$ is periodic, then there is some $M>0$ such that if $I$ an interval of length $>M$ then at least one multiple $T$ of $M$ lies in $I$, and hence
for any $t \in \mathbb{R}$ we have $f(t+T) - f(t)=f(t)-f(t)=0$. Thus, for every $\epsilon>0$, if $I$ is an interval of length $>M$ then there is an
$\epsilon$-period in $I$. Therefore, with a periodic function, the length of the intervals $I$ need not depend on $\epsilon$, while for an almost
periodic function they may.


\section{The Gottschalk-Hedlund theorem}
The \textbf{Gottschalk-Hedlund theorem} is stated and proved in Katok and Hasselblatt.\footnote{Anatole Katok and Boris Hasselblat, {\em Introduction to the Modern Theory of Dynamical Systems},
p.~102, Theorem 2.9.4.} The following case of the Gottschalk-Hedlund theorem is from Ghys. We denote by
\[
\pi_1:\mathbb{R}/\mathbb{Z} \times \mathbb{R} \to 
\mathbb{R} / \mathbb{Z}, \qquad \pi_2:\mathbb{R}/\mathbb{Z} \times \mathbb{R} \to 
\mathbb{R} 
\]
 the projection maps. 

\begin{theorem}[Gottschalk-Hedlund theorem]
Suppose that $u:\mathbb{R}/\mathbb{Z} \to \mathbb{R}$ is continuous, that 
\[
\int_0^1 u(x) dx =0,
\]
that $x_0 \in \mathbb{R} / \mathbb{Z}$, and that $\alpha$ is irrational. If there is some $C$ such that
\begin{equation}
\left| \sum_{k=0}^n u(x_0+k\alpha) \right| \leq C, \qquad n \geq 0,
\label{Cinequality}
\end{equation}
then there is  a continuous function  $v:\mathbb{R}/\mathbb{Z} \to \mathbb{R}$ such that
\[
u(x)=v(x+\alpha)-v(x), \qquad x \in \mathbb{R}/\mathbb{Z}.
\]
\end{theorem}
\begin{proof}
Say there is some $C>0$ satisfying \eqref{Cinequality}. Define $g:\mathbb{R}/\mathbb{Z} \times \mathbb{R} \to \mathbb{R}/\mathbb{Z} \times \mathbb{R}$
by 
\[
g(x,y) = (x+\alpha,y+u(x)), \qquad x \in \mathbb{R} / \mathbb{Z}.
\]
For $n \geq 0$,
\[
g^n(x_0,0)=\Big(x_0+n\alpha,\sum_{k=0}^n u(x_0+k\alpha) \Big)
\]
The set $\{g^n(x_0,0): n \geq 0\}$, namely the orbit of $(x_0,0)$ under $g$, is contained in $\mathbb{R} / \mathbb{Z} \times [-C,C]$. 
Let $K$ be the closure of this orbit. 
Because $K$ is a metrizable topological space, for $(x,y) \in K$  there is a sequence $a(n)$ such that $g^{a(n)}(x_0,0) \to (x,y)$. As $g$ is continuous we get 
$g^{a(n)+1}(x_0,0) \to g(x,y)$, which implies that $g(x,y) \in K$. This shows that $K$ is invariant under $g$. Let $\mathscr{K}$ be the collection of nonempty compact
sets contained in $K$ and invariant under $g$. Thus $K \in \mathscr{K}$, so $\mathscr{K}$ is nonempty. We order $\mathscr{K}$ by $A \prec B$ when
$A \subset B$. If $\mathscr{C} \subset \mathscr{K}$ is a chain, let $C_0=\bigcap_{C \in \mathscr{C}} C$. It follows from $K$ being compact that $C_0$ is nonempty,
hence $C_0 \in \mathscr{K}$ and is a lower bound for the chain $\mathscr{C}$. Since every chain in $\mathscr{K}$ has a lower bound in $\mathscr{K}$,
by Zorn's lemma there exists a minimal element $M$ in $\mathscr{K}$: for every $A \in \mathscr{K}$ we have $M \prec A$, i.e. $M \subset A$. 
To say that $M$ is invariant under $g$ means that $g(M) \subset M$, and $M$ being a nonempty compact set contained in $K$ implies that
$g(M)$ is a nonempty compact set contained in $K$, hence by the minimality of $M$ we obtain $g(M)=M$.


The set $M$ is nonempty, so take $(x,y) \in M$. Because
$M$ is invariant under $g$,
$\{g^n(x,y): n \geq 0\} \subset M$.
The set
\[
\pi_1 \{g^n(x,y): n \geq 0\} = \{ x+n\alpha: n \geq 0\}
\]
 is dense in $\mathbb{R} / \mathbb{Z}$, hence $\pi_1(M)$ is dense in $\mathbb{R} / \mathbb{Z}$. Moreover,
  $M$ being compact implies that $\pi_1(M)$ is closed, so $\pi_1(M)=\mathbb{R}/\mathbb{Z}$. 
 
 For $t \in \mathbb{R}$, define $\tau_t:\mathbb{R} / \mathbb{Z} \times \mathbb{R} \to \mathbb{R} / \mathbb{Z} \times \mathbb{R}$
 by $\tau_t(x,y)=(x,y+t)$. For any $t$,
 \[
 \tau_t \circ g(x,y) = \tau_t  (x+\alpha,y+u(x)) = (x+\alpha,y+u(x)+t) = g(x,y+t) = g \circ \tau_t (x,y),
 \]
 so $\tau_t \circ g = g \circ \tau_t$. Hence, if $A \subset \mathbb{R} / \mathbb{Z} \times \mathbb{R}$
and $g(A) \subset A$, then $g(\tau_t (A)) = \tau_t \circ g(A) \subset \tau_t (A)$, namely, if $A$ is invariant under $g$ then $\tau_t(A)$ is invariant under $g$.
Therefore $\tau_t(M)$ is invariant under $g$, and so $M \cap \tau_t(M)$ is invariant under $g$. 
This intersection is compact and is contained in $K$, so either $M \cap \tau_t(M) = \emptyset$ or by the minimality of $M$, $M \cap \tau_t(M)=M$.
Suppose by contradiction that for some nonzero $t$, $M \cap \tau_t(M)=M$. Then using $g(M)=M$ we get 
 $\tau_t(M)=M$, and hence for any positive integer $k$ we have 
 $\tau_{kt}(M)=\tau_t^k(M) = M$. But because $M$ is compact, $\pi_2(M)$ is contained in some compact interval $I$, and then there is some 
 positive integer $k$ such that $\pi_2(\tau_{kt}(M))$ is not contained in $I$, a contradiction. Therefore, when $t \neq 0$ we have $M \cap \tau_t(M) = \emptyset$. 
 Let $x \in \mathbb{R} / \mathbb{Z}$. If there were 
  distinct $y_1,y_2 \in \mathbb{R}$ such that
 $(x,y_1),(x,y_2) \in M$, then with $t=y_2-y_1 \neq 0$ we get $\tau_t(x,y_1)=(x,y_2) \in M$, contradicting 
 $M \cap \tau_t(M) = \emptyset$. This shows that for each $x \in \mathbb{R} / \mathbb{Z}$ there is a unique 
 $y \in \mathbb{R}$ such that
 $(x,y) \in M$, and we denote this $y$ by $v(x)$, thus defining a function $v:\mathbb{R} / \mathbb{Z} \to \mathbb{R}$. 
 Then $M$ is the graph of $v$, and because $M$ is compact, it follows that the function $v$ is continuous. 
Let $(x,v(x)) \in M$. As $M$ is invariant under $g$,
\[
 (x+\alpha,v(x)+u(x)) = g(x,v(x)) \in M,
\]
and as $M$ is the graph of $v$ we get $v(x)+u(x)=v(x+\alpha)$ and hence
$v(x+\alpha)-v(x)=u(x)$, completing the proof.
\end{proof}



\section{Cohomology}
In this section I am following Tao.\footnote{Terence Tao, {\em Cohomology for dynamical systems}, \url{http://terrytao.wordpress.com/2008/12/21/cohomology-for-dynamical-systems/}}
Suppose that a group $(G,\cdot)$ acts on a set $X$ and that $(A,+)$ is an abelian group. A \textbf{cocycle} is a function
$\rho:G \times X \to A$ such that 
\begin{equation}
\rho(gh,x)=\rho(h,x)+\rho(g,hx), \qquad g,h \in G, \quad x \in X.
\label{cocycle}
\end{equation}
If $F:X \to A$ is a function, we call the function $\rho(g,x)=F(gx)-F(x)$ a \textbf{coboundary}. This satisfies
\[
\rho(gh,x)-\rho(g,hx)=F((gh)x)-F(x)-F(g(hx))+F(hx)=F(hx)-F(x)=\rho(h,x),
\]
showing that a coboundary is a cocycle.
We now show how to fit the notions of cocycle and coboundary into a general sitting of cohomology. We show that
they correspond respectively to a $1$-cocycle and a $1$-coboundary. 


For $n \geq 0$, an \textbf{$n$-simplex}
is an element of $G^n \times X$, i.e., 
a thing of the form $(g_1,\ldots,g_n,x)$, for $g_1,\ldots,g_n \in G$ and $x \in X$. 
We denote by $C_n(G,X)$  the free abelian group generated by the collection of all $n$-simplices, and an element of $C_n(G,X)$ is called an
\textbf{$n$-chain}.
In particular, the elements of $C_0(G,X)$ are formal $\mathbb{Z}$-linear combinations of elements 
of
$X$. For $n<0$, we define $C_n(G,X)$ to be the trivial group.

For $n>0$, we define the \textbf{boundary map} $\partial:C_n(G,X) \to C_{n-1}(G,X)$ by
\begin{eqnarray*}
\partial (g_1,\ldots,g_n,x) &=& (g_1,\ldots,g_{n-1},g_n x)\\
&&
+\sum_{k=1}^{n-1} (-1)^{n-k} (g_1,\ldots,g_{k-1},g_k g_{k+1},g_{k+2},\ldots,g_n,x)\\
&&+(-1)^n (g_2,\ldots,g_n,x).
\end{eqnarray*}
For $n \leq 0$ we define $\partial:C_n(G,X) \to C_{n-1}(G,X)$ to be the trivial map. If $n \leq 1$ then 
of course $\partial^2=0$. If $n \geq 2$,
 one writes out $\partial^2 (g_1,\ldots,g_n,x)$ and checks that it is equal to $0$, and hence that $\partial^2=0$.
Thus the sequence of abelian groups $C_n(G,X)$ and the boundary maps 
$\partial:C_n(G,X) \to C_{n-1}(G,X)$ are a \textbf{chain complex}. 

We denote the kernel of $\partial:C_n(G,X) \to C_{n-1}(G,X)$ by $Z_n(G,X)$, and elements of
$Z_n(G,X)$ are called \textbf{$n$-cycles}. We denote the image of $\partial:C_{n+1}(G,X) \to C_n(G,X)$
by $B_n(G,X)$, and elements of $B_n(G,X)$ are called \textbf{$n$-boundaries}. Because
$\partial^2=0$, an $n$-boundary is an $n$-cycle. 
$Z_n(G,X)$ and $B_n(G,X)$ are abelian groups and $B_n(G,X)$ is contained in $Z_n(G,X)$, and we write
\[
H_n(G,X) = Z_n(G,X) / B_n(G,X),
\]
and call $H_n(G,X)$ the \textbf{$n$th homology group}. 

We define $C^n(G,X,A) = \Hom(C_n(G,X),A)$, which is an abelian group.
 Elements of $C^n(G,X,A)$ are called 
 \textbf{$n$-cochains}.
 That is, an $n$-cochain is a group homomorphism
$C_n(G,X) \to A$. Because $C_n(G,X)$ is a free abelian group
generated by the collection of all $n$-simplices, an $n$-cochain is determined by the values it assigns to $n$-simplices. 
We thus identity $n$-cochains with functions $G^n \times X \to A$. 

We define the \textbf{coboundary map} $\delta:C^{n-1}(G,X,A) \to C^n(G,X,A)$ by
\[
(\delta F)(c) = F(\partial c), \qquad F \in C^{n-1}(G,X,A), c \in C_n(G,X).
\]
Explicitly, for $F \in C^{n-1}(G,X,A)$ and for an $n$-simplex  $(g_1,\ldots,g_n,x)$,
\begin{eqnarray*}
(\delta F)(g_1,\ldots,g_n,x)&=&F(\partial (g_1,\ldots,g_n,x))\\
&=&F(g_1,\ldots,g_{n-1},g_n x)\\
&&+\sum_{k=1}^{n-1}(-1)^{n-k} F (g_1,\ldots,g_{k-1},g_k g_{k+1},g_{k+2},\ldots,g_n,x)\\
&&+(-1)^n F (g_2,\ldots,g_n,x).
\end{eqnarray*}
For $F \in C^{n-2}(G,X,A)$, write $G=\delta F$ and take and $c \in C_n(G,X)$. Then,
\[
(\delta^2 F)(c) = (\delta G)(c) = G(\partial c) = (\delta F)(\partial c) = F(\partial^2 c)=F(0)=0,
\]
showing that $\delta^2=0$. Thus the sequence of abelian groups $C^n(G,X,A)$ and the coboundary maps
$\delta:C^{n-1}(G,X,A) \to C^n(G,X,A)$ are  a \textbf{cochain complex}.

We denote the kernel of $\delta:C^n(G,X,A) \to C^{n+1}(G,X,A)$ by 
$Z^n(G,X,A)$, and elements of $Z^n(G,X,A)$ are called \textbf{$n$-cocycles}. We denote
the image of $\delta:C^{n-1}(G,X,A) \to C^n(G,X,A)$ by $B^n(G,X,A)$, and elements of $B^n(G,X,A)$ are called
\textbf{$n$-coboundaries}. Because $\delta^2=0$, an $n$-coboundary is an $n$-cocycle.
$Z^n(G,X,A)$ and $B^n(G,X,A)$ are abelian groups and $B^n(G,X,A)$ is contained in $Z^n(G,X,A)$, and we write
\[
H^n(G,X,A) = Z^n(G,X,A) / B^n(G,X,A),
\]
which we call the \textbf{$n$th cohomology group}.

Take $n=1$. We identify $C^1(G,X,A)$, the group of $1$-chains, with functions $G \times X \to A$. 
For $\rho \in C^1(G,X,A)$, to say that
$\rho$ is a $1$-cocycle is equivalent to saying that for any $(g,h,x) \in G^2 \times X$,
$(\delta \rho)(g,h,x)=0$, i.e.
$\rho(g,hx)-\rho(gh,x)+\rho(h,x)=0$, i.e.
\[
\rho(gh,x)=\rho(h,x)+\rho(g,hx).
\]
To say that $\rho$ is a $1$-coboundary is equivalent to saying that there is a $0$-chain $F$ (a function $X \to A$)
such that $\rho = \delta F$, i.e., for any $(g,x) \in G \times X$,
\[
\rho(g,x) = (\delta F)(g,x) = F(gx)-F(x).
\]


\section{Small divisors}
Suppose that $u:\mathbb{R} / \mathbb{Z} \to \mathbb{R}$ be $C^\infty$ and satisfies
\[
\int_0^1 u(x) dx=0.
\]
For each $n \in \mathbb{Z}$, let
\[
\widehat{u}(n) = \int_0^1 e^{-2\pi i nx} u(x) dx.
\]
We have $\widehat{u}(0)=0$. For any $x \in \mathbb{R} / \mathbb{Z}$,
\[
u(x) = \sum_{n \in \mathbb{Z}} \widehat{u}(n) e^{2\pi i nx},
\]
and $\sum_{n \in \mathbb{Z}} |\widehat{u}(n)|<\infty$; for these statements to be true it suffices merely that
$u$ be $C^\beta$ for some $\beta>\frac{1}{2}$. 

Let $\alpha$ be irrational.
We shall find conditions under which there exists a continuous function $v:\mathbb{R} / \mathbb{Z} \to \mathbb{R}$ such that
\begin{equation}
u(x)=v(x+\alpha)-v(x), \qquad x \in \mathbb{R} / \mathbb{Z}.
\label{uvformula}
\end{equation}
Supposing that for each $x$, $v(x)$ is equal to its Fourier series evaluated at $x$ and that its Fourier series converges absolutely,
\[
v(x) = \sum_{n \in \mathbb{Z}} \widehat{v}(n) e^{2\pi i nx},
\]
then for each $x \in \mathbb{R} / \mathbb{Z}$,
\[
v(x+\alpha)-v(x) = \sum_{n \in \mathbb{Z}} \widehat{v}(n)\left( e^{2\pi in(x+\alpha)}-e^{2\pi i nx}\right)
= \sum_{n \in \mathbb{Z}} \widehat{v}(n)( e^{2\pi in\alpha}-1)e^{2\pi i nx}.
\]
Then using $u(x)=v(x+\alpha)-v(x)$ we obtain
\[
\widehat{u}(n) =  \widehat{v}(n)( e^{2\pi in\alpha}-1), \qquad n \in \mathbb{Z},
\]
or,
\begin{equation}
\widehat{v}(n) = \frac{\widehat{u}(n)}{e^{2\pi in\alpha}-1}, \qquad n \neq 0;
\label{vfourier}
\end{equation}
because $\alpha$ is irrational, the denominator of the right-hand side is indeed nonzero for $n \neq 0$. The value of $\widehat{v}(0)$
is not determined so far. 
We shall find conditions under which the continuous function $v$ we desire can be defined using \eqref{vfourier}.


A real number $\beta$ is said to be \textbf{Diophantine} if there is some $r \geq 2$ and some $C>0$ such that for all
$q>0$ and $p \in \mathbb{Z}$,
\begin{equation}
\left|\beta - \frac{p}{q} \right| > C q^{-r}.
\label{Diophantine}
\end{equation}
It is immediate that a Diophantine number is irrational. Suppose that  $\alpha$ satisfies \eqref{Diophantine}. 
Let $n \neq 0$ and let $p_n$ be the integer nearest $n\alpha$. Then
\begin{eqnarray*}
|e^{2\pi i n\alpha}-1|& =& |e^{2\pi i (n\alpha-p_n)}-1| \\
&\geq& \frac{2}{\pi} |2\pi(n\alpha-p_n)|\\
&=& 4|n\alpha-p_n|\\
&=&4|n|\left|\alpha-\frac{p_n}{n}\right|\\
&>&4|n|\cdot C|n|^{-r}\\
&=&4C |n|^{-r+1}.
\end{eqnarray*}
Because $u \in C^\infty$, it is straightforward to prove that for each nonnegative integer $k$ there is some $C_k >0$ such that 
\[
|\widehat{u}(n)| \leq C_k |n|^{-k}, \qquad n \neq 0.
\]
Therefore, for each nonnegative integer $k$, using \eqref{vfourier} we have
\begin{equation}
|\widehat{v}(n)| = \frac{|\widehat{u}(n)|}{|e^{2\pi in\alpha}-1|} < C_k |n|^{-k} \cdot \frac{1}{4C |n|^{-r+1}}
=\frac{C_k}{4C} |n|^{r-k-1}, \qquad n \neq 0.
\label{vinequality}
\end{equation}
One can prove that if $h_n$ are complex numbers satisfying 
\eqref{vinequality} then the function
defined by
\[
h(x) = \sum_{n \in \mathbb{Z}} h_n e^{2\pi i nx}, \qquad x \in \mathbb{R} / \mathbb{Z}
\]
is $C^\infty$. Therefore, we have established that if $\alpha$ is Diophantine then there is some $v:\mathbb{R} / \mathbb{Z} 
\to \mathbb{R}$ that is $C^\infty$ and that satisfies \eqref{uvformula}.

On the other hand,  for $\alpha=\sum_{n=1}^\infty 10^{-n!}$, Ghys constructs a $C^\infty$ function $u:\mathbb{R} / \mathbb{Z}
\to \mathbb{R}$ such that there is no continuous function $v:\mathbb{R} / \mathbb{Z} \to \mathbb{R}$ satisfying
$u(x)=v(x+\alpha)-v(x)$ for all $x \in \mathbb{R} / \mathbb{Z}$. 



\end{document}