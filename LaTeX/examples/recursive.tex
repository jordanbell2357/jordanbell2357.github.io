% !TEX TS-program = pdflatex
% !TEX encoding = UTF-8 Unicode

% This is a simple template for a LaTeX document using the "article" class.
% See "book", "report", "letter" for other types of document.

\documentclass[11pt]{article} % use larger type; default would be 10pt

\usepackage[utf8]{inputenc} % set input encoding (not needed with XeLaTeX)

%%% Examples of Article customizations
% These packages are optional, depending whether you want the features they provide.
% See the LaTeX Companion or other references for full information.

%%% PAGE DIMENSIONS
\usepackage{geometry} % to change the page dimensions
\geometry{a4paper} % or letterpaper (US) or a5paper or....
% \geometry{margin=2in} % for example, change the margins to 2 inches all round
% \geometry{landscape} % set up the page for landscape
%   read geometry.pdf for detailed page layout information

\usepackage{graphicx} % support the \includegraphics command and options

% \usepackage[parfill]{parskip} % Activate to begin paragraphs with an empty line rather than an indent

%%% PACKAGES
\usepackage{booktabs} % for much better looking tables
\usepackage{array} % for better arrays (eg matrices) in maths
\usepackage{paralist} % very flexible & customisable lists (eg. enumerate/itemize, etc.)
\usepackage{verbatim} % adds environment for commenting out blocks of text & for better verbatim
\usepackage{subfig} % make it possible to include more than one captioned figure/table in a single float
% These packages are all incorporated in the memoir class to one degree or another...

%%% HEADERS & FOOTERS
\usepackage{fancyhdr} % This should be set AFTER setting up the page geometry
\pagestyle{fancy} % options: empty , plain , fancy
\renewcommand{\headrulewidth}{0pt} % customise the layout...
\lhead{}\chead{}\rhead{}
\lfoot{}\cfoot{\thepage}\rfoot{}

%%% SECTION TITLE APPEARANCE
\usepackage{sectsty}
\allsectionsfont{\sffamily\mdseries\upshape} % (See the fntguide.pdf for font help)
% (This matches ConTeXt defaults)

%%% ToC (table of contents) APPEARANCE
\usepackage[nottoc,notlof,notlot]{tocbibind} % Put the bibliography in the ToC
\usepackage[titles,subfigure]{tocloft} % Alter the style of the Table of Contents
\renewcommand{\cftsecfont}{\rmfamily\mdseries\upshape}
\renewcommand{\cftsecpagefont}{\rmfamily\mdseries\upshape} % No bold!

\usepackage{amsmath}
\usepackage{amssymb}
\usepackage{amsthm}
\usepackage{mathrsfs}

%%% END Article customizations

%%% The "real" document content comes below...

\newtheorem{theorem}{Theorem}


\title{Induction}
\author{Jordan Bell}
%\date{} % Activate to display a given date or no date (if empty),
         % otherwise the current date is printed 

\begin{document}
\maketitle

Let $\mathbb{N} = \{0,1,2,\ldots\}$

\begin{theorem}
For all $n \in \mathbb{N}$, $6^n-1$ is a multiple of 5.
\end{theorem}
\begin{proof}
\textbf{Base case} For $n=0$, $6^n-1=6^0-1=0$. $0$ is a multiple of 5: $0 = 0\cdot 5$. The claim is true for $n=0$.

\textbf{Inductive step} Assume the claim is true for some $n \geq 0$. That is, suppose that $6^n-1$ is a multiple of 5. Being a multiple of 5 means that
there is some $a \in \mathbb{N}$ such that $6^n-1=5a$.

\begin{align*}
6^{n+1}-1&=6(6^n)-1\\
&=6(6^n-1+1)-1\\
&=6(6^n-1) + 6 - 1\\
&=6(6^n-1) + 5\\
&=6(5a) + 5\\
&=(6a)5+5\\
&=(6a+1)5
\end{align*}
$6a+1 \in \mathbb{N}$, so the above shows that $6^{n+1}-1$ is a multiple of 5, completing the inductive step.

\textbf{Conclusion} By induction, for all $n \in \mathbb{N}$ it is true that $6^n-1$ is a multiple of 5.
\end{proof}

\newpage

\begin{theorem}
For $n \geq 5$, $2n+1 < 2^n$.
\end{theorem}
\begin{proof}
\textbf{Base case} For $n=5$, $2n+1 = 11$ and $2^n=32$, and it is true that $11<32$. The claim is true when $n=5$.

\textbf{Inductive step} Assume the claim is true for some $n \geq 5$. That is, suppose $2n+1 < 2^n$.

$$2(n+1) + 1 = (2n+1)+2< 2^n + 2$$.

$2^n+2 < 2^n+2^n$ for $n \geq 1$, and here $n \geq 5$ so this is true.

Then we have

$2(n+1)+1 < 2^n + 2^n$, and $2^n+2^n = 2(2^n) = 2^{n+1}$, so

$$2(n+1)+1 < 2^{n+1}$$

This shows that the claim is true for $n+1$, completing the inductive step.

\textbf{Conclusion}  By induction, for all $n \geq 5$ it is true that $2n+1 < 2^n$.
\end{proof}

\newpage

\begin{theorem}
For $n \geq 5$, $n^2 < 2^n$.
\end{theorem}
\begin{proof}
\textbf{Base case} Let $n=5$. $n^2 = 25$ and $2^n=32$, and it is true that $25<32$. The claim is true when $n=5$.

\textbf{Inductive step} Assume the claim is true for some $n \geq 5$. That is, suppose that $n^2<2^n$

\begin{align*}
(n+1)^2&=n^2+2n+1 \quad \text{expanding}\\
&< 2^n + 2n+1 \quad \text{because } n^2<2^n
&< 2^n + 2n + 1
\end{align*}

We proved in the previous theorem that for $n \geq 5$, $2n+1 < 2^n$. Therefore
,
$$2^n+2n+1  = 2^n + (2n+1) <2^n + 2^n = 2(2^n)=2^{n+1}$$ 

This show that the claim is true for $n+1$, completing the inductive step.

\textbf{Conclusion} By induction, for $n \geq 5$ it is true that $n^2 < 2^n$.
\end{proof}

\begin{theorem}
For $n \geq 1$, $n^3-n+3$ is a multiple of 3
\end{theorem}
\begin{proof}

\end{proof}

\begin{theorem}
For $n \in \mathbb{N}$, $2^{2n}-1$ is a multiple of 3
\end{theorem}
\begin{proof}
Let $n=0$. $2^{2n}-1 = 2^0 - 1 = 1-1=0$, and $0 = 0 \cdot 3$, so indeed $2^{2n}-1$ is a multiple of 3. Thus the claim is true when $n=0$.

Assume the claim is true for some $n \geq 0$. That is, $2^{2n}-1$ is a multiple of 3; say, $2^{2n}-1 = 3a$ for $a \in \mathbb{N}$. Then
rearranging, $2^{2n} = 3a+1$.

\begin{align*}
2^{2(n+1)} - 1&=2^{2n+2} - 1\\
&=(2^{2n}) (2^2) - 1\\
&=(3a+1)(4) - 1\\
&=12a+4-1\\
&=12a+3\\
&=3(4a+1)
\end{align*}
Since $4a +1 \in \mathbb{N}$, this shows that $2^{2(n+1)}-1$ is a multiple of 3. This completes the inductive step.

Therefore for all $n \in \mathbb{N}$, it is true that $2^{2n}-1$ is a multiple of 3.
\end{proof}
  

\end{document}
