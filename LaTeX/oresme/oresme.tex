\documentclass{article}
\usepackage{amsmath,amssymb,mathrsfs,amsthm}
\usepackage[T1]{fontenc}  
\usepackage[utf8]{inputenc}
\usepackage{textalpha}
\newtheorem{theorem}{Theorem}
\newtheorem{lemma}[theorem]{Lemma}
\newtheorem{proposition}[theorem]{Proposition}
\newtheorem{corollary}[theorem]{Corollary}
\theoremstyle{definition}
\newtheorem{definition}[theorem]{Definition}
\newtheorem{example}[theorem]{Example}
\begin{document}
\title{Nicole Oresme}
\author{Jordan Bell\\
jordan.bell@gmail.com}
\date{\today}

\maketitle

Molland 116--117:

\begin{quote}
Thus, because Oresme's theory is
mathematical, it also has considerable coherence; and the third and final
part of the treatise is devoted to exploring in more detail some of the in-
ternal features of the scheme. In it, as already mentioned, Oresme gives
a very simple proof of the ``Merton theorem of uniform acceleration'',
and his explicit representation of qualitative configurations by geometric
figures allows him to form a much clearer conception of what we may call
``quantity of heat'' than did, say, Richard Swineshead. Qualities (in the
case of a linear subject) are to be measured by the area of the representing
figure, and Oresme devotes much space to equating different qualities.
He is particularly at pains to show that a quality on an infinite subject or
a quality that is at one point infinitely intense may nevertheless be finite.
Such proofs involve him in processes analogous to those used in summing
infinite series, and they are often interpreted as such by modern writers.
\end{quote}

Mathematik im Abendland: Von den römischen Feldmessern bis zu Descartes
By Helmuth Gericke

Geschichte der Zeta-Funktion von Oresme bis Poisson

Die Vorläufer Galileis im 14. Jahrhundert, p. 128

An der Grenze von Scholastik und Naturwissenschaft

Nicolaus Oresmes Kommentar zur Physik des Aristoteles

Philoponus: On Aristotle On Coming to be 1.6-2.4
By C.J.F. William


Courtenay on Oresme's early career \cite{courtenay}. Clagett biography of Oresme \cite{dictionary}.
 
Grant \cite[pp.~106--107]{grant2001} gives the following scheme for medival {\em questiones}:
\begin{quote}
\begin{enumerate}
\item The statement of the question;
\item principal arguments ({\em rationes principales}), usually representing alternatives opposed to the author's
position;
\item opposite opinion ({\em oppositum}, or {\em sed contra}), a version of which the author will defend. In support
of this opinion, the author often cites a major authority, often Aristotle himself; or cites from a commentary on a work
of Aristotle; or invokes a theological authority in a theological treatise, such as a {\em Commentary on the Sentences of Peter Lombard};
\item qualifications, or doubts, about the question, or about some of its terms [optional];
\item body of arguments (author's opinions by way of a sequence of conclusions);
\item brief response to refute each principal argument.
\end{enumerate}
\end{quote}


 Kenny and Pinborg \cite{kenny}:
 \begin{quote}
 From about 1260 we find commentaries consisting only of a series of {\em quaestiones}, each of which
 has the basic form of a disputation. Presumably this reflects the development of a new kind of lecture.
 We do not know to what extent, if any, such {\em quaestiones} were staged as real disputations. Certainly
 from the fourteenth century onwards we have testimony that they were only read
 aloud by the master. (pp.~20--21)
 
 Perhaps the {\em disputatio} simply grew out of the other and older vehicle of professorial instruction:
 the {\em lectio}, or lecture. In the course of expounding a text a commentator, from time to time, is bound
 to encounter difficult passages which set special problems and need extended discussion. When we are dealing
 with a sacred or authoritative text, the difficult passages will have given rise to conflicting interpretations
 by different commentators, and the expositor's duty will be to set out and resolve the disagreements of previous
 authorities. Thus the {\em quaestio} arises naturally in the course of the {\em lectio}, and the disputation and
 the lecture are the institutionalised counterparts of these two facets of a method of study oriented
 to the interpretation of texts and the preservation of tradition. (p. 25)
 \end{quote}
 
 Kenny and Pinborg \cite[pp.~29--30]{kenny} write about {\em quaestiones}:
 \begin{quote}
 Most medieval philosophical literature reflects teaching practice and its form; even writings that were never delivered as lectures
 or held as disputations assume the traditional forms.
 
 See also Lawn \cite{lawn} on {\em quaestiones}.
 
 Accordingly, a large proportion of medieval philosophical literature is in the form of commentaries....

  Besides the literal commentaries there are commentaries in the form of a series of questions. Originally
 such questions formed only the latter part of lectures, but apparently they gradually became independent from the traditional
 lecture-form. From the latter half of the thirteenth century we find such commentaries, often consisting
 only of questions; sometimes short paraphrases of passages of the {\em littera}, otherwise neglected, are still
 given; but often only the opening words are left from the old structure.
 \end{quote}
 
 They describe the scheme of {\em quaestiones} \cite[pp.~30--32]{kenny}:
\begin{quote}
Such questions retain the simplest possible structure of a disputation. First a problem is stated in the {\em titulus quaestionis}
which is always formed as a question introduced by `{\em utrum}'....

After the {\em titulus quaestionis} follows a short series of principal arguments for one of the two possible
answers to the problem stated, frequently introduced by a formula such as `{\em et arguitur} ({\em videtur}) {\em quod sic/non}'.
These arguments normally defend the position eventually refuted. The normal number of arguments is two or three. Then follow
arguments on the opposite side of the issue. They are often fewer in number (often only one), and are frequently nothing
but references to authority. This is justifiable according to medieval tradition, even if an argument from authority was held to be
the weakest form of argument. Real arguments were normally given in the solution of the question, where the author
adopts the position he himself means to defend.

The solution (or {\em corpus quaestionis}) is introduced by phrases such as `{\em ad hoc dicendum/dico}' and states the conclusions of the
author, accompanied by some arguments and distinctions necessary to carry through the solution. These arguments are normally
more carefully organised and articulated, but still may take as their major premisses propositions which have not been or are not proved
`demonstratively' but are only regarded as generally acceptable. Frequently several previous opinions on the subject are
summarised and refuted, before the author states his own opinion....

The last part of a question contains the refutations of the arguments leading to the solution opposite to
the one advocated by the author. They often contain some distinctions which were thought not to be necessary to
the general solution of the problem but of importance only to solving one of the counter-arguments.
 \end{quote}


Murdoch \cite[pp.~567--568]{murdoch1982}:

\begin{quote}
Almost all the scholastics followed Aristotle's distinction of permissible from non-permissible infinites, formulating
a variety of alternative ways of expressing this distinction. The most popular of these alternative expressions
was the claim that the rejected actual infinite was a quantity so great that it could not be greater ({\em tantum quod non maius}),
while the permissible potential infinite was a quantity that was not so great but that it could be greater ({\em non tantum
quin maius}). The scholastics themselves often pointed out that the latter was really only an indefinite finite, as was made explicit
in any number of `expositions' of propositions involving this type of infinite.  
\end{quote}




Clagett, {\em Archimedes in the Middle Ages}, volume III, parts I and II

Johannes de Tinemue's redaction of Euclid's Elements, p.~35



Oresme's {\em Questiones super geometriam Euclidis} \cite{busard2010} (the first edition was reviewed in detail by Murdoch \cite[p.~69]{murdoch1964}) are 21 questions, and is dated by Grant to around 1350. 





The first two Questions are translated by Grant \cite[pp.~131--135]{grant1974}. Question 1:
``Concerning the book of Euclid, we inquire first about a certain statement by Campanus asserting that a magnitude decreases into infinity. First we inquire {\em whether a magnitude
decreases into infinity according to  proportional parts}.'' Oresme refers to the edition of Euclid's {\em Elements} by Campanus of Novara.

In his edition of the {\em Elements}, in Book I, after the common notions ({\em communi animi conceptiones})
and before the propositions, Campanus includes the following \cite[pp.~58--59]{campanus}:

\begin{quote}
Sciendum autem quod preter has communes scientias multas alias que numero sunt incomprehensibiles pretermisit Euclides
quarum hec est una: Si due quantitates equales ad quamlibet tertiam eiusdem generis comparentur, simul erunt ambe illa tertia aut eque maiores
aut eque minores aut simul equales. Item alia. Quanta est aliqua quantitas ad quamlibet aliam eiusdem generis,
tantam esse quamlibet tertiam ad aliquam quartam eiusdem generis. In quantitatibus continuis: hoc universaliter verum
est sive antecedentes maiores fuerint consequentibus suis sive minores. Magnitudo enim decrescit in infinitum. In numeris autem non sic,
sed si fuerit primus submultiplex secundi, erit quilibet tertius eque submultiplex alicuius quarti quoniam numerus crescit in infinitum, sicut magnitudo
in infinitum minuitur.
\end{quote}


Given quantities $a,b,c$, there is a quantity $d$ so that $a/b=c/d$, and if $b=na$ for some natural number $n>1$, then 
$d=nc$, and the fact that there are quantities such that $a/b$ is unbounded below is analogous to the fact the natural numbers are not bounded
above. 

Fix a magnitude $a$ and a proportion $r$. We remove $ra$ from the  magnitude $a$, leaving $a-ra=(1-r)a$. We remove $r(1-r)a$ from what is left,
leaving $(1-r)a-r(1-r)a=(1-r)^2a$. We remove $r(1-r)^2a$ from what is left, leaving $(1-r)^2a - r(1-r)^2a=(1-r)^3a$, and so on. What is left is unbounded below,
namely, ``decreases into infinity''. The total of all the magnitudes removed is
\[
ra+r(1-r)a + r(1-r)^2a + \cdots,
\]
and this is equal to $a$. Indeed,
\[
\sum_{n=0}^\infty r(1-r)^n a = ra \sum_{n=0}^\infty (1-r)^n = ra \cdot \frac{1}{1-(1-r)} = a.
\]


Question 2: ``Next we inquire {\em whether an addition to any magnitude could be made into infinity by proportional parts}.''


Oresme \cite[p.~135]{grant1974}:
\begin{quote}
The third proposition is this: It is possible that an addition could be made, though not proportionally, to any quantity by ratios of lesser
inequality, and yet the whole would become infinite; but if it were done proportionally, it would be finite, as was said. For example, let a one-foot
quantity be assumed to which one-half of a foot is added during the first proportional part of an hour, then one-third of a foot in another [or next proportional
part of an hour], then one-fourth [of a foot], then one-fifth, and so on into infinity following the series of [natural] numbers, I say that the whole would become
infinite, which is proved as follows: There exist infinite parts of which any one will be greater than one-half foot and [therefore] the whole will be infinite. The antecedent is obvious, since
$1/4$ and $1/3$ are greater than $1/2$; similarly [the sum of the parts] from $1/5$ to $1/8$ [is greater than $1/2$] and [also the sum of the parts]
from $1/9$ to $1/16$, and so on into infinity.
\end{quote}



Boyer \cite[pp.~80--89]{boyer} discusses infinite series in the work of Oresme.

Clagett \cite{clagett1968}

Pedersen \cite[p.~199]{pedersen}

Burton \cite{burton}

Kirschner \cite{kirschner}

Maieru \cite{maieru}

Mazet \cite{mazet}

Rommevaux \cite{rommevaux}

Wallace \cite[pp.~65--116, Chapter 3]{wallace} on University of Paris

Grant \cite{proportionibus}

Clagett \cite{singleton}

Murdoch \cite{murdoch1975} and \cite{elkana}

Gribaudo \cite{gribaudo}

Juschkewitsch \cite[pp.~405--413]{juschkewitsch}

\bibliographystyle{plain}
\bibliography{oresme}

\end{document}
