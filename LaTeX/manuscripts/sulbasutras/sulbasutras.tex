\documentclass{article}
\usepackage{amsmath,amssymb,graphicx,subfig,mathrsfs,amsthm,enumitem,xfrac,flexisym}
%\usepackage[larger]{skt}
\usepackage[polutonikogreek,english]{babel}
\newcommand{\Gk}[1]{\selectlanguage{polutonikogreek}#1\selectlanguage{english}}%\usepackage{caption}
%\usepackage{tikz-cd}
%\usepackage{hyperref}
\newcommand{\inner}[2]{\left\langle #1, #2 \right\rangle}
\newcommand{\tr}{\ensuremath\mathrm{tr}\,} 
\newcommand{\Span}{\ensuremath\mathrm{span}} 
\def\Re{\ensuremath{\mathrm{Re}}\,}
\def\Im{\ensuremath{\mathrm{Im}}\,}
\newcommand{\id}{\ensuremath\mathrm{id}} 
\newcommand{\gcm}{\ensuremath\mathrm{gcm}} 
\newcommand{\diam}{\ensuremath\mathrm{diam}} 
\newcommand{\sgn}{\ensuremath\mathrm{sgn}\,} 
\newcommand{\lcm}{\ensuremath\mathrm{lcm}} 
\newcommand{\supp}{\ensuremath\mathrm{supp}\,}
\newcommand{\dom}{\ensuremath\mathrm{dom}\,}
\newcommand{\norm}[1]{\left\Vert #1 \right\Vert}
\newtheorem{theorem}{Theorem}
\newtheorem{lemma}[theorem]{Lemma}
\newtheorem{proposition}[theorem]{Proposition}
\newtheorem{corollary}[theorem]{Corollary}
\begin{document}
\title{Turning a rectangle into a square in the Sulbasutras} 
\author{Jordan Bell\\ \texttt{jordan.bell@gmail.com}\\Department of Mathematics, University of Toronto}
\date{\today}

\maketitle

{\em Vedic Index of Names and Subjects}, Macdonell and Keith:

rajju, rope

{\em \'sa\.{n}ku}: ``\'Sa\.{n}ku in the Rigveda and later denotes a `wooden peg.' ''




Katyayana Sulbasutra \cite{khadilkar}

9, 1874-75: 292-298;
10, 1875-76: 17-22,
44-50,
72-74,
139-146,
166-170,
186-194,
209-218;
NS 1, 1876-77: 316-22,
556-578,
626-642,
692-706,
761-770.


Baudh\=ayana \'Sulbas\=utra 1.4 \cite[p.~77]{senbag}:

\begin{quote}
Having desired (to construct) a square, one is to take a cord of length equal 
to the (side of the) given square, make ties at both ends and mark it at its
middle. The (east-west) line (equal to the cord) is drawn and a pole is fixed
at its middle. The two ties (of the cord) are fixed in it (pole) and a circle is
drawn with the mark (in the middle of the cord). Two poles are fixed at both
ends of the diameter (east-west line). With one tie fastened to the eastern
(pole), a circle is drawn with the other. A similar (circle) about the western
(pole). The second diameter is obtained from the points of intersection of
these two (circles); two poles are fixed at two ends of the diameter (thus
obtained). With two ties fastened to the eastern (pole) a circle is drawn with
the mark. The same (is to be done) with respect to the southern, the western
and the northern (pole). The end points of intersection of these (four circles)
produce the (required) square.
\end{quote}

Baudh\=ayana \'Sulbas\=utra 1.5 \cite[p.~77]{senbag}:

\begin{quote}
Now another (method). Ties are made at both ends of a cord twice the
measure and a mark is given at the middle. This (halving of the cord) is for
the east-west line (that is, the side of the required square). In the other half
(cord) at a point shorter by one-fourth, a mark is given; this is the {\em nya\~ncana}
(mark). (Then) a mark is given at the middle (of the same half cord) for
purposes (of fixing) the corner (of the square). With the two ties fastened to
the two ends of the east-west line ({\em p\d{r}\d{s}\d{t}hy\=a}), the cord is to be stretched towards
the south by the {\em nya\~ncana} (mark); the middle mark (of the half cord) determines
the western and the eastern corners (of the square).
\end{quote}

Katyayana \'Sulbas\=utra 1.1 \cite[p.~95]{thibaut1882}, Thibaut:

\begin{quote}
We will explain the employment of the cords (by means of which the sacrificial areas are measured out).
\end{quote}

Katyayana \'Sulbas\=utra 1.2 \cite[p.~95]{thibaut1882}, Thibaut:

\begin{quote}
Having, on a level spot, planted a pole and having described (round it) a circle by means of a cord fastened to the pole (or else ``measured by the pole'')
one fixes a pole on each of the two points where the end of the pole's shadow touches the two halves of the circle; this (the line connecting these two points) is the east-line.
\end{quote}

Katyayana \'Sulbas\=utra 1.2 \cite[p.~120]{senbag}, Sen and Bag:

\begin{quote}
Having put a pole on a level ground and described a circle round it by means
of a cord (fastened to the pole), a pole is fixed on each of the two points where
the end of the pole's shadow touches (the two halves of the circle). This (line
joining the two points) is the east-west line ({\em pr\=ac\=\i}). Then after doubling (a
given) cord, two loops (made at its two ends) are fixed at the two poles (of
the {\em pr\=ac\=\i}), and (the cord is stretched towards south by its middle point where)
a pole is fixed; the same is repeated to the north. This (line joining the two 
poles) is the north-south line ({\em ud{\=\i}c{\=\i}}). 
\end{quote}

Katyayana \'Sulbas\=utra 1.3 \cite[p.~99]{thibaut1882}, Thibaut:

\begin{quote}
Having by a cord doubled their interval (i.e. having taken a cord whose length is double the interval of the two poles)
and having made slings at its two ends and fastened them to the two poles one stretches the cord towards the south
and plants a pole at its middle, the same is done on the north side, this is the north-line.
\end{quote}

Katyayana \'Sulbas\=utra 1.3 \cite[p.~120]{senbag}, Sen and Bag:

\begin{quote}
Two loops are fixed at the two ends of a cord. Marks are (to be given) at the
{\em \'sro\d{n}{\=\i}s}, the {\em a\d{m}sas}, the {\em nira\~nchana} and the
{\em sam\=asabha\.{n}gas}. A pole is fixed at each end of the east-west line (of desired length); likewise
(a pole is fixed at each of) the two {\em \'sro\d{n}{\=\i}s} (west corners) and the two
{\em a\d{m}sas} (east corners). Having fixed the loops at the two poles (on the east-west line),
the cord is to be stretched by the {\em nira\~nchana} mark towards the south-east corner. The same
is done towards the north-east corner. After interchanging (the loops of the 
cord on the poles), the same is repeated. This is the method (of construction
of squares and rectangles) in all cases.
\end{quote}

Katyayana \'Sulbas\=utra 1.3 \cite[p.~99]{thibaut1882}, Thibaut:

\begin{quote}

\end{quote}

Katyayana \'Sulbas\=utra 1.3 \cite[p.~99]{thibaut1882}, Thibaut:

\begin{quote}

\end{quote}

Katyayana \'Sulbas\=utra 1.3 \cite[p.~99]{thibaut1882}, Thibaut:

\begin{quote}

\end{quote}

\nocite{*}

\bibliographystyle{amsplain}
\bibliography{sulbasutras}

\end{document}