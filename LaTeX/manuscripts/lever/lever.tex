% trylinearb.tex    Linear B font
\documentclass{article}
%%\documentclass[12pt]{article}
\usepackage{amsmath}
\usepackage{linearb}

\title{The balance}
\author{Jordan Bell\\ \texttt{jordan.bell@gmail.com}\\Department of Mathematics, University of Toronto}
\date{\today}

\renewcommand{\baselinestretch}{1.2}
\begin{document}
\maketitle

Aristotle steelyard, Mechanical problems, 20.853b--854a, 1849b--850a, 848a--850b10

Heron OO II

Vitruvius 10,3,4

Isidore, Orig. 16,25,6

Hero, {\em Mechanica} 2,7

Pappus {\em Collectio} 8,2; 1068:20

Plutarch, {\em Vita Marcelli}, xiv.7


Singer, vol.~1, pp.~779--84.

Heron, {\em Dioptra} 308.19--20, simple machines, 310.26, 312.20. Heronis iii


Pappus, {\em Mathematical Collection} 8.52 \cite[p.~49]{humphrey}:

\begin{quote}
There are five machines by the use of which a given weight is moved by a
given force, and we will undertake to give the forms, the applications, and the
names of these machines. Now both Hero and Philo have shown that these
machines, though they are considerably different from one another in their
external form, are reducible to a single principle. The names are as follows:
wheel and axle, lever, system of pulleys, wedge, and, finally, the so-called
endless screw.
\end{quote}

Philon, {\em Belopoeica} 59.11--15 \cite[p.~123]{marsden}:

\begin{quote}
Larger circles have a bigger turning effect than smaller ones fixed about
the same centre, as we  demonstrated in our `Principles of Leverage'; 
similarly, men move loads with levers more easily when they place the
fulcrum as close as possible to the load (the fulcrum, of course, performs
the function of the centre; when brought close to the load it decreases the
circle and easy movement is the result).
\end{quote}


Pappus, {\em Mathematical Collection} 8.1--2 \cite[pp.~47--48]{humphrey}:

\begin{quote}
The science of mechanics, my son Hermodorus, is useful for many
applications in daily life. In addition, it is found worthy of great favour among
the philosophers and is eagerly pursued by all mathematicians. For it has very
nearly primary relevance to the science of nature, which concerns the material of
the elements making up the universe. Being the general, theoretical consideration
of inertia and mass and of movement through space, it not only examines the
causes of objects moved according to their nature, but also causes objects to
leave their own position against opposing forces, contrary to their nature,
devising it through the theories suggested to it by matter itself. The
mechanicians ({\em mechanikoi}) of Hero's school state that part of the science of
mechanics is theoretical, part is practical. The theoretical part consists of
geometry, arithmetic, astronomy, and physics, while the practical part consists
of metal-working, construction, carpentry, painting, and the manual practice of
these arts. They say, then, that he will be the best inventor and master builder of
mechanical contrivances who from his boyhood has been involved with the fields
of knowledge mentioned above and has received training in the above-mentioned
crafts, and having a nature inclined towards them. But since it is not possible for
the same person to grasp so many fields of knowledge and at the same time learn
the above-mentioned crafts, they advise the individual who desires to undertake
projects involving mechanics to use the particular crafts of which he himself has
a command for the purposes suitable to each.

Of all the mechanical arts the most indispensable with respect to human
needs are the following (the mechanical considered before the architectural): the
art of the ``conjurers'' ({\em manganarioi}), termed mechanicians ({\em mechanikoi}) by the
ancients. By means of machines ({\em mechanai}) they lift great weights to a height,
moving them with little force, contrary to nature. Also, the art of those who
construct the siege machines necessary for war ({\em organopoioi}), they too are termed
mechanicians. They design catapults that fire stone or iron missiles and other 
objects of this type a great distance. Additional to these is the art of those
properly called machine-builders ({\em mechanopoioi}). Water is quite easily raised
from a great depth by means of the water-lifting machines which they devise.
The ancients also termed the gadget-designers ({\em thaumasiourgoi}) mechanicians.
Some of them practice their art on the principles of air pressure, like Hero in his
{\em Pneumatika}, while others utilize sinews and cords to imitate the movements of
living creatures, like Hero in his {\em Automata} and {\em Balances}; still others depend
upon bodies floating on water, like Archimedes in his {\em On Floating Bodies}, or on
water clocks, which seem to be connected with the theory of sundials, like Hero
in his {\em On Water Clocks}. Finally, they also termed  mechanicians the sphere-makers,
who construct models of the heavens based on the even and circular
motion of water.
\end{quote}


{\em Timaeus} Cornford 62C, p.~262:

\begin{quote}
`Heavy' and `light' may be most clearly explained by
examining them together with the expressions `above' and `below'.
\end{quote}

Cornford, 63A--63E, pp.~263--264:

\begin{quote}
As to the source of these terms and the things to which
they really apply and which have occasioned our habit of
using the words to describe a division of the universe as a
whole, we may arrive at an agreement, if we make the
following supposition. Imagine a man in that region of the
universe which is specially allotted to fire, taking his stand
on the main mass towards which fire moves, and suppose it
possible for him to detach portions of fire and weigh them
in the scales of a balance. When he lifts the beam and
forcibly drags the fire into the alien air, clearly he will get
the smaller portion to yield to force more readily than the
greater; for when two masses at once are raised aloft by
the same power, the lesser must follow the constraint more
readily than the greater, which will make more resistance;
and so the large mass will be said to be `heavy' and to tend
`downwards', the small to be `light' and to tend `upwards'.
Now this is just what we ought to detect ourselves doing
here in our own region. Standing on the Earth, when we
are trying to distinguish between earthy substances or
sometimes pure earth, we are dragging the two things
into the alien air by violence and against their nature; both
cling to their own kind, but the smaller yields more readily
to our constraint than the larger and follows it more quickly
into the alien element. Accordingly we have come to call
it `light' and the region into which we force it `above' ;
when the thing behaves in the opposite way, we speak of
`heavy' and `below'. Consequently, the relation of these
things to one another must vary, because the main masses
of the kinds occupy regions opposite to one another: what
is `light' or `heavy' or `above' or `below' in one region
will all be found to become, or be, contrary to what is `light'
or `heavy' or `above' or `below' in the opposite region,
or to be inclined at an angle, with every possible difference
of direction. The one thing to be observed in all cases,
however, is that it is the travelling of each kind towards its
kindred that makes the moving thing `heavy' and the
region to which it moves `below', while the contrary names
are given to their opposites. So much for the explanation
of these affections.
\end{quote}


Early Physics and Astronomy: A Historical Introduction
By Olaf Pedersen

{\em Mechanica} \cite{aristotleVI}


Pappus, {\em Mathematical Collection} VIII.1--5, Cohen and Drabkin, pp.~183--186:

\begin{quote}
The science of mechanics, my dear Hermodorus, is not merely useful
for many important practical undertakings, but is justly esteemed by
philosophers and is diligently pursued by all who are interested in 
mathematics, since it is fundamentally concerned with the doctrine of nature
with special reference to the material composition of the elements in the
cosmos. For it examines bodies at rest, their natural tendency, and their
locomotion in general, not only assigning causes of natural motion, but
devising means of impelling bodies to change their position, contrary to
their natures, in a direction away from their natural places. In this the
science of mechanics uses theorems suggested to it by a consideration of
matter itself.

Now the mechanicians of Hero's school tell us that the science of
mechanics consists of a theoretical and a practical part. The theoretical
part includes geometry, arithmetic, astronomy, and physics, while the
practical part consists of metal-working, architecture, carpentry, painting,
and the manual activities connected with these arts. One who has had
instruction from boyhood in the aforesaid theoretical branches, and has
attained skill in the practical arts mentioned, and possesses a quick 
intelligence, will be, they say, the ablest inventor of mechanical devices and
the most competent master-builder. But since it is not generally possible
for a person to master so many mathematical branches and at the same
time to learn all the aforesaid arts, they advise a person who is desirous
of engaging in mechanical work to make use of those special arts which he
has mastered for the particular ends for which they are useful.

The most important of the mechanical arts from the point of view
of practical utility are the following. (1) The art of the {\em manganarii},
known also, among the ancients, as mechanicians. With their machines
they need only a small force to overcome the natural tendency of large
weights and lift them to a height. (2) The art of the makers of engines
of war, who are also called mechanicians. They design catapults to fling
missiles of stone and iron and the like a considerable distance. (3) The
art of the contrivers of machines, properly so-called. For example, they
build water-lifting machines by which water is more easily raised from a
great depth. (4) The art of those who contrive marvelous devices. They
too are called mechanicians by the ancients. Sometimes they employ air
pressure, as does Hero in his {\em Pneumatica}; sometimes ropes and cables to
simulate the motions of living things, e.g., Hero in his works on {\em Automata}
and {\em Balances}; and sometimes they use objects floating on water, e.g.,
Archimedes in his work {\em On Floating Bodies}, or water clocks, e.g., Hero
in his treatise on that subject, which is evidently connected with the theory
of the sun dial. (5) The art of the sphere makers, who are also considered
mechanicians. They construct a model of the heavens [and operate it]
with the help of the uniform circular motion of water.

Now some say that Archimedes of Syracuse mastered the principles
and the theory of all these branches. For he is the only man down to
our time who brought a versatile genius and understanding to them all,
as Geminus the mathematician tells us in his discussion of the relationship
of the branches of mathematics. But Carpus of Antioch says 
somewhere that Archimedes of Syracuse wrote only one book on a mechanical
subject, that on sphere-construction, but did not consider any of the other
mechanical branches worthy of literary treatment. Now this wonderful
man, a man so richly endowed that his name will be celebrated forever by
all mankind, is extolled by most people for his achievement in mechanics.
But his chief concern was the composition of works dealing with the 
principal matters of geometric and arithmetic theory, even those parts often
held to be least important. Evidently he was so devoted to these branches
that he did not permit himself to add to them anything extraneous.
But Carpus and others have made use of geometry as a basis for various
arts, and properly so. For in aiding numerous arts geometry is in no wise
harmed by the association with them. Since geometry is, so to speak, the
mother of these arts, it is not harmed by aiding in the construction of
engines or in the work of the master-builder, or by association with geodesy,
horology, mechanics, and scene-painting. On the contrary, geometry 
obviously promotes these arts and is justly honored and glorified by them.
Such, then, is the nature of mechanics, which is both a science and
an art, and such are the parts into which it is divided. Now I consider it
well to set forth more concisely, clearly, and rigorously than my 
predecessors have done, the most important theorems proved geometrically by the
old writers on the subject of the motion of heavy bodies, as well as the
theorems which I succeeded in discovering for myself. I cite as examples:

1. If a given weight is drawn by a given force on a horizontal plane, to
find the force by which the weight will be drawn up a plane inclined to the
horizontal at a given angle. This proposition is useful to those 
mechanicians who construct machines for lifting weights, for by adding a force of
men to the force found to be theoretically required they may be confident
that the weight will be drawn up;

2. Given two unequal straight lines to find two mean proportionals
in continued proportion. By this theorem every solid figure may be
augmented or decreased in any given ratio;

3. Given a wheel with a known number of cogs or teeth, to find the
diameter of a second wheel to be engaged with the first and having a given
number of teeth. This proposition is generally useful and in particular
for machine makers in connection with the fitting of cogged wheels.

Each of these propositions will be elucidated in its proper place along
with other propositions useful to the master-builder and the mechanician.
But first let us discuss those things which have to do with the matter of
centers of gravity.

We do not have to discuss at this time what is meant by the heavy
and the light, what is the cause of the upward or downward tendency of
bodies, and in fact what significance attaches to the terms up and down
and by what limits each is bounded. These matters have been treated by
Ptolemy in his {\em Mathematica}. But we should consider just what we mean
by the center of weight of a given body, for that is the fundamental element
m the whole subject of centers of gravity on which depend all the other
parts of mechanical theory. For the other theorems in this field can be
clear, in my opinion, if this fundamental concept is clear Now we define
the center of gravity of any given body as a point within the body such that,
if we imagine the body to be suspended from that point, the body will be
at rest, maintaining its original position without any tendency to turn.
Statics
\end{quote}

Jammer, Concepts of Force, 1962, p.~17:

\begin{quote}
The idea of force, in the prescientific stage, was formed most
probably by the consciousness of our effort, spent in voluntary
actions, as in the immediate experience of moving our limbs, or
by the consciousness of the feeling of a resistance to be overcome
in lifting a heavy object from the ground and carrying it 
from one place to another. 
\end{quote}


\bibliographystyle{amsplain}
\bibliography{lever}

\end{document}