\documentclass{amsart}
\usepackage{amsmath,amssymb,graphicx,subfig,mathrsfs,amsthm,enumitem}
%\usepackage[T1]{fontenc}
\newtheorem{theorem}{Theorem}
\newtheorem{lemma}[theorem]{Lemma}
\newtheorem{proposition}[theorem]{Proposition}
\newtheorem{corollary}[theorem]{Corollary}
\theoremstyle{definition}
\newtheorem{definition}[theorem]{Definition}
\newtheorem{example}[theorem]{Example}
\begin{document}
\title[Swineshead's {\em Liber calculationum}]{An infinite series in Richard Swineshead's {\em Liber calculationum}}
\author{Jordan Bell}
\email{jordan.bell@gmail.com}
\address{Department of Mathematics, University of Toronto, Toronto, Ontario, Canada}
\author{Viktor Bl{\aa}sj{\"o}}
\email{V.N.E.Blasjo@uu.nl}
\address{Mathematisch Instituut, Universiteit Utrecht, Utrecht, The Netherlands}
\date{\today}

\maketitle

\section{Introduction and commentary}

Tract II of Richard Swineshead's {\em Liber calculationum} (c.~1340--1350) contains a discussion of the series
\[ S = 1+ \frac{1}{2}+\frac{2}{4}+\frac{3}{8}+\frac{4}{16}+\frac{5}{32}+\cdots,\]
including a proof that $S=2$. The discussion is interesting in what it reveals about the motivations for considering infinite series in the scholastic tradition, and for how it relates to questions that we would characterise as pertaining to convergence and divergence. In this note we give a translation of the passage in question and analyse what we can learn from it in these regards.

Let us consider first the question of why Swineshead is interested in the series in the first place. It arises in an Aristotelian context as follows. Consider a quality which may vary in intensity, such as heat, light, density, or velocity. The problem arises of how to ``denominate'' a whole if constituent parts vary in intensity. For example, an object may be hotter in one part and cooler in another---what degree of heat should be assigned to the object as a whole? Or we may ask for the velocity of an object during a time interval, when the velocity at subintervals differ. Questions of this kind where prominent in the scholastic tradition.

Swineshead's solution to the denomination problem consists in the use of an average. This is detailed in {\S}{\S}1--3 of the translation (in our labelling of the paragraphs). If a quality has degree 8 in one half and 4 in the other, then the whole is denominated as 6, i.e., the average of the two. This was not an original suggestion. The same proposal had been made by Swineshead's contemporary William Heytesbury, who, like Swineshead, was a fellow of Merton College, Oxford, in his work {\em Regule solvendi sophismata} (c.~1335). (See \cite[p.~83]{wilson}.) In fact, Heytesbury's account includes the use of the weighted average where the division is unequal. Swineshead, by contrast, only discusses the case of equal division into halves in the passage of our interest.

Swineshead then, in {\S}4, goes on to raise an objection to this solution to the denomination problem. Suppose the division is repeated ad infinitum, dividing the whole into pieces of size
\[ \frac{1}{2}+\frac{1}{4}+\frac{1}{8}+\frac{1}{16}+\frac{1}{32}+\cdots, \tag{C} \]
and suppose further that the intensity of the quality is successively increasing by one in each of the pieces. According to the averaging rule, the denomination of the whole would then be $S$, which equals 2. But on the other hand the intensity grows without bounds, so it seem absurd that the denomination should be finite. This therefore suggests that the averaging solution to the denomination problem is not viable since it has absurd consequences.

Swineshead's proof that $S$ is in fact 2 occurs in the elaboration of this objection ({\S}{\S}5--7). For this proof he does not introduce the series $S$ directly as above. Instead he defines it indirectly, as follows. By construction we know that $C=1$ since the series is obtained by dividing the whole interval. Therefore 1 is also the denomination of the whole if the intensity is unity throughout. If we let the intensity double throughout we clearly have
\[ \frac{2}{2}+\frac{2}{4}+\frac{2}{8}+\frac{2}{16}+\frac{2}{32}+\cdots = 2 \tag{B} \]
for the denomination of the whole. We can arrive at $B$ from $C$ by increasing each numerator successively. Consider now another series $A$, which also starts out equal to $C$. Each time we increase one of the numerators of what is to become $B$, say of the $n$th term, we also increase by 1 all the numerators of $A$ from the $(n+1)$th term onwards. It follows by construction that $A=B$ since any tail-end sum of $C$ equals the preceding term (by definition of $C$ through successive halving). It is furthermore clear that 
\[ \begin{split} A = & \frac{1}{2}+\frac{1}{4}+\frac{1}{8}+\frac{1}{16}+\frac{1}{32}+\cdots \\
& \phantom{\frac{1}{2}}+\frac{1}{4}+\frac{1}{8}+\frac{1}{16}+\frac{1}{32}+\cdots \\
& \phantom{\frac{1}{2}+\frac{1}{4}}+\frac{1}{8}+\frac{1}{16}+\frac{1}{32}+\cdots \\
& \phantom{\frac{1}{2}+\frac{1}{4}+\frac{1}{8}}+\frac{1}{16}+\frac{1}{32}+\cdots \\
& \phantom{\frac{1}{2}+\frac{1}{4}+\frac{1}{8}+\frac{1}{16}}+\frac{1}{32}+\cdots = S \end{split}\]
Therefore $S=A=B=2$.

The above argument has been called ``the equivalent of a proof of the convergence of the infinite series [$S$]'' (\cite[p.~77]{boyer}), but as Murdoch and Sylla \cite[p.~193]{dictionary} aptly observe:
\begin{quote}
Such an interpretation is misleading. $\ldots$ Swineshead knows where he is going to end up before he even starts; he has merely redistributed what he already knows to be a given finite increase in the intensity of one subject over another subject, something that is found to be true in most instances of the occurrence of ``convergent infinite series'' in the late Middle Ages.
\end{quote}
Indeed, instead of starting with $S$ and trying to find a way of summing it up, Swineshead starts with a series that clearly has sum 2, and rewrites it in an equivalent form so as to arrive at $S$. In other words, the value of the sum is the starting point of the argument and the series the conclusion, not the other way around.

As noted, the result $S=2$ is part of an objection to Swineshead's averaging rule of denomination. Having thus explicated this objection at length, Swineshead goes on to dismiss it quite briefly ({\S}8). He denies the assumption that since the intensity grows beyond bounds the net denomination should be infinite. In the case of the halves having degree 8 and 4, Swineshead had taken care to argue that each part separately denominates the whole as $8/2$ and $4/2$ respectively. Thus the parts can be analysed separately and their contributions aggregated, rather than being added together first and only then averaged out over the whole. This perspective now serves him well when discussing the $S$ case. For from this point of view it becomes evident that each term contributes less and less to the whole, even though the intensity is increasing beyond bounds. Therefore the assumption that the whole should have infinite denomination is not warranted.

From a modern point of view one is inclined to see Swineshead's achievement as consisting in the summation of the infinite series $S$, and indeed Swineshead is in this respect something of a pioneer in Medieval Europe. However, this may seem at odds with the logic of Swineshead's own text. The result $S=2$ is on Swineshead's own account not presented as a triumph of any sort but rather as the key ingredient of a counterargument to his preferred stance. It is a source of problem rather than pride. As such it must be argued against and defused, as Swineshead indeed does. Isn't this a rather backward and unusual way of introducing one's main achievement?

Since the proof that $S=2$ occurs in this ``negative'' form, one must ask whether it is in fact due to Swineshead at all, as is generally assumed. After all, the rule of denomination that Swineshead defends had already been advanced before him, as we saw. Perhaps, then, the counterargument based on $S=2$ had been raised in the meantime also. Swineshead's own contribution would then be confined to the much more modest refutation of this argument in {\S}8. Certainly the manner in which Swineshead introduces the argument based on $S=2$ in {\S}4 is perfectly compatible with this hypothesis.

This hypothesis appears unlikely, however, since no previous source raising the counterargument based on $S=2$ is known. Furthermore Swineshead is noted for his prowess in calculation, as indeed his nickname (``Calculator'') and the title of his book attest. This suggests that the proof that $S=2$ is indeed due to Swineshead himself.

But if Swineshead really is writing to defend the rule of denomination, as he claims, then why does he seemingly devote so much more effort to explaining the counterargument based on $S=2$ than he does to answering it? One way of looking at it is that perhaps his interest was not so much in the problem of denomination as such as in the mathematics that it engendered. Perhaps, in other words, he is appropriating this scholastic discussion as a vehicle for getting to exercise his documented love of ``calculation.'' This would explain why he is seemingly not concerned that his main mathematical achievement is advanced in the negative, as a counterargument to the principle he is ostensibly concerned with defending.

It is also evident that the $S=2$ objection is just one example of a potentially more general problem. The objection is that the averaging rule of denomination gives the value 2, whereas the true value should perhaps be infinity. Swineshead argues that 2 is in fact correct in this case. Nevertheless one may ask more generally whether the averaging rule of denomination always gives the correct value for any series, i.e., whether there are in fact series that sum to infinity but for which the averaging rule gives a finite value. We know that this is not so according to the modern theory of series, for the averaging rule amounts to the modern definition of the sum of a series, so it will always give the correct value for the series whenever it is convergent and infinity when the sum diverges to infinity.

One may ask to what extent Swineshead was aware of considerations along such lines. In {\S}8, where he argues that the contributions of each term of $S$ become smaller and smaller, he also notes that if the fourth term was $\frac{8}{8}$ instead of $\frac{3}{8}$ then its contribution to the whole would be equal to that of the first term. This strongly hints at the series
\[ 1+ \frac{2}{2}+\frac{4}{4}+\frac{8}{8}+\frac{16}{16}+\frac{32}{32}+\cdots\]
However, there is no need for him to discuss this series in this context since in this case the averaging rule of denomination does not give a finite value, so it cannot be used for the purposes of an objection along the lines of the $S=2$ case. Nevertheless, it shows that questions pertaining to convergence and divergence certainly do arise from within Swineshead's own line of reasoning.

One may speculate that Swineshead's restriction of his denomination rule to halving only, instead of arbitrary division with weighted averaging as his colleague Heytesbury had already considered, is designed to avoid opening the door for many more complicated series. With halving as the only form of division covered, $C$ is essentially the only possible infinite division of an interval possible, and of course also the simplest to treat mathematically. Series corresponding to other divisions of the interval would be vastly more complicated and certainly not treatable with the methods Swineshead used for the $S$ case. Perhaps Swineshead had considered such a possibility and did not feel confident, or did not feel that he could prove, that a weighted average rule of denomination could not lead to paradoxical results in such cases (i.e., lead to a finite value of a series which should properly be called infinite or divergent).







\section{Text}
The following is a complete translation of the passage from tract II, ``De difformibus'', of Richard Swineshead's {\em Liber calculationum} pertaining to the series $S$. Our translation expands on and incorporates, in slightly modified form, the partial translation previously given by Clagett in \cite[pp.~59--61]{clagett1968}. Our translation is based on 6va--7ra of the 1520 Venice edition \cite{venice1520}. Clagett gives a recension of the Latin text based on the 1498 Pavia edition and manuscripts. Clagett's translation begins:

\begin{quote}
[{\S}1] The first opinion regarding a difform quality in which each half is uniform can, however, be sustained, namely that it corresponds to the middle degree between these qualities, and the argument is based on this: A quality extended through the whole subject is twice as productive for the denomination of the whole subject as all the quality extended through one half, which is argued as follows. Let $a$ denote something that has a heat of 4 through the whole: then the whole is hot through the whole as 4; but one half of this does as much for the denomination of the whole subject as the other half. Therefore that whole quality denominates the whole by twice as much as one of its parts or halves denominates the whole, which was to be proved. From this it follows that the denomination of the whole subject by a quality extended only through half of it is half of that quality; it denominates the whole by half as much as that half through which it extends; and it denominates that half by its highest degree. Consequently it denominates the whole by that quality to a degree that is half of that quality. And if it extended only through one quarter of the whole, it would denominate the whole to a degree of one fourth of that quality. And so correspondingly, as it extends proportionally through a smaller part than the whole, thus it denominates the whole to a lesser degree than the part through which it extends.
\end{quote}

The 1520 Venice edition continues:\footnote{{\it Istis concessis faciliter probatur positio. Sit enim tale uniformiter difforme seu difforme cuius utraque medietas est uniformis una ut .iiii. [sic but should be .viii.] et alia ut .iiii., gratia argumenti. Tunc illa qualitas ut .viii. extenditur per medietatem totius per predicta. Ergo solum denominat totum ut .iiii. Per idem illa qualitas ut .iiii. per aliam medietatem extensa solum facit ut duo ad totius denominationem. Igitur iste due qualitates totum precise denominabunt ut .vi. qui est gradus medius inter illas medietates. Sequitur igitur positio sic in speciali.

Arguitur tamen ad hoc generaliter sic: signato tali difformi. Tunc utraque qualitas in istis medietatibus est dupla ad gradum per quem totum denominatur ab illa qualitate. Et si aliquid fieret ex istis qualitatibus simul existensis, illa qualitas esset dupla ad illas duas denominationes simul aggregatas: que denominationes denominationem totius constituunt: et per consequens cum iste qualitates sint ineque intense patet quod denominatio totius erit media equaliter inter istas duas qualitates, eo quod omne compositum ex duobus inequalibus est precise duplum ad medium inter illa ut est argutum in secunda suppositione conclusionis xxxviii de motu locali et postea de inductione gradus summi. Arguitur scilicet quod omne compositum ex duobus inequalibus erit plus quam duplum ad omne minus medio et minus quam duplum ad omnem maius medio. Patet ergo ex predictis quod denominatio totius est in gradu medio inter illas qualitates. Quod fuit probandum.} \cite[6va]{venice1520}}

\begin{quote}
[{\S}2] Once this is granted, the assertion is easily proved. Let the uniformly difform or difform [quality] be such that each half of it is uniform: for the sake of the argument one as [8],
the other as 4. Then that quality extends as 8 through half of the whole, by what was said above: therefore, taken alone, it denominates the whole as 4. By the same, that quality which extends as 4 through the other half, taken alone, contributes to the denomination of the whole as 2. Consequently those two quantities will denominate the whole precisely as 6, which is the middle degree between those halves. Therefore the assertion follows in the special case.

[{\S}3] But in general one argues for this as follows: when such a difform [quality] is designated, each quality in the halves is double of the degree to which the whole is denominated by that quality; and if something were made from these qualities extended together, that quality would be double of those two denominations joined together, and these denominations make up the denomination of the whole; and consequently, since these qualities are unequally intense, it is clear that the denomination of the whole will be equally intermediate between these two qualities, because any compound of two unequals is precisely the double of the middle [term] between them, as has been argued in the second assumption of Conclusion 38 [of the treatise] on motion, and is argued subsequently on the attainment of the highest degree: i.e., that any compound of two unequals is more than double of anything less than the middle [term] and less than double of anything more than the middle. So it is clear from what has been said that the denomination of the whole is in the middle degree between those qualities, which was to be proved.
\end{quote}

Clagett's translation continues:

\begin{quote}
[{\S}4] Against this assertion and its foundation it is argued thus: it follows that if the first proportional part of something were intense to a certain degree and the second twice as intense, the third three times, and so on to infinity, the whole would be precisely equally intense as the second proportional part, which however does not seem to be true. For it appears that this quality is infinite; hence if it is without a contrary, it will denominate its subject infinitely.
\end{quote}

The 1520 Venice edition continues:\footnote{{\it Et quod conclusio sequatur arguitur sic: sint .a. .b. duo equalia et uniformia eodem gradu, et dividatur .a. .b. in partes proportionales proportione dupla: et etiam illa hora ita quod partes maiores terminentur seu incipiant ab hoc instanti, et ponatur quod in prima parte proportionali illius hore intendatur prima pars .b. ad duplum, et similiter in secunda parte proportionali hore intendatur secunda pars proportionalis illius ad duplum, et sic in infinitum, ita quod in fine erit .b. uniforme sub gradu duplo ad gradum nunc habitum. Et ponatur quod .a. in prima parte proportionali illius hore intendatur totum residuum a prima parte proportionali .a. acquirende tantam latitudinem sicut tunc acquirit prima pars proportionalis .b. et in secunda parte proportionali eiusdem hore intendatur totum residuum .a. a prima parte proportionali et secunda illius .a. acquirendo tantam latitudinem sicut tunc acquiret pars proportionalis secunda .b. et in tertia parte proportionali intendatur residuum a prima parte proportionali et secunda et tertia acquirendo tantam latitudinem sicut tunc acquiret tertia pars proportionalis .b. et sic in infinitum sic quod quandocumque aliqua pars proportionalis .b. intendetur, pro tunc intendatur .a. secundum partes proportionales subsequentes partem correspondentem in .a. acquirendo tantam latitudinem sicut acquiret pars illa in .b. et sint .a. et .b. consimilis quantitatis continue. Quo posito sequitur quod .a. et .b. continue equevelociter intendentur quia .a. continue per partem equalem proportionalem intendetur sicut .b. quia residuum a prima parte proportionali .a. est equale prime parti proportionali eiusdem .b..

Cum igitur .b. in prima parte proportionali illius hore continue intendetur per primam partem proportionalem et similiter .a. per totum residuum a prima sua parte proportionali, patet quod .a. in prima parte proportionali equevelociter intendetur cum .b. et sic de omni alia parte eo quod quandocumque .b. intendetur per aliquam partem proportionalem .a. intendetur per totum interceptum inter partes correspondentes sui et extremum ubi partes terminantur, scilict minores. Cum ergo quelibet pars proportionalis cuiuslibet continui sit equalis toti intercepto inter eandem et extremum ubi partes minores terminantur,  sequitur ergo quod .a. continue per partem eque proportionalem sic intendetur sicut .b. Igitur patet quod .a. continue [this was an error in Padua where the compositor�s eye skipped from one �continue� to the next] equevelociter intendetur cum .b. et nunc est eque intensum cum .b. ut ponitur in casu et hoc ubi partes sint proportionales proportione dupla. Ergo in fine  .a. erit eque intensum cum .b. et .a. tunc est tale cuius prima pars proportionalis erit aliqualiter intensa, et secunda pars proportionalis in duplo intensior, et tertia in triplo intensior et sic in infinitum. Et .b. erit uniforme sub gradu sub quo erit secunda pars proportionalis .a. Ergo sequitur conclusio. Minor sic arguitur que fuit hec: `et .a. tunc erit tale cuius prima proportionalis etc.'  Sit .c. gradus quem nunc habent .a. et .b.. Tunc in fine erit prima pars proportionalis .a. sub .c. gradu, quia non intendetur. Et secunda pars proportionalis .a. tunc erit sub gradu duplo ad .c., quia in omni parte proportionali hore acquiretur tanta latitudo sicut est .c., et secunda pars proportionalis .a. solum intendetur per primam partem proportionalem illius hore. Ergo ille secunde parti proportionali .a. solum acquiretur unum .c. et per consequens, tunc erit sub duplo gradu ad .c. et tertia pars proportionalis .a. solum intendetur per duas partes proportionales hore, quia solum per primam partem et secundam. Et quarta pars proportionalis .a. solum intendetur per tres partes proportionales illius hore, et sic in infinitum, ut satis per casum patet. Ergo tertia pars proportionalis .a. solum acqauiret duo .c. et erit in fine in triplo intensior quam est .c. gradus. Et quarta acquiret tria .c. et sic in infinitum. Igitur patet quod tunc prima pars proportionalis .a. erit aliqualiter intensa, quia uniformis .c. gradui. Et secunda pars in duplo intensior, quia mediantibus duobus .c. gradibus. Et tertia in triplo intensior, quia mediantibus tribus .c. et sic in infinitum. Ergo patet ista minor. et .b. tunc erit eque intensum cum secunda parte proportionali .a. quia uniforme duobus .c. gradibus, quia quelibet eius pars proportionalis solum intendetur ad duplum. Ergo sequitur conclusio que est concedenda.} \cite[6va--6vb]{venice1520}}

\begin{quote}
[{\S}5] And that the conclusion follows is argued thus: let $a$ and $b$ be two equal [things] uniform to the same degree, and let $a$ and $b$ be divided into proportional parts in
double proportion, and also at that hour, so that the major parts end or begin from this instant; and let it be so that in the first proportional part of that hour the first part of $b$
intensifies to the double, and similarly in the second proportional part of the hour the second proportional part of it intensifies to the double, and so on to infinity, so that at the end
$b$ will be uniform by a degree double of that which it has now. And let it be so that in the first proportional part of that hour all the residue left by the first proportional part of $a$ intensifies, acquiring such a latitude as the first proportional part of $b$ then acquires, and in the second proportional part of the same hour the residue left by the first proportional part and the second one of $a$ intensifies, acquiring such a latitude as the second proportional part of $b$ then acquires, and in the third proportional part the residue left by the first proportional part and the second and the third one intensifies, acquiring such a latitude as the third proportional part of $b$ then acquires, and so on to infinity, so that, whenever some proportional part of $b$ intensifies, then also $a$ intensifies according to the proportional parts following the part that corresponds in $a$, acquiring such a latitude as that part of $b$ will acquire, and that $a$ and $b$ are continually of a quantity similar to each other. Once this is settled, it follows that $a$ and $b$ will continually intensify at equal velocity, since $a$ will continually intensify by an equal proportional part as $b$, since the residue left by the first proportional part of $a$ is equal to the first proportional part of $b$.

[{\S}6]Consequently, since in the first proportional part of that hour $b$ will continually intensify by the first proportional part and similarly $a$ by the whole residue left by its first proportional part, it is clear that in the first proportional part $a$ will intensify at equal velocity with $b$, and so for any other part, because whenever $b$ will intensify by some proportional part,
$a$ will intensify by the entire interval between the corresponding parts of itself and the terminal point where the parts -- i.e., the minor ones -- end. Therefore, since any proportional part of an arbitrary continuum is equal to all the interval between the same and the terminal point where the minor parts end, it follows that $a$ will continually intensify by an equally proportional part as $b$. Consequently it is clear that $a$ will continually intensify at equal velocity with $b$ and is now equally intense as $b$, as it is also proposed in this case where the parts are proportional in double proportion. Consequently at the end $a$ will be equally intense with $b$, and then $a$ is such that its first proportional part is of some intensity, and the second proportional part doubly more intense, and the third triply more intense, and so on to infinity; and $b$ will be uniform with the degree that the second proportional part of $a$ has. Therefore the conclusion follows. 

[{\S}7] The minor one, which was ``and then $a$ will be such that its first proportional...'', is argued thus: Let $c$ be the degree that $a$ and $b$ now have. Then at the end the first proportional part of $a$ will be at degree $c$, because it will not intensify; and the second proportional part of $a$ will then be at a degree double that of $c$, since in each proportional part of the hour as much latitude will be acquired as $c$ is, and the second proportional part of $a$ will only intensify through the first proportional part of that hour; therefore in that second proportional part $a$ will acquire only one $c$ and consequently it will then be at a degree double that of $c$; and the third proportional part of $a$ will only intensify through two proportional parts of the hour, since only through the first and the second part; and the fourth proportional part of $a$ will only intensify through three proportional parts of that hour, and so on to infinity, as is clear enough by the case. Consequently the third proportional part of $a$ will acquire two $c$'s and will be triply more intense at the end than the degree of $c$ is; and the fourth will acquire three $c$'s and so on to infinity. It is therefore clear that then the first proportional part of $a$ will be intense to some degree, since it is uniform with the degree $c$; and the second part doubly more intense, since two degrees $c$ come in between; and the third part triply more intense, since three $c$'s come between, and so on to infinity. Consequently that minor [conclusion] is clear, and $b$ will then be equally intense as the second proportional part of $a$ since it is uniform with two degrees $c$, because any of its proportional parts intensifies only to the double. Therefore the conclusion which must be admitted follows.
\end{quote}

Clagett's translation continues:

\begin{quote}
[{\S}8] For an argument in favour of the opposite, the consequences are denied: The quality is infinitely intense, therefore it denominates the whole subject infinitely. This infinite quality produces, if it is extended in this way, something infinitely modest with respect to that subject, in as much as the quality of the fourth proportional part is doubly more intense than the quality of the second proportional part and the subject is four times less, so it produces less by half than the second. Indeed if the fourth proportional part were eight times more intense than the first, just as it is eight times smaller than it [in extension], then it would produce just as much for the denomination of the whole as the first. However � as is known � the fourth part is actually less intense by half as it would then be. Consequently the fourth proportional part contributes less by half to the whole in comparison than the first proportional part does, and the first contributes only as much to the whole in comparison as the second, as is evident. Consequently the fourth proportional part produces less by half than the second for the intensity of the whole, and yet its quality is twice as intense. And proceeding in this way, any quality extended through a later part contributes less than a quality extended through an earlier part, calling those parts earlier which are closer to the endpoint where the larger parts terminate. And this is true of all the proportional parts of $a$ except for the first and the second, which contribute equally to the denomination of the whole.
\end{quote}




\section{Gloss}




\section*{Acknowledgments}
The recension of the Latin text from Swinehead's {\em Liber calculationum} is by Edith Dudley Sylla.

\bibliographystyle{amsplain}
\bibliography{swineshead}

\end{document}