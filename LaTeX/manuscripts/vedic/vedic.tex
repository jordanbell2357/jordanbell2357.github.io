\documentclass{article}
\usepackage{amsmath,amssymb,graphicx,subfig,mathrsfs,amsthm,enumitem,xfrac,flexisym}
\usepackage[polutonikogreek,english]{babel}
\newcommand{\Gk}[1]{\selectlanguage{polutonikogreek}#1\selectlanguage{english}}%\usepackage{caption}
%\usepackage[LGR,T1]{fontenc}
%\newcommand{\textgreek}[1]{\begingroup\fontencoding{LGR}\selectfont#1\endgroup}
%\usepackage{tikz-cd}
%\usepackage{hyperref}
\newcommand{\inner}[2]{\left\langle #1, #2 \right\rangle}
\newcommand{\tr}{\ensuremath\mathrm{tr}\,} 
\newcommand{\Span}{\ensuremath\mathrm{span}} 
\def\Re{\ensuremath{\mathrm{Re}}\,}
\def\Im{\ensuremath{\mathrm{Im}}\,}
\newcommand{\id}{\ensuremath\mathrm{id}} 
\newcommand{\gcm}{\ensuremath\mathrm{gcm}} 
\newcommand{\diam}{\ensuremath\mathrm{diam}} 
\newcommand{\sgn}{\ensuremath\mathrm{sgn}\,} 
\newcommand{\lcm}{\ensuremath\mathrm{lcm}} 
\newcommand{\supp}{\ensuremath\mathrm{supp}\,}
\newcommand{\dom}{\ensuremath\mathrm{dom}\,}
\newcommand{\norm}[1]{\left\Vert #1 \right\Vert}
\newtheorem{theorem}{Theorem}
\newtheorem{lemma}[theorem]{Lemma}
\newtheorem{proposition}[theorem]{Proposition}
\newtheorem{corollary}[theorem]{Corollary}
\begin{document}
\title{Vedic texts} 
\author{Jordan Bell\\ \texttt{jordan.bell@gmail.com}\\Department of Mathematics, University of Toronto}
\date{\today}

\maketitle

Kane \cite[p.~1224]{kane22}, Chapter XXXV


Olivelle \cite[pp.~xxiii--xxiv]{olivelle}:

\begin{quote}
Although the earliest texts, such as the hymns of the \d{R}gveda,
were composed in verse, the liturgical works ({\em br\=ahma\d{n}a}) of the
middle vedic period were composed in prose. This practice was
continued in the literature of the expert traditions; most ancient
works falling within the Vedic Supplements are in prose. Probably
because instruction in the expert traditions was carried out
orally and the pedagogy of these schools was based on first
memorizing the basic texts and then delving into their meaning
with the aid of the teacher, the basic texts came to be composed
in an aphoristic style known as {\em s\=utra}.
 A {\em s\=utra} is a sentence
from which most non-essential elements have been removed.
Individual {\em s\=utras} are often syntactically connected to the
preceding, words of earlier {\em s\=utras} being implicit in later ones, a
convention technically called {\em anuv\d{r}tti}. This convention makes
the entire composition similar to a chain and each {\em s\=utra} a link
in that chain. It is this characteristic that probably gave it the
name {\em s\=utra} (lit., `thread'), the composition being compared to
a thread on which each aphorism is strung like a pearl. Given
the brevity of each {\em s\=utra}, it is frequently impossible to understand
the meaning without the benefit of either an oral explanation
or a commentary.
\end{quote}

Olivelle \cite[p.~3]{olivelle}:

\begin{quote}
The Dharmas\=utra forms part of the voluminous Kalpas\=utra of
\=Apastamba containing thirty {\em pra\'snas} (lit., `questions') or books.
The first twenty-four comprise the \'Srautas\=utra. Books 25--6
contain the Mantrap\=a\d{t}ha or the collections of ritual formulas to
be used in domestic rites, and book 27 contains the G\d{r}hyas\=utra.
The two books of our Dharmas\=utra occupy books 28--9, and
the final book contains the \'Sulvas\=utra, a treatise on principles of
geometry needed for the vedic sacrifice. \=Apastamba belongs to
the Taittir{\=\i}ya branch of the Black Yajurveda. Opinion is divided
as to whether the entire Kalpas\=utra was composed by a single
individual (Kane 1962--75, i. 54). The Kalpas\=utra of \=Apastamba
has been preserved better than most probably because commentaries
were written on it at a relatively early date.
\end{quote}

Kane = P. V. Kane, {\em History of Dharma\'s\=astra}, 5 volumes. 

Olivelle \cite[p.~127]{olivelle}:

\begin{quote}
The Dharmas\=utra of Baudh\=ayana, as that of \=Apastamba,
forms part of the Kalpas\=utra ascribed to this eponymous author
and divided into {\em pra\'snas} (lit., `questions') or books. Unlike
\=Apastamba's, however, the ritual texts of Baudh\=ayana have been
tampered with repeatedly and contain numerous additions and
interpolations. The extent and structure of the entire Kalpas\=utra
are not altogether clear. It appears that the first twenty-nine
books contain the \'Srautas\=utra and other ritual treatises; book
30 contains the \'Sulvas\=utra (vedic geometry); and the next four
books comprise the G\d{r}hyas\'utra. The last four books are the
Dharmas\=utra. 
\end{quote}

In the introduction to his translation of the Baudhayana Srautasutra, Kashikar \cite[p.~xi]{kashikarI} writes:

\begin{quote}
The Baudh\=ayana \'Srautas\=utra forms the initial and prominent part of
the Baudh\=ayana corpus. This corpus comprises the \'Srauta, Pr\=aya\'scitta,
\'Sulba, G\d{r}hya, Pi\d{t}rmedha, Pravara and Dharma S\=utras. Tradition ascribes all
these sutras to Baudh\=ayana. While preparing his critical edition of the
Baudh\=ayana \'Srautas\=utra (Baudh\'SS), W. Caland examined all available
manuscripts of the above-mentioned types of the S\=utra-texts ascribed to
Baudh\=ayana, and formulated the order of the Baudh\=ayana corpus. In the 
printed edition of the \'Srautas\=utra the \'Sulbas\=utra forms the Pra\'sna XXX.
The Pravaras\=utra printed at the end is without the consecutive Pra\'sna 
number.
\end{quote}

Vitruvius, {\em De Architectura}, 1.6.4--9.

Hyginus Gromaticus, Establishment 152.4--22

Arthashastra 1.19.7,8; 2.20.10,41,42,39,40

\begin{quote}
During the ay
\end{quote}

White Yajurveda \cite[p.~50]{white}, VI, 36:

\begin{quote}
East, west, north, south, from every side to meet thee let the regions run.\\
Fill him, O Mother, let the noble meet together.
\end{quote}

Rigveda \cite[p.~405]{rigvedaI}, III, 1, 2:

\begin{quote}
East have we turned the rite; may the hymn aid it. With wood and worship shall they honour
Agni.\\
From heaven the synods of the wise have learnt it: e'en for the quick and strong they
seek advancement.
\end{quote}

Rigveda \cite[p.~277]{rigvedaIII}, VIII, 54:

\begin{quote}
Though, Indria, though art called by men from east and west, from north
and south,\\
Come hither quickly with fleet steeds;
\end{quote}

Sankhayana Grihyasutra \cite[pp.~22--24]{grihyaI}, Adhyaya 1, Khanda 7:

\begin{quote}
1. When assent has been declared (by the girl's father, the bridegroom) sacrifices.

2. He besmears a quadrangular space with cow-dung.

3. (Let him consider in the ceremonies to be performed,) of the two eastern intermediate directions,
the southern one as that to which (the rites) should be directed, if the rites belong to the Manes,

4. The northern one, if the rites belong to the gods,

5. Or rather the east (itself) according to some (teachers).

6. He draws in the middle (of the sacrificial ground) a line from south to north,

7. Upwards from this, turned upwards, to the south one line, in the middle one, to the north one.

8. These he sprinkles (with water),

9. Carries forward the fire with the verse, `I carry forward Agni with genial mind; may he be the assembler
of goods. Do no harm to us, to the old nor to the young; be a saviour to us, to men and animals,'

10. Or (he carries it forward) silently,

11. Then he wipes with his wet hand three times around the fire, turning his right side to it. This
they call Sam\^uhana (sweeping together).

12. Once, turning his left side to it, in the rites belonging to the Manes.
\end{quote}

Hiranyakesin Grihyasutra \cite[pp.~141--142]{grihyaII}, Prasna 1, Patala 1, Section 2:

\begin{quote}
1. And lays the (three) pegs round (the fire).

2. On the west side (of the fire) he places the middle (peg), with its broad end to the north,

3. On the south side (of the fire the second peg), so that it touches the middle one, with its broad end
to the east,

4. On the north side (of the fire the third peg), so that it touches the middle one, with its broad end
to the east.

5. To the west of the fire (the teacher who is going to initiate the student), sits down with his face
turned towards the east.

6. To the south (of the teacher) the boy, wearing the sacrificial cord over his left shoulder, having
sipped water, sits down and touches (the teacher).

7. Then (the teacher) sprinkles water round the fire (in the following way):

8. On the south side (of the fire he sprinkles water) from west to east with (the words), `Aditi! Give
thy consent!'--

9. On the west side, from south to north, with (the words), `Anumati! Give thy consent!' On the
north side, from west to east, with (the words), `Sarasvati! Give thy consent!'--

10. On all sides, so as to keep his right side turned towards (the fire), with (the Mantra), `God Savitri!
Give thy impulse!' (Taitt. Samh. I, 7, 7, 1).
\end{quote}

Asvalayana Grihyasutra \cite[p.~212]{grihyaII}, Prasna 1, Patala 1, Section 2:

\begin{quote}
1. Now he should examine the ground in the following ways.

2. He should dig a pit knee-deep and fill it again with the same earth (which he has taken out of it).

3. If (the earth) reaches out (of the pit, the ground is) excellent; if it is level, (it is) of middle quality;
if it does not fill (the pit, it is) to be rejected.

4. After sunset he should fill (the pit) with water and leave it so through the night.

5. If (in the morning) there is water in it, (the ground is) excellent; if it is moist, (it is) of middle
quality; if it is dry, (it is) to be rejected.

6. White (ground), of sweet taste, with sand on the surface, (should be elected) by a Brahmana.
7. Red (ground) for a Kshatriya.

8. Yellow (ground) for a Vaisya.

9. He should draw a thousand furrows on it and should have it measured off as quadrangular, with
equal sides to each (of the four) directions;

10. Or as an oblong quadrangle.
\end{quote}

Seidenberg \cite{seidenberg1983}

Taittiriya Samhita \cite[p.~119]{black1}, I,8,7,c:

\begin{quote}
The gods that sit in the east, led by Agni; that sit in the south,
led by Yama; that sit in the west, led by Savit\d{r}; that sit in the north,
led by Varuna; that sit above, led by B\d{r}haspati; that slay the Rak\d{s}ases;
may they protect us, may they help us; to them homage; to them
hail!
\end{quote}

Taittiriya Samhita \cite[p.~506]{black2}, VI,2,4:

\begin{quote}
All this earth is the 
Vedi, but they measure off and sacrifice on so much as they deem they can
use. The back cross-line is thirty feet, the eastern line is thirty-six feet,
the front cross-line is twenty-four feet.
\end{quote}

Staal \cite[p.~]{agniI}

\begin{quote}

\end{quote}

Staal \cite[p.~]{agniI}

\begin{quote}

\end{quote}



Dominik Wujastyk, Mathematics and Medicine in Sanskrit

Satapatha Brahmana \cite[pp.~62--63]{eggelingI}, I,2,5,14:

\begin{quote}
`Let is (the altar) measure a fathom across 
on the west side,' they say: that, namely, is the
size of a man, and it (the altar) should be of (the)
man's size. `Three cubits long (should be) the
``easterly line,'' for threefold is the sacrifice,' (so they say, but)
in this there is no (fixed) measure: let
him make it as long as he thinks fit in his own mind!
\end{quote}

fathrom=vy\^ama, distance between ends of outstretched arms, namely the size of a man.

{\em \'Satapatha Br\=ahma\d{n}a} \cite[pp.~111--112]{eggelingII}, third k\=a\d{n}da, fifth adhy\=aya, first br\=ahma\d{n}a:

\begin{quote}
1. From that post which is the largest on the east side (of the hall) he now
strides three steps forwards (to the east), and there drives in a peg,--this is the intermediate (peg).

2. From that middle peg he strides fifteen steps to the right, and there drives in a peg,--this is the
right hip.

3. From that middle peg he strides fifteen steps
northwards, and there drives in a peg,--this is the left hip.

4. From that middle peg he strides thirty-six steps eastwards, and there drives in a peg,--this is
the fore-part.

5. From that middle peg (in front) he strides twelve steps to the right, and there drives in a peg,--
this is the right shoulder.

6. From that middle peg he strides twelve steps to the north, and there drives in a peg,--this is the 
left shoulder. This is the measure of the altar.
\end{quote}

Satapatha Brahmana \cite[p.~199]{eggelingIII}, VI,3,1,30:

\begin{quote}
On the right (south) side is the \^Ahavan{\^\i}ya
fire, and on the left (north) lies that spade; for the 
\^Ahavan{\^\i}ya (m.) is a male, and the spade (abhri, f.)
a female, and the male lies on the right side of the female. [It lies] at a cubit's distance, for at a
cubit's distance the male lies by the female.
\end{quote}

Satapatha Brahmana \cite[pp.~327--329]{eggelingIII}, VII,2,2:

\begin{quote}
9. On the right (south) side of the fire-altar, he ploughs first a furrow eastwards inside the enclosing-stones, \ldots

10. Then on the hindpart (he ploughs a furrow) northwards, \ldots

11. Then on the left (north) side (he ploughs a furrow) eastwards, \ldots

12. Then on the forepart (he ploughs a furrow) southwards, \ldots. He first ploughs thus (south-west to
south-east), then thus (south-west to north-west), then thus (north-west to north-east), then thus (north-east to
south-east): that is (sunwise), for thus it is with the gods.
\end{quote}

Satapatha Brahmana \cite[pp.~372--373]{eggelingIII}, VII,4,1,35:

\begin{quote}
Behind the altar (he offers) while seated with
his face towards the east; then on the left (north) 
side (looking) to the south; then in front (looking) to the west;
then going round behind (he offers) on
the right (south) side while sitting with his face towards the north. Thus (he moves)
to the right, for that (leads) to the gods. Thereupon, going back, (he offers) while sitting behind,
with his face towards the east; and in this way that performance of his takes place towards the east.
\end{quote}

Satapatha Brahmana \cite[p.~139]{eggelingIV}, VIII,7,3,5:

\begin{quote}
The body (of the altar) he covers first, for of (a bird) that is produced,
the body is the first to be produced; then the right wing, then the tail, then
the left wing: that is in the rightward (sunwise) way, for this is (the way) with the gods.
\end{quote}

Satapatha Brahmana \cite[p.~147]{eggelingIV}, VIII,7,4,10:

\begin{quote}
First (he scatters them) at the back whilst
standing with his face towards the east; then on
the left (north) side towards the south; then in
front whilst facing the west; then, having gone
round the back, from the south whilst facing the
north: this is from left to right (sunwise), for that
is (the way) with the gods. Then, having gone
round, (he scatters chips) at the back whilst 
standing with his face to the east, for in this way
that performance of him took place.
\end{quote}

Satapatha Brahmana \cite[pp.~299--300]{eggelingIV}, X,2,1:

\begin{quote}
1. Pragapati was desirous of going up to the
world of heaven; but Pragapati, indeed, is all the
(sacrificial) animals--man, horse, bull, ram, and
he goat:--by means of these forms he could not
do so. He saw this bird-like body, the fire-altar,
and constructed it. He attempted to fly up, without
contracting and expanding (the wings), but could
not do so. By contracting and expanding (the wings)
he did fly up: whence even to this day birds
can only fly up when they contract their wings and
spread their feathers.

2. He measures it (the fire-altar) by finger-breadths;
for the sacrifice being a man, it is by
means of him that everything is measured here. [``The sacrifice, being the substitute of (the sacrificing) man, is represented
as identical with the Sacrificer\ldots'']
Now these, to wit, the fingers, are his lowest
measure: he thus secures for him (the sacrificial
man) that lowest measure of his, and therewith
he thus measures him.

3. He measures the twenty-four finger-breadths,--
the Gayatri (verse) consists of twenty-four syllables,
and Agni is of Gayatra nature: as great as Agni
is, as great is his measure, by so much he thus
measures him.
\end{quote}

Satapatha Brahmana \cite[pp.~310--311]{eggelingIV}, X,2,3:

\begin{quote}
8. He now measures off a cord thirty-six steps
(yards) long, and folds it up into seven (equal) parts:
of this he covers (the space of) the three front
(eastern) parts (with bricks), and leaves four (parts)
free.

9. He then measures (a cord) thirty steps long,
and lays it sevenfold: of this he covers three parts
(with bricks) behind, and leaves four (parts) free.

10. He then measures (a cord) twenty-four steps
long, and lays it sevenfold: of this he covers three
parts in front (with bricks), and leaves four (parts)
free. This, then, is the measuring out of the Vedi.
\end{quote}

Yano \cite[pp.~145--146]{yano}:

\begin{quote}
When we read the {\em \'sulbas\=utras} we have an impression that what has
been transmitted by the text is only a part of the whole instruction.
The rest of the instruction must have been transmitted by so-called
{\em guru\'si\d{s}yaparampar\=a}, `uninterrupted succession from teacher ({\em guru}) to
student ({\em \'si\d{s}ya})', and it was not open to the general public, or even
kept secret.
\end{quote}

\'Srautasutras \cite{kashikarI}

Pingree \cite[pp.~4]{pingree}:

\begin{quote}
The \'Srautasutras containing \'Sulbas\=utras are those of Baudh\=ayana,
 in which
the \'Sulbas\=utra is {\em pra\'sna} 30;
 of \=Apastamba,
 in which the \'Sulbas\=utra is also
{\em pra\'sna} 30;
 of Var\=aha,
 whose \'Sulbas\=utra is said to survive in a manuscript
at Madras;
 of Manava,
 in which the \'Sulbas\=utra is adhy\=aya 10
 (a recension
of this is entitled the Maitr\=aya\d{n}{\=\i}ya\'sulbas\=utra); of Var\=aha,
 whose \'Sulbas\=utra
survives in a manuscript at Mysore;
 and of K\=aty\=ayana,
 in which the \'Sulbas\=utra
is {\em pari\'si\d{s}\d{t}a} 7.
 The last of these belongs to the \'Suklayajurveda (the
V\=ajasaneyisa\d{m}hit\=a), the first five to the K\d{r}\d{s}\d{n}ayajurveda (Baudh\=ayana,
\=Apastamba, and V\=adhula to the Taittir{\=\i}yasa\d{m}hit\=a, and M\=anava and Var\=aha
to the Maitr\=aya\d{n}{\=\i}yasa\d{m}hit\=a).

Precise dating of any of these texts is impossible. The earliest, that of
Baudh\=ayana, was perhaps written before 500 B.C., and the remainder presumably
antedate the Christian era. It was, indeed, during this period also, probably
in the second century B.C., that the most striking {\em \'syenaciti} of which
remains survive was built in Kau\'s\=amb\=\i.
 The \=Apastamba appears to be the
second oldest of the major \'Sulbas\=utras, and the K\=aty\=ayana, which consists of
a {\em s\=utra} section (to a large extent repeating {\em s\=utras} of the \=Apastamba verbatim),
followed by a verse section, is among the latest; the M\=anava has apparently
copied some verses from the K\=aty\=ayana.

Each of the basic altars must be constructed with five layers of bricks, and
there must be a fixed number of bricks in each layer; moreover, the bricks in
the second and fourth layers must not be directly above or below those in the
first, third, and fifth layers. And the surface covered by the altar, regardless
of its shape, must cover an area of seven and one half square {\em puru\d{s}as} or, for
certain purposes, that area increased by specified numbers of square {\em puru\d{s}as},
or it must be multiplied by a given factor. Finally, the altar must be correctly
oriented with respect to the cardinal directions. The task faced by the authors
of the \'Sulbas\=utras was to prescribe rules for laying out these altars with only
a rope ({\em rajju} or {\em \'sulba}) of determined length and posts or gnomons ({\em \'sa\.nku}).
The geometrical problems that were solved by these altar-builders are indeed
impressive, but it would be a mistake to see in their works the unique origin of
geometry;
 others in India and elsewhere, whether in response to practical or
theoretical problems, may well have advanced as far without their solutions
having been committed to memory or eventually transcribed in manuscripts.
\end{quote}

Winternitz \cite[pp.~55--56]{winternitz}:

\begin{quote}
The expression ``Veda'' is justified only for this literature
which is regarded as revealed. However, there is another
class of works, which has the closest connection with the
Vedic literature, but yet cannot be said to belong to the
Veda. These are the so-�called Kalpas\=utras (sometimes also
called briefly ``S\=utras'') or manuals on ritual, which are
composed in a peculiar, aphoristic prose style. These
include:

1. The \u{S}rautas\=utras, which contain the rules for the
performance of the great sacrifices, which often lasted many
days, at which many sacred fires had to burn and a great
number of priests had to be employed.

2. The G\d{r}hyas\=utras, which contain directions for the
simple ceremonies and sacrificial acts of daily life (at birth,
marriage, death, and so on).

3. The Dharmas\=utras, books of instruction on spiritual
and secular law--the oldest law�books of the Indians.

Like Brahma\d{n}as, \={A}ra\d{n}yakas and Upani\d{s}ads, these works,
too, are connected with one of the four Vedas ; and there
are \u{S}rauta, G\d{r}hya, and Dharmas\=utras which belong to the
\d{R}gveda, others which belong to the S\=amaveda, to the
Yajurveda, or the Atharvaveda. As a matter of fact, they
originated in certain Vedic schools which set themselves the
task of the study of a certain Veda. Yet all these books of
instruction are regarded as human work, and no longer as
divine revelation ; they do not belong to the Veda, but to
the ``Ved\=a\.{n}gas,'' the ``limbs,'' i.e. ``the auxiliary sciences
of the Veda.''
\end{quote}

Parpola \cite[p.~187]{parpola}:

\begin{quote}
The Yajurvedic texts connect the {\em nak\d{s}atras} and
the full and new moon with specific ``bricks'' laid down in brick-built fire altars ({\em agni-citi}),
which are unknown in the Rigveda and belong to the archaic layer of Vedic rituals connected
with the Atharvavedic tradition. The relatively late Vedic \'Sulvas\=utras describe
the elaborate geometry and rules of orientation used in the construction of these altars. This
\'Sulvas\=utra tradition probably goes back to the Indus people, who lived for a millenium in
brick-built cities and needed solar time-reckoning, as did other agriculturally based riverine
and urban early cultures, for example, in Mesopotamia and Egypt, in which astronomy and
astrology invariably form an important part of religion.
\end{quote}

Parpola \cite[pp.~195--196]{parpola}:

\begin{quote}
The Baudh\=ayana-\'Sulvas\=utra (1,22--28) describes a method to construct an oriented 
square and thus to define the cardinal and intermediate directions by using a gnomon and 
a cord with marked midpoint. This produces a pattern of ``intersecting circles,'' which is an
important motif on Early and Mature Harappan painted pottery, and also a favorite motif
on Harappan bathroom floors and ``bath tubs.'' I suspect this motif relates to the necessity 
of leveling the ground for the gnomon by means of water. In later Indian astronomical texts
water is recommended for making ground perfectly level.
\end{quote}

\nocite{*}

\bibliographystyle{amsplain}
\bibliography{vedic}

\end{document}