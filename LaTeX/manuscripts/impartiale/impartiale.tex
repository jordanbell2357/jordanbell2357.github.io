\documentclass{amsart}
\usepackage{amsmath,amssymb,mathrsfs,amsthm}
\newtheorem{theorem}{Theorem}
\newtheorem{lemma}[theorem]{Lemma}
\newtheorem{proposition}[theorem]{Proposition}
\newtheorem{corollary}[theorem]{Corollary}
\theoremstyle{definition}
\newtheorem{definition}[theorem]{Definition}
\newtheorem{example}[theorem]{Example}
\begin{document}
\title{Proof of the pentagonal number theorem}
\author{Jordan Bell}
\email{jordan.bell@gmail.com}
\address{Department of Mathematics, University of Toronto, Toronto, Ontario, Canada}
\author{Viktor Bl{\aa}sj{\"o}}
\email{V.N.E.Blasjo@uu.nl}
\address{Mathematisch Instituut, Universiteit Utrecht, Utrecht, The Netherlands}
\date{\today}


\maketitle

1752 Biblioth\`eque impartiale (juillet et ao\^ut), tome VI, premi\`ere partie, article
IX, ``Demonstration de la Loi d'une suite de termes de la Quantit\'e compos\'ee qui
est faite par la multiplication des Binomes $1-x$, $1-x^2$, $1-x^3$, \& c.'', pp. 111–126.

\[
y-2y1'+y2'=0
\]
That is, $y_p-2y_{p+1}+y_{p+2}=0$, $p \geq 0$. 

Let
\[
(1-x)(1-x^2)(1-x^3)\cdots = 1+zx+z'x^2+z''x^3+z'''x^4+z^{\textrm{IV}}z^5+\cdots
\]
In other words,
\[
(1-x)(1-x^2)(1-x^3)\cdots = 1 + \sum_{p \geq 0} z_p x^{p+1}.
\]

Let
\[
S=(1-x)(1-x^2)(1-x^3)(1-x^4)\cdots
\]

Let $i=1+x$, $i'=1+x^2$, $i''=1+x^3$, $i'''=1+x^4$, etc.
In other words,
\[
i_p=1+x^{p+1},\qquad p \geq 0.
\]

Let
$k=1$, $k'=5$, $k''=12$, $k'''=22$, $k^{\textrm{IV}}=35$, etc.
In other words,
\[
k_p=\frac{(p+1)(3p+2)}{2}, \qquad p \geq 0.
\]

We shall prove that 
\[
S=1-ix^{k'}-i''x^{k''}+i'''x^{k'''}-i^{\textrm{IV}}x^{k^{\textrm{IV}}}+\cdots
\]

Let $m=1-x$, $m'=1-x^2$, $m''=1-x^3$, $m'''=1-x^4$, $m^{\textrm{IV}}=1-x^5$, etc.
In other words,
\[
m_p=1-x^{p+1}, \qquad p\geq 0.
\]




%\bibliographystyle{amsplain}
%\bibliography{impartiale}

\end{document}