\documentclass{amsart}
\usepackage{amsmath,amssymb,graphicx,subfig,mathrsfs}
\newcommand{\norm}[1]{\left\Vert #1 \right\Vert}
\def\Re{\ensuremath{\mathrm{Re}}\,}
\def\Im{\ensuremath{\mathrm{Im}}\,}
\def\sgn{\ensuremath{\mathrm{sgn}}\,}
\newtheorem{theorem}{Theorem}
\newtheorem{lemma}[theorem]{Lemma}
\newtheorem{proposition}[theorem]{Proposition}
\newtheorem{corollary}[theorem]{Corollary}
\begin{document}
\title{Riemann}
\author{Jordan Bell}
\email{jordan.bell@gmail.com}
\address{Department of Mathematics, University of Toronto, Toronto, Ontario, Canada}
\date{\today}

\maketitle


Zeitschrift f\"ur Mathematik und Physik, Volume 34
J. Frischau

"Riemann's Example" of a continuous nondifferentiable function in the light of two letters (1865) of Christoffel to Prym, Bulletin de la Societe Mathematique de Belgique, Serie A, t. XXXVIII, 1986, p. 45-73.

Introduction to the theory of Fourier's series and integrals, Horatio Scott Carslaw

Grattan-Guinness, p.~158

Hobson, Ch. VII,
The Theory of Functions of a Real Variable and the Theory of, Volume 1,
p. 730

Define $(x)$ to be $0$ if $x  \in \mathbb{Z}+\frac{1}{2}$; if $x \not \in \mathbb{Z}+\frac{1}{2}$ then there is
an integer $m_x$ for which $|x-m_x| < |x-n|$ for all integers $n \neq m_x$, and we define $(x)$ to be $x-m_x$. 

Jahnke


22

31

Riemann \cite[p.~105, \S 6]{riemann} defines 
\[
f(x)=\sum_{n=1}^\infty \frac{(nx)}{n^2};
\]
for any $x$, the series converges absolutely because $|(-nx)|<\frac{1}{2}$. 
Riemann states that if  $p$ and $m$ are relatively prime and $x=\frac{p}{2m}$,  then
\[
f(x^+)=\lim_{h \to 0^+} f(x+h) = f(x)-\frac{\pi^2}{16m^2}, \qquad f(x^-)=\lim_{h \to 0^-} f(x+h) = f(x)+\frac{\pi^2}{16m^2},
\]
thus
\[
f(x^-)-f(x^+)=\frac{\pi^2}{8m^2},
\]
and hence that $f$ is discontinuous at such points, and says that at all other points $f$ is continuous; see Neuenschwander \cite{neuenschwander} about Riemann's work on pathological functions, and also \cite[p.~37]{pringsheim}.
For any interval $[a,b]$ and any $\sigma>0$, it is apparent from the above that there are only finitely many $x \in [a,b]$ for which
$f(x^-)-f(x^+)>\sigma$, and Riemann deduces from this that $f$ is  Riemann integrable on $[a,b]$; cf. Hawkins  \cite[p.~18]{hawkins} on the history
of Riemann integration.
Later in the same paper  \cite[p.~129, \S 13]{riemann}, Riemann states that the function
\[
x \mapsto \sum_{n=1}^\infty \frac{(nx)}{n},
\] 
is not Riemann integrable in any interval.


Hankel 1871 199-200

\bibliographystyle{amsplain}
\bibliography{riemann}

\end{document}