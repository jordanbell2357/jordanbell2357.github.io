\documentclass{article}
\usepackage{amsmath,amssymb,mathrsfs,amsthm,xfrac}
\usepackage[LGR,T1]{fontenc}
\newcommand{\textgreek}[1]{\begingroup\fontencoding{LGR}\selectfont#1\endgroup}
%\usepackage{tikz-cd}
%\usepackage{hyperref}
\newcommand{\inner}[2]{\left\langle #1, #2 \right\rangle}
\newcommand{\tr}{\ensuremath\mathrm{tr}\,} 
\newcommand{\Span}{\ensuremath\mathrm{span}} 
\def\Re{\ensuremath{\mathrm{Re}}\,}
\def\Im{\ensuremath{\mathrm{Im}}\,}
\newcommand{\id}{\ensuremath\mathrm{id}} 
\newcommand{\gcm}{\ensuremath\mathrm{gcm}} 
\newcommand{\diam}{\ensuremath\mathrm{diam}} 
\newcommand{\sgn}{\ensuremath\mathrm{sgn}\,} 
\newcommand{\lcm}{\ensuremath\mathrm{lcm}} 
\newcommand{\supp}{\ensuremath\mathrm{supp}\,}
\newcommand{\dom}{\ensuremath\mathrm{dom}\,}
\newcommand{\norm}[1]{\left\Vert #1 \right\Vert}
\newcommand*\rfrac[2]{{}^{#1}\!/_{#2}}
\newtheorem{theorem}{Theorem}
\newtheorem{lemma}[theorem]{Lemma}
\newtheorem{proposition}[theorem]{Proposition}
\newtheorem{corollary}[theorem]{Corollary}
\begin{document}
\title{Calendars}
\author{Jordan Bell\\ \texttt{jordan.bell@gmail.com}\\Department of Mathematics, University of Toronto}
\date{\today}

\maketitle

Plutarch, {\em Platonic Questions} Question VIII \cite[pp.~81--83]{LCL427}, 1006 E--F:

\begin{quote}
It is best, then, to understand that the earth
is an instrument of time not by being in motion as
the stars are but by remaining always at rest as they
revolve about her and so providing them with risings
and settings, which define days and nights, the
primary measures of times. That is also why he
called her strict guardian and artificer of night and
day, for the pins of sun-dials too have come to be
instruments and measures of time not by changing
their position along with the shadows but by standing
still, imitating the earth's occultation of the sun when
he moves down around her, as Empedocles said
\begin{quote}
Night is produced by the earth when she stands in the way
of the daylight.
\end{quote}
Such, then, is the explanation of this point.
\end{quote}

Timaeus Locrus 97 D.

Proclus, {\em In Platonis Timaeum} iii, pp.~139, 23--140, 5

Timaeus 40 C

Plutarch, {\em De Facie} 937 E, 938 E.

Empedocles Fragment B 48.

Solomon Gandz, Jewish calendar week

Parker, Richard A. and Waldo H. Dubberstein. Babylonian Chronology 626 BC.�AD. 75. Providence, RI: Brown University Press, 1956.

Philo of Alexandria, Quaestiones ad Exodum, Special Laws, Life of Moses

Josephus 3.10.5

Johannes Lydus, de Mensibus

The Roman Calendar from Numa to Constantine: Time, History, and the Fasti, J\"org R\"upke

Alexandria and the Moon: An Investigation into the Lunar Macedonian Calendar of Ptolemaic Egypt (Studia Hellenistica), C. Bennett

Ancient Egyptian Science: A Source Book. Volume Two: Calendars, Clocks, and Astronomy, Clagett

Calendar of the Roman Republic, Agnes Kirsopp Michels

The Sacred and Civil Calendar of the Athenian Year, Jon D. Mikalson

Greek and Roman Sundials, Sharon Gibbs, 1976

Parker, Richard A. The Calendars of Ancient Egypt, Studies in Ancient Oriental Civilization, 26. Chicago: University of Chicago Press, 1950.

\end{document}