\documentclass{amsart}
\usepackage{amsmath,amssymb,graphicx,subfig,mathrsfs,amsthm,enumitem}
\usepackage[LGR,T1]{fontenc}
\newcommand{\textgreek}[1]{\begingroup\fontencoding{LGR}\selectfont#1\endgroup}
\newtheorem{theorem}{Theorem}
\newtheorem{lemma}[theorem]{Lemma}
\newtheorem{proposition}[theorem]{Proposition}
\newtheorem{corollary}[theorem]{Corollary}
\theoremstyle{definition}
\newtheorem{definition}[theorem]{Definition}
\newtheorem{example}[theorem]{Example}
\begin{document}
\title{Extant writings in antique and medieval mathematics}
\author{Jordan Bell}
\email{jordan.bell@gmail.com}
\address{Department of Mathematics, University of Toronto, Toronto, Ontario, Canada}
\date{\today}

\maketitle

S Gandz (ed.), The geometry of al-Khwarizmi (Berlin, 1932).

F Rosen (trs.), Muhammad ibn Musa Al-Khwarizmi : Algebra (London, 1831).

J P Hogendijk (ed. and trans.), Ibn Al-Haytham's Completion of the Conics (1985).

D S Kasir (ed. and trans.), The Algebra of Omar Khayyam (1931, reprinted 1972).

M Levey (ed. and trans.), The Algebra of Abu Kamil (1966).

M Levey and M Petruck (eds. and trans.), Principles of Hindu Reckoning (1965).

F Rosen (ed. and trans.), The Algebra of Mohammed ben Musa (1831, reprinted 1986).

A S Saidan (ed. and trans.), The Arithmetic of al-Uqlidisi (1978).

Roger Cotes -- Natural Philosopher, Ronald Gowing

The Arabic Version of Euclid's Optics: Edited and Translated with Historical Introduction and Commentary Volume I (Sources in the History of Mathematics and Physical Sciences) (v. 16),
Elaheh Kheirandish 

A Critical Edition of Ibn al-Haytham's On the Shape of the Eclipse: The First Experimental Study of the Camera Obscura (Sources and Studies in the ... Sciences) (English, Arabic and Greek Edition) 1st ed. 2016, Dominique Raynaud

Abu Kamil Alg\`ebre et analyse diophantienne. \'Edition, traduction et commentaire. Roshdi Rashed, Scientia Graeco-Arabica 9

Roshdi Rashed, Les math\'ematiques infinit\'esimales du IXe au XIe si\`ecle

Thabit ibn Qurra: Science and Philosophy in Ninth-Century Baghdad (Scientia Graeco-Arabica), Roshdi Rashed

Sources in the History of Mathematics and Physical Sciences 7,
Ibn al-Haytham�s Completion of the Conics,
J. P. Hogendijk 

A. I. Sabra,
The Optics of Ibn al-Haytham, books I-III, On Direct Vision. Translated with introduction and commentary, in 2 volumes

Ibn al-Haytham's Theory of Conics, Geometrical Constructions and Practical Geometry: A History of Arabic Sciences and Mathematics Volume 3 (Culture and Civilization in the Middle East),
Roshdi Rashed

Ibn al-Haytham and Analytical Mathematics: A History of Arabic Sciences and Mathematics Volume 2 (Culture and Civilization in the Middle East),
Roshdi Rashed

Thabit ibn Qurra: Science and Philosophy in Ninth-Century Baghdad (Scientia Graeco-Arabica),
Roshdi Rashed

Histoire De L'analyse Diophantienne Classique: D'abu Kamil a Fermat (Scientia Graeco-Arabica), Roshdi Rashed

Les ``Arithmetiques'' de Diophante: Lecture Historique Et Mathematique (Scientia Graeco-Arabica),
 Roshdi Rashed and Christian Houzel
 
Oeuvre mathematique d'al-Sijzi Volume 1: Geometrie des coniques et theorie des nombres au Xe siecle,
 Roshdi Rashed
 
Ptolemy and the Foundations of Ancient Mathematical Optics: A Guided Study (Transactions of the American Philosophical Society),
 A. Mark Smith
 
Alhacen's Theory of Visual Perception (First Three Books of Alhacen's De Aspectibus), Volume One--Introduction and Latin Text (Transactions of the American Philosophical Society),
 A. Mark Smith
 
Porphyry's Commentary on Ptolemy's Harmonics: A Greek Text and Annotated Translation,
 Andrew Barker
 
Theories of Vision from Al-Kindi to Kepler, David C. Lindberg

\nocite{*}
\bibliographystyle{amsplain}
\bibliography{extant}

\end{document}