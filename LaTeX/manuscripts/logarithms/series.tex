\documentclass{article}
\usepackage{amsmath,amssymb,mathrsfs,amsthm}
%\usepackage{tikz-cd}
\usepackage{hyperref}
\newcommand{\inner}[2]{\left\langle #1, #2 \right\rangle}
\newcommand{\tr}{\ensuremath\mathrm{tr}\,} 
\newcommand{\Span}{\ensuremath\mathrm{span}} 
\def\Re{\ensuremath{\mathrm{Re}}\,}
\def\Im{\ensuremath{\mathrm{Im}}\,}
\newcommand{\id}{\ensuremath\mathrm{id}} 
\newcommand{\var}{\ensuremath\mathrm{var}} 
\newcommand{\Lip}{\ensuremath\mathrm{Lip}}
\newcommand{\Res}{\ensuremath\mathrm{Res}}  
\newcommand{\GL}{\ensuremath\mathrm{GL}} 
\newcommand{\diam}{\ensuremath\mathrm{diam}} 
\newcommand{\sgn}{\ensuremath\mathrm{sgn}\,} 
\newcommand{\lcm}{\ensuremath\mathrm{lcm}} 
\newcommand{\sinc}{\ensuremath\mathrm{sinc}\,} 
\newcommand{\Si}{\ensuremath\mathrm{Si}} 
\newcommand{\supp}{\ensuremath\mathrm{supp}\,}
\newcommand{\dom}{\ensuremath\mathrm{dom}\,}
\newcommand{\upto}{\nearrow}
\newcommand{\downto}{\searrow}
\newcommand{\norm}[1]{\left\Vert #1 \right\Vert}
\newtheorem{theorem}{Theorem}
\newtheorem{lemma}[theorem]{Lemma}
\newtheorem{proposition}[theorem]{Proposition}
\newtheorem{corollary}[theorem]{Corollary}
\theoremstyle{definition}
\newtheorem{definition}[theorem]{Definition}
\newtheorem{example}[theorem]{Example}
\begin{document}
\title{Alternating series}
\author{Jordan Bell\\ \texttt{jordan.bell@gmail.com}\\Department of Mathematics, University of Toronto}
\date{\today}

\maketitle

Jahnke, A History of Analysis

Isaac Barrow

H W Turnbull, James Gregory Memorial Volume (London, 1939)

C J Scriba, Gregory's converging double sequence: a new look at the controversy between Huygens and Gregory over the 'analytical' quadrature of the circle, Historia Math. 10 (3) (1983), 274-285.

Montmort, De seriebus infinitis tractatus in Philosophical Transactions of the Royal Society, 30 (1720), 633--675

Henry Briggs (1624) Arithmetica Logarithmica

B Burn, Gregory of St. Vincent and the Rectangular Hyperbola, The Mathematical Gazette 84 (501) (2000), 480-485.

Ch Naux, G de St. Vincent et les propri�t�s logarithmiques de l'hyperbole \'equilat\`ere, Rev. Questions Sci. 143 (2) (1972), 209-221.

S. J. Riguaud, ed., Correspondence of scientific Men of the seventeenth Century, I (Oxford, 1841)

Joost B\"urgi,
Arithmetische und geometrische Progress-Tabulen, sambt gr\"undlichem Unterricht, wie solche n\"utzlich in allerley Rechnungen zu gebrauchen, und verstanden werden sol (Prague, 1620)

Christiaan Huygens (1651) Theoremata de quadratura hyperboles, ellipsis et circuli, in Oeuvres Compl\`etes, Tome XI.
CH. HUYGENS, �uvres comple`tes, publie?es par la Socie?te? Hollandaise des Sciences, 19 vol., La Haye (Martinus Nijhoff), 1888-1937: a) Exa- men de ?Vera Circuli et Hyperboles Quadratura . . . ?, ; b) Horologium Oscillatorium, t. XVIII; c) Theoremata de Quadratura Hyperboles, Ellipsis et Circuli . . . , t. XI, p. 271-337; d) De Circuli Magnitudine Inventa . . . , t. XII, p. 91-180.

William Brouncker (1667) The Squaring of the Hyperbola, Philosophical Transactions of the Royal Society of London, abridged edition 1809, v. i, pp 233--6

E. Voellmy, Jost B\"urgi und die Logarithmen, in Beihefte zur Zeitschrift f\"ur Elemente der Mathematik, no. 5 (1948)

Cataldi, {\em Trattato del modo brevissimo di trovare la radice quadra delli numeri}, 1613, Bologna, appresso Bartolomeo Cochi 
square root series. E. Bortolotti. Le antiche regole empiriche del calcolo approssimato dei radicali quadratici e le prime serie infinite, in Bollettino della mathesis, 11 (1919), 14--29

J. NEPER, Mirifici logarithmorum canonis constructio, Lyon, 1620

B. CAVALIERI: a) Geometria indivisibilibus continuorum quadam ratione promota, Bononiae, 1635 (2e ed., 1653); b) Exercitationes geometricae sex, Bononiae, 1647.

E.TORRICELLI, Opere, 4vol., ed. G. Loria et G.Vassura, Faenza(Montanari), 1919.

E. TORRICELLI, in G. LORIA, Bibl. Mat. (III), t. I, 1900, p. 78-79.

Gregory Saint-Vincent [25, p. 102], book II, part I, scholion to proposition
LXXXVII. Whiteside VIII.300].

R. DESCARTES, �uvres, ed. Ch. Adam et P. Tannery, 11 vol., Paris (L. Cerf), 1897-1909.

P. GREGORII A SANCTO VINCENTIO, Opus Geometricum Quadraturae Circuli et Sectionum Coni . . . , 2 vol., Antverpiae, 1647.

N.MERCATOR,Logarithmotechnia...cuinuncacceditveraquadratura hyperbolae . . . Londini, 1668 (reproduced in F. MASERES, Scriptores, Logarithmici . . . , vol. I, London, 1791, p. 167-196).

LORD BROUNCKER, The Squaring of the Hyperbola by an infinite se- ries of Rational Numbers, together with its Demonstration by the Right Honourable the Lord Viscount Brouncker, Phil. Trans, v. 3 (1668), p. 645-649 (reproduced in F. MASE` RES, Scriptores, Logarithmici . . . , vol. I, London, 1791, p. 213-218).

J. WALLIS, Opera Mathematica, 3 vol., Oxoniae, 1693-95: a) Arith- metica Infinitorum, t. I, p. 355-478.

J. WALLIS, LOGARITHMOTECHNIA NICOLAI MERCATORIS: discoursed in a letter written by Dr. J. Wallis to the Lord Viscount Brouncker, Phil. Trans., v. 3 (1668), p. 753-759 (reproduced in F. MASE` RES, Scriptores Logarithmici . . . , vol. I, London, 1791, p. 219-226).

J. GREGORY: a) Vera Circuli et Hyperbolae Quadratura . . . , Pataviae, 1667; b) Geometriae Pars Universalis, Pataviae, 1668; c) Exercitationes Geometricae, London, 1668.

James Gregory Tercentenary Memorial Volume, containing his corre- spondence with John Collins and his hitherto unpublished mathematical manuscripts . . . , ed. H. W. Turnbull, London (Bell and Sons), 1939.

1684-1691, D. T. Whiteside, p. 33

Westfall [30, p. 33]

Vincent Leotaud, Examen circuli quadraturae, 1654

I point out that Torricelli gave a geometric proof of the sum of a geometric series
in his De dimensione Parabolae [1644]. For Torricelli's proof, I refer to Panza
[1992, 307?308].

R. Ayoub, What is a Napierian Logarithm? American Mathematical Monthly,100 (1993) pp. 351-364.



\section{Gr\'egoire de Saint-Vincent}
Gregory Saint-Vincent, {\em Opus Geometricum Quadraturae Circuli et Sectionum Coni}, 1647: 51--177.,






\section{Brouncker}
The squaring of the hyperbola, by an infinite series of rational numbers, together with its demonstration
Philosophical Transactions No 34 (13 April 1668), 645-9.


\section{Newton}
I.112-5,134-42

II.166,246

III


Collins to Gregory Decembry 1760 on sine,  Gregory to Collins February 1761.

Gregory {\em Vera Circuli} 1667

Gregory, {\em Exercitationes Geometriae}, ``Appendicula ad Veram Circuli et Hyperbole Quadraturam'' 1668

In both the Vera Quadratur aand the ``Appendicuta'' GREGORY was concerned more generally with sectors of central conics. See Vera Quadratura Proposition 20 for 
the circle and the ellipse, Proposition 25 for the hyperbola and Proposition 29 for the calculation. HUYGENS had already given the result for the circle in Proposition 5
of his De Circuli Magnitudine Inventa (Leiden, 1654). That GREGORY failed to acknowledge this was the main point of contention in the bitter controversy that ensued 
between him and HUYGENS. GREGORY's methods are discussed by D. T. WHITESIDE in op. cit. (see especially his pages 226-7 and 266-70).

As well as inscribed figures GREGORY considered circumscribed figures and combinations of both to produce a series of upper and lower bounds for areas of
sectors in the ``Appendicula'' - twenty three in all. For an analysis of this work see J. E. HOFMANN, ``Uber Gregorys systematische N/iherungen ffir den Sektor
eines Mittelpunktkegelschnittes'', Centaurus 1 (1950), 24--37.

\section{Wolff}
October 1674, {\em Schediasma de serierum summis, et seriebus quadraticibus}.

Wolff to Leibniz June 12, 1712 \cite[pp.~143, Letter LXX]{wolf}

Leibniz to Wolff July 13 1712 \cite[pp.~147, Letter LXXI]{wolf}:

\begin{quote}
Respondissem citius, si prius vacasset elegantissimam tuam
meditationem considerare attentius, qua ostendere aggrederis, ut
$1-1+1-1$ etc. in infinit. est $\frac{1}{2}$, ita $1-2+4-8+16-32$ etc.
esse $\frac{1}{3}$, et $1-3+9-27+81$ etc. esse $\frac{1}{4}$, et ita porro; in quo
ego haesi, quia summationes serierum infinitarum solent postulare
descrecentiam terminorum.
\end{quote}


\section{Wallis}
Wallis, {\em Arithmetica Infinitorum}, Propositions 39--41 \cite[p.~39]{wallis}.

%\begin{quote}
%
%\end{quote}


\section{Leibniz}
Leibniz, {\em De vera proportione Circuli ad Quadratum circumscriptum in Numeris rationalibus expressa},
Acta Erud February 1682, \cite[pp.~118--122]{gerhardtV}

De quadratura arithmetica, Proposition XLIX, p.~657 \cite[p.~657]{LeibnizVII6}.

Leibniz to Hermann, June 26, 1705 \cite[pp.~272--275, Letter VII]{gerhardtIV}

\begin{quote}
Videtur mihi determinatio limitum pars esse essentialis doctrinae de seriebus infinitis plene tradendae.
\end{quote}

Leibniz \cite[p.~922]{gerhardtIII}

Leibniz to Johann Bernoulli, January 10, 1714 \cite[pp.~925--927, Letter CCLI]{gerhardtIII}, 

Johann Bernoulli to Leibniz, February 28, 1714 \cite[pp.~927--930, Letter CCLII]{gerhardtIII}.


Leibniz, {\em Epistola ad. V. Cl. Christianum Wolfium, Professorem Matheseos Halensem,
circa Scientiam infiniti}, Acta Eruditorum Supplementa, Volume V, 1713 \cite[pp.~382--386]{gerhardtV}.


Leibniz, AE, February 1682, {\em De vera proportione Circuli ad Quadratum circumscriptum in Numeris rationalibus
expressa} \cite[pp.~118--122]{gerhardtV}.

Grandi, {\em Quadratura circuli et hyperbolae per infinitas hyperbolas geometrice exhibita}, Pisa, 1703

\section{Nicolaus Bernoulli}
Nicolaus Bernoulli to Leibniz, October 25, 1712 and April 7, 1713.

Leibniz to Nicolaus Bernoulli, June 28, 1713 \cite[p.~983]{gerhardtIII}







\section{Alternating series test}
\begin{theorem}[Alternating series test]
Suppose that $0<a_{k+1}<a_k$, $k \geq 1$, and that $a_k \to 0$ as $k \to \infty$. 
Let $S_m = \sum_{1 \leq k \leq m} (-1)^{k-1} a_k$. 
Then (i) 
\[
S_2<S_4<\cdots<S_3<S_1,
\]
and (ii) there is some $S$ such that
\[
S_m \to S, \qquad m \to \infty,
\]
with
\[
S_2<S_4<\cdots<S<\cdots<S_3<S_1,
\]
and for $r \geq 1$,
\[
S-a_{2r+1} <S_{2r} < S,\qquad S<S_{2r-1}<S+a_{2r}.
\]
\end{theorem}
\begin{proof}
(i) As $a_{k+1}<a_k$,
\[
S_{2r+2}-S_{2r} = -a_{2r+2} + a_{2r+1} > 0
\]
and
\[
S_{2r+1}-S_{2r-1} = a_{2r+1}-a_{2r} < 0. 
\]
Thus for $r \geq 1$,
\begin{equation}
S_{2r+2}>S_{2r},\qquad S_{2r+1}<S_{2r-1}.
\label{S2r}
\end{equation}

Furthermore, for $r \geq 1$,
\begin{equation}
S_{2r}=-a_{2r}+S_{2r-1}<S_{2r-1}
\label{evenodd}
\end{equation}
Fix $r$.
For $1 \rho \leq r$, by \eqref{S2r} we have $S_{2\rho-1} \geq S_{2r-1}$ and thus by
\eqref{evenodd} we have
$S_{2r} < S_{2\rho-1}$. For
$\rho \geq r$, by \eqref{S2r} we have $S_{2r} \leq S_{2\rho}$ and thus by \eqref{evenodd} we have
$S_{2r}<S_{2\rho-1}$. Therefore
$S_{2r}<S_{2\rho-1}$ for all $\rho \geq 1$. That is, if $n$ is even and $m$ is odd then $S_n<S_m$.
Thus
\[
S_2<S_4<\cdots<S_3<S_1.
\]


(ii) From (i) we have
\[
S_2<S_4<\cdots<S_3<S_1.
\]
In particular, $S_{2r} < S_1$ for all $r \geq 1$ and $S_{2r-1}>S_2$ for all $r \geq 1$. 
Let
\[
A=\sup\{S_{2r}: r \geq 1 \} \leq S_1,\qquad B=\inf\{S_{2r-1}:r \geq 1\} \geq S_2.
\]
Then $S_{2r-1}-S_{2r} \to B-A$. But $S_{2r-1}-S_{2r}=-a_{2r}$ and
$a_{2r+1} \to 0$. Therefore $B-A=0$, i.e. $A=B$. 
Let
\[
S = A = B,
\]
so $S_{2r}<S$ and $S_{2r-1}>S$ for all $r \geq 1$. 
Then
\[
S - S_{2r} < S_{2r+1} - S_{2r} = a_{2r+1},
\]
giving $S_{2r}>S-a_{2r+1}$,
and
\[
S_{2r-1} - S < S_{2r-1}-S_{2r} = a_{2r},
\]
giving $S_{2r-1}<S+a_{2r}$,
completing the proof.
\end{proof}

Jakob Bernoulli, 1689 limits

Mengoli, Geometriae Speciosiae 1659

The Mathematical Papers of Isaac Newton:, Volume 4; Volumes 1674-1684, p.~611

Cauchy's Cours d'analyse : An Annotated Translation (2009), page 85-on; p.~125 in original

James Gregory 1668 convergent and divergent

James Gregory letter to Collins, February 15, 1671, $\arctan x$.

Knobloch \cite{knobloch2006}



\bibliographystyle{amsplain}
\bibliography{logarithms}

\end{document}