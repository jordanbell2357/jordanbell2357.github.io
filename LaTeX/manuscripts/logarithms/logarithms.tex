\documentclass{article}
\usepackage{amsmath,amssymb,mathrsfs,amsthm}
%\usepackage{tikz-cd}
\usepackage{hyperref}
\newcommand{\inner}[2]{\left\langle #1, #2 \right\rangle}
\newcommand{\tr}{\ensuremath\mathrm{tr}\,} 
\newcommand{\Span}{\ensuremath\mathrm{span}} 
\def\Re{\ensuremath{\mathrm{Re}}\,}
\def\Im{\ensuremath{\mathrm{Im}}\,}
\newcommand{\id}{\ensuremath\mathrm{id}} 
\newcommand{\var}{\ensuremath\mathrm{var}} 
\newcommand{\Lip}{\ensuremath\mathrm{Lip}}
\newcommand{\Res}{\ensuremath\mathrm{Res}}  
\newcommand{\GL}{\ensuremath\mathrm{GL}} 
\newcommand{\diam}{\ensuremath\mathrm{diam}} 
\newcommand{\sgn}{\ensuremath\mathrm{sgn}\,} 
\newcommand{\lcm}{\ensuremath\mathrm{lcm}} 
\newcommand{\sinc}{\ensuremath\mathrm{sinc}\,} 
\newcommand{\Si}{\ensuremath\mathrm{Si}} 
\newcommand{\supp}{\ensuremath\mathrm{supp}\,}
\newcommand{\dom}{\ensuremath\mathrm{dom}\,}
\newcommand{\upto}{\nearrow}
\newcommand{\downto}{\searrow}
\newcommand{\norm}[1]{\left\Vert #1 \right\Vert}
\newtheorem{theorem}{Theorem}
\newtheorem{lemma}[theorem]{Lemma}
\newtheorem{proposition}[theorem]{Proposition}
\newtheorem{corollary}[theorem]{Corollary}
\theoremstyle{definition}
\newtheorem{definition}[theorem]{Definition}
\newtheorem{example}[theorem]{Example}
\begin{document}
\title{Logarithms}
\author{Jordan Bell\\ \texttt{jordan.bell@gmail.com}\\Department of Mathematics, University of Toronto}
\date{\today}

\maketitle

Henry Briggs (1624) Arithmetica Logarithmica

S. J. Riguaud, ed., Correspondence of scientific Men of the seventeenth Century, I (Oxford, 1841)

Joost B\"urgi,
Arithmetische und geometrische Progress-Tabulen, sambt gr\"undlichem Unterricht, wie solche n\"utzlich in allerley Rechnungen zu gebrauchen, und verstanden werden sol (Prague, 1620)

E. Voellmy, Jost B\"urgi und die Logarithmen, in Beihefte zur Zeitschrift f\"ur Elemente der Mathematik, no. 5 (1948)

J. NEPER, Mirifici logarithmorum canonis constructio, Lyon, 1620

R. Ayoub, What is a Napierian Logarithm? American Mathematical Monthly,100 (1993) pp. 351-364.





\bibliographystyle{amsplain}
\bibliography{logarithms}

\end{document}