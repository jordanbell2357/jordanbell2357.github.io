\documentclass{amsart}
\usepackage{amssymb,latexsym,amsmath,amsthm,mathrsfs}
%\usepackage{graphicx}
\newcommand{\Res}{\mathrm{Res}}
\newcommand{\Ei}{\mathrm{Ei}}
\def\Re{\ensuremath{\mathrm{Re}}\,}
\def\Im{\ensuremath{\mathrm{Im}}\,}
\newtheorem{theorem}{Theorem}
\newtheorem{remark}[theorem]{Remark}
\newtheorem{lemma}[theorem]{Lemma}
\begin{document}
\title{E794}
\author{Jordan Bell}
\email{jordan.bell@gmail.com}
\address{Department of Mathematics, University of Toronto, Toronto, Ontario, Canada}
\date{\today}
\maketitle


Sandifer, no. 22 \cite{sandifer}

{\em Institutionum calculi integralis}, vol. 2, \S 1169.

Introductio, 38-46

 The method of integrating rational functions R(x) by decomposition into partial fractions
was first indicated by Leibniz in 1702 and 1703; however, Leibniz was not able to remove all
obstacles and in particular could not find the decomposition of x4 + a4 into real quadratic
factors (for a discussion in the context of the Fundamental Theorem of Algebra, see infra
no 58, note 14). This led him to the assertion that  R(x) dx could not always be expressed
using quadratures of the circle and the hyperbola (in modern terms: by algebraic, trigonometric
and logarithmic functions), but that an infinitely ascending order of independent,
transcendental quadratures of rational formulae would be needed.
Johann I Bernoulli (1704) circumvented Leibniz?s difficulties by a different method of integration
and succeeded in solving his examples; in his final paper on the subject, Op. CXIV
from 1719, he also shows how to proceed with partial fraction decomposition when the denominator
has complex zeros and multiple factors.
Euler discussed the decomposition into partial fractions in his Introductio (E. 101, t. I, �� 39?
45bis and �� 199?210: O. I/8, p. 41?58, 213?228) and in Part II, chapter 18 of the Institutiones
calculi differentialis (E. 212: O. I/10, p. 648?676). The integration of rational functions is described
in �� 56?86 of the first chapter of the Institutiones calculi integralis (E. 342: O. I/11,
p. 28?51).

Euler?s technique of integrating rational functions by partial fraction decomposition is masterfully
explained in his letter from September 1st, 1742 (R 236: O. IVA/3, p. 515?516). 


{\em Speculationes analyticae}, E475.

{\em Nova methodus fractiones quascumque rationales in fractiones simplices resolvendi}, E540.

{\em Theorema arithmeticum eiusque demonstratio}, E794.


November 9, 1762 letter from Euler to Goldbach.

Serre \cite[p.~56, ch. III, \S 6, Lemma 2]{serre}

Knuth \cite[p.~472, \S 1.2.3, Exercise 33]{knuth}.

E540

Waring, Phil Trans 69 1779 64-67

\[
\sum_{j=1}^n \frac{x_j^r}{\prod_{1 \leq k \leq n, k \neq j}} (x_j-x_k) = 
\begin{cases}
0&0 \leq r < n-1,\\
1&r=n-1,\\
\sum_{j=1}^n x_j&r=n.
\end{cases}
\]

\bibliographystyle{amsplain}
\bibliography{rational}

\end{document}