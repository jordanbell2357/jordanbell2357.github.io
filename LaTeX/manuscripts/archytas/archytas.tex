\documentclass{amsart}
\usepackage{amsmath,amssymb,graphicx,subfig,mathrsfs,amsthm,enumitem}
\usepackage[LGR,T1]{fontenc}
\newcommand{\textgreek}[1]{\begingroup\fontencoding{LGR}\selectfont#1\endgroup}
\newtheorem{theorem}{Theorem}
\newtheorem{lemma}[theorem]{Lemma}
\newtheorem{proposition}[theorem]{Proposition}
\newtheorem{corollary}[theorem]{Corollary}
\theoremstyle{definition}
\newtheorem{definition}[theorem]{Definition}
\newtheorem{example}[theorem]{Example}
\begin{document}
\title{Greek music theory and Archytas's theorem}
\author{Jordan Bell}
\email{jordan.bell@gmail.com}
\address{Department of Mathematics, University of Toronto, Toronto, Ontario, Canada}
\date{\today}

\maketitle

Athenaeus, {\em Deipnosophistae} 3.60 quotes the comic writer Damoxenus, {\em Syntrophoi} \cite[p.~215]{edmonds}:

\begin{quote}
What are akin\\
By octaves, fifths, or fourths I weave all in\\
At the proper `intervals' and suitably\\
To their `resolution.'
\end{quote}

octave: \textgreek{di`a pas\~wn},  ``through all'', ``through the whole'', e.g. 2 Corinthians 8:18. fifth: \textgreek{di`a pente},

fourth: \textgreek{di`a tett'aron}. intervals:
\textgreek{diast'hmata}. A \textgreek{di'asthma} is an interval, extent, extension, distance, line segment, space between things, gap; see LXX Ezek. 41:8, Acts 5:7,
Asclepiodotus, {\em Tactics} iv.1.

Plato, {\em Parmenides} \cite[p.~38]{allen}

Plato, {\em Timaeus} 35b--36b \cite[pp.~59--60]{barker}, about the Demiurge cutting off pieces from a long strip:

\begin{quote}
This is how he began to divide. First he took away one part from the
whole, then another, double the size of the first, then a third, hemiolic with
respect to the second and triple the first, then a fourth, double the second, then
a fifth, three times the third, then a sixth, eight times the first, then a seventh,
twenty-seven times the first. Next he filled out the double and triple intervals,
once again cutting off parts from the material and placing them in the
intervening gaps, so that in each interval there were two means, the one
exceeding [one extreme] and exceeded [by the other extreme] by the same part
of the extremes themselves, the other exceeding [one extreme] and exceeded [by
the other] by an equal number. From these links within the previous intervals
there arose hemiolic, epitritic and epogdoic intervals; and he filled up all the
epitritics with the epogdoic kind of interval, leaving a part of each of them,
where the interval of the remaining part had as its boundaries, number to
number, 256 to 243. And in this way he had now used up all the mixture from
which he cut these portions.
\end{quote}

intervals: \textgreek{diast'hmata}.
hemiolic: \textgreek{<hmiol'ian}, ``half as much again'',
epitritic: \textgreek{>epitr'iton}, ``a third as much again'',
epogdoic: \textgreek{>ep'ogdoos}, ``an eighth as much again''. The words \textgreek{>epitr'iton} and
\textgreek{>ep'ogdoos} are used in many extant writings to talk about lending money, e.g. Demonsthenes, {\em Against Polycles} 50.17.

Aristotle, {\em Rheotric} 3.10.7, 1411a \cite[p.~220]{rhetoric}:

\begin{quote}
And Moerocles said he was no more
wicked than--(naming someone of the upper class); for that person
was wicked ``at thirty-three and a third percent interest'' he himself
``at ten.''
\end{quote}

third-three and a third percent interest: \textgreek{>epitr'itwn t'okwn}

Aristotelian {\em Problemata} XIX.32, 920a \cite[pp.~198]{GMWI}:

\begin{quote}
Why is the {\em dia pas\={o}n} so called, instead of being called {\em di' okt\={o}} to correspond
with the number, like the {\em dia tettar\={o}n} and the {\em dia pente}? Is it because in old
times there were seven strings? Then Terpander took away {\em trit\={e}} and added
{\em n\={e}t\={e}}, and that is why it was called {\em dia pas\={o}n} and not {\em di' okt\={o}}, since there
were seven strings.
\end{quote}

Aristotelian {\em Problemata} XIX.35, 920a \cite[pp.~93]{barker}:

\begin{quote}
Why is the octave the finest concord? Is it because its ratios are
between terms that are wholes, while those of the others are not between
wholes? For {\em n\={e}t\={e}} is double {\em hypat\={e}}, so that if {\em n\={e}t\={e}} is 2, {\em hypat\={e}}
is 1, if {\em hypat\={e}} is
2, {\em n\={e}t\={e}} is 4, and so on invariably. But {\em n\={e}t\={e}} is the hemiolic of {\em mes\={e}}:
for the fifth, which is hemiolic, is not in whole numbers -- if, for instance, the smaller term
is 1, the greater is the same quantity and the half in addition. Hence wholes are
not being compared [or `combined', {\em synkrinetai}] with wholes, but parts are
added. The case is similar with the fourth, for epitritic ratio is so much and one
of the three in addition. Or is it because the completest concord is that
constituted out of both, and because the measure of melody\dots
\end{quote}

octave: \textgreek{di`a pas\~wn},
concord: \textgreek{sumfwn'ia}.

{\em n\={e}t\={e}}, {\em mes\={e}}, and {\em hypat\={e}} are names for strings on a seven string lyre.
West \cite[p.~219]{west}: from high to low pitch the seven strings are
{\em n\={e}t\={e}}, ``bottom'',
{\em paranet\={e}}, ``alongside-bottom'',
{\em trit\={e}}, ``third'',
{\em mes\={e}}, ``middle'',
{\em lichanos}, ``forefinger'',
{\em parhypat\={e}}, ``alongside-topmost'',
{\em hypat\={e}}, ``topmost''.
The interval from {\em hypat\={e}} to {\em mes\={e}} is the fourth.

Aristotle, {\em Metaphysics} V.15, 1020b--1021a, William of Moerbeke's translation:

\begin{quote}
Dicuntur autem prima quidem secundum numerum aut simpliciter aut determinate, ad ipsos aut ad unum. Ut duplum quidem ad unum numerus determinatus; multiplex vero secundum numerum ad unum, non determinatum autem, ut hunc aut hunc; emiolium autem ad subemiolium secundum numerum ad numerum determinatum; superparticulare autem ad subsuperparticulare secundum indeterminatos, ut multiplex ad unum.
\end{quote}

{\em emiolium}: \textgreek{<hmi'olion},
{\em subemiolium}: \textgreek{<ufhmi'olion},
superparticular: \textgreek{>epim'orion},
subsuperparticular: \textgreek{<upepim'orion}.
The term \textgreek{>epim'orion} is used in Galen, {\em De pulsuum differentiis}, K\"uhn 5.516 \cite{galeni}

Aristoxenus, {\em Elementa harmonica} 15 \cite[p.~136]{barker}:

\begin{quote}
Now that this is understood we must say what a note [{\em phthongos}] is. To
put it briefly, a note is the incidence of the voice on one pitch: for it is when
the voice appears to rest at one pitch that there seems to be a note capable
of being put into a position in a harmonically attuned melody [{\em melos
h\={e}rmosmenon}]. That, then, is the sort of thing a note is.

An interval [{\em diast\={e}ma}] is that which is bounded by two notes which do not
have the same pitch, since an interval appears, roughly speaking, to be a
difference between pitches, and a space capable of receiving notes higher than
the lower of the pitches which bound it, and lower than the higher of them.
Difference between pitches lies in their having been subjected to greater or
lesser tension.
\end{quote}

Aristoxenus, {\em Elementa harmonica} 21 \cite[p.~140]{barker}:

\begin{quote}
The tone is the difference in magnitude between the first two
concords. It is to be divided in three ways, since the half, the third and the
quarter of it should be considered melodic. All intervals smaller than these are
to be treated as unmelodic. Let the smallest of them be called the least
enharmonic diesis, the next the least chromatic diesis, and the greatest the
semitone.
\end{quote}

tone: \textgreek{t'onos},
concords: \textgreek{sumf'wnwn},
magnitude: \textgreek{m'egejos},
semitone: \textgreek{<hmit'onion}.

Pliny, {\em Natural History} 2.20, 84 \cite[pp.~226--229]{LCL330}:

\begin{quote}
But occasionally Pythagoras draws on the
theory of music, and designates the distance between
the earth and the moon as a whole tone, that between
the moon and Mercury a semitone, between Mercury
and Venus the same, between her and the sun a tone and a half, between the sun and Mars a tone (the same as the distance between the earth and the
moon), between Mars and Jupiter half a tone,
between Jupiter and Saturn half a tone, between
Saturn and the zodiac a tone and a half: the seven
tones thus producing the so-called diapason, {\em i.e.} a universal harmony;
in this Saturn moves in the Dorian mode, Jupiter in the Phrygian, and similarly
with the other planets--a refinement more entertaining
than convincing.

{\em Sed Pythagoras interdum ex musica ratione appellat tonimi quantum absit a terra luna, ab ea ad
Mercurium dimidium eius spatii, et ab eo ad Venerem
tantundem, a qua ad solem sescuplum, a sole ad
Martem tonum, id est quantum ad lunam a terra,
ab eo ad lovem dimidium, et ab eo ad Saturnum
dimidium, et inde sescuplum ad signiferum ; ita septem tonis effici quam diapason harmoniam vocant,
hoc est universitatem concentus ; in ea Saturnum
Dorio moveri phthongo, lovem Phrygio, et in reliquis
simiha, iucunda magis quam necessaria subtilitate.}
\end{quote}

Nicomachus, {\em Introduction to Arithmetic} I.19.1--3 \cite[p.~215]{nicomachus}:

\begin{quote}
The superparticular, the second species of the greater both naturally
and in order, is a number that contains within itself the whole of the
number compared with it, and some one factor of it besides. 

If this factor is a half, the greater of the terms compared is called
specifically sesquialter, and the smaller subsesquialter; if it is a
third, sesquitertian and subsesquitertian; and as you go on throughout
it will always thus agree, so that these species also will progress to
infinity, even though they are species of an unlimited genus. 

For it comes about that the first species, the sesquialter ratio, has as
its consequents the even numbers in succession from 2, and no other at
all, and as antecedents the triples in succession from 3, and no other.
These must be joined together regularly, first to first, second to second, 3
third to third -- $3:2$, $6:4$, $9:6$, $12:8$ -- and the analogous numbers to
the ones corresponding to them in position. 
\end{quote}

superparticular: \textgreek{>epim'orios}, 
sesquialter: \textgreek{<hmi'olios},
subsesquialter: \textgreek{<ufhmi'olios},
sesquitertian: \textgreek{>ep'itrit'os},
subsesquitertian: \textgreek{<upep'itritos}.

Nicomachus, {\em Introduction to Arithmetic} I.20.1 \cite[p.~220]{nicomachus}:

\begin{quote}
It is the superpartient relation when a number contains within
itself the whole of the number compared and in addition more than
one part of it; and more than one starts with 2 and goes on to all
the numbers in succession.
\end{quote}

superpartient: \textgreek{>epimer`hs}

Nicomachus, {\em Introduction to Arithmetic} II.29.4 \cite[p.~286]{nicomachus}:

\begin{quote}
Moreover $8 : 6$ or $12 : 9$ is the diatessaron, in sesquitertian ratio; $9 : 6$
or $12 : 8$ is the diapente in the sesquialter; $12 : 6$ is the diapason in the
double. Finally, $9 : 8$ is the interval of a tone, in the superoctave
ratio, which is the common measure of all the ratios in music, since it
is also the more familiar, because it is likewise the difference between
the first and most elementary intervals. 
\end{quote}

Nicomachus, {\em Enchiridion} Chap. 9 \cite[p.~261]{barker}:

\begin{quote}
Even the most ancient writers show agreement with what we have explained.
Their name for the octave is `{\em harmonia}', for the fourth `{\em syllaba}' (since it is the
first concordant combination [{\em syll\={e}psis}] of notes), and for the fifth `{\em di' oxeian}'
(since the fifth is continuous with the concord first generated and goes on
upwards); and the combination of both {\em syllaba} and {\em di' oxeian} together is the
{\em dia pas\={o}n}, and was given the name `{\em harmonia}' because it is the first concord
to be fitted together out of concords. Their agreement with what we have said
is made clear by Philolaus, the disciple of Pythagoras, who writes roughly as
follows in the first book of his {\em Physics}: pressure of time demands that we rest
content with just one witness, though many people say similar things in various
ways about the same subject. Philolaus' statement goes like this.

`The magnitude of harmonia is {\em syllaba} and {\em di' oxeian}. The {\em di' oxeian} is
greater than the {\em syllaba} in epogdoic ratio. From {\em hypat\={e}} to {\em mes\={e}}
is a {\em syllaba},
from {\em mes\={e}} to {\em neat\={e}} is a {\em di' oxeian}, from {\em neat\={e}} to
{\em trit\={e}} is a {\em syllaba}, and from {\em trit\={e}}
to {\em hypat\={e}} is a {\em di' oxeian}. The interval between {\em trit\={e}}
and {\em mes\={e}} is epogdoic, the
{\em syllaba} is epitritic, the {\em di' oxeian} hemiolic, and the {\em dia pas\={o}n} is duple. Thus
{\em harmonia} consists of five epogdoics and two dieses; {\em di' oxeian} is three
epogdoics and a diesis, and {\em syllaba} is two epogdoics and a diesis.'
\end{quote}

Heiberg and Menge \cite{euclidVIII}

von Jan \cite[p.~252]{MSG}: octave: \textgreek{di`a pas\~wn}, 
fourth: \textgreek{di`a tessarwn},
fifth: \textgreek{di`a p'ente},
magnitude: \textgreek{m'egejos}.

{\em syllaba}: span, interval from {\em hypat\={e}} to {\em mes\={e}}.
{\em di' oxeian}: ``across the high strings'', interval from {\em mes\={e}} to {\em net\={e}} \cite[p.~219]{west}.

Domninus, {\em Enchiridion} \cite{domninus}

Iamblichus, {\em in Nicomachi arithmeticam introductionem} \cite{iamblichus}

Klein \cite{klein}











Heath \cite{euclidII}

Euclid, {\em Elements} VII Def. 2, part: \textgreek{m'eros}, Def. 3, parts: \textgreek{m'erh}.
\textgreek{m'eros}: part, share, portion, e.g. John 19:23, four soldiers make four parts of the garments of Jesus,
and each soldier gets a part.















Barbera \cite{barbera}

Euclid, {\em Sectio canonis} 149 \cite[p.~192]{barker}:

\begin{quote}
Hence notes
that are higher than what is required are slackened by the subtraction of
movement and so reach what is required, while those which are too low are
tightened by the addition of movement, and so reach what is required. We
must therefore assert that notes are composed of parts, since they attain what
is required through addition and subtraction. Now all things that are composed
of parts are spoken of in a ratio of number with respect to one another, so that
notes, too, must be spoken of in a ratio of number to one another. Some
numbers are spoken of in multiple ratio with respect to one another, some in
epimoric ratio, and some in epimeric ratio, so that notes must also be spoken
of in these kinds of ratio to one another:  and of these, the multiple and the
epimoric are spoken of in relation to one another under a single name.
\end{quote}

Multiple and epimoric ratios are described by a word involving a single number, while epimeric
ratios are described by a word involving two numbers.

Euclid, {\em Sectio Canonis} Prop. 1, 150 \cite[p.~194]{barker}:

\begin{quote}
If a multiple interval put together twice makes some
interval, this interval too will be multiple.

Let there be an interval BC, and let B be a multiple of C,
and let B be to D as is C to B. I assert then that D is a
multiple of C. For since B is a multiple of C, C therefore
measures B. But B was to D as C was to B, so that C
measures D too. Therefore D is a multiple of C.
\end{quote}

Euclid, {\em Sectio Canonis} Prop. 1, 150 \cite[p.~194]{barker}:

\begin{quote}
If a multiple interval put together twice makes some
interval, this interval too will be multiple.

Let there be an interval BC, and let B be a multiple of C,
and let B be to D as is C to B. I assert then that D is a
multiple of C. For since B is a multiple of C, C therefore
measures B. But B was to D as C was to B, so that C
measures D too. Therefore D is a multiple of C.
\end{quote}

Euclid, {\em Sectio Canonis} Prop. 2, 151 \cite[p.~194]{barker}:

\begin{quote}
If an interval put together twice makes a whole that
is multiple, then that interval will also be multiple.

Let there be an interval BC, let B be to D as is C to B, and let D be a multiple
of C. I assert that B is also a multiple of C. For since D is a multiple of C, C
therefore measures D. But we have learned that where there are numbers in
proportion -- however many of them -- and where the first measures the last, it
will also measure those in between. Therefore C measures B, and B is therefore
a multiple of C.
\end{quote}

Euclid, {\em Sectio Canonis} Prop. 3, 152--153 \cite[p.~195]{barker}:

\begin{quote}
In the case of an epimoric interval, no mean number,
neither one nor more than one, will fall within it proportionally.

Let BC be an epimoric interval. Let DE and F be the smallest numbers in the
same ratio as are B and C. These then are measured only by the unit as a
common measure. Take away GE, which is equal to F. Since DE is the
epimoric of F, the remainder DG is a common measure of DE and F. DG is
therefore a unit. Therefore no mean will fall between DE and F. For the
intervening number will be less than DE and greater than F, and will thus divide
the unit, which is impossible. Therefore no mean will fall between DE and F.
And however many means fall in proportion between the smallest numbers,
there will fall in proportion exactly as many between any others which have the
same ratio. But none will fall between DE and F; nor will one fall between
B and C.
\end{quote}

Euclid, {\em Sectio Canonis} Prop. 4, 153--154 \cite[p.~196]{barker}:

\begin{quote}
If an interval which is not multiple is put together
twice, the whole will be neither multiple nor epimoric.

Let BC be an interval which is not multiple, and let B be to D as C is to B. I
say that D is neither a multiple nor an epimoric of C. First let D be a multiple
of C. Now we have learned that if an interval put together twice makes a whole
that is multiple, that interval is also multiple. Then B will be a multiple of C:
but it was not. Hence it is impossible for D to be a multiple of C. Nor is it an
epimoric: for within an epimoric interval there falls no mean in proportion.
But B falls within DC. Therefore it is impossible for D to be either a multiple
or an epimoric of C.
\end{quote}

Euclid, {\em Sectio Canonis} Prop. 8, 156--157 \cite[p.~198]{barker}:

\begin{quote}
If from a hemiolic interval an epitritic interval is
subtracted, the remainder left is epogdoic.

Let A be the hemiolic of B, and let C be the epitritic of B.
I say that A is the epogdoic of C. Since A is the hemiolic
of B, A therefore contains B and a half of B. Therefore
eight A's are equal to twelve B's. Again, since C is the
epitritic of B, C therefore contains B and a third of B.
Therefore nine C's are equal to twelve B's. But twelve B's
are equal to eight A's, and therefore eight A's are equal to
nine C's. A is therefore equal to C and an eighth of C, and
A is therefore the epogdoic of C.
\end{quote}

Euclid, {\em Sectio Canonis} Prop. 10, 158 \cite[p.~199]{barker}:

\begin{quote}
Let A be {\em n\={e}t\={e} hyperbolai\={o}n}, let B be {\em mes\={e}} and let C be {\em proslambanomenos}.
Then the interval AC, being a double octave, is concordant. It is therefore
either epimoric or multiple. It is not epimoric, since no mean falls
proportionally within an epimoric interval. Therefore it is multiple. Thus
since the two equal intervals AB and BC put together make a whole that is
multiple, AB is therefore multiple too.
\end{quote}

Euclid, {\em Sectio Canonis} Prop. 13, 160 \cite[p.~201]{barker}:

\begin{quote}
It remains to consider the interval of a tone, to
show that it is epogdoic.

We have learned that if an epitritic interval is subtracted from a hemiolic
interval, the remainder left is epogdoic. And if the fourth is taken from the
fifth, the remainder is the interval of a tone. Therefore the interval of a tone
is epogdoic.
\end{quote}

Euclid, {\em Sectio Canonis} Prop. 16, 161 \cite[p.~202]{barker}:

\begin{quote}
The tone will not be divided into two or more
equal intervals.

It has been shown that it is epimoric. Within an epimoric interval there falls
neither one nor more than one mean in proportion. Therefore the tone will
not be divided into equal intervals.
\end{quote}

Vat. gr. 221, 274--280

BNF grec 2456, 206




























Vitruvius, {\em De architectura} V.4.4 \cite{vitruvius}:

\begin{quote}
igitur intervalla tonorum et hemitoniorum tetrachordo in voce divisit natura finiitque terminationes eorum mensuris intervallorum quantitate, modisque certis distantibus constituit qualitates, quibus etiam artifices qui organa fabricant ex natura constitutis utendo comparant ad concentus convenientes eorum perfectiones.
\end{quote}

Vitruvius, {\em De architectura} V.4.8 \cite{vitruvius}:

\begin{quote}
ideoque et a numero nomina ceperunt, quod cum vox constiterit in una sonorum finitione ab eaque se flectens mutaverit et pervenerit in quartam terminationem, appellatur diatessaron, in quintam diapente [in sextam diapason, in octavam et dimidiam diapason et diatessaron, in nonam et dimidiam diapason et diapente, in XII disdiapason].
\end{quote}

Plutarch, {\em De animae procreatione in Timaeo} 17 ({\em Moralia} XIII, 1020--1021) \cite[pp.~303--309]{LCL427}:

\begin{quote}
What the ``leimma'' is and what is Plato's meaning you will perceive more clearly, however, after having first been reminded briefly of the customary statements in the Pythagorean treatises. For an interval in music is all that is encompassed by two sounds dissimilar in pitch; and of the intervals one is what is called the tone, that by which the fifth is greater than
the fourth. The harmonists think that this, when divided in two, makes two intervals, each of which they call a semitone; but the Pythagoreans denied that it is divisible into equal
parts and, as the segments are unequal, name the lesser of them ``leimma'' because it falls short of the half. This is also why among the consonances the fourth is by the former 
made to consist of two tones and a semitone and by the latter of two and a ``leimma.'' Sense-perception seems to testify in favour of the harmonists but in favour of the
mathematicians demonstration, the manner of which is
as follows. It has been found by observation with instruments that the octave has the duple ratio and the fifth the sesquialteran and the fourth the sesquitertian and the tone
the sesquioctavan. It is possible even now to test the truth of this either by suspending unequal weights from two strings or by making one of two pipes with equal cavities double
the length of the other, for of the two pipes the larger will sound lower as hypat\^e to n\^et\^e and of the strings the one stretched by the double weight will 
sound higher than the other as n\^et\^e to hypat\^e. This is an octave. Similarly too, when lengths and weights of three to two are taken, they will produce the fifth and of four to three
the fourth, the latter of which has sesquitertian ratio and the former sesquialteran. If the inequality of the weights or the
lengths be made as nine to eight, however, it will produce an interval, that of the tone, not concordant but tuneful because,to put it briefly, the notes it gives, if they are struck 
successively, sound sweet and agreeable but, if struck together, harsh and painful, whereas in the case of consonances, whether they be struck together or alternately, the sense
accepts with pleasure the combination of sounds. What is more, they give a rational demonstration of this too. The reason is that in a musical scale the octave is composed of the 
fifth and the fourth and arithmetically the duple is composed of the sesquialter and the sesquiterce, for twelve is four thirds of nine and half again as much as eight and twice as
much as six. Therefore the ratio of the duple is composite of the sesquialter and the sesquiterce just as that of the octave is of the fifth and the fourth, but in that case the fifth is 
greater than the fourth by a tone and in this the sesquialter greater than the sesquiterce by a sesquioctave. It is apparent, then, that the octave
has the duple ratio and the fifth the sesquialteran and the fourth the sesquitertian and the tone the sesquioctavan,
\end{quote}

18 \cite[pp.~309--315]{LCL427}:

\begin{quote}
Now that this has been demonstrated, let us see whether the sesquioctave is susceptible of being
divided in half, for, if it is not, neither is the tone. Since nine and eight, the first numbers producing the sesquioctavan ratio, have no intermediate interval but between them when both
are doubled the intervening number produces two intervals, it is clear that, if these intervals are equal, the sesquioctave is divided in half. But now twice nine is eighteen and twice
 eight sixteen; and between them these numbers contain seventeen, and one of the intervals turns out to be larger and the other smaller, for the former is eighteen seventeenths and
 the second is seventeen sixteenths. It is into unequal parts, then, that the sesquioctave is divided; and, if this is, the tone is also. Neither of its segments, therefore, when it is divided, 
 turns out to be a semitone; but it has rightly been called by the mathematicians ``leimma.''  This is just what Plato says god in filling in the sesquiterces with the sesquioctaves leaves
  a fraction of each of them, the ratio of which is 256 to 243. For let the fourth be taken as expressed by two numbers comprising the sesquitertian
ratio, 256 and 192; and of these let the smaller, 192, be placed at the lowest note of the tetrachord and the larger, 256, at the highest. It is to be proved that, when this is filled in with
 two sesquioctaves, there is left an interval of the size that numerically expressed is 256 to 243. This is so, for, when the lower note has been raised a tone, which is a sesquioctave,
 it amounts to 216; and, when this has been raised again another tone, it amounts to 243, for the latter exceeds 216 by 27 and 216 exceeds 192 by 24, and of these 27 is an eighth
 of 216 and 24 an eighth of 192. Consequently, of these three numbers the largest turns out to be sesquioctavan of the intermediate and the intermediate sesquioctavan of
the smallest; and the interval from the smallest to the largest, i.e. that from 192 to 243, amounts to an interval of two tones filled in with two sesquioctaves. When this is subtracted, 
 however, there remains of the whole as an interval left over what is between 243 and 256, that is thirteen; and this is the very reason why they named this number ``leimma.'' 
So I, for my part, think that Plato's intention is most clearly explained by these numbers.
\end{quote}

19 \cite[pp.~315--317]{LCL427}:

\begin{quote}
As terms of the fourth, however, others put the high note at 288 and the low at 216 and then determine proportionally those that come next, except that they take the ``leimma'' to be
 between the two tones. For, when the lower note has been raised a tone, the result is 243 and, when the higher has been lowered a tone, it is 256, for 213 is nine eighths of 216 and
 288 nine eighths of 256, so that each of the two intervals is that of a tone and there is left what is between 243 and 256; and this is not a semitone but
is less, for 288 exceeds 256 by 32 and 243 exceeds 216 by 27 but 256 exceeds 243 by thirteen, which is less than half of both the excesses 32 and 27. Consequently it turns out
that the fourth consists of two tones and a ``leimma,'' not of two tones and a half. Such, then, is the demonstration of this point. As to the following point, from what has been said
before it is not very difficult either to see why, after Plato had said that there came to be intervals of three to two and of four to three and of nine to eight, when saying that those of four
to three are filled in with those of nine to eight he did not mention those of three to two but omitted them. The reason is that the sesquialter {$\langle$}is greater than{$\rangle$} the 
sesquiterce by the sesquioctave {$\langle$}so that with the sesquioctave's{$\rangle$} addition to the sesquiterce the sesquialter is filled in as well.
\end{quote}

Pseudo-Plutarch, {\em De Musica} \cite{LCL428}

Ptolemy, {\em Tetrabiblos} I.13 \cite{LCL435}:

\begin{quote}
Of the parts of the zodiac those first are familiar75 one to another which are in aspect. These are the ones which are in opposition, enclosing two right angles, six signs, and 180 degrees; those which are in trine, enclosing one and one-third right angles, four signs, and 120 degrees; those which are said to be in quartile, enclosing one right angle, three signs, and 90 degrees, and finally those that occupy the sextile position, enclosing two-thirds of a right angle, two signs, and 60 degrees.

We may learn from the following why only these intervals have been taken into consideration. The explanation of opposition is immediately obvious, because it causes the signs to meet on one straight line. But if we take the two fractions and the two superparticulars most important in music, and if the fractions one-half and one-third be applied to opposition, composed of two right angles, the half makes the quartile and the third the sextile and trine. Of the superparticulars, if the sesquialter and sesquitertian be applied to the quartile interval of one right angle, which lies between them, the sesquialter makes the ratio of the quartile to the sextile and the sesquitertian that of trine to quartile. Of these aspects trine and sextile are called
harmonious because they are composed of signs of the same kind, either entirely of feminine or entirely of masculine signs; while quartile and opposition are disharmonious because they are composed of signs of opposite kinds.
\end{quote}

Ptolemy, {\em Harmonics} 11 \cite[pp.~284--285]{barker}:

\begin{quote}
Perception accepts as concords the fourth, as it is called, and the fifth, the
difference between which is called a tone [{\em tonos}], and the octave; and also the
octave and a fourth, the octave and a fifth, and the double octave. Let us ignore
the concords greater than these for present purposes. The theory of the
Pythagoreans rules out one of these, the octave and a fourth, by following its
own special assumptions, ones which the leaders of the school put forward on
the basis of ideas such as the following. They laid down a first principle of their
method that was entirely appropriate, according to which equal numbers
should be associated with equal-toned notes, and unequal numbers with
unequal-toned; and from this they argued that just as there are two primary
classes of unequal-toned notes, that of the concords and that of the discords,
and that of the concords is finer, so there are also two primary distinct classes
of ratio between unequal numbers, one being that of what are called `epimeric'
or `number to number' ratios, the other being that of the epimorics and
multiples; and of these the latter is better than the former on account of the
simplicity of the comparison, since in this class the difference, in the case of
epimorics, is a simple part, while in the multiples the smaller is a simple part
of the greater. For this reason they fit the epimorics and multiple ratios to the
concords, and link the octave to duple ratio [2:1], the fifth to hemiolic [3:2],
the fourth to epitritic [4:3].
\end{quote}

Ptolemy, {\em Harmonics} 12 \cite[p.~285]{barker}:

\begin{quote}
But they do not adopt the magnitude put together from
the octave and the fourth, because it makes the ratio of 8 to 3, which is neither
epimoric nor multiple.
\end{quote}

Ptolemy, {\em Harmonics} 12 \cite[p.~286]{barker}:

\begin{quote}
Since the tone is accordingly shown to be
epogdoic, they reveal that the half-tone is unmelodic, because no epimoric ratio
divides another proportionately as a mean, and melodic magnitudes must be in
epimoric ratios.
\end{quote}

Ptolemy, {\em Harmonics} 16 \cite[p.~290]{barker}:

\begin{quote}
Next after the epitritic ratio, those closer to equality will be those that come
together to compose it and whose excesses are commensurable, that is, the
epimoric ratios that are smaller than these, and following the concords in
excellence come the melodies, such as the tone and all those that come together
to compose the smallest of the concords; so that to these we should fit the
epimoric ratios that are smaller than the epitritic.
\end{quote}

If $A>B$,
the distance of the ratio $A:B$ from equality is the ratio $(A-B):B$.

Ptolemy, {\em Harmonics} 22--23 \cite[pp.~296--297]{barker}:

\begin{quote}
But as soon as the tone has been shown to be epogdoic and the fourth
epitritic, it is obvious from this very fact that reason [{\em logos}] entails that the
difference by which the fourth exceeds the ditone, called the {\em leimma}, is smaller
than a half-tone. For if the first number is taken which is capable of displaying
the proposition, which is of 1,536 units, its epogdoic is the number of 1,728
units, and the epogdoic of that is the number of 1,944 units, which will
obviously have the ratio of a ditone to that of 1,536 units. Now the number of
2,048 units is the epitritic of that of 1,536: hence the {\em leimma} is in the ratio of
2,048 units to 1,944. But if we also take the epogdoic of 1,944, we shall have
the number of 2,187 units, and the ratio of 2,187 to 2,048 is greater than that
10 of 2,048 to 1,944. For 2,187 exceeds 2,048 by more than a fifteenth part of 2,048
and less than a fourteenth. But 2,048 exceeds 1,944 by more than a nineteenth
part of 1,944 and less than an eighteenth. Hence the smaller segment of the
third tone is included within the fourth in addition to the ditone, so that the
magnitude of the {\em leimma} comes to less than a half-tone, and the whole fourth
is less than two and a half tones. And the ratio of 2,048 to 1,944 is in fact
the same as that of 256 to 243.
\end{quote}

{\em leimma}: \textgreek{le\~imma}, remainder, residue, intermission, deficiency

Ptolemy, {\em Harmonics} 24 \cite[pp.~297--298]{barker}:

\begin{quote}
For since
neither the epogdoic ratio nor any other of the epimorics is divided into two
equal ratios, and the most nearly equal ratios that make the epogdoic are
17:16 and 18:17, the half-tone would be in the ratio that lies somehow between
these, that is, greater than 18:17 and smaller than 17:16. Now 15 is greater
than a seventeenth part of 243 and less than a sixteenth part, so that when
these, 243 and 15, are added together, the half-tone is in a ratio very close to
that of 258 to 243. The ratio of the {\em leimma} was shown to be that of 256 to 243;
and hence the half-tone will differ from the {\em leimma} in the ratio of 258 to 256,
that is, 129:128.
\end{quote}

Bacchius Geron, {\em Introduction to the Art of Music} 8 \cite[pp.~273--274]{bacchius}:

\begin{quote}
Which is the smallest of the intervals? -- The diesis.

What is a diesis? -- The smallest degree we can by nature melodically relax or tense
 the voice.
 
What is twice a diesis? -- A semitone.
  
What is twice a semitone? -- A tone.
\end{quote}

Bacchius Geron, {\em Introduction to the Art of Music} 68 \cite[p.~287]{bacchius}:

\begin{quote}
 In how many senses do we say {\em tonos} exists in music? -- In two senses: one in terms of height of pitch, the other in terms
 of interval.
 
 What is {\em tonos} in terms of height of pitch?
 -- That one person sings higher or lower than another, or that a
 higher or lower instrument relates to the tuning of another by
 some interval, whatever it may be.
 
 What is {\em tonos} in terms of interval?
 -- The degree by which the consonance of the fifth is bigger than
 that of the fourth.
\end{quote}

Cleonides \cite{strunk}

Gaudentius, {\em Harmonica Introductio} 13,15 \cite{strunk}, \cite[p.~327]{MSG}

Aristides Quintilianus, {\em De Musica} I.7, 10--11 \cite[p.~410]{barker}:

\begin{quote}
The term `interval' is used in two ways, one general and one
specific. In general, an interval is any magnitude bounded between limits: in
the usage specific to music, an interval is a magnitude of sound circumscribed
by two notes. Of intervals some are composite, some incomposite, the
incomposite being those bounded by successive notes, the composite those
bounded by notes which are not successive, and capable ofbeing divided in a
melody into several intervals. Of these intervals the smallest, so far as their
use in melody is concerned, is the enharmonic diesis, followed -- to speak
rather roughly -- by the semitone, which is twice the diesis, the tone, which is
twice the semitone, and finally the ditone, which is twice the tone.
\end{quote}

Aristides Quintilianus, {\em De Musica} III.1, 95--96 \cite[pp.~495--496]{barker}: 

\begin{quote}
Since they also wanted to find the ratios of the intervals that are smaller than
the ditone, of the tone, that is, and the semitone and the diesis, they proceeded
as follows. They knew that the fifth exceeds the fourth by a tone. Hence they
put together a sequence of three numbers, of which the first stands to the
second in epitritic ratio and to the third in hemiolic. The numbers are these:
6, 8, 9. Now 8 is in epitritic ratio with 6, 9 is in hemiolic ratio with 6, and 9
is in epogdoic ratio with 8. But it was agreed that the fifth exceeds the fourth
by a tone: and they therefore concluded that the ratio of the excess of the fifth
over the fourth, which is a tone, is in epogdoic ratio.

They wanted to know also the ratio of the semitones. Since there is no
number between 8 and 9, they doubled the original terms to make 16 and 18,
and found that between them there lies the number 17. By this number, they
said, the tone is divided into semitones. They found, however, that this was not
a division into equal parts, but into a larger and a smaller, since 18 stands to
17 in a ratio which is not equal to that of 17 to 16, but is smaller than it. This
is why in the notation by semitones there is set out a double series of symbols,
so that when the smaller semitone is required to sound, we ascend or descend
to the nearer symbol, and to the further one when the larger semitone is needed.
For this reason, people of ancient times called this interval the {\em leimma}, because
its exact value [{\em isot\={e}s}, lit. `equality'] is hard to determine.
\end{quote}

Censorinus, {\em De Die Natali} 10.7 \cite[p.~18]{censorinus}: according to Aristoxenus the octave is 6 tones, while
according to the Pythagoreans the octave is 5 tones and 2 semitones:

\begin{quote}
so Pythagoras and the mathematicians, who pointed
out that two semi-tones do not necessarily add up to a full tone.
\end{quote}

Theon of Smyrna, {\em Expositio} \cite{dupuis}

Theon of Smyrna, {\em Expositio} 51 \cite[pp.~214--215]{barker}:

\begin{quote}
Notes are in concord with one another if, when one or the other is struck on
a stringed instrument the other one also sounds with it, through some sort of
kinship and sympathy: under the same conditions, if both are struck together,
a sweet and agreeable sound arising from the mixture is heard. Of notes
attuned in a continuous series, those fourth in order from one another, firstly,
are in concord with one another, and form the concord which for that very
reason is called the fourth [{\em dia tessar\={o}n}, lit. `through four']; secondly, those
fifth in order form the concord of a fifth [{\em dia pente}, `through five']; and next,
the notes bounding both those concords, and being eighth in order from one
another, form the concord of an octave [{\em dia pas\={o}n}, `through all'], so called
because, to begin with, the first and lowest note of the eight-stringed {\em lyra}, called
{\em hypat\={e}}, in relation to the last and highest, which is {\em n\={e}t\={e}}, was found to contain
this same concord, in correspondence. And when music had been augmented
and instruments had become many-stringed and many-noted, by the addition
of several other notes, both lower and higher, to the original eight, nevertheless
the names of the first concords were preserved, `through four', `through five',
and `through all'.
\end{quote}

Theon of Smyrna, {\em Expositio} 53 \cite[p.~215]{barker}:

\begin{quote}
The most clearly recognisable part, and the measure of the range we have
mentioned and of every interval within it, is what is called the interval of a tone
[{\em toniaion diast\={e}ma}], just as the foot length is the measure of the strictly
`spatial' distance [{\em topikon diast\={e}ma}] which bodies in movement traverse. The
interval of a tone is most recognisable because it is the difference between the
first and most recognisable concords; for the fifth exceeds the fourth by a
tone.
\end{quote}

Theon of Smyrna, {\em Expositio} 56 \cite[p.~217]{barker}:

\begin{quote}
It seems that Pythagoras was the first to have identified the concordant
notes in their ratios to one another, those at the fourth in epitritic ratio [4:3],
those at a fifth in hemiolic [3:2], those at an octave in duple [2:1]; those at an
octave and a fourth in a ratio of 8 to 3, which is multiple-epimeric, since it is
duple and double-epitritic; those at an octave and a fifth in triple ratio [3:1],
those at a double octave in quadruple [4:1]; and of the other attuned notes,
those bounding the tone in epogdoic ratio [9:8], those bounding what is now
called the semitone but was then called the diesis in a ratio of number to
number, that of 256 to 243.
\end{quote}

Theon of Smyrna, {\em Expositio} 59 \cite[p.~218]{barker}:

\begin{quote}
Some people thought it proper to derive these concords from weights, some
from magnitudes, some from movements and numbers, and some from vessels.
Lasus of Hermione, so they say, and the followers of Hippasus of Metapontum,
a Pythagorean, pursued the speeds and slownesses of the movements, through
which the concords arise\dots Thinking that\dots in numbers, he constructed ratios
of these sorts in vessels. All the vessels were equal and alike. Leaving one
empty and filling the other up to halfway with liquid, he made a sound on each,
and the concord of the octave was given out for him. Then, again leaving one
of the vessels empty, he poured into the other one part out of the four, and
when he struck it the concord of the fourth was given out for him, as was the
fifth when he filled up one part out of the three. The one empty space stood
to the other in the octave as 2 to 1, in the fifth as 3 to 2, and in the fourth as
4 to 3.
\end{quote}

concords: \textgreek{sumfwn'ias},
magnitudes: \textgreek{megej\~wn},
octave: \textgreek{di`a pas\~wn},
fourth: \textgreek{di`a tess'arwn},
fifth: \textgreek{di`a p'ente}.  

Theon of Smyrna, {\em Expositio} 66--68 \cite[pp.~221--222]{barker}:

\begin{quote}
The ancients took the first interval of the voice to be the tone, for they did
not recognise the semitone and diesis. The tone was found to be in epogdoic
ratio [9:8] in contrivances involving discs, vessels, strings, {\em auloi}, weights, and
several other things: for nine items in relation to eight made one hear the
interval of a tone. The reason why the tone is the first interval is that down as
far as this interval the voice in its movement keeps the hearing free from error,
but after that the hearing is not able to grasp the interval accurately. After all,
people dispute about the interval next in order, the so called semitone, some
saying that it is a complete half-tone, others that it is a {\em leimma}. Now the fourth,
which is epitritic [4:3], is filled up by the tone, that is by the epogdoic interval,
as follows. It is agreed by everyone that the fourth is greater than a ditone but
smaller than a tritone. But Aristoxenus says that it consists of two and a half
complete tones, while Plato says that it consists of two tones and what is called
the {\em leimma}. He says that this {\em leimma} is not incapable of being expressed, and
that it is in the ratio of number to number in which 256 stands to 243. That is
the interval, and the difference [{\em hyperoch\={e}}, `excess'] is 13. It is found as
follows. The number 6 could not be the first term, since it has no eighth,
through which its epogdoic could come into being. Nor indeed could it be 8, for
while it has an epogdoic, 9, the 9 in its turn has no epogdoic. But one must take
an epogdoic of an epogdoic, since the epitritic interval of a fourth is greater
than a ditone. So we take the fundamental epogdoic, 8 and 9. Taking 8 with
itself we find 64; then we take 8 with 9 and 72 is produced; then 9 with itself
and 81 is produced. Then again let each of these be taken three times: three
times 64 will be 192, three times 72 will be 216, and three times 81 is 243. Thus:
8, 9, 64, 72, 81, 192, 216, 243. Then we add, beyond 243, the epitritic based
upon 192, which is 256. Thus the setting-out [{\em ekthesis}] is as follows: the
fundamental epogdoic, 8, 9; the second epogdoics, 64, 72, 81; the two that are
third epogdoics of one another, 192, 216, 243; and let there be also the epitritic
of 192, which is 256. This last will be the epitritic filled out by two tones and
the so called {\em leimma}.
\end{quote}

Porphyry, {\em Commentary on Ptolemy's Harmonics} \cite{porphyry}

Chalcidius l: semitone \cite{calcidius}

Favonius Eulogius, {\em Disputatio de somnio Scipionis} \cite{favonius}

Hyginus Gromaticus, {\em De limitibus constituendis} \cite[p.~146]{campbell}:

\begin{quote}
to the planet Saturn is a distance that the Greeks call a {\em hemitonion}; from Saturn to
Jupiter another {\em hemitonion}; from Jupiter to Mars a {\em tonon}; from Mars to the sun is
three times as far as the distance from the pole to Saturn, that is, a {\em trihemitonion}; from
the sun to Venus is as far as from Saturn to Jupiter, a {\em hemitonion}; from Venus to
Mercury a {\em hemitonion}; from Mercury to the moon the same, a {\em hemitonion}; from the
moon to the earth as far as the distance from the pole to Jupiter, a {\em tonon}. In this way
\end{quote}

Macrobius, {\em Commentary on the Dream of Scipio} \cite[pp.~188--189]{macrobius} 2.1.21--23:

\begin{quote}
[21] The ancients chose to call the interval smaller than a tone a semitone, but this must not be taken to mean half a tone
any more than we would call an intermediate vowel a semivowel. [22] The tone by its
very nature cannot be divided equally: inasmuch as it originates in the number nine, which cannot be equally divided,
the tone refuses to be divided into two halves; they have merely called an interval smaller than a full tone a semitone,
but it has been discovered that there is as little difference between it and a full tone as the difference
between the numbers 256 and 243. [23] The early Pythagoreans called the semitone {\em diesis}, but those who came later
decided to use the word {\em diesis} for the interval smaller than the semitone. Plato called the semitone
{\em leimma}.
\end{quote}

Martianus Capella, {\em The Marriage of Philology and Mercury} \cite{martianusII} 757 \cite[p.~293]{martianusII}:

\begin{quote}
The Greeks call numbers multiplied {\em pollaplasioi}
[multiples]; numbers divided {\em hypopollaplasioi} [submultiples];
numbers exceeding other numbers by a member or members {\em epimoroi}
[superparticulars]; and numbers smaller than other numbers by a 
member or members {\em hyperepimorioi} [subsuperparticulars] \dots.
\end{quote}

Augustine, {\em De musica} Chap. 9 \cite{taliaferro}

Boethius, {\em De Institutione Musica} \cite{bower}

Friedlein \cite{friedlein}

Boethius, {\em De Institutione Musica} III.11 \cite[pp.~451--470]{archytas}, Archytas of Tarentum, Fragment A19: 

\begin{quote}
Archytas' proof that a superparticular ratio cannot be divided into equal parts and a critique of it.

A superparticular ratio is not able to be divided into equal parts by a mean proportional placed between. This indeed will be proved securely later. For the proof which Archytas puts forth is too loose. At any rate, his proof is of the following sort:

Let there be, he says, a superparticular ratio $A:B$. I take $C:D + E$ as the least numbers in the same ratio. Therefore, since $C:D + E$ are the least numbers
in the same ratio and are superparticulars, the number $D + E$ exceeds the number $C$ by one of its own parts and by a part of $C$. Let this be $D$. I say that $D$ will not be a number but a unit. For, if $D$ is a number and is a part of $D + E$, the number $D$ measures the number $D + E$, wherefore it will also measure the number $E$, by which it comes about that it also measures $C$. Therefore, the number $D$ will measure both the numbers $C$ and $D + E$, which is impossible. For those numbers which are the least in the same ratio to any other numbers whatever are prime to one another. Therefore, $D$ is a unit. Therefore, the number $D + E$ exceeds the number $C$ by a unit. Wherefore, no number falls in the middle which divides that ratio equally. By which it comes about that, in the case of those numbers which have the same ratio as these, it is also not possible that a number be placed in the middle, which divides the same proportion equally.
And indeed, according to the reasoning of Archytas, it is for this reason that no term, which equally divides the ratio, falls in the middle of a superparticular ratio, namely that the least numbers in the same ratio differ by a unit alone, as if indeed the least numbers in a multiple ratio [did not] also have allotted to them the same difference of a unit, although we see more multiple ratios besides those which are grounded in roots, in which a middle term can be fit, which divides the same proportion equally. But one who has examined my arithmetical books diligently understands this very easily. It must be added that it does turn out in the way Archytas thinks in the superparticular ratio alone; however, it must not be asserted universally. Now let us turn to what follows.
\end{quote}

John Lydus, {\em De Mensibus} March 36, \cite[p.~94]{lydus}:

\begin{quote}
Ptolemy, in his {\em Harmonics}: The numbers have been defined through which there arises a concordant harmony in all those things which are in agreement and attunement with each other. And nothing at all is able to harmonize with anything except by virtue of these numbers. They are as follows:

\textgreek{epitritos hmiolios opondeios diplasios triplasios tetraplasios}.
\end{quote}

Proclus, {\em Commentary on Plato's Timaeus} \cite{timaeum3II}

Proclus, {\em Commentary on Plato's Republic} \cite{festugiereII}

Cassiodorus, {\em Institutions of Divine and Secular Learning} \cite{cassiodorus}

Isidore of Seville, {\em Etymologies} III.xvii.1, III.xix.2, III.xix.5 \cite[p.~96]{isidore}:

\begin{quote}
Music has three parts, that is, the harmonic ({\em harmonicus}), the rhythmic ({\em rhythmicus}), and the metric ({\em metricus}). The harmonic part is that which differenti- ates high and low sounds.

Voice ({\em vox}) is air beaten ({\em verberare}) by breath, and from this also words ({\em verbum}) are named.

Diastema ({\em diastema}) is the appropriate vocal interval between two or more sounds. 
\end{quote}

Jordanus Nemorarius, {\em De elementis arithmetice artis} IX.61 \cite[pp.~193--194]{jordanus}:

\begin{quote}
Nulla superparticularis proportio in aliquot equales proportiones est
divisibilis.

Sit inter $a$ et $d$ proportio superparticularis. Ponanturque unus vel duo medii, si
possibile est, sintque $b$ et $c$. Sint autem minimi eiusdem proportionis $e f g h$. Et
quia $h$ ad $e$ sicut $d$ ad $a$, tunc $h$ continebit $e$ et eius partem que sit $z$. Numerabit
igitur $z$ et $h$, erunt ergo $h$ et $e$ commensurabiles. Quod est contrarium premissis.
Aliter. Sit numerus denominans partem $t$ addaturque ei unitas et sit totus $y$. Et
quia cuiuslibet numeri pars est unitas ab ipso dicta, erit $y$ ad $t$ sicut $d$ ad $a$. Itaque
per xii\textsuperscript{am} quarti proveniet impossibile.
\end{quote}

Campanus VIII.8 \cite{campanusI}

{\em Quaestiones in Musica}, pars secunda, chap. 7 \cite[p.~77]{steglich}:

\begin{quote}
Quae superparticularis.

Superparticularis vocatur inaequalitatis, in qua numerus ad numerum
comparatus habet in se totum minorem et eius aliquam partem. Qui, si
minoris habeat medietatem, vocatur sesqualter, ut tres ad duos, ut VIIII
ad sex. Si partem terciam, vocatur sesquitercius, sicut octo ad VI, sic XII
ad VIIII. Si quartam, vocatur sesquiquartus. Si quintam, sesquiquintus.
Sed maior est ad minorem suum pars media quam tertia,
tercia quam quarta, quarta quam quinta. Et sic in infinitum pars a maiore
numero denominata ipsa decrescit. Unde adtende hanc inaequalitatem de
illo quantitatis esse genere, cuius maiorem partem constat in infinitum
decrescere.
\end{quote}

{\em Quaestiones in Musica}, pars secunda, chap. 8 \cite[p.~77]{steglich}:

\begin{quote}
Quae superpartiens.

Superpartiens inaequalitatis appellatur, ubi numerus ad alium comparatus
inferiorem numerum totum continet in se ut super hoc aliquas partes eius,
aut duas, aut III, aut IIII, aut V, aut quotlibet alias, ut verbia gratia tria
continentur a quinque cum aliis duobus et vocatur superbipartiens. Quatuor
continentur a VII cum tribus partibus suius et vocatur supertripartiens, et
sic deinceps. 
\end{quote}

{\em Quaestiones in Musica}, pars secunda, chap. 23 \cite[p.~89]{steglich}:

\begin{quote}
De divisione semitonii.

[I]tem Philolaus minima semitonii spatia talibus diffinitionibus includit:
Diesis, inquit, [est spatium, quo maior est sesquitertia proportio duobus
tonis. Comma vero] est spatium, quo maior est sesquioctava proportio
duabus diesibus, id est duobus semitoniis minoribus. Scisma est dimidium
commatis, diascisma vero dimidium dieseos, id est semitonii minoris.
Ex quibus illud colligitur, quoniam tonus quidem dividitur principaliter
in semitonium minus atque apotomen, dividitur etiam in duo semitonia
et comma. Quo fit, ut dividatur in quatuor diascismata et comma.
Integrum vero dimidium toni, quod est semitonium, constat ex duobus
diascismatibus, quod est unum semitonium minus, et scismate, quod est
dimidium commatis. Quoniam enim totus tonus ex duobus semitoniis
minoribus et commate coniunctus est, si quis id integre dividere velit,
faciet unum semitonium minus commatisque dimidium. Sed unum semitonium
minus dividitur in duo diacismata, dimidium vero commatis unum
scisma. Recte igitur dictum est, integre dimidium tonum in duo diascismata
atque unum scisma posse partiri. Quo fit, ut integrum semitonium
minore semitonio uno scismate differe videatur. Apotome autem
a minore semitonio duobus scismatibus differt. Differt enim commate,
sed duo scismata unum perficiunt comma.
\end{quote}

Guillaume d'Auberive, {\em Regule arithmetice} XV \cite[p.~109]{gersh}:

\begin{quote}
The superparticular number is that which, when compared with
another, contains the whole of the smaller number together with same part
of the latter. If it has its half, it is called ``sesquialter'', if its third part ``sesquitertius'',
if its fourth ``sesquiquartus'', the form of the superparticular
also proceeding to infinity with the infinite extension of such names.
\end{quote}

Roger Bacon, {\em Summulae dialectices} para. 73--75 \cite[pp.~43--44]{MST47}

\begin{quote}
[73] {\em Multiple} is what contains something else added more than once. Its lowest species are doubleness, tripleness, and so on. And, likewise, other species can be proposed, but it is more appropriate to give concrete names for abstract ones, since abstract names are not imposed [for the names of species].

[74] A {\em superparticular number} is a number related to another number and containing all of that plus some part of it, namely, any one part.
And under it are these kinds of parts: {\em sesquialter}, which contains the whole [of another number] plus its other part or half, as three stands to two; and {\em sesquitertius}, which contains something plus its third part, as four stands to three; and {\em sesquiquartus}, which contains something plus its fourth part, as five stands to four; and so on.

[75] A superpartient number is one that contains something, [another number,] plus some parts of it. Its species are: {\em superbipartient}, which contains a number plus two-thirds of it, as five stands to three; or two-fifths of it, as seven stands to five; and so on for all uneven numbers taken
in order. But this is not so with respect to even numbers, because, should it contain two halves, or two-fourths, or two-sixths, or two-eighths, it will not be a superbipartient number, but a multiple number or a superparticular number. Should it contain two halves, it will be a multiple number, because it would be a double multiple, as four stands to two. Should it contain a number [plus] two-fourths of it, then there will be a sesquialter proportion, as six stands to four; should it contain a number plus two-sixths of it, then there will be a sesquitertius proportion, as eight stands to six; and should it contain a number plus two-eighths of it, then there will be a sesquiquartus proportion, as ten stands to eight. Another species of superpartient number
is what is called {\em supertripartient}. It is such that it contains something and three-fourths of it, as seven stands to four; or three-fifths of it, as eight stands to five; or three-sevenths of it, as ten stands to seven; or three-eighths of it, as eleven stands to eight. But not three-sixths, because then there will be a sesquialter proportion, as nine stands to six; nor three-ninths of it, because then there will be a sesquitertius proportion, as twelve stands to nine. Another species of superpartient numberis said to be {\em superquadripartient number}, which contains something and four-fifths of it, as nine stands to five; or four-sixths of it, as ten stands to six; or four-sevenths of it, as eleven stands to seven; or four-ninths of it, as thirteen stands to nine.
But not four-eighths, because then there will be a sesquialter proportion; and one is to keep such things in mind from here on. Similarly, other species of superpartient number can be added without limit in accord with multiplicity. But it would take a long time to enumerate them and they are clear to one who diligently inquires. Thus they are to be omitted.
\end{quote}

Roger Bacon, {\em Summulae dialectices} para. 84 \cite[p.~49]{MST47}

\begin{quote}
[84] One should note that the names `relative', `related', and `relation' differ. A relation ({\em relatio}) is the possession itself, like paternity, doubleness, etc. A relative ({\em relativum}) is something taken concretely from such, e.g., a father, double. A thing related ({\em relatum}) is that whose possession is designated by a relative or a relation, like Socrates, to whom paternity or that he be a father accrues.
\end{quote}

Aquinas, {\em in Metaph.}, on {\em Metaphysics} 3, 997a15--997a34:

\begin{quote}
396. But sometimes it happens to be the function of some science to demonstrate from certain principles that a thing is so, and sometimes it happens to be the function of some science to demonstrate the principles from which it was demonstrated that a thing is so, sometimes to the same science and sometimes to a different one.

An example of its being the function of the same science is seen in the case of geometry, which demonstrates that a triangle has three angles equal to two right angles in virtue of the principle that the exterior angle of a triangle is equal to the two interior angles opposite to it; for to demonstrate this belongs to geometry alone. And an example of its being the function of a different science is seen in the case of music, which proves that a tone is not divided into two equal semitones by reason of the fact that a ratio of 9 to 8, which is super-particular, cannot be divided into two equal parts. But to prove this does not pertain to the musician but to the arithmetician. It is evident, then, that sometimes sciences differ because their principles differ, so long as one science demonstrates the principles of another science by means of certain higher principles.
\end{quote}

Johannes de Muris, {\em Quadripartitum numerorum} III \cite[p.~270]{huillier}

Prosdocimo de' Beldomandi, {\em Musica speculativa} II.15 \cite[pp.~198]{herlinger}






\section{Conclusions}
Knorr \cite{knorr}

van der Waerden \cite{waerden}

Fowler \cite{fowler}


\bibliographystyle{amsplain}
\bibliography{archytas}

\end{document}